\documentclass[11pt]{scrartcl}
%\usepackage{amsmath, amssymb, amsthm}
\usepackage{enumerate}
%\usepackage{todonotes}
%\usepackage{bbm}
%\usepackage{calligra}
%\usepackage{mathrsfs}
% Don't indent paragraphs, leave some space between them
\usepackage{parskip}

\usepackage{bm}
%\usepackage{hyperref}
%\hypersetup{
%  colorlinks=true,
%  linktoc=all,
%}% Already defined in Evan.sty
\usepackage[sexy]{../ddist/evan} %% Evan.sty to make it look pretty

%% Useful when using report/book to have deeper ramifications

%%\usepackage{titling}
%%\newcommand{\subtitle}[1]{%
%%  \posttitle{%
%%    \par\end{center}
%%    \begin{center}\large#1\end{center}
%%    \vskip0.5em}%
%%}
%%\setcounter{tocdepth}{4}
%%\setcounter{secnumdepth}{4}

%% Figures stuff
\usepackage{import}
\usepackage{xifthen}
\usepackage{pdfpages}
\usepackage{transparent}

%\newcommand{\incfig}[1]{%
%  \def\svgwidth{\columnwidth}
%  \import{./media/}{#1.pdf_tex}
%}
\pdfsuppresswarningpagegroup=1

\DeclareTextFontCommand{\emph}{\bfseries\em}

%% Maths Useful commands
\usepackage{epsfig}
\def\acts{\mathrel{\reflectbox{$\righttoleftarrow$}}}
\newcommand{\nt}[1]{\textrm{#1}}


%\newcommand{\ii}[1]{\item #1}
% \DeclareMathOperator{\null}{null}
\newcommand{\ee}{\bm{e}}
\newcommand{\e}{\bm{e}}
\newcommand{\uu}{\bm{u}} 
\newcommand{\vv}{\bm{v}} 


%% Theorems
%%% Styles
\usepackage{thmtools}



\newtheorem{thm}{Theorem}[section]
\newtheorem{lem}[thm]{Lemma}
\newtheorem{prop}[thm]{Proposition}
\newtheorem{cor}{Corollary}[thm]

%\theoremstyle{definition}
%\newtheorem{definition}[thm]{Definition}

\newtheorem{ex}[thm]{Example}

\theoremstyle{rem}
\newtheorem*{rem}{Remark}

%\usepackage{xcomment}
\usepackage[
  active,
  generate=alts_notes,
  extract-env={definition},
  extract-cmd={section,subsection}
]{extract}
\usepackage{tikz-cd}

\title{Algebra}
\subtitle{Taught by Alex Bartel}
\author{Allan Perez\thanks{Original Notes} \and Lewis Foskett\thanks{Adaptations and Additional Content}}
\date{Last Updated: October 31, 2023}
\let\ltxcup\cup

%\xcomment{theorem,section,subsection}
%\xcomment{definition,section,subsection}
%\xcomment{theorem,proof,\\section,\\subsection}
%\renewcommand{\xcommentchar}{section}
%\newxcomment[]{|\meta{section}|}
\begin{document}
%\frontmatter
% \renewcommand{\thepage}{\arabic{page}}
 \maketitle
 
% Notable Changes:
% Changed commutative diagram for THM 9.7
% Provided alternative direct proof for proving injectivity in THM 9.7
% Made proofs nicer/ more concise for THM 8.8
% Fixed all the proofs for THM 8.7 and fixed mistake in the theorem statement
% Added proofs for THM 1.3
% Added proof for all subgroups of metabelian subgroups being metabelian
% Rewrote proof of the second isomorphism theorem to make it clearer.
% Added an extension to Example 10.2
% Provided helpful commutative diagram for the third isomorphism theorem with a remark explaining it.
% Removed bloat from third isomorphism theorem, made it easier to understand
% Added aside on free groups after group actions
% Fixed theorem 13.4 part 2, proof was borked.
% Most group action proofs were waffle or unreadable, fixed
% Rewrote proof of Cauchy's Thm and added supporting definitions / lemmas to make it more understandable

\tableofcontents
\newpage
\listoftheorems
\newpage

\section{Lecture 1} 
\begin{definition}
  A \emph{group} is a pair $(G,*)$, where $G$ is a set and $*:G\times G\to G$ is a binary
  operation, that satisfies the following axioms:
  
       (G1) Associativity: for any $g,h,k\in G$,
      \[(g*h)*k = g*(h*k);\]
      
       (G2)  Existence of identity: there exists $e\in G$ such that for any $g\in G$,
      \[e*g=g*e = g;\]
      
       (G3) Existence of inverses: for any $g\in G$ there exists $h\in G$ s.t. 
      \[g*h=h*g=e.\]
  
  \label{group}
\end{definition}
% When notating groups, we omit the binary operation if it is well-understood.
% Note that the second axiom implies that the identity element is unique, since assuming
% that two different identity elements exist, it automatically leads to a contradiction
% saying that the two identity elements are the same. Moreover, note that closure is implied
% by the definition of binary operation.

We should note that we are assuming $G$ is closed under the binary operation, but 
%(this is something that must be verified about an operation on element). 
in general we should verify that the product of two elements of $G$ stays in $G$. Often we will omit the binary operation so that $g*h$ is shortened to just $gh$. 

Also, from the second axiom we can immediately deduce that the identity in a group must be unique. We will typically use $e$ or $1$ to denote the identity element of a group, but when there is the possibility of ambiguity we will use $1_{G}$ to denote the identity element in the group $G$.

% \begin{theorem}
%   Let $G$ be a group with identity $e\in G$ and an element $g\in G$ with left inverse
%   $h\in G$ and right inverse $h'\in G$ s.t. $hg = gh'=e$. Then $h=h'$.
% \end{theorem}

\begin{theorem}
    Let $G$ be a group and $g \in G$ with left inverse $h$ and right inverse $h'$, i.e. $$h*g = g*h'=e \in G.$$ Then $h=h'$. 
\end{theorem}

\begin{proof}
  % We have $hg=gh'$ so $hhg=hgh'$, which implies by associativity $hhg=he=h=eh'=h'$.

  We have that
  $$h = h*e = h*(g*h') = (h*g)*h' = e*h' = h',$$
  by direct application of the group axioms.
\end{proof}
Note that the above also implies that the inverse is unique, and we need not make the distinction between left and right inverses in a group. We denote the inverse of an element $g\in G$ as $g^{-1}.$


% \begin{theorem}
%   Let $G$ be a group. Then
%   \begin{itemize}
%       \ii $\forall g\in G$ one has $(g^{-1})^{-1} = g$.
%       \ii $\forall g,h,k\in G$ the following are equivalent:
%       \begin{itemize}
%         \ii $gh = gk$
%         \ii $h=k$
%         \ii $hg=kg$
%       \end{itemize}
%       \ii $\forall  g,h\in G$, we have $(gh)^{-1} = h^{-1} g^{-1}$.
%   \end{itemize}
% \end{theorem}
\begin{proposition}
     Let $G$ be a group and $g,h \in G$. Then we have
     \begin{enumerate}
        \item $(g^{-1})^{-1} = g$,
        \item $(gh)^{-1} = h^{-1} g^{-1}.$
    \end{enumerate}
\end{proposition}
\begin{proof}
    % \begin{enumerate}
        (1) The inverse of $g^{-1}$ is an element $x \in G$ that satisfies $xg^{-1} = g^{-1}x = e.$ 
        
        Taking the second equality, we can left multiply by $g$ to get
        $$
        gg^{-1}x = g \implies x = g.
        $$
        % and we can notice that $g$ is exactly the element that satisfies this condition.
        (2) We can verify that
        \begin{align*}
            (gh)( h^{-1} g^{-1}) &= g(h h^{-1})g^{-1} \\
            &= g e g^{-1} \\
            &= g g^{-1} \\
            &= e,
        \end{align*}
        meaning that the element $h^{-1} g^{-1}$ is the unique inverse of $gh$.
    % \end{enumerate}
\end{proof}

\begin{proposition}
    Let $G$ be a group and $g,h,k \in G$. Then the following are equivalent:
    \begin{enumerate}
        \item $gh = gk,$
        \item $h=k,$
        \item $hg=kg$.
    \end{enumerate}
\end{proposition}

\begin{proof}
    If $gh = gk,$ left multiplying by $g^{-1}$ and applying the inverse axiom we see that $h = k$. Similarly, given that $h=k$, we can left multiply by $g$ so that $gh = gk.$

    Applying a similar argument for right multiplication shows that all three statements are equivalent.
\end{proof}

% \begin{proof}
%     Let $g,h,k \in G$
%     \begin{itemize}
%         \ii By definition of an inverse $g^{-1} (g^{-1})^{-1}=1_G=(g^{-1})^{-1}g^{-1}$ and left multiplying (or right) by $g$ gives $(g^{-1})^{-1} = g$.
%         \ii $gh=gk \iff h=k \iff hg=kg$ by left cancellation and right multiplication (backwards implication follows by right cancellation and left multiplication)
%         \ii $h^{-1} g^{-1} gh = 1_G = gh h^{-1} g^{-1}$ so by definition $h^{-1} g^{-1}$ is the inverse of $gh$ so $h^{-1} g^{-1} = (gh)^{-1}$
%     \end{itemize}
% \end{proof}

\begin{definition}
  A group $G$ is \emph{abelian} if the group operation is commutative, i.e. $\forall g,h\in G$ we have
  $gh=hg$.
  \label{abelianGroup}
\end{definition}
When a group is abelian, we often use additive notation to denote the group operation, as opposed to the typical multiplicative notation (e.g. powers).


\begin{definition}
  We say a group is \emph{finite} or \emph{countable} if the underlying set is finite or countable,
  respectively.
\end{definition}


\subsection{Symmetric groups}
% Since using examples of symmetric groups is recurrent in this course, I thought convenient
% to include the Baker course notes on this topic. 

% The n-th symmetric group $S_n$ consists of all bijections (or permutations) of the set
% ${1,2,\cdots , n}$ under function composition. The group has $n!$ elements (permutations).
% Each of these elements can be represented by the disjoint cycle notation. Each $\sigma\in
% S_n$ can be written as a composition of permutations $\rho$, of the form
% \[r_1\to r_2=\rho(r_1) \to \cdots \to r_l=\rho(r_{l-1})\]
% Omitting the $r_k\to r_k=\rho(r_k)$. The usual name of this $l$-cycle $\rho$ is
% $(r_1, r_2, \cdots, r_l)$. It's usual to omit $1$-cycles, since they correspond to the
% identity element. Every element in $S_n$ can be written as a composition of $l$-cycles
% (disjoint) of length $1\leq l \leq n$. This factorisation of $\rho$ is unique.

One recurring class of groups we will consider are symmetric groups.

\begin{definition} 
    Let $X$ be a set. The \emph{symmetric group} of $X$, denoted $\operatorname{Sym}(X)$ is the set of all permutations of the elements of $X$. Symbolically, $$\operatorname{Sym}(X) = \{\sigma: X \to X \mid \sigma \text{ is a bijection}\}.$$ 

    If $X = \{1, 2, \ldots, n\}$ for some $n \in \NN$, we say $\operatorname{Sym}(X)$ is the symmetric group on $n$ letters, denoted $S_n$.
\end{definition}

\begin{remark}
    Symmetric groups are indeed groups under composition of permutations. It is straightforward to check that the composition of permutations is indeed a permutation, and that the remaining group axioms hold.
    We will later see that all finite groups can be understood within the context of a symmetric group.
\end{remark}

%cycle stuff...
Any permutation can be decomposed into a product of disjoint cycles. For example, the bijection 
% $$\{1,2,3,4,5\} \longmapsto \{2,3,1,5,4\}$$ 
$$
\begin{pmatrix}
1 & 2 & 3 & 4 & 5\\
2 & 3 & 1 & 5 & 4
\end{pmatrix}
$$

is equivalent to the composition of cycles $(123)(45).$ The notation $(123)$ is shorthand for the permutation sending 1 to 2, 2 to 3, and 3 to 1.

\section{Lecture 2 - 22 Sep 2021}
 \begin{definition}
   A subgroup of a group G, $H\subset G$, is a subset of $G$ with 
   \begin{itemize}
       \ii It contains the $G$ group identity element.
       \ii For all $a,b\in H$ we have $ab\in H$
       \ii for any $a\in H$ we have $a^{-1}\in H$.
   \end{itemize}
   \label{subgroup}
 \end{definition}
Basically, $H$ is a subset that is a group on its own, with the same identity and same
group operation. We write $H<G$ or $H\leq G$ to denote a subgroup.

\begin{proposition}[Test for subgroup]
  Let $H$ be a subset of the group $G$. Then $H$is a subgroup iff the following hold:
  \begin{enumerate}
    \item $H\neq\emptyset$
    \item If $x,y\in H$ then $x^{-1}y\in H$.
  \end{enumerate}
  \label{subgroupTest}
\end{proposition}
\begin{proof}
  The first holds since $e\in H$. 

  ($\implies$) Since we have that $H$ is a subgroup, then we must have $x\in H \implies
  x^{-1}\in H$ and $x^{-1}y\in H$ for some $y\in H$.

  ($\Leftarrow$) We have that $H$ is non-empty and that for any $x,y\in H$ we have
  $x^{-1}y\in H$. We claim that $H$ contains the identity, is closed, associative, and
  every element has an inverse. Note that the identity is in $H$, since when $x=y$, the
  result follows. Also, note that it's closed, since $x^{-1}y=h\implies
  h^{-1}=y^{-1}x\in H$. This also implies that every element has an inverse. Finally, to
  show associativity, we claim $a,b,c\in H$ we have $(ab)c=a(bc)$. Note that given the
  above, we have $(ab)c=a(bc)\iff a'(ab)c=a'a(bc)=bc=(a'a)bc \iff a'(ab)c=(a'a)bc= a'abc$.
\end{proof}

\begin{definition}
  A group $G$ is called cyclic if $\exists g\in G$ such that $G=\left\{ g^n : n\in\ZZ
  \right\}$. If $G$ is cyclic, then an element $g$ as above is called a generator, and we
  say $G$ is generated by $g$, $G=<g>$.
  \label{cyclicGroup}
\end{definition}

\begin{theorem}
  Every cyclic group is abelian.
\end{theorem}
\begin{proof}
  $g^n g^m = g^{n+m} = g^{m+n} = g^m g^n$
\end{proof}

\begin{theorem}
  All subgroups of a cyclic group are cyclic.
\end{theorem}
\begin{proof}
  Let $H<G$ where $G$ is a cyclic subgroup. We claim $\exists h\in H$ s.t. $H=\left\{
  h^n | n\in\ZZ  \right\}$. When $H=\left\{ 1_G \right\}$, the proof is trivial. If
  $\exists n\in\ZZ$ s.t. $g^n\in H$, assume such $n$ is the lowest possible, without
  loss of generality. Now assume $g^a\in H$ for some $a=qn+r$ for $q,r\in \ZZ$ and
  $0\leq r < n$. Then $g^a=(g^n)^q g^r$, and we know $g^nq\in H$ since $H$ is supposed
  to be a group on its own. Then we must have $g^r\in H$, but $n$ was the smallest
  possible power of $g$, so $r=0$, i.e. $g^r=e$, and hence $g^a = g^nq$, i.e. $H$ must be
  cyclic.
\end{proof}


\begin{definition}
  The order of an element $g\in G$ of a group $G$, written $|g|$, is the least positive
  integer $n$ s.t. $g^n=e$. If such $n$ doesn't exist, it has infinite order. The order
  of a group $G$, $|G|$, is the cardinallity of the underlying set.
  \label{orderGroup}
\end{definition}

\begin{theorem}
  Let $G$ be a group, and let $g\in G$. The order of $g$ is the same as the order of the
  subgroup $<g> < G$.
\end{theorem}
\begin{proof}
  If $|g|=\infty$, then $g^i=g^k \iff i=k$. If $|g|=m<\infty$, then $<g>=\left\{ 1,g,g^2,
  \cdots, g^{m-1} \right\}$.
\end{proof}

\begin{theorem}
  Let $G$ be a group, and let $g\in G$, and $n\in\ZZ$. Then $g^n=e$ iff $n$ is multiple
  of $|g|$.
\end{theorem}
\begin{proof}
  $n\Big | |g| \implies g^n = e$: We have $n=k|g|$ for some integer $k$. Then
  $g^n=(g^{|g|})^k = e^k = e$.

  $g^n=e \implies n\Big | |g|$: Let $n=|g|m + r$ for integers $m,r$ and $0\leq r < |g|$,
  we have $g^n=g^{|g|m} g^r = e$. We know $g^{|g|}=e$, so $g^n = g^r=e$. However, this can
  be true only if $r=0$, hence $n\Big | |g|$.
\end{proof}

\subsection{Cyclic groups structure}
For a cyclic group of infinite order, the structure of the group resembles
$(\ZZ,+)$, whereas for a cyclic group of finite order $n$, the structure resembles
$(\ZZ_n,
+\mod n)$.

\section{Lecture 3 - 28 Sep 2021}
Today we cover more on cyclic groups and multuplication tables.
\begin{theorem}
  Let $g\in G$ have order $n$, finite, and fix some $k\in\ZZ$. Then
  \[|g^k|= \frac{n}{d} \iff d=\gcd{(n,|k|)}\]
\end{theorem}
\begin{proof}
  The order of $g^k$ is the smallest integer $i$ s.t. $g^{ik}=e$. However, we know that the
  order of $g$ is $n$, so we must have $g^{mn}=e=g^{ik}$, so we must have
  $i=\frac{nm}{k}$, and such $i$ must be the smallest possible. Since we have fixed $n$,
  we have that $i=\frac{n}{d}$ only if $d=\gcd{(n,k)}$, as required.
\end{proof}

\begin{cor}
  Let $G=\langle g \rangle$ be a cyclic group generated by $g$, and have order $n<\infty$. Fix
  $k\in\ZZ$. Then
  \[G=\langle g^k \rangle \iff \gcd{(n,k)}=1\]
\end{cor}
\begin{proof}
  We have that $g^k$ is a generator for $G$ $\iff \langle g^k \rangle = \langle g \rangle \iff |g^k|=n \iff
  \gcd{(k,n)}=1$.
\end{proof}


\subsection{Multiplication table}
Next multuplication tables are explained and how they encode the entire structure of the
group, where the (i,j) entry encodes the result of $g_ig_J$. However, we should be aware
that different multiplication tables can encode essentially the same group. An example is
given, considering the cyclic group of order $4$, where two elements' labels are swapped
and seeing how the multuplication table differs.
\begin{theorem}
  In a multiplation table, each row contains every element only once, and so does each
  column. 
\end{theorem}
\begin{proof}
  We have $g_ig_j = g_i g_k \iff g_j=g_k$, and the same argument can be used for the
  columns. For countable groups, this holds (even if it's infinite).
\end{proof}

Next we're shown an example of how there are only 2 groups of order 4, namely Klein's
4-group and the typical cyclic group of order 4 isomorphic of $\ZZ_4$.

\section{Lecture 4}
\subsection{Cosets}
\begin{definition}
  Let $G$ be a group and $H$ a subgroup of $G$. Let $g\in G$. The \emph{left coset} of $H$
  containing $g$ is the set $gH=\left\{ gh \mid h\in H \right\}$. Similarly, the \emph{right coset} of $H$
  containing $g$ is the set $Hg=\left\{ hg \mid h\in H \right\}$.
  \label{coset}
\end{definition}

\begin{example}
  Let $G$ be the symmetric group $S_4$ and let $H=\langle(1,2,3)\rangle=\left\{
e,(1,2,3), (1,3,2) \right\}$ and let $g=(1,4)$. Then we have 
\[gH = \left\{ (1,4), (1,2,3,4), (1,3,2,4) \right\} \subset G.\]
Note that this is not a group itself since it doesn't have the identity; it is simply a
subset of $G$. Moreover, the right coset
\[Hg = \left\{ (1,4), (1,4,2,3), (1,4,3,2) \right\}\]
is a different coset that only has in common $g$. So $gH\neq Hg$.
\end{example}

\begin{theorem}
  Let $G$ be a group, $H\leq G$, and $g,g'\in G$. Then one has $gH=g'H \iff g'^{-1}g\in H$.
\end{theorem}
\begin{proof}
  ($\implies$) Suppose $gH=g'H$. Then there
  exists $h\in H$ s.t. $g=g'h$, and so $g'^{-1}g=h \implies g'^{-1}g\in H$, as required.

  ($\impliedby$) Let $gh\in gH$. We have that $g'^{-1}g = h'$, for some $h'\in H,$ so $g = g'h',$ meaning $gh = g'h'h \in g'H,$ so $gH \subseteq g'H$. We can use a symmetric argument to conclude that $g'H \subseteq gH,$ so $gH = g'H$.
  % Suppose we have $g'^{-1}g\in H$.  Then $ \exists h\in H \text{ s.t. } g'^{-1}g =h$, so $g=g' h$. Similarly, $\exists h' \in H$ s.t. $g'=gh'.$ Therefore $gH=g'H.$ 
\end{proof}

\begin{example}
  Let $G=(\RR,+)$ and $H=\ZZ$. We see that for $x,y \in \RR$, we have $x+\ZZ=y+\ZZ$ if and only if
  $y-x\in \ZZ$, i.e.  $x$ and $y$ differ by an integer.
\end{example}
% \begin{example}
%   A very illustrating example is the group $G=(\RR^3, +)$ and $H=\langle(1,0,0),(0,1,0)\rangle$ (note
%   $H$ is just the $xy$-plane). Let $g,g'\in G$ be vectors. Following the above theorem,
%   we have $g+H=g'+H \iff g-g'\in H$, so we must have $g_3=g'_3$ (so that $g$ and $g'$
%     cancel out in the z-axis and land on the xy-plane).
% \end{example}

\begin{corollary}[Absorption rule]
  Let $G$ be a group and $H\leq G$, and $g\in G$. Then $gH=H \iff g\in H$.
\end{corollary}

\begin{corollary}
  The relation $\sim$ on $G$ defined by $g\sim g'$ iff $gH=g'H$ is an equivalence relation.
  % and so this partitions the whole group $G$ into equivalence classes, which
  % are all the cosets of $H$.
  In particular, the equivalence classes are all the left cosets of $H$ in $G$.
  \label{leftCosetsEqRel}
\end{corollary}

\begin{proof}
     The relation is trivially reflexive, symmetric and transitive by properties of equality.
\end{proof}

\begin{remark}
  Note that this gives the natural conception that cosets partition the group --- in a sense it allows us to look at the group on a larger scale. However, this partition is only another set; it need not necessarily conserve group structure.
\end{remark}


\begin{theorem}
  All cosets of $H$ have the same cardinality. That is $\forall g\in G$, $|gH|=|H|$.
  \label{cosetsCardinality}
\end{theorem}
\begin{proof}
  % We claim there exists a bijection $H\to gH$. This bijection can be $h\mapsto gh$.
  % This is surjective almost by definition, since $gH=\left\{ gh \mid h\in H \right\}$.
  % Injectivity rises from $gh=gh'\implies h=h'$.

  Fix $g \in G$ and let $\varphi:H \longrightarrow gH$ be the mapping $h \longmapsto gh$. We claim that $\varphi$ is a bijection.

  This is surjective since every element of $gH$ can be recovered as $gh$ for some $h \in H$.

  If we suppose $gh = gh',$ then clearly $h = h'$ after we left multiply by $g^{-1}$, so $\varphi$ is injective.
\end{proof}

\section{Lecture 5 - 29 Sep 2021}
\subsection{Counting groups and Lagrange}
\begin{definition}
  Let $G$ be a group and $H$ a subgroup. The set of left cosets of $H$ in $G$ is denoted
  by $G/H$, also called the set of equivalence classes generated by the equivalence
  relation defined in Corollary \ref{leftCosetsEqRel} . Similarly the set of right cosets of
  $H$ in $G$ is denoted by $H\setminus G$.
  \label{cosets}
\end{definition}
Everything we have said about left cosets applies symmetrically to right cosets, since we can define a natural injective map $Ha\mapsto a^{-1}H$.
However, it is important to note that the way $H$ partitions $G$ into left and right cosets is different in general. 
%of a given subgroup $H$ may be different between left and right cosets.
%This is because we can define a natural injective map $Ha\mapsto a^{-1}H$.

\begin{theorem}[Lagrange's theorem]
  Let $G$ be a finite group, and let $H$ be a subgroup. Then $|H|$ divides $|G|$
\end{theorem}

\begin{proof}
    We have that $G$ is the disjoint union of its left cosets. Since $\forall g \in G, |gH| = |H|,$ we can observe that $|G|$ must be a multiple of $|H|$; in other words $|H|$ divides $|G|$.
\end{proof}

\begin{definition}
  The number of (left) cosets of $H$ in $G$ is called the index of $H$ in $G$, written
  $[G:H]$.
\end{definition}
The index may be infinite.

\begin{corollary}
  Let $G$ be a finite group and $H$ be a subgroup. Then we have that $|G|=|H| \cdot [G:H]$.
  \label{lagrange}
\end{corollary}
\begin{proof}
  By Theorem \ref{cosetsCardinality}, we have that $G$ is the union of cosets, which are
  disjoint. I.e. $G=\bigcup_{g\in G} gH\implies |G|=|gH|[G:H]=|H|[G:H]$.
\end{proof}

\begin{theorem}
  The number of left cosets of $H$ in $G$ equals the number of right cosets. Note that $G$
  may be an infinite-order group.
\end{theorem}
\begin{proof}
  Define a bijection $G/H\to H\setminus G$ as $gH\mapsto Hg^{-1}$. To show injectivity,
  suppose $Hg^{-1}=Hg'^{-1}$ for some $g,g'\in G$. Then $\exists h\in H$ s.t.
  $hg^{-1} = g'^{-1}$, so $g' = gh^{-1}\in gH \implies g'H=gH$. Surjectivity follows from
  the fact that every element $g \in G$ has an inverse, since the left coset $g^{-1}H$ will map to $H\left(g^{-1}\right)^{-1} = Hg$.
\end{proof}

Note that we may have infinite-order groups but with finite index (finite number of
cosets).

\begin{corollary}
  Let $G$ be a finite group, and let $g\in G$. Then the order of $g$ divides $|G|$.
  \label{lagrangeConsequence}
\end{corollary}
\begin{proof}
  Let $H=\langle g \rangle$ so that $H\leq G$. By Lagrange we have $|G|=|H|\cdot[G:H]=|g| \cdot [G:H]$, so $\lvert g\rvert \text{ divides } |G|$ since $[G:H] \in \NN$.
\end{proof}

\begin{example}
  A particular example where Lagrange's theorem is useful is the following. Let
  $n\in\NN$. The set $\left\{ i\in \left\{ 1,2,\cdots,n-1 \right\} : gcd(i,n)=1 \right\}$
  forms a group under multiplication mod $n$. This group is denoted by
  $(\ZZ/n\ZZ)^{\times}$. Note that if $n$ is prime, then the group has order $n-1$. Let us
  work out the order of $3\in (\ZZ/7\ZZ)^{\times}$. By Lagrange, we know that $|3|$ must
  divide $|G|=6$, so $|3|$ can be either $2,3,$ or $6$. It's not hard to see that
  $3^2=9=2\mod 7$ and $3^3=2*3\mod 7=6\mod 7$, so we must have $|3|$ has indeed order $6$. Since the group
  is order $6$ and we found an element of order $6$, we also discovered that the group is
  cyclic ($3$ is a generator).
\end{example}

\section{Lecture 6 - 4 Oct 2021}
\subsection{Normal subgroups and Quotients}
This lecture we will start the revision on normal subgroups and quotients.
Recall from last year that, if $V$ is a vector space and $U$ is a subspace, then the set
of cosets $\left\{ v+ U: v\in V \right\}$ is a vector space in its own right. For example,
the addition of cosets is defined by $(v+U)+(v'+U)=(v+v')+U$. The lecturer gave as an
intuitive example the vector space $\RR^2$, where we take a line throught the origin as a
subspace, and then show how adding some vector form $\RR^2$ just shifts the line around
the plane (no longer through the origin, so not a subspace), and adding these two lines we
add the two elements used to create that line ($v,v'$), find the new vector ($v+v'$), and
create a new line.

In order to be sure that this addition of cosets works, we need to show that it's well
defined (i.e. it preserves under addition of different representative of the same coset).


Let us try to generalise this idea of reconstructing the parent structure from quotients,
but to any group. Let $G$ be a group and $H$ a subgroup of $G$. We would like the set of
left cosets of $H$ in $G$ to \emph{inherit} the group structure -- spoiler alert a
homomorphism!. We could try to define for $x,y\in G$, $xH yH=(xy)H$ (observe the
homomorphic structure). We need to check that this is well defined, as above. Let $e,h\in
H$ denote the identity element and some element of $H$, respectively. Let $y\in G$ denote
some element not in $H$. We want that $(eH)(yH)=yH = (hH)(yH)=(hy)H \iff y^{-1}hy\in H$. 

Hence, a necessary and sufficient condition for multiplication of cosets
to be well defined is that, for every $g\in G$ and $h\in H$ we get $ghg^{-1}\in H$.
Another way of stating this condition is,
\[\forall g\in G, gHg^{-1}\subseteq H \iff H\subseteq g^{-1}H g \iff H=g^{-1}H g\]

\begin{definition}
  Let $G$ be a group. A subgroup $H$ of $G$ is called normal, written $H\triangleleft G$
  or $H\trianglelefteq G$, if $\forall g\in G$ we have $gHg^{-1}=H$.
  \label{normalSubgroup}
\end{definition}

Hence for the set of left (or right) cosets of $H<G$ to have group structure with the
above operation, we require $H$ to be normal in $G$.

\begin{example}
  Consider the group $S_3$. We claim the subgroup generated by $(1,2,3)$ is normal.
  Indeed, it is the group consisting of the identity and all 3-cycles in $S_3$. Since
  conjugation preserves the cycle type of a permutation, the claim follows (since any
  conjugation will be a 3-cycle, which is contained in the subgroup).

  By contrast, the subgroup generated by $(1,2)$ is not normal. Get for instance
  \[(1,3)(1,2)(1,3)=(2,3)\not\in<(1,2)>\]

\end{example}

Note: Conjugation preserves the order of an element in any arbitrary group (we prove this
last year, since the conjugating elements get cancelled out). However, the claim above is
stronger: conjugating in a symmetric group element preserves the \emph{cycle type}, i.e.
if an element is 3-4-2-cycle (e.g. $(1,2,3)(4,5,6,7)(8,9)$), conjugating it will yield
again an element of 3-4-2-cycle.

\begin{example}
  Recall that we can write a permutation as a product of transpositions (2-cycle
  permutations), and the \emph{parity} of the number of transpositions is invariant
  (doesn't depend how you write the product). So a permutation is even if they can be
  written as an even number of transpositions, and it's odd otherwise. Hence we define the
  sign of a permutation to be $+1$ or $-1$ depending on whether it's even or odd,
  respectively.

  Let $n\in\NN$ and let $A_n\subset S_n$ be the set of even permutations. This is a normal
  subgroup, since for any $\sigma, \tau\in S_n$ we have
  \[\sign [\sigma \tau \sigma^{-1}] =
  \sign[\sigma]\sign[\tau]\sign[\sigma]=\sign[\tau]\]
\end{example}

\begin{example}
  Let $D_{2n}$ be the dihedral group of order $2n$, and let $H$ be the subgroup of $n$
  rotations. Then $H$ is normal. On the other hand, the subgroup generated by a refletion
  is not normal.

  Another example is any abelian group, since every subgroup is normal.

  Finally, for a group $G$, any trivial subgroups of the group itself are always normal,
  no matter the group.
\end{example}

Once we've demonstrated well-definedness and provided some examples of the criteria, we
have a method for creating a group structure out of the quotients of the group. 

\begin{definition}
  Let $G$ be a group, and $N$ be a normal subgroup. The set of left cosets $G/N$ together
  with the binary operation $(gN)(hN)=(gh)N$ for $g,h\in G$ is called the \emph{quotient
  group} or \emph{factor group} of $G$ by $N$.
  \label{quotientGroup}
\end{definition}
Note that by the definition of normal subgroups, we have that normal subgroup's left
cosets equal that subgroup's right cosets, since $gNg^{-1}=N\iff gN=Ng$.
\begin{remark}
  Neumann Groups and Geometry book gives a nice and direct view of how normal groups
  naturally give raise to the \emph{group operation preservation} condition. Since for a
  normal subgroup $N\trianglelefteq G$ we have for any $a,b$ it follows that $(Na)(Nb)$ is
  again a coset (recall direct product of groups), since $(Na)(Nb)=N(aN)b=(NN)(ab)=N(ab)$,
  making $G/N$ a group, the quotient group.
\end{remark}

\begin{remark}
  Note that, as previously said, cosets partition the group, making a \emph{smaller set}
  where the elements of those sets are collections of many group elements. However, the
  fact that normal subgroups form quotient \emph{groups}, it makes this partion a group
  itself. That's why it's so useful, it helps understanding the group, in a simpler light.
\end{remark}

\section{Lecture 7}
In this section we will focus on examples of quotient groups.

\begin{example}
  Let $G=S_n$ and $N=A_n$. There are exactly two left cosets of $A_n$ in $S_n$: $1A_n,
  (1,2)A_n$, the latter consisting of all odd permutations. Hence the quotient $S_n/A_n$
  is cyclic of order $2$.
\end{example}

\begin{example}
  Consider the group $S_4$. The subgroup $V_4 = \{e, (12)(34), (13)(24), (14)(23) \}$ is normal. The set $X=\left\{ e,(1,2),(1,3),(2,3),(1,2,3),(1,3,2) \right\}$ is a \emph{full set of left coset representatives of $V_4$ in $S_4$}, i.e. every coset of $V_4$ in $S_4$ contains exactly one element in $X$. This happens to be a subgroup of $S_4$, actually it's $S_3$. Hence we immediately identify the quotient group $S_4/V_4$ is isomorphic to $S_3$.
\end{example}

\begin{example}
  Consider the group $\ZZ$ and the normal subgroup $n\ZZ$ for $n\in\NN$. The quotient
  $\ZZ/n\ZZ=\left\{ 0+n\ZZ, 1+n\ZZ, \cdots, n-1+n\ZZ \right\}$ is cyclic of order $n$,
  generated by $1+n\ZZ$. This quotient group may be written as $C_n$.
\end{example}



\section{Lecture 8 - 8 Oct 2021}
\subsection{Group Homomorphisms, Types and Facts}
On group homomorphisms. 
\begin{definition}
  Let $G,G'$ be groups. A group homomorphism from $G$ to $G'$ is a function $\phi:G\to G'$
  s.t. for all $g,h\in G$ one has $\phi(gh)=\phi(g) \phi(h)$.
  \label{groupHomomorphism}
\end{definition}
I.e. a homomorphism preserves the structure of the binary operation.

\begin{example}
  For any two groups, there's always at least one group, being the trivial homomorphism
  (sending every element to the identity element).
\end{example}

\begin{example}
  Write $\RR^{\times}$ as the non-zero reals as a group under multiplication. For every $n\in\NN$
  we have the group homomorphism
  \[ \phi: \GL_n \RR \to \RR^{\times}:X\mapsto \det X\]
  Where $\phi(XY)=\phi(X)\phi(Y)$. This works for any field, let it be $\QQ,\CC$.
\end{example}

\begin{example}
  For every $n\in\NN$, the function $S_n\to \left\{ \pm 1 \right\}; \sigma\mapsto
  \sgn\sigma$ is a group homomorphism. I.e. $\sgn (\sigma\tau) = \sgn \sigma \sgn
  \tau$.
\end{example}

\begin{example}
  Let $G$ be a group and $H$ be a normal subgroup. The quotient map $G\to G/H : g\to gH$
  is a surjective group homomorphism. Note that it's surjective because every element is
  mapped, and homomorphism follows from the defining group operation of the quotient
  group. In particular, for every $n\in\NN$ there is a surjective homomorphism $\ZZ\to
  n\ZZ$, $k\mapsto k+n\ZZ$.
\end{example}

Recall that we established multiplicative notation for non-abelian group operations and
additive notation for abelian group operations. What if a group homomorphism sends from
one non-abelian to an abelian group? Then the notation in the homomorphism has to be
changed accordingly.
\begin{example}
  Te set $\RR_{>0}$ is a group under multiplication. The homomorphism $\RR_{>0}\to
  \RR:x\mapsto \log x$. We have $\log xy = \log x + \log y$. Similarly we have a
  homomorphism $\RR\to\RR_{>0} : x\mapsto e^{x}$.
\end{example}

The following theorem is foundational. The first two points establlish that groups form a
category. 
\begin{theorem} [Groups form a category]
  We have the following
  \begin{enumerate}
    \item Let $G,G',G''$ be groups and $\phi:G\to G'$ and $\phi':G'\to G''$ be group
      homomorphisms. Then the composition $\phi'\circ\phi:G\to G''$ form also a group
      homomorphism.
    \item Let $G$ be a group. The identity map $G\to G,g\mapsto g$ is a group
      homomorphism.
    \item Let $G,G'$ be groups, and define a bijective group homomorphism $\phi:G\to G'$.
      Then the inverse function $\phi^{-1}:G'\to G$ is also a group homomorphism.
  \end{enumerate}
  \label{groupsCategories}
\end{theorem}
\begin{proof}
  Let $g,h\in G, g',h'\in G'$ throughout.
  \begin{enumerate}
    \item Let $\omega = \phi' \circ \phi $ We have $\omega(gh)= \phi'(\phi(gh))=
      \phi'(\phi(g)\phi(h)) =\phi'(\phi(g)) \phi'(\phi(h)) = \omega(g)\omega(h) $. As
      required.
    \item We have $\phi(gh) = gh = \phi(g)\phi(h)$ as required.
    \item Without loss of generality define $\phi(g) = g'$ and $\phi(h) = h'$. We have 
    $\phi(gh) = \phi(g)\phi(h) \iff gh = \phi^{-1}(\phi(g)\phi(h)) \iff \phi^{-1}(g')\phi^{-1}(h') = \phi^{-1}(g'h')$ As required.

  \end{enumerate}
\end{proof}

\begin{theorem}
  Let $\phi:G\to G'$ be a group homomorphism. Then
  \begin{enumerate}
    \item $\phi(1_G)=1_{G'}$
    \item for every $g\in G$, $\phi(g^{-1})= \phi(g)^{-1}$
  \end{enumerate}
  \label{homIdInv}
\end{theorem}
\begin{proof}
Let $g \in G$.
  \begin{enumerate}
    \item Then
    $$\phi(g) = \phi(1_G g) = \phi(1_G)\phi(g)$$
    and right multiplying by $\phi(g)^{-1}$ gives $\phi(1_G) = 1_{G'}$
    \item Furthermore, we have
    $$\phi(g)\phi(g^{-1}) = \phi(gg^{-1}) = \phi(1_G) = 1_{G'},$$
    and symmetrically $\phi(g^{-1})\phi(g) = 1_{G'}.$ Therefore, we have
    $$\phi(g^{-1}) = \phi(g)^{-1}$$.
  \end{enumerate}
\end{proof}


\begin{definition}[Morphism Zoo]
  \begin{enumerate}
  
    \item A group \emph{isomorphism} is a group homomorphism $\phi:G\to G'$ that has a
      2-sided inverse. This is, $\phi\circ\phi^{-1}=1_{G'},\phi^{-1}\circ\phi=1_G$. If
      there exists a group isomorphism between groups $G,G'$, we say these groups are
      isomorphic and write $G\cong G'$. By the previous results (Theorems
      \ref{groupsCategories} \ref{homIdInv}) we have that being isomorphic is an
      equivalence relation. Two isomorphic groups are structurally the same.

    \item A group automorphism is a group isomorphism to itself.
    \item A group endomorphism is a group homomorphism to itself. E.g. the trivial
      homomorphism to itself is an endomorphism for non-trivial groups.
  \end{enumerate}
  \label{morphismZoo}
\end{definition}

The following is what in last year we were taught to be the definition of an isomorphism,
but it really is a theorem.
\begin{theorem}
  Let $G,G'$ be groups and $\phi:G\to G'$ be group homomorphism. Then $\phi$ is an
  isomorphism if and only if $\phi$ is bijective.
  \label{isomorphismBijective}
\end{theorem}
\begin{proof}
  If the homomorphism is an isomorphism, we have that it must have a two-sided inverse,
  and it follows that $\phi$ is bijective. 

  If $\phi$ is bijective, we claim that $\phi^{-1}$ exists and that it is a group
  homomorphism. This follows from theorem \ref{groupsCategories}.
\end{proof}

\section{Lecture 9 - 11 Oct 2021}
\subsection{First Isomorphism Theorem}
More group homomorphisms and the first isomorphism theorem.
\begin{definition} [Kernel of Homomorphisms]
  Let $\phi:G\to G'$ be a group homomorphism. Then the \emph{kernel} $\ker\phi$ is the set
  of all elements of $G$ that are mapped to the identity of $G'$, 
  \[\ker \phi = \left\{ g\in G : \phi(g)=1_{G'} \right\}\]
  \label{kernel}
\end{definition}

\begin{theorem}
  Let $\phi:G\to G'$ be a group homomorphism. Then $\phi$ is injective if and only if
  $\ker \phi = \left\{ 1_G \right\}$, i.e. the kernel is trivial.
  \label{kernelHomomorphism}
\end{theorem}
\begin{proof}
  ($\Rightarrow$) Suppose $\exists g,h\in G$ s.t. $\phi(g)=\phi(h)$. Then we have
  $\phi(gh^{-1})=1_{G'}$ and so $gh^{-1}\in\ker\phi$. The proof follows by injectivity. 
  Symbolically we have
  \[ (\phi(g)=\phi(h) \implies g=h) \iff (gh^{-1}\in\ker\phi \implies g=h) \iff \ker\phi
  = \left\{ 1_{G'} \right\}\]

  ($\Leftarrow$) $gh^{-1}\in\ker\phi \iff \phi(g)=\phi(h)$. The fact that the kernel is
  trivial automatically implies injectivity.
\end{proof}

\begin{remark} [Groups as symmetries formalised]
  If groups are thought of as symmetries, then homomorphisms are changes in the focus of
  attention of the same symmetry.
\end{remark}

\begin{example}
  Consider the dihedral group $D_{12}$ of order 12. We can describe this by the
  presentation
  \[D_{12}=\langle \sigma, \tau \mid \sigma^{6}=\tau^2=1, \tau\sigma\tau^{-1} =
  \sigma^{-1}\rangle\]
  If we label the vertices of the hexagon as $1,2,\cdots,6$ we can view this dihedral
  group inside $S_6$, through its action on the set of vertices (see group actions). Then
  we observe how the rotation $\sigma$ corresponds to $(1,2,3,4,5,6)$ and reflecion
  $\tau$ to $(1,6)(2,5)(3,4)$, say. This is really saying that an injective group
  homomorphism $D_{12}\to S_6$ exists, defined by $\sigma\to (1,2,3,4,5,6), \tau\to
  (1,6)(2,5)(3,4)$. Observe that we haven't really done much. We just changed the focus of
  attention from the whole hexagon to only the set of vertices,
  \[D_{12}=\langle (1,2,3,4,5,6),(1,6)(2,5)(3,4)\rangle \leq S_6\]
  However, we can also consider the action of $D_12$ on teh set of diagonals of an
  hexagon. Then, for diagonals $1,2,3$, we have 
  \[\sigma\mapsto (1,2,3)\]
  \[\tau\mapsto (1,3)\]
  I.e. this action gives us a homomorphism $D_{12}\to S_3$, which is non-injective
  clearly. Hence we must have a non-trivial element in the kernel, naming $\sigma^3$ which
  rotates $180$ degrees, however in $S_3$ it gives back the identity. Although the
  homomorphism is non-injective, it is surjective. In fact, $\ker\phi = \left\{ 1,\sigma^3
  \right\}$.
\end{example}
This example illustrated the idea of changing the focus of attention, like deciding which
question you're going to ask about your object.

\begin{example} [Quotient homomorphisms, Injections]
  Recall that for $G$ group and $N$ normal subgroup, the quotient map is a surjective
  homomorphism, $\phi:G\to G/N : g\mapsto gN$.

  Let $H< G$. The map $H\to G:h\mapsto h$ is an injective group homomorphism.
\end{example}

\begin{example}[A silly isomorphism]
  Let $G$ be a group s.t. $\left\{ g_1,g_2,\cdots\right\}$ (although it may not be
  countable). Let us keep the group operation but relabel all elements to form a new set
  $H$ with elements $\left\{ h_1,h_2,\cdots \right\}$ and define a binary operation on $H$
  as follows. Whenever $i,j,k$ are such that $g_ig_j=g_k$, define $h_k:=h_ih_j$. Then $H$
  is isomorphic to $G$ by the homomorphism $g_i\mapsto h_i$ for all $i$.
\end{example}
Note that all isomorphisms are essentially like the example above!

Then, the first big theorem of this group is a combination of the above examples of this
lecture: Every homomorphism is a combination of quotient maps, injections, and
isomorphisms.
\begin{theorem} [First isomorphism theorem, Part 1]
  Let $\phi:G\to G'$ be a group homomorphism. Then 
  \begin{enumerate}
    \item $\Img\phi < G'$
    \item $\ker\phi \trianglelefteq G$
  \end{enumerate}
  \label{firstIso1}
\end{theorem}
\begin{proof}
  For the first claim, we have
  \[x,y\in\Img\phi \implies \exists g,h\in G \text{ s.t. } \phi(g)=x,\phi(h)=y \implies
  \phi(gh^{-1})=xy^{-1}\]
  And recall that by the subgroup test, since $1\in\Img\phi$ by definition of
  homomorphism, and $xy^{-1}\in\Img\phi$, then $\Img\phi$ is a subgroup of $G'$, as
  required.

  To prove the second claim, note that $e_G\in\ker\phi$. Moreover, let $x,y\in\ker\phi$.
  Then we have $\phi (x) = \phi (y) =e_{G'}\iff \phi (xy^{-1}) = e_{G'} \implies
  xy^{-1}\in\ker\phi$. So $\ker\phi$ is a subgroup. Next, we claim $\forall g\in G,
  g\ker\phi g^{-1}= \ker\phi$, or $\forall g\in G, h\in\ker\phi,ghg^{-1}\in\ker\phi$.
  Indeed we observe that 
  \[\phi(ghg^{-1})= \phi (g)\phi (h) \phi (g)^{-1} = (\phi g)e_{G'} (\phi g)^{-1} = e_{G'}\]
  So $ghg^{-1}\in\ker\phi$, as required.
\end{proof}

Now, note that for a some homomorphism $\phi:G\to G'$, it may not be surjective, but it
will be surjective to its image (that's the definition of image). It may not be injective,
but we can brutally define an equivalence relation as 
\[g\sim g' \iff \phi(g)=\phi(g')\iff gg^{-1}\in \ker\phi\]
Then $\phi(g)$ depends only on the equivalence class of $g$, and is injective as a
function on equivalence classes. However, now think about what equivalence classes are:
quotient groups! We have by the definition $g\sim g' \iff gg'^{-1}\in \ker\phi$. We already
said $\ker\phi$ is a subgroup. So we have $gg'^{-1}\in \ker\phi \iff
g\ker\phi=g'\ker\phi$. Writing $N=\ker\phi$ we have,
\[g\sim g' \iff gN = g'N\]
Observe that by defining the above we're saying that $\phi$ is well-defined on cosets, and
is injective in the set of cosets. In Theorem \ref{firstIso1} we said the kernel of a
homomorphism is a normal subgroup of $G$.

An attempt to illustrate this has been made in Figure 1:

\begin{center}
\[\begin{tikzcd}
	G && {G'} \\
	\\
	{G/\ker \varphi} && {\operatorname{Im}\varphi}
	\arrow["\varphi", from=1-1, to=1-3]
	\arrow["\pi"', two heads, from=1-1, to=3-1]
	\arrow["{\exists ! \widetilde{\varphi}}"', dashed, from=3-1, to=3-3]
	\arrow["\iota"', hook, from=3-3, to=1-3]
\end{tikzcd}\]
\end{center}

\begin{figure}[htpb]
    \centering
    \caption{First isomorphism theorem, illustrated. Note how every homomorphism is really
    a composition of a trivial and surjective quotient map, an bijective (injective and
    surjective) homomorphism from the quotient to the image.} 
    \label{fig:isomorphism-theorem}
\end{figure}


\begin{theorem} [First Isomorphism theorem]
  Let $\phi:G\to G'$ be a group homomorphism. Then,
  \begin{enumerate}
    \item The image $\Img\phi=\left\{ \phi(g) : g\in G \right\}$ is a subgroup of $G'$
    \item The kernel $\ker\phi$ is a normal subgroup of $G$ (also written
        $\phi^{-1}(1_{G'})$ denoting the pre-image of the identity).
    \item The bijection 
      \[\psi: G/\ker\phi\to\Img\phi : g \ker\phi\mapsto \phi(g)\]
      Is well defined. Moreover, it defines a group isomorphism,
      \[G/\ker\phi \cong \Img\phi\]
  \end{enumerate}
  \label{firstIso}
\end{theorem}
\begin{proof}
  Note that we proved statements $1,2$ in Theorem \ref{firstIso1}. We're left with proving
  the third statement. Note that the operation $\psi$ is well-defined. Observe that for
  any $g,g'\in G$ s.t. $g\ker\phi=g'\ker\phi$. Then we have $g=g'h$ for some
  $h\in\ker\phi$, and so $\phi(g)=\phi(g'h) \iff \phi(g)=\phi(g')$. Hence we shown that
  $g\ker\phi=g'\ker\phi \implies \phi(g)=\phi(g')$, i.e. $\psi$ is well-defined.

  Furthermore, we claim that $\psi$ is surjective and injective. Note that surjectivity
  follows from the definition of $\Img\phi$. To prove injectivity, let us proceed by
  contradiction. Assume $g,g'\in G$ are s.t. $g\ker\phi\neq g'\ker\phi$ and
  $\phi(g)=\phi(g')$. So $g\neq g'h$ for any $h\in\ker\phi$, and it follows that
  $g^{-1}g'h=l$ for some non-identity element $l$ of $G$ not necessarily in $\ker\phi$. Hence,
  \[\phi(g^{-1}g'h) = \phi(l) \iff \phi(g^{-1})\phi(g') = \phi(l) \iff
  \phi(g)\phi(l)=\phi(g')\]
  But $\phi(l)\neq 1_{G'}$, a contradiction

  Alternative direct proof for injectivity (nicer imo)
  
  Assume $g,g'\in G$ and $\psi(g\ker\phi) = \psi(g'\ker\phi)$. Hence,
  \[\phi(g) = \phi(g') \iff 1_{G'} = \phi(g)^{-1}\phi(g') = \phi(g^{-1}g') \iff g^{-1}g' \in \ker\phi \iff g\ker\phi = g'\ker\phi\]
  Therefore, $\psi$ is a bijection.
\end{proof}

\section{Lecture 10}
On isomorphism theorems. The first isomorphism in action.
\begin{proposition}
  A subgroup $H\leq G$ is normal if and only if $H$ is the kernel of some homomorphism
  $\phi: G\to G'$.
\end{proposition}
\begin{proof}
  Note that every kernel is a normal subgroup by the first isomorphism theorem.
  Conversely, if $N\trianglelefteq G$, we can define the quotient map $\phi:G\to
  G/N, \text{ } g\mapsto gN.$ 
  This is a group homomorphism with kernel $N$ since $\phi(g) = 1_{G/N} \iff gN = N \iff g\in N.$
  %all $g\in G$ s.t. $\phi(g)=N$, which is all $g\in N.$ In other words, $\ker \phi=N.$
\end{proof}

\begin{remark}
    For a group $G,$ it is essentially equivalent to list normal subgroups as it is to list group homomorphisms out of $G.$
\end{remark}
% Then, taking the quotient by a normal subgroup $N$ is essentially equivalent to applying a
% homomorphism from $G$ to some other gorup whose kernel is $N$.
\begin{definition}
  Let $(H,*),(K,\cdot)$ be groups. We define their \emph{direct product} as the group
  with underlying set $H\times K = \left\{ (h,k) : h\in H, k\in K \right\}$ with operation
  of point-wise multiplication
  $(h,k)(h',k')=(h*h',k\cdot k').$
  \label{directProduct}
\end{definition}


\begin{proposition}
    Let $H$ and $K$ be groups, and let $G=H \times K$. The subset $H \times\left\{1_K\right\}$ is a normal subgroup of $G$ isomorphic to $H$, and one has $G / H \times\left\{1_K\right\} \cong K$. Similarly, $\left\{1_H\right\} \times K$ is a normal subgroup of $G$ isomorphic to $K$, and one has $G /\left(\left\{1_H\right\} \times K\right) \cong H$.
\end{proposition}

\begin{proof}
    Consider the map
    $$\phi :H\times \{1_K\} \to H$$ defined by $\phi ((h,1))=h.$ This is clearly an isomorphism.
    Now consider the canonical projection map
    $$\pi: H\times K \to K$$ defined by $\pi ((h,k))=k.$
    Then $\ker \pi =\{(h,1_K):h\in H\}=H\times 1_K.$ Since kernels are normal subgroups, we have shown that $H\times /{1_K}$ is a normal subgroup of $G.$ We have $\operatorname{Im} \pi =K$ trivially. Therefore, by the FIT, we have $H\times K / H\times \{1_K\} \cong K,$ or equivalently, $G/H\times {1_K} \cong K.$ 
    Similarly, we have $\{1_H\}\times K \cong K,$ is a normal subgroup of $G,$ and $G /\left(\left\{1_H\right\} \times K\right) \cong H.$
\end{proof}

\begin{example}
  We have previously seen (Example \ref{D12}) that there exists a homomorphism from $D_{12}$ to $S_3$. The kernel of 
  this homomorphism is the subgroup $H$ generated by the rotation by $180^o$, and the
  image is all of $S_3$. By FIT, we have $D_{12}/H \cong S_3$.
\end{example}
\begin{example}
  Let $\CC^{\times}$ be the non-zero complex numbers as a group under multiplication. The
  function $f:\RR\to\CC, \text{ } x\mapsto e^{2\pi i x}$ is a group homomorphism. The image is the
  unit circle in $\CC$, denoted $S^1$, and the kernel is $\ZZ$. Then,
  \[\RR/\ZZ \cong S^1.\]
  This can be extended to show that 
  \[\RR^n/\ZZ^n \cong \bigoplus_{i=1}^n S^1.\]
\end{example}

\begin{example}
  Consider the homomorphism from $\CC^{\times}$ to $\RR_{>0}$ viewed as groups under multiplication given by $z\mapsto |z|$. The image is $\RR_{>0}$ and the kernel is
  $S^1$. Hence by FIT,
  \[\CC^{\times} / S^1 \cong \RR_{>0}\]
\end{example}
\begin{example}
  How many distinct homomorphisms $\phi:G\to G'$ are there from $G=C_4=\langle g \mid g^4=1\rangle$ to $ G'= C_{10}=\langle
  g' \mid (g')^{10}=1\rangle$?  

 We will begin by listing our options for possible kernels and images, and then classify our homomorphisms by the choices of kernel.
First note that since $G$ is cyclic, it is abelian and therefore all of its subgroups are normal and can be kernels of the map.

The group $G$ has three subgroups: the trivial subgroup $\{e\},$ the subgroup $\langle g^2 \rangle =\{e,g^2\},$ and the group itself, $G.$

Our options for image are subgroups of $G'.$ We have: $\{e\}$, $\langle (g')^5 \rangle =\{e,(g')^5\},$ and the group itself, $G'.$

Now consider a map $\phi$ with $\ker \phi =\{e\}$. By FIT, we must have $G/\{e\} \cong G \cong \operatorname{Im} \phi,$ but $G'$ has no subgroup of order 4, so there is no such map.\\ Now consider $\phi$ with $\ker \phi =\{e,g^2\}.$ Again, by FIT, we must have $C_4/\{1,g^2\}=\{\{1,g^2\},\{g,g^3\}\} \cong \operatorname{Im}.$ We can take $\operatorname{Im} \phi =\{e,(g')^5\}$.\\ Finally, consider $\ker \phi =G.$ We must have $G/G \cong \{e\} \cong \Img\phi.$ We can take $\Img\phi =\{e\},$ and this is in fact the trivial homomorphism sending every element to the identity in $G'.$ \\Therefore, there are only two possible homomorphisms.
\end{example}

\section{Lecture 11 - 13 Oct 2021}
\subsection{Second Isomorphism Theorem}
On Isomorphism theorems. We will continue applying the FIT. First we will apply it to the
second isomorphism theorem.
\begin{theorem}[Second Isomorphism Theorem]
  Let $G$ be a group, $H<G$, let $N\trianglelefteq G$. Then,
  \begin{enumerate}
    \item The set $HN=\left\{ hn |h\in H,n\in N \right\} <G$ .
    \item The intesection $H\cap N \trianglelefteq H$, not necessarily of $G$.
    \item There is an isomorphism $H/H\cap N \cong HN/N$.
  \end{enumerate}
  \label{secondIso}
\end{theorem}
\begin{proof}
  \todo{Part 1 and 2 are exercise sheet 2, q 2}
  To prove the third point, we use the FIT. We claim there exists an homomorphism $\phi$
  where $\Img\phi= HN/N$ and $\ker\phi = H\cap N$. Consider the composition $\phi:H\xhookrightarrow{}
  G\to G/N: h\mapsto hN$. Note that the operation is a homomorphism because it's a
  composition of two homomorphisms: the embedding of $H$ in $G$ and the quotient map.
  Moreover note that $h\in\ker\phi \iff h\in H \text{ and } hN=N \iff h\in H \text{ and } h\in N$, i.e.
  $h\in H\cap N$. On the other hand, note that $xN\in\Img\phi \iff \exists h\in H : hN=xN
  \iff  h^{-1}x\in N \iff x = hn \in HN$.
  Therefore $\Img\phi = HN/N$, and by the FIT it follows
  that $H/H\cap N \cong HN/N$. 
\end{proof}
Recall from the previous lecture that the first isomorphism theorem says that taking the
quotient of $G$ by some normal subgroup $N$ is equivalent to applying a homomorphism $\phi$
from $G$ whose kernel is $N$.

The second isomorphism theorem addresses the question: What if I apply the same $\phi$ to
a subgroup $H$ of $G$? 

\begin{example}
  Let $GL_2(\RR)$ be a multiplicative group. Let $SL_2(\RR)\leq GL_2(\RR)$ be the subgroup
  of matrices with determinant $1$. This is a normal subgroup. What's the structure of
  $GL_2(\RR)/SL_2(\RR)$? By the first isom theorem we have that $SL_2(\RR)$ must be the
  kernel of some group homomorphism, namely the determinant map. The image of the
  determinant is all $\RR^{\times}$, so by the FIT, we deduce that
  $GL_2(\RR)/SL_2(\RR)\cong \RR^{\times}$. Consider the subgroup
  \[Z= \left\{ \diag (a,a) : a\in \RR^{\times} \right\} \leq GL_{2}(\RR)\]
  Observe that $z\in Z, |z|=a^2$, so we see that $Z\cap SL_{2}(\RR) = \left\{ \pm
  \diag(1,1) \right\}$. 

  We claim that $Z\cdot SL_2(\RR)$ is the subgroup $GL_2^+(\RR)$
  consisting of all matrices with positive determinant. The proof is left as an exercise
  to the reader (it's not too hard). By the SIT applied with $H=Z$ and $N=SL_2(\RR)$ we
  have
  \[ Z/\left\{ \pm \diag(1,1) \right\} \cong GL_2^+(\RR)/ SL_2(\RR) \cong \RR_{>0}\]

\end{example}

\subsection{Non-examinable aside}
Bartel makes a non-examinable aside to try to explain how SIT will be useful eventually.
First, he introduces the idea of \emph{atoms of group theory}. If we want to understand a
big and complicated group $G$, we can do so by finding a normal subgroup $N$ and
understand that and its quotient $G/N$ (which is in essence a group with similar structure
to $G$ but with less elements). Note that his reduction step relies on the existence of a
proper non-trivial normal subgroup. The SIT is very useful in this approach
\begin{definition}[Simple Group]
  A group is \emph{simple} if it is non-trivial and has no proper non-trivial normal
  subgroups.
\end{definition}

If we understand all simple groups, and how bigger groups are made up of smaller normal
groups and their quotients, we would understand all groups. All finite simple groups have
been classified. For instance, for any $n\geq 5$, the alternate group $A_n$ is simple.

\begin{theorem}
  Let $G$ be a metabelian group, meaning there exists an abelian normal subgroup $N$ of
  $G$ sucht that $G/N$ is also abelian. Then every subgroup of $G$ is also metabelian.
  \label{<+label+>}
\end{theorem}

\begin{proof}
    Let $H \leq G$. Since $N$ is a normal abelian subgroup of $G$, $H \cap N$ is abelian as its a subgroup of $N$. Using the SIT we have that $H/H \cap N \cong HN/N$, since $G/N$ is abelian $HN/N$ is abelian and so $H/H \cap N$ is abelian. So H has an abelian normal subgroup $H \cap N$ such that the quotient is abelian, and as $H$ was an arbitrary subgroup of $G$ every subgroup of $G$ is metabelian.
\end{proof}

\section{Lecture 12 - 18 Oct 2021}
On the 3rd isomorphism theorem. Last lecture about group theory. From next lecture on
we'll talk about group actions. 
\subsection{Third Isomorphism Theorem}
\begin{lemma}[Sheet 2, Q5]
  Let $\phi:G\to G'$ be a homomorphism and $H'<G'$. We claim that the preimage
  $\phi^{-1}(H')<G$ is a subgroup of $G$. 
  \label{lem:preImgSubgroup}
\end{lemma}
\begin{proof}
  Note that we have $1_{G'}\in H'$ since $H'$ is a subgroup of $G'$, so
  $\phi(1_{G})=1_{G'}$ and it follows tht $1_{G}\in \phi^{-1}(H')$. Next we claim that for
  any $x,y\in\phi^{-1}(H')$ we have $xy^{-1}\in\phi^{-1}(H')$. Note that we have, for any
  $h,h'\in H'$, $h'h^{-1}\in H'$ since $H'$ is a subgroup. Moreover, we have that
  $h=\phi(x),h'=\phi(y)$ for some $x,y,\in\phi^{-1}(H')$, hence it follows that
  $\phi(x)\phi(y)^{-1} = \phi(xy^{-1})\in H'$, or equivalently $xy^{-1}\in\phi^{-1}(H')$,
  as required.
\end{proof}

\begin{lemma}
  Let $\phi:G\to G'$ be a surjective homomorphism and $H'<G'$. We claim 
  \[H'\trianglelefteq G' \iff \phi^{-1}(H')\trianglelefteq G\]
  \label{lem:preImgNormSub}
\end{lemma}
\begin{proof}
  Let $g\in G$ and $h\in \phi^{-1}(H')$. Then define $g'=\phi(g), h'=\phi(h)\in H'$ and $\bar{h'} = \phi(\bar{h})$.
  We have $g'h' = \bar{h}'g'$ for some $\bar{h}'\in H$,
  \[g'h' = \bar{h}'g' \iff \phi(g)\phi(h)  =\phi(\bar{h})\phi(g) \iff \phi(ghg^{-1}) = g'h'(g')^{-1} \in H'.\]
  \[\iff ghg^{-1}\in \phi^{-1}(H')\]
  Hence $\phi^{-1}(H')\trianglelefteq G$, as required.
\end{proof}

\begin{theorem} [Third Isomorphism Theorem]
If $N$ and $K$ are normal in $G$ with $N \subseteq K$, then we have an isomorphism of groups $G/K \cong (G/N)/(K/N)$
%  Let $G$ be a group, and $N\trianglelefteq G$. Define the quotient map $q_N:G\to G/N$.
%  Then
%  \begin{enumerate}
%    \item For every subgroup $U$ of $G/N$, the preimage $q_N^{-1}(U)$ is a subgroup of
%      $G$.
%    \item If $U$ is a subgroup of $G/N$, then $U$ is normal in $G/N$ if and only iff
%      $q_N^{-1}(U)$ is normal in $G$.
%    \item The function $U\mapsto q_N^{-1}(U)$ defines a bijection between the set of
%      subgroups of $G/N$ and the set of subgroups of $G$ that contain $N$.
 %   If $K\trianglelefteq G$ containing $N$, then there exists a unique isomorphism $\widetilde{f}: G/K \to (G/N)/(K/N)$ which gives 
   % $$
   %     G/K \cong (G/N)/(K/N)
  %  $$
   % By $gK\mapsto (gN)(K/N)$.
%  \end{enumerate}
  \label{thm:tit}
\end{theorem}

This is easier to understand with the following commutative diagram:

\[\begin{tikzcd}
	{G} &&&&& {G/K} \\
	\\
	& {G/N} &&& {(G/N)/(K/N)}
	\arrow["{q_K}", two heads, from=1-1, to=1-6]
	\arrow["{q_{K/N}}", two heads, from=3-2, to=3-5]
	\arrow["{q_N}"{description}, two heads, from=1-1, to=3-2]
	\arrow["{\widetilde{f}}"{description}, dashed, from=3-5, to=1-6]
\end{tikzcd}\]

\begin{remark}
    In other words, the third isomorphism theorem says that the quotient homomorphism $q_K : G \rightarrow G/K$ can be factored through the quotient homomorphism $q_N : G \rightarrow G/N$ to give $q_K \simeq q_{K/N} \circ q_N$.
\end{remark}


\begin{proof}
  % We'll prove these part by part. For the first part, we have that
  % $N=\pi^{-1}(1N)\subset\pi^{-1}(U)$ for every subset of $U$, since $U$ is a subgroup of
  % $G/N$ it must contain the identity element, naming $N$. Note that the pre-image of $N$
  % is $N$ itself (recall the absorption rule from 2F). Moreover, we have that for a
  % homomorphism $\phi:G\to G/N$, the preimage of any subgrup of $G/N$ is a subgroup of $G$
  % by Lemma \ref{lem:preImgSubgroup}.

  % For the second part, we have that the quotient map is surjective by definition. The
  % proof then follows by Lemma \ref{lem:preImgNormSub}.

  % For the third part, we find an inverse of the function $U\to \pi^{-1}(U)$ to show that
  % it's indeed a bijection. Let $\SH$ be the set of subgroups of $G$ that contain $N$, i.e.
  % $\SH=\{H<G | H\cap N = N\}$, and let $\SH'$ be the st of subgroups of $G/N$, i.e.
  % $\SH'=\{gN | g\in G\}$. Then we find $\SH\to\SH'$ by $H\mapsto \pi (H)=H/N$. Hence the
  % defined function is a bijection.
  We claim that  $\widetilde{f}: G/K \to (G/N)/(K/N)$ is an isomorphism of groups.
  First we check it's well defined, suppose $gK=g'K$ for some $g,g'\in G$. Then
  $g^{-1}g'\in K$, so $g^{-1}g' N \in K/N$ hence $gN(K/N) = g'N(K/N)\in (G/N)/(K/N)$.
  Surjectivity is given since every coset representative of $G/N$ is hit by some
  representative of $G/K$. Injectivity proof is the same as the well-definedness proof but
  read backwards (notice how the proof goes both ways!)
  \todo{provide alternate proof using FIT}
\end{proof}
The above theorem says that if $G$ is a group, $K$ is a normal subgroup, and $\phi$ is a
homomorphism from $G$ whose kernel ($N$ above, since every normal subgroup is a kernel of
a homomorphism) is contained in $K$, then $G/K \cong \phi(G)/\phi(K)$. Note that
$\phi(G)=\Img \phi:G\to G/N$ and $\phi(K)=\Img \phi_{K}:K\to K/N$. This, in order to
understand $G/K$ we may apply any homomorphism $\phi$ whose kernel ($N$) is contained in
$K$ and instead understand that quotient on the other side of the homomorphism, which is
often simpler.

\begin{example}
  Let $G=\GL_{2}(\RR)$ and let $K=\GL_2^+(\RR)=\{X\in\GL_2(\RR) : \det X>0\}$. Then it's
  easy to see that $K$ is normal in $G$. What is $G/K$? THe gorup $K$ contains the normal
  subgroup $N=\SL_2(\RR)=\ker\det$. We have surjective group homomorphisms
  $G\to\RR^{\times}$ and $\GL_2^+(\RR)\to\RR_{>0}$, with kernel $\SL_2(\RR)$ and by the
  FIT we have $G/N\cong \RR^{\times}$ and $K/N\cong \RR_{>0}$. By the TIT we have
  $G/K\cong \det\GL_2(\RR)/\det\GL_2^+(\RR)\cong \RR^{\times}/\RR_{>0}\cong \{\pm 1\}$.
\end{example}

Get used to thinking about normal subgroups as kernels of homomorphisms, and quotients
$G/N$ as images $\phi(G)$ of homomorphisms $\phi$ whose kernels are that normal subgroup
$N=\ker\phi$.

\section{Lecture 13 - 20 Oct 2021}
\subsection{Group Actions}
On group actions. Group actions is in a way the birth of groups.
\begin{definition} [Left action]
  Let $G$ be a group, and let $X$ be a set. A left action of $G$ on $X$ is a function
  \[G\times X \to X\]
  \[(g,x)\mapsto g\cdot x\in X\]
  Such that 
  \begin{enumerate}
    \item $1_G \cdot x=x$ for any $x\in X$
    \item $g_1 \cdot (g_2\cdot x) = (g_1g_2)\cdot x$ for any $g_1,g_2\in G$ and $x\in X$.
  \end{enumerate}
  Note that $\cdot$ is used to denote the action and the multiplication notation used for
  group operation. A set equipped with an action of a group $G$ is called $G$-set.
  \label{def:leftAction}
\end{definition}

\begin{definition}[Right action]
  Analogously, a right action of a group $G$ on a set $X$ is a function
  \[X\times G \to X\]
  \[(x,g)\mapsto x\cdot g\]
  Such that
  \begin{enumerate}
    \item for all $x\in X$ one has $x\cdot 1_G=x$
    \item for all $g_1,g_2\in G$ and for all $x\in X$ one has $(x\cdot g_1)\cdot g_2 =
      x\cdot (g_1g_2)$.
  \end{enumerate}
  \label{def:rightAction}
\end{definition}

\begin{remark}
  Note that in a left action, first acting by $g_2$ and then acting on the result by $g_1$
  is the same as acting by the product $g_1g_2$. For a right action, first acting by $g_2$
  and then by $g_1$ is the same as the product $g_2g_1\neq g_1g_2$.
\end{remark}

\begin{theorem}
  Let $G$ be a group and $X$ a $G$-set. Then
  \begin{enumerate}
    \item For every $g\in G$, the function $\sigma_g:X\to X:x\mapsto g\cdot x$ is
      injective. I.e. every $g\in G$ induces a permutation of $X$.
    \item The function $G\to $ \{permutations of X\} given by $g\mapsto\sigma_g$ is a
      group homomorphism
  \end{enumerate}
  \label{<+label+>}
\end{theorem}
\begin{proof}
  To prove the first claim, we have $\sigma_g x = \sigma_g y \implies g\cdot x= g\cdot y
  \implies (g^{-1}g)\cdot x = (g^{-1}g)\cdot y \implies 1_G\cdot x = 1_G\cdot y \implies
  x=y$.
  
  The second claim follows by the second axiom in Definition \ref{def:leftAction}. Let
  $S_X$ be the set of permutations of $X$, and we have $\phi:G\to S_X: g\mapsto g\cdot x$ Let $g,g'\in G$ and $x \in X$.
  Then $\phi(gg')(x) = \sigma_{gg'}(x)= (gg')\cdot x = g\cdot \phi (g')(x) = (\phi(g)(\phi (g')(x)) = (\phi(g) \circ \phi(g'))(x)$,
  as required.
\end{proof}

\begin{example}
  Let $n\in\NN$. The group $S_n$ acts on $\left\{ 1,\cdots, n \right\}$.

  For $n\in\NN_{\geq 3}$, the dihedral group of order $2n$ acts on the set of vertices and
  also on the set of edges of a regular $n$-gon. Moreover, if $n$ is even, it acts on the
  set of diagonals.

  Let $G$ be the group of rotations of a cube. Then $G$ acts on the set of diagonals of
  the cube. This defines a group homomorphism $G\to S_4$. Bartel then shows an animation
  to illustrate the fact that rotations diagonals is equivalent to a $3$-cycle
  permutation, the rotation about axis through edges is equivalent to the product a
  transposition, and the rotation about axis through faces is just a product of two
  transpositions. This shows that every possibility is hit by an element of $S_4$. In fact
  this defines an isomorphism.
\end{example}

\begin{definition}
  Let $G$ be a group. A $G$-action on a set $X$ is called transitive if for any $x,y\in X$
  there exists $g\in G$ s.t. $y=g\cdot x$.
  \label{def:transitiveAction}
\end{definition}
\begin{example}
  Let $G$ be a group and $H$ be a subgroup. The set of left cosets of $H$ in $G$ is a
  transitive $G$-set under action
  \[G\times G/H \to G/H: (g,xH)\mapsto (gx)H\]
\end{example}

\begin{definition} [Isomorphism on group action]
  Let $G$ be a group and $X,Y$ be $G$-sets. An isomorphism from $X$ to $Y$ is a bijection
  $\phi:X\to Y$ s.t. $\forall x\in X, g\in G$, we have $\phi(g\cdot x)=g\cdot \phi(x)$.
  \label{defi:isomorphismAction}
\end{definition}

\begin{theorem}
  Let $G$ be a group and let $H,K$ be subgroups. The $G$-sets $G/H$ and $G/K$ are
  isomorphic if and only if there exists $g\in G$ such that $H=gKg^{-1}$.
  \label{thm:cosetsIsomorphic}
\end{theorem}
\begin{proof} 
  Note that the trivial map $G/H\to G/K:gH\mapsto gK$ is not well-defined, as it fails
  with $g=1$.

  ($\Leftarrow$) Define $\phi:G/H\to G/K$. How do we find a mapping? Let us define the
  placeholder $H\mapsto gK$ and we want to find $g\in G$ s.t. for any $h\in H$,
  $hgK = gK$, i.e. $g^{-1}hg\in K$ or equivalently $h\in gKg^{-1}$. In our assumption we
  have $g\in G$ with $H=gKg^{-1}$, However, we need a map for any
  $g$, not just the identity. We can define $\phi:xH\mapsto xgK$ for any $x\in G$.
  We claim this is well-defined and injective. We claim $xH=yH$ and it follows
  \[\iff x^{-1}y\in H=gKg^{-1} \iff x^{-1}yg = gk, \quad \exists k\in K \]
  \[\iff xgk= yg \iff xgK = ygK\]
Hence this is well defined and injective.

  We also claim that this is surjective, which follows from the fact that $G$ is a group,
  so every element of $x$ will have $xg\in G$. Finally, let $\alpha\in G$ and $xH\in
  G/H$. Then $\phi(\alpha \cdot xH)=\alpha \cdot xgK = \alpha \cdot \phi(xH)$.

  ($\Rightarrow$) Let $\phi: G/H\to G/K$ be an isomorphism of $G$-sets. Let $g\in G$ be
  s.t. $\phi(H)=gK$. Claim: $H=gKg^{-1}$. We have 
  \[\alpha\cdot H=H \iff \alpha\in H \]
  \[\iff \phi(\alpha\cdot H)=\alpha\cdot \phi(H)=gK\iff \]
    \[\alpha g K = gK \iff g^{-1}\alpha g\in K \iff \alpha \in gKg^{-1}\]
    Note that the second line follows by Definition \ref{defi:isomorphismAction}.
\end{proof}

\section{Lecture 14 - 22 Oct 2021}
\subsection{Orbit-Stabiliser Theorem}
Orbit-stabilizer theorem.
\begin{definition}[Obrit and Stabiliser]
  Let $G$ be a group and $X$ be a $G$-set. Let $x\in X$. The orbit of $x$ in $G$ is
  $\Orb_G (x)= G\cdot x = \left\{ g\cdot x :g\in G\right\}$.
  The stabiliser of $x$ in $G$ is $\Stab_G (x)=\left\{ g\in G | g\cdot x=x \right\}$
  \label{def:orbStab}
\end{definition}


\begin{theorem}
  Let $G$ be a group and $X$ be a $G$-set. Then 
  \begin{enumerate}
    \item The $G$-orbit of $x$ is a transitive $G$-set.
    \item The stabiliser of $x$ is a subgroup of $G$.
  \end{enumerate}
  \label{<+label+>}
\end{theorem}
\begin{proof}
  For the first claim, we have to show that $\Orb_G(x)$ is indeed a $G$-set and that it's
  transitive. To show that it's a $G$-set note that $\forall g\in G, y\in\Orb_G(x)$ we
  have $g\cdot y = g\cdot (h\cdot x)$ for some $h\in G$. By Definition
  \ref{def:leftAction} it follows that $g\cdot y = (g\cdot h)\cdot x \in\Orb_G(x)$, as
  required. Moreover, we claim it's transitive, i.e. $\forall a,b\in\Orb_G(x)\exists h'\in
  G : b=h'\cdot a$. Note that $b=g\cdot x, a=g'\cdot x$ for some $g,g'\in G$, so it
  follows that $a=g'\cdot x = g'\cdot (g^{-1}\cdot b)$, and again it follows that
  $a=(g'g^{-1})\cdot b$, as required.

  For part 2 we show that the stabilizer is indeed a subgroup. Note that the identity
  $e_G\in\Stab_G(x)$, so the set is non-empty. Next, consider some $a,b\in\Stab_G(x)$, so
  $a\cdot x=x=b\cdot x \iff (b^{-1}a)\cdot x =x$, so $b^{-1}a\in\Stab_G(x)$, as required.
\end{proof}


\begin{theorem}[Orbit-Stabiliser Theorem]
  Let $G$ be a group and $X$ be a $G$-set. Let $x\in X$. Then there is an isomorphism of
  $G$-sets as $\phi:G/\Stab_{G}(x)\to \Orb(x):g\Stab_G(x)\mapsto g\cdot x$.
  \label{thm:orbStab}
\end{theorem}
\begin{proof}
  We first show that $\phi$ is well defined (doesn't depend on coset representatives) and
  then show how it's a bijective map with $\phi(g\cdot x)=g\cdot \phi(x)$. We show
  well-definedness and injectivity in one statement. Let $H=\Stab_G(x)$, so
  \[gH=hH \iff h^{-1}g\in H \iff h^{-1}g \cdot x = x \iff g\cdot x = h\cdot x\]
  Note that surjectivity is clear by the definition of $\Orb_G(x)$. Hence $\phi$ is a
  bijection. Next, observe that $\phi(g\cdot xH)=(gh)\cdot x = g\cdot (h\cdot x)$ by
  axioms of group actions, and it follows $\phi(g\cdot xH)=g\phi(xH)$, as required.
\end{proof}


\begin{corollary}
  Let $G$ be a group and $X$ be a $G$-set. Let $x\in X$. We have $|\Orb_{G}(x)|=
  [G:\Stab_G(x)]$. In particular, if $G$ is finite, then $|\Orb_{G}(x)|=|G|/|\Stab_G(x)|$,
  by Lagrange, and the size of every orbit divides $|G|$.
  \label{cor:orbStab}
\end{corollary}


\begin{theorem}
  Let $G$ be a group and $X$ be a $G$-set. Then lying in the same orbit is an equivalence
  relation on $X$. In particular, $X$ is a union of disjoint orbits (equivalence classes).
  \label{thm:eqRelOrb}
\end{theorem}
\begin{proof}
  We prove reflexivity, symmetry, and transitivity. Note that the relation is reflexive
  since $x=1\cdot x$, the definition of left action. Moreover, note that the relation is
  symmetric since $x\sim y \iff x=hz$ $y=gz\exists h,g\in G \iff y\sim x$. Finally, we
  have transivity trivially, since $x=hc,y=gc\exists h,g\in G$ and $y=h'c,z=g'c\exists
  h',g'\in G$ it follows that $x\sim z$.
\end{proof}


\begin{theorem}
  Let $G$ be a group, and let $X$ be a transitive $G$-set. Then any two point stabilisers are
  conjugate in $G$. That is, for any $x,y\in X$ there exists $g\in G$ with
  $\Stab_G(x)=g\Stab_G(y)g^{-1}$.
  \label{<+label+>}
\end{theorem}
\begin{proof}
  Since $X$ is transitive, there is $g\in G$ with $x=gy$. Let $h\in\Stab_G(x)$, and
  observe
  \[hx=x=hgy=gy\]
  \[\iff g^{-1}hgy = y\]
  \[\iff g^{-1}hg\in\Stab_G(y)\]
  \[\iff h\in g\Stab_G(y)g^{-1}\]
  Hence $\Stab_G(x)=g\Stab_G(y)g^{-1}$, as required.
\end{proof}

Recall that since orbits define equivalence relations, we can define equivalence classes,
and since the orbit of $x$ is isomorphic to the set of left cosets of the stabilizer of
$x$ in $G$, $G/\Stab_G(x)$, we can find equivalence classes also in $G/\Stab_G(x)$
\begin{theorem}
  Let $G$ be a group. Then there is a bijection between conjugancy classes of subgroups of
  $G$ and isomorphism classes of transitive $G$-sets.
  \begin{enumerate}
    \item ($\rightarrow$) Given a subgroup $H\leq G$, assign to it the set $G/H$ of left
      cosets of $H$ in $G$ -- which will be a transitive $G$-set since by Theorem \ref{thm:orbStab} we
      have $G/H\cong \Orb(x)$.
    \item ($\leftarrow$) Given a transitive $G$-set $X$, assign to it $H=\Stab_G(x)\forall
      x\in X$
  \end{enumerate}
  \label{<+label+>}
\end{theorem}
\begin{proof}
  We first claim that the assignment from subgroups $H$ of $G$ to the set of left cosets
  $G/H$ is well defined, i.e. if I have two subgroups $H,K$ that are conjugate, then they
  will map to the same transitive $G$-set. By Theorem \ref{thm:cosetsIsomorphic} if $H,K$
  are conjugate, we have that the $G$-sets $G/H$ and $G/K$ are isomorphic. Hence this is
  given.

  More over, if $X$ is a transitive $G$-set, then $\Stab_G(x)$ for any $x\in X$ is a
  well-defined conjugacy class (independent of $x$).
   
  Finally we claim the two assignment sabove are inverses of each other. Fix $H\leq
  G$ and consider the map $\phi:H\mapsto G/H$. Note how the map
  $\psi:G/H\mapsto\Stab_G(1\cdot H)$ is $\psi=\phi^{-1}$, since $g\cdot 1H=H \iff g\in H$.
  Conversely, consider the map from a transitive $G$-set $X\mapsto \Stab(x)$, which has
  inverse $\Stab(x)\mapsto G/\Stab(x)$ where by the Orbit-Stabilizer theorem we have
  $G/\Stab(x)\cong X$.
  
\end{proof}

All transitive $G$-sets look like sets of left cosets, $G/H$ for a suitable $H$. What $H$?
Given a transitive $G$-set say $X$, then the subgroup it corresponds to (the conjugacy
class of subgroups really) is the conjugacy class of point stabilizers.
\todo{Understand this}
Note that if you take two different points in $X$, the stabilizer of each point are
conjugate. Two conjugate subgroups give isomorphic $G$-sets, and hence a bijection arises
transitive $G$-sets and conjugate classes of subgroups of $G$.

\section{Lecture 15}
\subsection{Applications of Orbit-Stabiliser Theorem and Cauchy's Theorem}

\begin{definition}
    Let XX be a GG-set. We write XG:={x∈X∣∀g∈G,g⋅x=x}X^G:=\{x \in X \mid \forall g \in G, g \cdot x=x\}
    % g \cdot x=x \:\: \forall g \in G\} \subset X$ 
    to denote the set of fixed points of the action. In other words, XGX^G consists of all xx in XX with \OrbG(x)={x}.\Orb_G(x)=\{x\}. 
    % so x∈XG⟹\OrbG(x)={x}x \in X^G \Longrightarrow \Orb_G(x)=\{x\}.
    In particular, if XX is finite then we have
    $$
    |X|=\left|X^G\right| + \sum_{i=1}^r \left|\Orb{}_{G}(x_i)\right|,
    $$
    where the sum runs over those orbits whose size are greater than one. This formula is called the class equation of the group action. \label{classeq}
\end{definition}

\begin{lemma}
    Let GG be a group of order pnp^n for some prime pp and n≥1n \geq 1. If GG acts on a finite set XX, then
    $$
        \left|X^G\right| \equiv|X| \bmod p. \label{usefullemma}
    $$
\end{lemma}
\begin{proof}
    Since $G$ is a finite group, by the orbit-stabiliser theorem $|\Orb_G(x)|$ divides $|G| = p^n$ for all $x$. In particular, if $\Orb_G(x_i)$ is an orbit with $|\Orb_G(x_i)| > 1$ then we must have $|\Orb_G(x_i)| = p^k$ for some $1 \leq k \leq n$. Hence the result follows from considering the class equation of the group action (\ref{classeq}):
    $$ 
    |X|-\left|X^G\right| =\underbrace{\sum_{i=1}^r \left|\Orb(x_i)\right|}_{\textrm{multiple of pp}}
    \implies \left|X^G\right| \equiv |X| \bmod p.
    $$
\end{proof}

\begin{theorem}[Cauchy's Theorem]
  Let $G$ be a finite group and $p$ be a prime divisor of $|G|$. Then $G$ contains an element of order $p.$
  \label{thm:cauchy}
\end{theorem}
\begin{proof}
    Consider the set
    $$
    X=\left\{\left(x_1, \ldots, x_p\right) \in G^p \mid x_1 \cdots x_p=1\right\}
    $$
    of $p$-tuples of elements $x_i$ of $G$ whose product is the identity. Notice that such a $p$-tuple is uniquely determined by $p-1$ of its components. Indeed, if $x_1, \ldots, x_{p-1}$ is an arbitrary collection of elements in $G$ then $x_p$ is forced to be $x_p=\left(x_1 \cdots x_{p-1}\right)^{-1}$. Thus, we see that $X$ has $|G|^{p-1}$ elements and hence $|X|$ is divisible by $p$ (as $p$ is a divisor of $|G|$).
    
    Now observe that the cyclic group $\mathbb{Z}_p=\langle\sigma\rangle$, for $\sigma=(123 \cdots p) \in S_p$, acts on $X$ by
    $$
    \sigma \cdot\left(x_1, \ldots, x_p\right)=\left(x_{\sigma(1)}, \ldots, x_{\sigma(p)}\right)=\left(x_2, \ldots, x_p, x_1\right)
    $$
    The RHS does indeed remain in $X$
    since $x_1 \cdots x_p=1$ implies $x_1^{-1}=x_2 \cdots x_p$ and hence $x_2 \cdots x_p \cdot x_1=1$. Moreover, we can apply Lemma \ref{usefullemma} to this action to conclude that $\left|X^{\mathbb{Z}_p}\right| \equiv 0 \bmod p$. 
    The set of fixed points of this action is given by
    % $$
    % X^{\mathbb{Z}_p}=\{(x, \ldots, x) \in X \mid x \in G\}.
    % $$
    $$
    X^{\mathbb{Z}_p} = \{(x, \ldots, x) \in G^p \mid x\cdots x = 1\}
    $$
    since $\sigma$ fixes $\left(x_1, \ldots, x_p\right)$ iff $x_1=x_2=\cdots=x_p$. This set is non-empty, since $(1, \ldots, 1) \in X^{\mathbb{Z}_p}$, and so $\left|X^{\mathbb{Z}_p}\right| \geq p$. 

    This implies that there exists an $x \in G$ with $x \neq 1$ such that $(x, \ldots, x) \in X^{\mathbb{Z}_p}$; that is, $x^p=1$.
\end{proof}

\begin{theorem}
  Let $G$ be a finite group and $p$ be the smallest prime dividing $|G|$. Let $H$ be a
  subgroup of index $p$. Then $H$ is normal in $G$.
  \label{thm:cauchyGeneral}
\end{theorem}
\begin{proof}
    Consider the action of $H$ on the set of left cosets $G/H$. By the orbit-stabiliser theorem, the size of every orbit of cosets divides $|H|$, and hence also divides $|G|$. Since there are exactly $p$ elements of $G/H$, any orbit must simultaneously divide $|G|$ and have cardinality at most $p$, so either we have a single orbit of size $p$ or there are $p$ different orbits of size 1, since $p$ is the smallest prime divisor of $|G|$.
    
    
    
    % cosets of $H$, and $p$ is the smallest prime dividing $|G|$, we must have that there is either a single orbit of size $p$ or there are $p$ different orbits of size 1.
    
    Clearly the first option is impossible, since $H \in G/H$ is a fixed point under the action; 
    % for every $h \in H, hH = H$. Which means the action fixes the coset corresponding to the identity. 
    there is an orbit of size 1, so they must all be of size 1. This means that all of our cosets are fixed points, and for every $h \in H,\text{ } g \in G,$ we have $h g^{-1} H=g^{-1} H.$ So, $\exists h' \in H \text { s.t. } h g^{-1}=g^{-1} h',$ and hence $g h g^{-1}=h' \in H,$ so $H$ is normal in $G.$
    % $$
    %     \begin{gathered}
    %     h g^{-1} H=g^{-1} H \Longrightarrow \exists h' \in H \text { s.t. } h g^{-1}=g^{-1} h' \Longrightarrow \\
    %     g h g^{-1}=h' \in H.
    %     \end{gathered}
    % $$
\end{proof}
\begin{remark}
    We are already familiar with this result for index 2 subgroups by Theorem \ref{thm:index2}. We should 
    note that a subgroup of index $p$ is not guaranteed to exist.
\end{remark}

\section{Lecture 16}
\subsection{Not Burnside's Lemma}
Applications of Cauchy's Theorem and Burnside's lemma (not Burnside's work!)
\begin{example} [Classify groups of order 6]
  Let $G$ be a group of order $6$. Then $G$ is either cyclic or isomorphic to $S_3\equiv
  D_{6}$.
\end{example}
\begin{proof}
  Let $G$ be a group of order 6. By Cauchy's Theorem, there exists an element $h\in G$ of order $2$ and an
  element $k\in G$ of order $3$. Notice that $K=\langle k \rangle$ has index $2$ by
  Lagrange, and so by Theorem \ref{thm:cauchyGeneral} it must be normal in $G$. 
  %say here that G is generated by h and k, how to prove?
  Also, considering orders we can see that $G = \{1_G, k, k^2, h, hk, hk^2 \}$ since each of these elements must be distinct. In other words, $G = \langle h, k \rangle.$
  
  % $\langle h \rangle \cap \langle k \rangle = \{1_{G}\}$
  
  Recall that an automorphism is an isomorphism from a group to itself. Since $K$ is of prime order $p$, the set of automorphisms of $K,\text{ } \Aut(K)$, must have size $p-1$ (proved in ex sheet 2 Q11), so $|\Aut(K)|=2$;
  note that we have actually seen this directly using multiplication tables in Example \ref{ex:multTable}. Let $\varphi \in \Aut(K)$ be the identity map $k \mapsto k$ and $\psi \in \Aut(K)$ be the map defined by $k \mapsto k^2.$
  
  %conjugation by h defines an auto
  %case 1: identity auto
  %case 2: other auto

  Furthermore, we claim that 
  \begin{align*}
      \phi_K: K &\to K \\
      \tilde{k} &\mapsto h \tilde{k} h^{-1}
  \end{align*}
  is an automorphism of $K$, i.e. a bijective homomorphism. 
  % If we can show this, then we have only two options for $\phi_K$.

  Since $K$ is normal in $G$, we have $h\tilde{k}h^{-1}\in K$ for $\tilde{k} \in K$, so $K$ is indeed the codomain. Furthermore, for any two $k_1,k_2\in K,$ we have
  $$\phi_K(k_1k_2)=h(k_1k_2)h^{-1} = 
  hk_1(h^{-1}h)k_2h^{-1} = \phi_K(k_1)\phi_K(k_2),$$ hence $\phi_K$ is a group homomorphism.
  Finally, because $h$ has order $2$,
  $$\phi_K^2(\tilde{k}) = h(h\tilde{k}h^{-1})h^{-1}= h^2 \tilde{k} h^{-2} = \tilde{k},$$ and so $\phi_K$ is self-inverse (bijective), so it is indeed an automorphism.
  % shows that the inverse of $\phi_K$ is itself, so $\phi_K$ is indeed a bijection.

  Now since $\phi_K \in \Aut(K) = \{\varphi, \psi\},$ we have two cases to check.

  First, suppose $\phi_K = \varphi,$ i.e. $hkh^{-1} = k$. Then $hk = kh$, and so $h$ and $k$ commute.
  %say more stuff, conclude that G is cyclic
  %multiplication table????
  Imposing this condition, we can see that the element $hk \in G$ has order 6, since $(hk)^n = h^nk^n,$ and so $|hk| = \lcm(|h|,|k|) = 6.$
  Therefore $hk$ is a generator for $G$, and hence $G$ is cyclic.
  % The fact that $h$ and $k$ commute makes it straightforward to construct the multiplication table of $G$:

  % and one is able to check that this is the multiplication table for $Z_6$, so $G$ is cyclic.

  On the other hand, suppose $\phi_K = \psi,$ i.e. $hkh^{-1} = k^2 = k^{-1}.$
  %cook and conclude that G is D_6
  Then $$G = \langle h, k \mid h^2 = k^3 = 1_G, hk = k^{-1}h \rangle \cong D_6 \cong S_3.$$

  
  
  % We claim that conjugation by $h$ defines an automorphism on $K$. 

  % We will now show that the following is an automorphism,
  % \begin{align*}
  %     \phi_K:K&\to K
  %     \\ k^i &\mapsto hk^ih^{-1}.
  % \end{align*}
  % Since $K$ is normal, we have $hk^ih^{-1}\in K$ for any $i$. Furthermore, for $k_1,k_2\in K,$ we have, 
  % $$\phi_K(k_1k_2)=hk_1k_2h^{-1}=
  % hk_1h^{-1}hk_2h^{-1}=\phi_K(k_1)\phi_K(k_2),$$ hence $\phi_K$ is a group homomorphism.
  % Finally, because $h$ has order $2$,
  % $\phi_K^2(k)=h(hkh^{-1})h^{-1}= h^2 k h^{-2} = k$. This shows that the inverse of $\phi_K$ is itself, so $\phi_K$ is indeed a bijection.
  
  % But since $K$ is of prime order $p$,
  % the set of automorphisms of $K,\text{ } \Aut(K)$, must have size $p-1$ (proved in ex sheet 2 Q11), so $|\Aut(K)|=2$. We have automorphisms defined by $k\mapsto k$ and
  % $k\mapsto k^2$. We showed that $\phi_K\in \Aut(K).$ 
  
  % First, we consider $hkh^{-1}=k$, so $hk=kh$, so 
  % \[G=\langle k,h \mid k^3=e=h^2, kh=hk \rangle\]
  % So $G=\langle k \rangle \times \langle h \rangle$, which is cyclic by exercise sheet 2.
  % In particular, $G=\langle kh \rangle$.

  % Then we consider $hkh^{-1}=k^{2}=k^{-1}$, so we have
  % \[G=\langle k,h \mid k^3=e=h^2, hk=k^{-1}h\rangle\]
  % Which is the dihedral group of order $6$, and this is isomorphic to $S_3$. 
\end{proof}

\begin{definition}
    Let $G$ be a group, $g \in G$ and $X$ the associated $G$-Set. Then we denote $X^{g} := \{x \in X \mid g \cdot x = x\}$, that is, all the points in $X$ that are fixed by a specific element $g\in G.$
\end{definition}
\begin{remark}
    Do not confuse this with $X^G,$ the set of fixed points of an action.
\end{remark}
\begin{definition}
  Let $G$ be a group and $X$ be a $G$-set. The set of $G$-orbits of $X$ is denoted by
  $X/G$. Thus we have $X=\bigsqcup_{\mathcal{O}\in X/G} \mathcal{O}$.
\end{definition}
\begin{remark}
    Note that the above definition uses disjoint union.
\end{remark}
\begin{theorem}[Not Burnside's Lemma]
  Let $G$ be a group and $X$ be a $G$-set. We have 
  \[|X/G|=\frac{1}{|G|}\sum_{g\in G} |X^g|.\]
  \label{thm:notBurnside}
\end{theorem}
\begin{proof}
  Consider the set $S=\left\{ (g,x)\in G\times X \mid g \cdot x=x \right\}$. We will show the required equality by counting $S$ in two different ways.
  On the one side, for each element $g\in G$, we can count the number of elements $x\in X$ satisfying
  %for which that $g$ has 
  $g \cdot x=x$; this is exactly $|X^g|$ by definition. 
  But in the definition of $S$ we are free to choose $g \in G$, so 
  %is defined to be
  % doing this for all g, so |S| = ....
  % Hence, all possible such tuples is 
  $|S|=\sum_{g\in G} |X^g|$.

  On the other hand, we can count the $g\in G$ for each $x\in X$ such that 
  $g\cdot x=x$. Notice that this is just $|\Stab{}_G(x)|,$ which is $|G|/|\Orb_G(x)|$ by the Orbit-Stabiliser Theorem. Let $\mathcal{O}=\Orb_G(x)$ for this section only. Then, 
  % \begin{align*}
  $$
      |S|= \sum_{x\in X} |\Stab{}_G(x)|= \sum_{x\in X} \frac{|G|}{|\mathcal{O}|}.
  $$
  We can translate this sum over all elements of $X$ to a sum over all elements of each orbit:
    
      
  $$ |S| = |G|
  \sum_{\mathcal{O}\in X/G} \sum_{x\in \mathcal{O}} \frac{1}{|\mathcal{O}|}= |G|\sum_{\mathcal{O}\in X/G} 1 = |G||X/G|.
  $$
  % \end{align*}
  Hence
  \[|G||X/G| = \sum_{g\in G} |X^g|,\]
  as required.
\end{proof}


Recall that an orbit of a G-set is a collection of elements that can be transformed into each other by the group action. In other words, it is a set of points that become
indistinguishable under the symmetries of the set under the group action. This is important as it allows us to count each distinct arrangement only once, omitting any symmetrical equivalents. 
\begin{remark}
    An important application of group actions and orbit counting is simplifying counting problems. The next example demonstrates this by counting the distinct ways to color a triangle, considering its symmetries.
\end{remark}
\begin{example}
  How many essentially different ways are there of coloring the sides of a triangle using
  four colours?  Essentially different in that one coloring cannot be obtained from the
  other by applying symmetries of the triangle.
\end{example}
\begin{proof}
  This can be done using non-Burnside's Lemma. Let $X$ be the set of all genuinely different colourings for the triangle sitting still. Then $|X|=4^3=64$ as we have four colours to choose from for each side. 
  Consider the group $D_6$ acting on $X$. Two colourings are essentially different if one cannot be obtained from the
  other by applying symmetries of the triangle, i.e. if they lie in different orbits under
  the action of $D_6$. So the number of essentially different colourings is just the number of orbits, $|X/G|$, which is given by
  \[|X/G|=\frac{1}{|G|}\sum_{\sigma\in D_6}|X^{\sigma}|.\]

    We will now count $|X^{\sigma}|$ for each $\sigma \in D_6.$ In other words, we are counting the number of triangle colourings that are fixed by each $\sigma \in D_6.$ 
  
    First consider $\sigma=e.$ The identity fixes all $x\in X,$ so $|X^e|=|X|=64$.
    
    Next, consider $\sigma$ a reflection through an axis of symmetry. For the triangle to be fixed, we require the two faces we're swapping to be the same colour. We therefore have 4 choices in the first face, then the opposite w.r.t. the axis of symmetry is decided (no choice), and we have 4 choices for the third face. Hence $|X^{\sigma}|=16$. Note that we have three such $\sigma \in D_6$ corresponding to each axis of symmetry so we count $3\cdot 16=48$.

    Finally, consider $\sigma$ a (single) rotation. For the triangle to stay the same, we require all 3 sides to be the same; we have one choice from four colors. Therefore, $|X^{\sigma}|=4$. We have 2 different rotations (the third rotation in the triangle is just the identity), so we count $2\cdot 4= 8$.
  
  Then, by Burnside's,
  \[|X/G|=\frac{1}{6}(64+8+48) = 20.\]
\end{proof}

\section{Lecture 17 - 29 Oct 2021}
\subsection{Semi-direct product}
On semidirect product. This is a generalisation of the direct product, which is the
group theory analogous to the cartesian product in set theory. Recall that we could make
groups like $\ZZ/n\ZZ \times \ZZ/n\ZZ$, this is an example of direct product. 
\begin{definition}
  Let $G$ be a group and let $H\leq G$, $N\trianglelefteq G$ such that $H\cap N = \{e\}$
  and $HN=G$. Then $G$ is called an \emph{(internal) semi-direct product} of $H,N$.
  \label{def:intSemidirProd}
\end{definition}
Note that for every $h\in H$ we can define an automorphism $\phi_h$ in $N$ given by
conjugation of by $h$,
\[\phi_h:n\mapsto hnh^{-1}:N\to N\]
Where the inverse of $\phi_h$ is $\phi_{h^{-1}}$. Moreover, we have $\phi:H\to \Aut N:h\to
\phi_h$ is a group homomorphism undercomposition. Note, $\phi_{h_1h_2}:n\to h_1h_2 n
h_2^{-1}h_{1}^{-1} =\phi_{h_1}(\phi_{h_2}(n))= (\phi_{h_1}\circ\phi_{h_2})(n)$. The entire
group structure of $G$ is determined by the structures of $N$ and $H$ and the homomorphism
$\phi$. Indeed, let $g,g'\in G$, and say $g=nh, g'=n'h'$ for $n,n'\in N, h,h'\in H$, then 
\[gg'=nhn'h'=nhn'h^{-1}hh'=n\phi_h(n')hh'=n''h''\]
For $n''\in N, h''\in H$, since $\phi_h(n')\in N$. Note that the assumption implies that
$G$ is in bijection with $N\times H$ via $(n,h)\mapsto nh$. It's injective and well
defined since $nh=n'h' \iff (n')^{-1}n=h'h^{-1} \iff (n')^{-1}n,h'h^{-1}\in H\cap N =
\{e\} \iff n=n',h=h'$.

\begin{definition}
  Given groups $N$ and $H$, and a group homomorphism $\phi:H\to \Aut N$ we define the
  \emph{(external) semidirect product} of $N$ and $H$ (wrt $\phi$), written $N\rtimes H$
  or $N\rtimes_{\phi} H$, with the multiplication operation,
  \[(n,h)(n',h')=(n\phi_h(n'), hh')\]
  \label{def:exSemidirProd}
\end{definition}
If $G=N\rtimes_{\phi} H$ is as above, we have that $N\times \{1\}$ forms a normal subgroup
of $G$, $\{1\}\times H$ forms a subgroup of $G$, and $G$ is the internal semidirect
product of these two groups.

\begin{example}
  If the definition of the semi direct product we take $\phi:H\to\Aut N$ to be the trivial
  homomorphism (taking every $h\in H$ to $1\in \Aut N$, where $1:n\to n$ for all $n\in
  N$, and note that saying is is saying $hn=nh$ for any $n\in N, h\in H$). Then we recover
  the definition of direct product.
  \label{ex:trivialHomDirecProd}
\end{example}

\begin{example}
  For $n\in\ZZ_{\geq 3}$ the dihedral group $D_{2n}$ is a semidirect product of the normal
  subgroup $\langle \sigma \rangle$ and the subgroup $\langle \tau \rangle$, where $\sigma
  ^n=\tau^2=e$. Here $\phi:\langle \tau \rangle\to \Aut \langle \sigma \rangle$ sends
  $\tau$ to the inversion automorphism, $\phi_{\tau}:\sigma^i\mapsto \sigma^{-1}$.
\end{example}

\begin{theorem}
  Every group of order $15$ is cyclic.
  \label{<+label+>}
\end{theorem}
\begin{proof}
  Let $G$ be a group of order $15$. By Cauchy's we have $g, h\in G$ where $g$ is an
  element of order $5$ and $h$ of order $3$. Moreover, $3$ is the smallest prime divisor,
  and $\langle g \rangle$ has order $5$ so by Lagrange it follows that it has index $3$
  and by Theorem \ref{thm:cauchyGeneral} it follows that $N=\langle g
  \rangle\trianglelefteq G$. 

  Let $H=\langle h \rangle$, then $H\cap N$ is a normal subgroup of
  $H$ by the second isomorphism theorem and by Lagrange it follows that $|H\cap N|$ must
  divide $3$, i.e. $|H\cap N|$ is either $3$ or $1$. Furthermore, $H\cap N$ lies in $N$
  and it must be a subgroup of $N$, hence its order must also divide $5$ or $1$. Hence
  $H\cap N=\{1\}$. Therefore, $NH=G$. 

  Note this implies $G= N\rtimes H$. Hence conjugation by $H$
  defines a group homomorphism $\phi:H\to\Aut N$, but $|\Aut N|=4$. Moreover, $\Img\phi
  \leq \Aut N$, hence $|\Img\phi|$ divides $|\Aut N|=4$ but note that by the First
  Isomorphism Theorem and Lagrange, $|\Img \phi|$ also divides $|H|=3$, hence
  $|\Img\phi|=1$, i.e. the image of $\phi$ is trivial. Therefore every element of $H$
  gives rise to the trivial automorphism of $N$, and since conjugating by every element of
  $h$ leaves $n\in N$ unchanged, it implies that $h$ and $n$ commute.

  Hence, by Example \ref{ex:trivialHomDirecProd} $G=N\times H$. Since $N$ is cyclic of
  order $5$ and $H$ is cyclic of order $3$, note they're coprime, from the second Exercise
  Sheet, $G$ itself is cyclic, $G=\langle n\cdot h \rangle$.
\end{proof}

\subsection{An Aside on Free Groups (Non-Examinable)}

\begin{definition}[Free Group]
Let $S$ be a set, and let $T$ be a set such that each element of $T$ is designated as the inverse of a unique element of $S$, and vice versa. Consider the set of all 'words' that can be formed by concatenating elements of the combined set $S \cup T$, including an empty word (denoted by $e$) which serves as the identity.

We impose the relation of 'reduction' which allows the removal of any adjacent pair of elements $s \in S$ and $t \in T$ where $t$ is the inverse of $s$ (or vice versa), aiming to reduce words to their simplest form. 

The \textbf{free group} on $S$, denoted by $F_S$, is then defined as the set of all such reduced words, equipped with the operation of concatenation followed by reduction. This structure satisfies the group axioms: closure, associativity, identity existence (the empty word), and invertibility (each element can be followed by its inverse leading to their reduction).

Specifically, the relations satisfied by the elements of $F_S$ are precisely those necessary to meet the definition of a group, no more and no less. The group $F_S$ is characterized by the property that any function from $S$ to a group $G$ extends uniquely to a group homomorphism from $F_S$ to $G$. That is the following diagram commutes:
\[
\begin{tikzcd}
S \arrow[r, hook, "i"] \arrow[rd, "f"'] & F_S \arrow[d, "\varphi"] \\
 & G
\end{tikzcd}
\]
\end{definition}

\begin{theorem}
    Every group $G$ is the quotient of a Free group $F_X$ by some normal subgroup $N$.
\end{theorem}

\begin{proof}
    We know that by the universal property of Free groups, $f$ extends to a unique $\varphi$ where $\varphi$ is a group homomorphism. Take $f: X \to G$ to be the canonical inclusion map with $X$ the underlying set of $G$ and $\varphi(x) = f(x).$ $\varphi$ is clearly surjective onto G so by the FIT, $F_X/\ker \varphi \cong G$. So $N = \ker \phi$.
\end{proof}

\begin{theorem}[Uniqueness of Free Groups]
    Let $F$ and $F'$ be free groups on the set $X$. Then $F \cong F'$
\end{theorem}

\begin{proof}
    Suppose \( F\) and \( F' \) are both free groups on the set \( X \). We want to show that \( F \) and \( F' \) are isomorphic.

By the universal property of $F$, for any function $\iota_2: X \to F'$, there exists a unique group homomorphism $\varphi_1: F \to F'$ such that $\varphi_1 \circ \iota_1 = \iota_2$, where $\iota_1: X \to F$ is the inclusion map. Similarly for the universal property of $F'$ we get $\varphi_2 \circ \iota_2 = \iota_1$. This gives the following diagram.

\begin{center}
\begin{tikzcd}
& X \arrow[ld, "\iota_1"'] \arrow[rd, "\iota_2"] & \\
F \arrow[rr, "\varphi_1", bend left=25] & & F' \arrow[ll, "\varphi_2", bend left=25]
\end{tikzcd}
\end{center}

Substituting gives $\iota_1 = (\varphi_2 \circ \varphi_1) \circ \iota_1$ and $\iota_2 = (\varphi_1 \circ \varphi_2) \circ \iota_2$. So by the uniqueness of the universal property, $\varphi_2 \circ \varphi_1 = \id_{F}$ and $\varphi_1 \circ \varphi_2 = \id_{F'}$ so $F \cong F'$ as required. 
\end{proof}




\section{Lecture 18 - 1 Nov 2021}
\subsection{Intro to rings}
\begin{definition}
  A ring is a collection $(R,+,\cdot)$ where $R$ is a set, and $+,\cdot$ are two binary
  operations on $R$ such that 
  \begin{enumerate}
    \item $(R,+)$ is an abelian group (labeled as the additive identity $0$)
    \item The operation $\cdot$ is associative, i.e. for any $a,b,c\in R$ one has
      $(a\cdot b)\cdot c = a\cdot (b\cdot c)$
    \item $\cdot$ distributes over $+$, i.e. for all $a,b,c\in R$ one has 
      \[(a+b)\cdot c = a\cdot c+b\cdot c \text{ and } c\cdot (a+b)=c\cdot a + c\cdot b\]
  \end{enumerate}
  A \emph{unital ring} or a \emph{ring with unity} is a ring as above such that there
  exists $1\in R\setminus \{0\}$ satisfying $1\cdot a = a\cdot 1 = a$ for all $a\in R$.
  A ring is called \emph{commutative} if the operation $\cdot$ is commutative. Note that
  $+$ is always commutative.
  \label{def:ring}
\end{definition}
\begin{remark}
  Note that many authors will refer to a unital ring simply as a ring, and they would
  state otherwise if the ring in question is not unital.
\end{remark}

\begin{example}
  We have already seen examples of rings such as $\ZZ,\QQ,\RR,\CC$. Other more interesting
  ones are,
  \begin{enumerate}
    \item For every $n\in\ZZ_{>1}$, the set of cosets $\ZZ/n\ZZ$ forms a ring under
      addition and multiplication $\mod n$.
    \item If $R$ is commutative ring, then one can form the ring $R[X]$ of polynomials
      over $R$ in one variable. The underlying set is just a set of symbols, not
      necessarily any definite operation. The polynomial hence is just a formal finite
      linear combination of symbols of the form $X^i$ for $i\in\NN$. Multiplication would
      be defined as $X^iX^j=X^{i+j}$. The ring consists of elements of the form
      $\sum_{i=o}^d a_iX^i$, $d\in\ZZ_{\geq 0}$ and $a_i\in R$ for all $i$, with the usual
      addition and multiplication rules.
    \item if $R$ is a ring, then one can form the ring $M_n(R)$ of $n\times n$ matrices
      over $R$ under matrix addition and multiplication.
    \item If $R$ is a ring and $S$ is some set, the set $R^S$ of functions $f:S\to R$ is
      a ring under pointwise addition and multiplication: $(f+g)(s)=f(s)+g(s), (f\cdot
      g)(s)=f(s)\cdot g(s)$. If $S=\NN$ then the function $f$ assigns every natural number
      some element of $R$, i.e. some finite sequence, and we can add and multiply
      pointwise elements of the sequence.
  \end{enumerate}
\end{example}

\begin{remark}
  Let $R$ be a ring, and $a,b\in R$, $n\in\ZZ\setminus \{0\}$.
  \begin{enumerate}
    \item Like in the case of additive groups, we write $-a$ for the additive inverse of
      $a$, $na=a+a+\cdots+a$ if $n>0$ and $na=-a-a-\cdots-a$ if $n<0$.
    \item Write $a-b=a+(-b)$.
    \item If $R$ is unital, we set $a^0=1$ for any $a\in R\setminus\{0\}$. Note however
      that $a^{-1}$ won't exist in general.
    \item We usually drop $\cdot$ and instead write $ab$ rather than $a\cdot b$.
  \end{enumerate}
\end{remark}

\begin{remark}
  Note that in rings, $ab$ do not need to commute, the multiplicative identity need does
  not exist, and the cancellation rule does not need to hold. Be careful with these
  intuitive ideas. Try to derive everything from first principles. Go to the axioms and
  prove it.
\end{remark}


\begin{theorem}
  Let $R$ be a ring, let $a,b\in R$, and let $m,n\in\ZZ$. Then we have 
  \begin{enumerate}
    \item $0a = a0 =0$
    \item $a(-b)=(-a)b=-(ab)$
    \item $(-a)(-b)=ab$
    \item $(m+n)a = ma+na$ (not in the axioms, mind that these are integers, not ring
      elements)
    \item $(mn)a=m(na)$
    \item $m(a+b)=ma+mb$
    \item $m(ab)=(ma)b = a(mb)$
    \item $(ma)(nb)=(mn)(ab)$
  \end{enumerate}
  \label{<+label+>}
\end{theorem}

\begin{definition}
  Let $R$ be a unital ring. An element $u$ of $R$ is called a \emph{unit} if there exists
  $u^{-1}\in R$ such that $uu^{-1}u^{-1}u=1$. The set of units of $R$ is denoted by
  $R^{\times}$. The set of units of $R$ forms a group under multiplication.
  \todo{Note that we defined $R$ to be closed under multiplication but it's not obvious
  that $R^{\times}$ is closed}
  \label{<+label+>}
\end{definition}

\begin{definition}
  A unital ring in which every non-zero element is a unit is called a \emph{division
  ring}. A commutative division ring is called a \emph{field}.
  \label{<+label+>}
\end{definition}

\begin{example}
  The Hamilton quaternions $\HH$ are real vector space with basis $1, \vi, \vj, \vk$, with
  multiplication defined by 
  \[\vi^2=\vj^2=\vk^2=1 \quad \vi\vj=-\vj\vi=\vk, \vj\vk=-\vk\vj=\vi, \vk\vi=-\vi\vk=\vj\]
  And extended to $\RR$-linear combinations by distributivity. The inverse of an element
  $a+b\vi+c\vj+d\vk \neq 0$ has inverse $\frac{a-b\vi-c\vj-d\vk}{a^2+b^2+c^2+d^2}$.
\end{example}


\section{Lecture 19 - 3 Nov 2021}
\subsection{Subrings, ideals, and quotients}
\begin{definition}
  Let $R$ be a ring. A subring is an additive subgroup $S\subset R$ s.t. for every
  $a,b\in S$, $ab\in S$. We usually write $S\leq R$. If $R$ is unital, a subring $S\leq R$
  is called a unital subring if $1\in S$. If $R$ is unital, then we assume every subring
  to be unital, unless otherwise stated.
  \label{<+label+>}
\end{definition}
There's a counter example to the claim that every subring of a unital ring is unital. 

\begin{example}
  We have $\ZZ \leq \QQ \leq \RR \leq \CC \leq \HH$.

  Another interesting example is if $R$ is a commutative ring, then $R$ itself is a
  subring of the ring of polynomials $R[X]$.

  Let $S$ be a subring of the unital ring $R$. Then for every $n\in\ZZ_{>0}$ we have
  $M_{n\times n}(S)$ is a subring of the unital ring $M_{n\times n}(R)$.

  Consider the set of all functions $f:\NN\to\RR$, i.e. all sequences of real numbers.
  Define the pointwise addition and multiplication of elements of the sequence. The subset
  of all sequence that converge to $0$ is a subring itself. The sequence of zeroes is the
  additive identity. This subring is not unital. Hence a counter example to the claim
  under the defintion above, even though the ring is unital, namaing the multiplicative
  identity to be the sequence with all ones. 
\end{example}


Question: What substructures do we need to consider to form quotients?
That depends on the kind of quotient structure we aim for: additive $(a+S)$ or
multiplicative cosets $(bS)$? We can look at existing cosets, naming $\ZZ/n\ZZ$. We will
try to build cosets whose elements consist of additive cosets. From the special case of
additive groups, we know that for the operation of cosets to be well defined we need the
substructure to be abelian. Note that as a group, rings are abelian, so every subring will
also be an abelian subgroup. Let $I\subset R$ be an additive subgroup, and let $a,b\in R$.
We want to define $(a+I)(b+I)=(ab)+I$. Let $a'=a+x, b'=b+y$ for $x,y\in I$ so that $a'$ is
a representative of the same coset that $a$ form, and similarly for $b'$. Then
\[(a+x+I)(b+y+I)= (ab+ay+xb+xy) + I\]
In the one hand, if we take $y=0$, we require $xb\in I$ to have $ab+xb+I=ab+I$. If we take
$x=0$, we require $ab+ay+I=ab+I$, i.e. $ay\in I$. Note that if we have $xb\in I$ and
$ay\in I$ it follows that $xy\in I$.

\begin{definition}
  Let $R$ be a ring. A \emph{left ideal} of $R$ is an additive subgroup $I$ of $R$ s.t.
  for every $a\in I$ and $r\in R$, $ra\in I$. That is, for every $r\in R$, we have
  $rI\subseteq I$.

  A \emph{right ideal} of $R$ is an additive subgroup $I$ of $R$ s.t. for every $a\in I$
  and $r\in R$, $ar\in I$. That is, for every $r\in R$, we have $rI\subseteq I$.

  A \emph{two-sided ideal} is an additive subgroup $I$ of $R$ s.t. it's a left ideal and a
  right ideal. We write $I\trianglelefteq R$.

  An ideal of $R$ is called \emph{proper} if it's not equal to $R$.
  \label{<+label+>}
\end{definition}
Spoiler alert: The two-sided ideals are exactly the conditions we need to create
well-defined addition of cosets. 
\todo{Is it the only condition? Do we need more conditions? Is there a way of building
addition of cosets in a different way than two-sided ideals?}
Note that for an ideal to be an proper, it's necessary for it to not contain the
multiplicative identity, since if it did, then for some $r\in R$ not in the ideal will
produce $r1=r$, which we said is not in the ideal. Hence it can't be an ideal. 

\begin{definition}
  Let $R$ be a unital ring, and $I$ be a proper two-sided ideal in $R$. The
  \emph{quotient ring} has, as its underlying set, the set of cosets $\left\{ r+I : r\in R
  \right\}$. We define the addition of cosets as $(r+I)+(s+I)=(r+s)+I$ and multiplication
  as $(r+I)(s+I)=(rs)+I$.
  \todo{Think about how two-sided proper ideals implies that the operations are
  well-defined, i.e. do not depend on quotient representatives, where two elements
$r+I,s+I$ are said to be the same if $s=r+x \exists x\in I$.}
  \label{<+label+>}
\end{definition}

Note that the quotient ring, which was guided and motivated by $\ZZ/n\ZZ$ is a bit easier
to think about than group quotients, since quotient rings will look like $\ZZ/n\ZZ$, which
is rather easy to visualize, in contrast with quotient groups.
\begin{example}
  Consider the polynomial ring $R=\RR[X]$. The subset $I=X^2 R$, the set of real
  polynomials of degree $2$ or greater, is ideal. Think that for every element $r$ of $R$, the
  product $rh$ for some $h\in I$ will still be in $I$, since the degree of $rh$ will be at
  least $2$ (by $h$). Two polynomials $f=a_0+a_1x+\cdots$, $g=b_0+b_1x+\cdots$ are in the
  same quotient $R/I$ if and only if $a_0=b_0$, $a_1=b=1$ so that $f-g\in I$. I.e. for
  some $a,b\in \RR$, the coset $a+bx+I$ is unique. For any $c\in\RR$, the coset
  $a+bx+cx^2+I$ will be equal for every $c$. Thus $R/I$ can be seen as a $2$-dimensional
  vector space while $R$ is an infinite dimensional vector space, over $\RR$. Here $R/I$
  is spanned by $\hat{1}=1+I$, $\hat{X}=X+I$, with property $\hat{X}^2=0$, since
  $X^2+I=0+I=I$.
\end{example}

\section{Lecture 20 - 5 Nov 2021}
\subsection{Ring homomorphisms}
\begin{example}
  Consider the subring $I=(X^2+1)R$, which is an two-sided ideal subring (think why?). We
  claim that every coset $f+I$ contains a unique polynomial of the form $a_0+a_1x$. It
  follows that $R/I$ is a 2-dimensional vector space $\RR$ spanned by $\hat{1}=1+I$,
  $\hat{X}=X+I$, with property $\hat{X}^2=(X^2+1)+I - (1+I)$ (why? think about it in terms
  of sets, the answer follows), so $\hat{X}^2=I-(1+I)= -\hat{1}$.
\end{example}

\begin{definition}
  Let $R,S$ be rings. Let a ring homomorphism from $R$ to $S$ be a function $\phi:R\to S$
  s.t. for every $a,b\in R$ we have 
  \begin{enumerate}
    \item $\phi(a+b)=\phi(a)+\phi(b)$ (note that this defines a group homomorphism under
      the operation of addition).
    \item $\phi(ab)=\phi(a)\phi(b)$. (note no necessary a group homomorphism bc $R$ is not
      a multiplicative groups, but it must respect the operation).
  \end{enumerate}
  If $R,S$ are unital, then a ring homomorphism is called \emph{unital} if
  $\phi(1_R)=1_S$. Unless otherwise stated, a homomorphism between unital rings will be
  assumed to be unital.

  We say that a homomorphism is an isomorphism if it has a two-sided inverse
  $\phi\inv:S\to R$ s.t. it's also a ring homomorphism. 
  \[\phi \circ \phi\inv = \id_S \quad \phi\inv\circ\phi = \id_R\]
  \label{def:ringHomUnital}
\end{definition}

\begin{theorem}
  A ring homomorphism is an isomorphism if and only if it is bijective.
  \label{<+label+>}
\end{theorem}
\begin{proof}
  If $\phi:R\to S$ is a bijective ring homomorphism, if and only if $\phi\inv$ is also a group
  homomorphism of addition. 

  We claim that $\phi\inv$ also preserves multiplication, i.e.
  $\forall g,h\in S$, $\phi\inv(gh)=\phi\inv(g)\phi\inv(h)$. Note that we have
  $g=\phi(x),h=\phi(y)$ for some $x,y\in R$, hence it follows that
  $gh=\phi(x)\phi(y)=\phi(xy)$. Note that $x=\phi\inv(g),y=\phi\inv(h)$, hence
  $\phi\inv(gh)=\phi\inv(\phi(xy))=xy$ and $xy=\phi\inv(g)\phi\inv(h)$, hence it follows
  that $\phi\inv(gh)=\phi\inv(g)\phi\inv(h)$.

  The reverse direction is trivial.
\end{proof}


\begin{example}
  Let $R$ be a unital ring and $I$ be a proper two-sided ideal. The quotient map $R\to
  R/I, r\mapsto r+I$ is a ring homomorphism.

  Another example is for a commutative ring $R$. Let $r\in R$. The evaluation map
  $\phi_r:R[X]\to R:f\mapsto f(r)$ is a group homomorphism.

  Finally, a more interesting example. Recall that for $I=(X^2+1)\RR[X]$, we had the
  property that $\hat{X}^2=-\hat{1}$, hence it might not be suprising that
  $\RR[X]/I\to\CC$ is a ring isomorphism, by sending $\hat{X}$ to $i\in\CC$.
\end{example}

\section{Lecture 21 - 8 Nov 2021}
\subsection{Isomorphism of rings and Cancellation}
On first isomorphism theorem for rings, zero divisors, and cancellation.

Recall that a subring $U\subset R$ is an additive subgroup of $R$ such that for all
$r,r'\in U$ we have $rr'\in U$. Morover, if $R$ is unital, we require $1_R\in U$. 

% A \emph{two-sided ideal $I$ in a ring $R$ } is an additive subgroup of $R$ such that for
% all $x\in I, r\in R$ we have $xr\in I$, $rx\in I$. Note that if $1\in I$, an ideal, then
% for every $i\in I$ and $r\in R$, then $ir\in R$, but note that if $i=1$, then $1x=x$, so
% we'd require that $I=R$, not a proper ideal. Hence, being an ideal is not a stronger
% condition than being a subgroup.

If a two-sided ideal $I$ of $R$ contains $1$, then $I = R$, since by properties of ideals, $x \in R \implies x = 1x \in I$. 

Recall that if $I$ is an ideal, then we define the quotient $R/I$ to be a ring whose
underlying additive group is the quotient of additive groups, with multiplication
$(r+I)(r'+I) = rr'+I$.

\begin{definition}
  Let $\phi:R\to S$ be a ring homomorphism. The \emph{kernel} of $\phi$ is defined as
  $\ker\phi=\left\{ r\in R : \phi(r)=0_S \right\}$ and the \emph{image} of $\phi$ is
  defined as $\img\phi=\left\{ \phi(r) : r\in R \right\}\subset S$
\end{definition}

\begin{theorem}
  Let $\phi:R\to S$ be a ring homomorphism of (unital) rings. Then
  \begin{enumerate}
    \item The kernel $\ker\phi$ is a proper ideal of $R$;
    \item The image $\img\phi$ is a subring of $S$;
    \item $\phi$ induces a ring isomorphism $R/\ker\phi \to \img\phi$.
  \end{enumerate}
\end{theorem}
\begin{proof}
  \emph{Kernel:} Note that $\ker\phi$ is an additive subgroup by FIT of groups. Moreover,
  note that for $a\in\ker\phi$ and $r\in R$ we have $\phi(ra)=\phi(r)\phi(a)=\phi(r)0=0$
  so $ra\in\ker\phi$ and similarly for $ar\in\ker\phi$. Then note that $1\not\in\ker \phi$
  since we have that $\phi(1_R)=1_S\neq 0_S$. Hence $\ker\phi\lhd R$.

  \emph{Image:} Note that $\Img\phi$ is an additive subgroup by FIT of groups. Morover, let
  $r,r'\in\img\phi$, so $\phi(ss')=\phi(s)\phi(s')=rr'\in S$ since $S$ is a ring the
  product of two elements of the ring will still be in $S$. Hence the claim follows.

  \emph{Isomorphism:} By FIT for groups, we have that the isomorphism exists for the
  additive group structure with $\psi:R\to \operatorname{Im} \phi$ defined by $\psi(r+\ker \phi)=\phi(r)$. As $\psi$ is bijective, we're left with showing that the homomorphism also holds for
  the multiplicative operation. Consider $r_1,r_2\in R$,
\begin{align*}
    \psi(r_1+\ker\phi)\psi(r_2+\ker\phi)&=\phi(r_1)\phi(r_2)
    \\&=\phi(r_1 r_2)
    \\&=\psi(r_1 r_2+\ker\phi)
    \\&=\psi((r_1+\ker\phi)(r_2+\ker\phi)).
\end{align*}
Therefore, we have a bijective ring homomorphism, hence the claim follows.
\end{proof}

\begin{example}[In algebraic number theory]
  Consider the homomorphism $\phi: \RR[X]\to \CC:f\mapsto f(i)$. This is a ring
  homomorphism. Its image is all $\CC$, and its kernel is all multiples of $X^2+1$. Hence
  we have $\ker\phi=(X^2+1)\RR[X]$ and by the FIT, $\RR[X]/(x^2+1)\RR[X]\cong \CC$. Note
  that as a vector space, $\CC$ is 2-dim, while $\RR[X]$ is infinitely dimensional. Hence
  we're saying that we just need two basis to represent every coset. 
\end{example}
\begin{example}[In analysis]
  Consider the set of continuous functions on $\RR$, $C^0(\RR)$. This is a ring under
  pointwise addition and multiplication. The evaluation map $\phi:C^0(\RR)\to \RR:f\mapsto
  f(0)$ is a ring homomorphism, where the image is $\RR$ and the kernel is
  $\ker\phi=\{f\in C^0(\RR): f(0)=0\}$, i.e. all the functions that pass through the
  origin. Hence $C^0(\RR)/\ker\phi \cong \RR$.
\end{example}


\begin{theorem}
  Let $F$ be a \emph{field} (a ring which is also a group under multiplication when
  restricted without the addition identity, like $\RR$), and let $R$ be a non-trivial ring. Then every
  homomorphism $F\to R$ is injective.
\end{theorem}
\begin{proof}
  Let $\phi:F\to R$ be a ring homomorphism. Assume that $\ker\phi$ is not trivial, and hence $\phi$ is not injective. Then there exists non-zero element $f\in F$, $f\in\ker \phi$.
  Then, $\phi(f)=0_R$. Since $F$ is a field, we have $f^{-1}\in F$ s.t. $f^{-1}f=e_F$. So,
  $$1_R=\phi(1_F)=\phi(f f^{-1})=\phi(f)\phi(f^{-1})=0_R.$$ Hence $R$ must be trivial, a contradiction.
\end{proof}
\begin{remark}
  Note that we didn't use the fact that $F$ is commutative. In fact, $F$ does not need to
  be field, but just a division ring (non-commutative field).
\end{remark}

\begin{definition}
  Let $R$ be a ring. An element $a\in R$ is a \emph{left zero divisor} if $a\neq 0$ and
  there exists $b\in R\setminus \{0\}$ s.t. $ab=0$. A right zero divisor is defined
  analogously.
\end{definition}

\begin{example}
  Consider $\ZZ/8\ZZ$. Then $2+8\ZZ$ is a zero divisor since it's non-zero and
  $(2+8\ZZ)(4+8\ZZ)=8+8\ZZ=0+8\ZZ$. 

  Consider $R=\RR[X]/X^2\RR[X]$, so the element $X+X^2\RR[X]$ is a zero divisor, since
  the square of itself is $0+X^2\RR[X]$.
\end{example}


\begin{theorem}
  Let $R$ be a ring. The following are equivalent
  \begin{enumerate}
    \item $R$ has no left zero divisors;
    \item $R$ has no right zero divisors;
    \item For all $a,b,c\in R$ with $a\neq 0$, one has $ab=ac$ if and only if $b=c$.
    \item For all $a,b,c\in R$ with $a\neq 0$, one has $ba=ca$ if and only if $b=c$.
  \end{enumerate}
  \label{<+label+>}
\end{theorem}
\begin{proof}
  We have 1 $\iff$ 2 because the definition of $a$ being a left zero divisor means that $b$
  is a right zero divisor.  We have 1 $\implies$ 3 since, for $ab=ac$ then $ab-ac=a(b-c)=0$,
  where $a\neq 0$ and we have that there are no zero divisors. Hence $b=c$.
  Next show that 3 $\implies$ 1 by contrapositive. Suppose that $a$ is a zero divisor, hence
  $ab=ac$. Let $b$ be the element such that $ab=0$ which is non-zero. Then the case where
  $c=0$ we have $ab=0=ac$ but $b\neq 0$. 
\end{proof}


\section{Lecture 22}
\subsection{Integral domains}
\begin{definition}
  A commutative unital ring with no zero divisors is called an \emph{integral domain}.
  \label{def:integralDomain}
\end{definition}

\begin{corollary}
  Every field is an integral domain.
  \label{<+label+>}
\end{corollary}
\begin{proof}
  Recall that a field is a commutative unital ring in which every non-zero element is a unit. Hence, by the Theorem \ref{thm:unitZero}, a field has no zero divisors and is thus an integral domain.
\end{proof}
\begin{remark}
  The converse is not true. An example of an integral domain that is not a field is
  $\ZZ$.
  \label{<+label+>}
\end{remark}

\begin{theorem}
  A finite integral domain is a field.
\end{theorem}
\begin{proof}
  It suffices to show that every element of a finite integral
  domain has an inverse. Let $R$ be a finite integral domain, and let $r\in R\setminus\{0\}$. Consider the set $\{r^n \mid n\in\NN\}\subseteq R$. Since $R$ is
  finite, there exists $k\in\NN_{\geq 2}$ s.t. $r^k=r$, so $r(r^{k-1}-1)=0$. Since
  $R$ is an integral domain, we must have $r^{k-1}-1 = 0$ since $r \neq 0.$ Equivalently, we can say that $r^{k-2}r=1$, so $r$ is invertible.
  \end{proof}

\begin{corollary}
  Let $p$ be a prime number. Then $\ZZ/p\ZZ$ is a field.
  \label{<+label+>}
\end{corollary}
\begin{proof}
  Since $\ZZ/p\ZZ$ is finite, it suffices to show that it is an integral domain. Let
  $a+p\ZZ, b+p\ZZ\in \ZZ/p\ZZ$. Then observe that $ab+p\ZZ=0+p\ZZ$ if and only if $ab\in
  p\ZZ$, thus either $p \mid a $ or $p \mid b$, i.e. $a \in p\ZZ$ or $b \in p\ZZ$, so either $a+p\ZZ = 0 + p\ZZ$ or $b+p\ZZ = 0 +p\ZZ $.
  % so $ab=np$ for some $n\in\ZZ$. Since $p$ is a prime, it must be the case that
  % either $a=p$, $b=p$, or $n=0$ and $b=0$ or $a=0$. Hence either $a+p\ZZ = 0+p\ZZ$ or
  % $b+p\ZZ=0+p\ZZ$. 
  Hence no zero divisor exists. Therefore $\ZZ/p\ZZ$ is an integral
  domain.
\end{proof}

This is exactly what it means to be prime, as if $ab$ is a multiple of $p$, then one of $a$ or $b$ must be a multiple of $p$. We generalise this definition below in terms of ideals.

\begin{definition}[Prime Ideal]
  Let $R$ be a ring. An ideal $I$ of $R$ is called \emph{prime} if $I$ is a proper ideal,
  and whenever $a,b\in R$ are s.t. $ab\in I$, then one has $a\in I$ or $b\in I$.
  \label{<+label+>}
\end{definition}

\begin{definition}[Maximal ideal]
  Let $R$ be a ring. An ideal $I$ of $R$ is called \emph{maximal ideal} if $I$ is a proper
  ideal and for every other ideal $J$ s.t. $I\subseteq J \subseteq R$ we have that $J=I$
  or $J=R$.
  \label{def:maximalIdeal}
\end{definition}



\section{Lecture 23}
\subsection{Prime and maximal ideals}
\begin{definition}
  Let $R$ be a ring, and $r\in R$. The set $Rr=\{xr \mid x\in R\}$ is called the \emph{left
  ideal generated by $r$}, and this is the smallest left sided ideal containing $r$. The
  \emph{two sided ideal generated by $r$} is
  \[(r)=RrR= \{\text{finite sums }\sum_i x_i r y_i \mid x_i,y_i\in R\}.\]
  This is the smallest two-sided ideal containing $r$. 
   Moreover, if $S$ is a subset of $R$, then
  the ideal generated by $S$ is
  \[(S)=RSR = \{\text{finite sums }\sum_{s\in S} x_s s y_s \mid x_s,y_s\in R \}.\]
\end{definition}

\begin{definition}
     Let $I\unlhd R.$ Then $I$ is called a \emph{principal ideal} if there exists an $r\in R$ such that $I=(r).$
  
\end{definition}
\begin{example}
  Every ideal of $\ZZ$ is principal. This means for any ideal in $\mathbb{Z}$, there exists an integer $n$ such that the ideal can be expressed as $(n) = \{ n \cdot k \mid k \in \mathbb{Z} \}$, the set of all multiples of $n$.
  
  Furthermore, an ideal $(n)$ is a prime ideal if and only if $|n|$ is a prime number. This is because if $|n|$ is not prime, say $n = pq$ with $p,\text{ }q\in \ZZ_{>1}$ both less than $|n|$, then $pq$ belongs to $(n)$ but neither $p$ nor $q$ would be in $(n)$, contradicting the definition of a prime ideal.
  
  Also, note that $(n)$ is prime if and only if it is maximal. This is because any other proper ideal that contains $(n)$ would have to be generated by a divisor of $n,$ and if $n$ is prime, its only divisors are 1 and itself.
\end{example}

\begin{theorem}
  Let $R$ be a ring and let $I,J$ be ideals of $R$. Then $I+J=\{i+j\mid i\in I, j\in J\}$ is
  an ideal of $R$. Furthermore, it is the smallest ideal containing $I$ and $J$.
  \label{<+label+>}
\end{theorem}
\begin{proof}
  Let $x,y\in I+J$, so there exists $i,i'\in I$ and $j,j'\in J$ s.t. $x=i+j$ and $y=i'+j'$. Then $$x-y=(i+j)-(i'+j')=(i-i')+(j-j')\in I+J,$$ so $I+J$ is an additive subgroup by the subgroup test. For $r\in R$, we have that
  $$r(i+j)=ri+rj\in I+J,$$
  since $I$ and $J$ are ideals.
  Therefore, $I+J$ is closed
  under left multiplication. Similarly, $I+J$ is closed under right multiplication, hence it is an ideal.

  Let $L$ be an ideal containing $I$ and $J$. Then clearly $I+J \subseteq L$ since an ideal is closed under addition, so $I+J$ must be the smallest one.
\end{proof}

\begin{theorem}
  Let $R$ be a commutative unital ring, and let $I$ be a proper ideal. Then $I$ is prime
  if and only if $R/I$ is an integral domain.
  \label{<+label+>}
\end{theorem}
\begin{proof}
  Suppose that $R/I$ is an integral domain. Let $a,b\in R$ and suppose 
  $$(a+I)(b+I)=0+I.$$
  Since $R/I$ is an integral domain, we must have that either $a\in I$ or $b\in I$, which is exactly the definition of $I$ being prime.

  Conversely, assume that $I$ is prime and $ab \in I$ for $a,b \in R$. 
  % Let $a+I, b+I\in R/I$, 
  It follows that $$ab+I=(a+I)(b+I)=0+I,$$
  and since $I$ is prime we must have either $a \in I$ or $b \in I.$ Equivalently, $a+I=0+I$ or $b+I=0+I$, as required.
  % From our assumption we have that either $a\in I$ or $ b\in I$, or equivalently, $a+I=0+I$ or $b+I=0+I$, as required.
\end{proof}


\begin{example}
  Let $R=\ZZ[X]$ and consider $I=(X)$. Consider the evaluation map
  \begin{align*}
  \phi: \ZZ[X]&\to \ZZ
  \\ f &\mapsto f(0).
  \end{align*}
  We have that $R/I\cong\ZZ$ by the FIT since $\phi$ has kernel $I$ and image $\ZZ$. Since $\ZZ$ and therefore $R/I$ is an integral domain, $I$ is a prime ideal.
\end{example}

\begin{theorem}
  Let $R$ be a commutative unital ring, and let $I$ be a proper ideal. Then $I$ is maximal
  if and only if $R/I$ is a field.
  \label{<+label+>}
\end{theorem}
\begin{proof}
  Suppose $I$ is maximal. Let $a+I\in R/I$ and assume $a\not\in I$. Consider $I+(a) \supset I$, but since $I$ is
  maximal, then $I+(a)= R$. In particular, $1\in I+(a)$, so there exists $i\in I, r\in R$
  s.t. $1=i+ra\implies ra+I=1+I \implies (r+I)(a+I)=1+I$, meaning $a+I$ is invertible. Hence, since our choice of $a$ was arbitrary, $R/I$ is a field.

  % Suppose $R/I$ is a field. Let $I \subset J$ be an ideal. So there exists $a \in J$ which implies $a+I \eq 0 + I$. As $R/I$ is a field then there exists a $b + I \in R/I$ such that $(a+I)(b+I) = 1+I$. So there exists a $c \in I$ such that $1 = i+ab$

  Suppose $R/I$ is a field and let $J\supset I$ be an ideal of $R$. Then there exists $x\in J$ s.t.
  $x\not\in I$, so $x+I\neq 0+I\in R/I$. Since $R/I$ is a field, $$\exists y+I\in R/I
 \text{ s.t. } (x+I)(y+I)=xy+I = 1+I.$$ In other words, $\exists i\in I$ s.t. $xy+i=1$. But $x\in J\implies xy\in J$, and 
  $i\in I\subset J$, hence $xy+i\in J\implies 1\in J \implies J=R$, so $I$ is a maximal ideal.
\end{proof}


% The following is an immediate consequence of the previous theorems.
\begin{corollary}
  % Let $R$ be a commutative unital ring. Then every maximal ideal of $R$ is prime.
  Every maximal ideal in a commutative unital ring is prime.
  \label{cor:maximalPrime}
\end{corollary}
\begin{proof}
  Let $R$ be a commutative unital ring and let $I$ be a maximal ideal of $R$. Then $R/I$ is a field, so $R/I$ is an integral
  domain, hence $I$ is prime.
\end{proof}

\begin{remark}
  The converse is false in general. For example, consider $\ZZ[X]$ and the the prime ideal $(X)$. We know $\ZZ[X]/(X)\cong \ZZ$ is an integral domain, but $(X)$ is not maximal since $\ZZ$ is not a field. 
\end{remark}
\begin{example}
    Let us consider an ideal that is maximal in $\ZZ[X]$. Let
  $p$ be a prime number, so the ideal $(p,X)=\{pf+Xg \mid f,g\in\ZZ[X]\}$ is the polynomials
  where constant terms are multiples of $p$. The ideal is maximal as $\ZZ[X]/(p,X)$ is a field.
\end{example}

\section{Lecture 24 - 15 Nov 2021}
\subsection{Prime and maximal ideals - division with remainder in polynomials}
The following is an immediate consequence of the previous theorems.
\begin{corollary}
  Let $R$ be a commutative unital ring. Then every maximal ideal of $R$ is prime.
  \label{cor:maximalPrime}
\end{corollary}
\begin{proof}
  Let $I$ be a maximal ideal of $R$. Then $R/I$ is a field, so $R/I$ is an integral
  domain, hence $I$ is prime.
\end{proof}

\begin{remark}
  The converse is false. For example consider $\ZZ[X]$ and the the prime ideal $(X)$. Since $\ZZ[X]/(X)\cong
  \ZZ$ is an integral domain, but $(X)$ is not maximal since $\ZZ$ is not a field. 
\end{remark}
\begin{example}
    Let us consider an ideal that is maximal in $\ZZ[X]$. Let
  $p$ be a prime number, so the ideal $(p,X)=\{pf+Xg: f,g\in\ZZ[X]\}$ is the polynomials
  where constant terms are multiples of $p$. The ideal is maximal as $\ZZ[X]/(p,X)$ is a field.
\end{example}
\begin{definition}
  Let $F$ be a field, and let $f(X)=a_0+a_1X+\cdots + a_d X^d\in F[X]$ be a polynomial
  with $a_d\neq 0$. The degree $\deg f$ of $f$ is defined as $d$.
  \label{<+label+>}
\end{definition}

\begin{theorem}[Division with remainder in polynomial rings]
  Let $F$ be a field, and let $f(X)$, $g(X)\in F[X]$. Then there exist unique $q(X),
  r(X)\in F[X]$ s.t. 
  \[f(X)=g(X)q(X)+r(X)\]
  and s.t. either $r(X)=0$ or $\deg r < \deg g$.
  \label{thm:divisionReminderPolynomial}
\end{theorem}
\begin{proof}
  \emph{Proof of existence.} Note $\deg g> \deg f \implies q=0, r=f$. Next, consider
  $f(x)=a_nx^n+\cdots+a_1x+a_0$, and $g(x)=b_mx^m+\cdots+b_1x+b_0$, with $a_n,b_m\neq 0$
  and $n\geq m$. Assume the statement is true for all $f_0$ with $\deg f_0<\deg f$. Let 
  \[f_0(x)=f-\frac{a_n}{b_m}x^{n-m}\cdot g = a'_{n-1}x^{n-1}+\cdots+a'_0\]
  So $\deg f_0< \deg f$. Then by induction we have that there exists $q_0, r\in F[X]$ with
  either $r=0$ or $\deg r< \deg g$ s.t.  $f_0=q_0g + r \implies f=
  (q_0+\frac{a_n}{bm}x^{n-m})g + r$

  \emph{Proof of uniqueness}. For the sake of contradiction, assume that there exists
  $q,q',r,r'\in F[X]$ s.t. $f(x)=qg+r=q'g+r'$ and each $r,r'=0$ or $\deg r,r'<\deg g$.
  However, not that then 
  \[qg+r=q'g+r' \implies g(q-q')=r'-r.\]
  Note that $\deg g(q-q')\geq \deg g$ since $q\neq q'$ and $g$ is not the $0$ polynomial
  since no polynomial has divisor $0$. Henceforth, $\deg r'-r\geq \deg g$, a
  contradiction.
\end{proof}

\begin{corollary}
  Let $F$ be a field and let $f\in F[X]$, $a\in F$. Then $f(a)=0$ if and only if $\exists
  h\in F[X]$ s.t. $f(X)=(X-a)h(X)$.
  \label{<+label+>}
\end{corollary}
\begin{proof}
  %backwards
  We have $f(X) = (X-a)q(X)+r(X) \implies f(a)=0$ trivially. 
  %forwards
  Assume $f(a)=0.$ By the previous theorem we have that there exists $q(X),r(X)\in F[X]$, such that $f(X) = (X-a)q(X)+r(X)$. We must have $\deg r < \deg (X-a) = 1$, so $r(X)$ is a constant. We have from our assumption that $f(a)=r(X)=0$, hence the result follows.
\end{proof}

\section{Lecture 25 - 17 Nov 2021 }
\subsection{Prime ideals in polynomial rings}
\begin{corollary}
  Let $F$ be a field and let $f\in F[X]$ be a non-zero polynomial with $n=\deg f$. Then
  $f$ has at most $n$ roots.
  \label{<+label+>}
\end{corollary}
\begin{proof}
  Applying the previous corollay inductively. If $a_1,\cdots, a_n$ are distinct zeros of
  $f$, then by the previous corollary we have $f(x)=\prod_{i=1}^n (x-a_i) h(x)$, and since
  the linear polynomial term already will have, after expanding, degree $n$, then the
  product with $h$ will have degree at least $n$.
\end{proof}

\begin{definition}
  Let $f\in F[X]$ be a polynomial over the field $F$ with $\deg f>0$. Then $f$ is
  \emph{irreducible} if whenever $g,h\in F[X]$ are such that $f=gh$, one has either $\deg
  g=0$ or $\deg h=0$. If $f$ is not irreducible, then it's called \emph{reducible}.
\end{definition}
\begin{remark}
  The notion of irreducability is relative to the field you're working on.
\end{remark}
\begin{example}
  Consider $f(x)=X^2+1$. This is irreducible in $\RR[X]$. However, in $\CC[X]$, we can
  write $f(x)=(X-i)(X+i)$, hence $f$ is reducible in $\CC[X]$.
\end{example}

\begin{theorem}
  Let $F$ be a field and let $I\subseteq F[X]$ be an ideal. Then there exists $f\in
  F[X]$ s.t. $I=(f)=\{gf : g\in F[X]\}$.
  \label{thm:idealPrinciplas}
\end{theorem}
\begin{proof}
  If $I=\{0\}$, then $I=(0)$. Next, consider $g\in I\setminus\{0\}$ be s.t. $\forall g'\in
  I\setminus\{0\}$, $\deg g \leq \deg g'$. Then since $I$ is an ideal,  $\forall h\in
  I\exists f\in F[X]$ s.t.  $gf=h$. In order words, 
  \[I= \{ 0, g, f_1g, f_2g, \cdots f_i\in F[X] \}\]
  Since $I$ is an ideal, we have that for every $h=f_k g \in I$ and for every $l\in F[X]$
  we have $lh\in I$, so $lfg \in I$. In other words, $I=\{fg | \forall f\in F[X]\}$, or
  $I=(g)$.
\end{proof}
\begin{proof}[Bartel's proof]
  If $I=\{0\}$, then $I=(0)$. Consider $f\in I\setminus \{0\}$ be of the smallest degree
  that is non-zero. We calim $I=(f)$. Let $g\in I$ and note that by Theorem
  \ref{thm:divisionReminderPolynomial} we have that there exists unique $q,r\in F[X]$ s.t.
  $g=qf+r$ and $r=0$ or $\deg r < \deg f$. Since we're considering the ideal $I$, it
  follos that $qf\in I$ and $g\in I$ by assumption, hence $r\in I$ and $r=g-qf$. But since
  we have $\deg f$ is the minimal, it follows that $r=0$ (because $0<\deg r<\deg f$ would
  contradict that $f$ was the smallest non-zero degree). Since $g$ was arbitrary, it
  follows that $I=(f)$.
\end{proof}
\begin{remark}
  This theorem does not hold if $F$ is not a field.
\end{remark}

\begin{example}
  Let $I$ be the ideal of $\QQ[X]$ generated by $(X^2+X)$ and $(X^4+X^3+X)$, 
  \[I=\{f(X^2+X) + g(X^4+X^3+X) : f,g\in \QQ[X]\}\]
  By the above theorem, we have that 
  \[I=(X)\]
\end{example}

\section{Lecture 26}
\subsection{Irreducible polynomials}
\begin{theorem}
  Let $F$ be a field, and let $f\in F[X]$ be non-zero. Then the following are equivalent
  \begin{enumerate}
    \item The ideal $(f)=\{fg \mid g\in F[X]\}$ is maximal;
    \item The ideal $(f)$ is prime;
    \item The polynomial $f$ is irreducible in $F[X]$.
  \end{enumerate}
  \label{<+label+>}
\end{theorem}
\begin{proof}
(1$\implies$2) Suppose $(f)$ is maximal. Then $(f)$ is maximal $\implies$ $F[X]/(f)$ is a field $\implies$ $F[X]/(f)$ is an integral domain $\implies$ $(f)$ is prime. (Corollary \ref{cor:maximalPrime})

(2$\implies$3) Suppose the ideal $(f)$ is prime. Consider $f = gh$ for some $g,h \in F[X]$, so $g \in (f)$ or $h \in (f)$. Assume without loss of generality that $g \in (f)$, i.e., there exists $k \in F[X]$ such that $g = fk$. Now, we have that $\deg f = \deg g + \deg h$ and $\deg g = \deg f + \deg k$. Hence, $\deg h = \deg k = 0$, so $h,k \in F$ and are units. Therefore, $f$ is irreducible in $F[X]$.    

\\(3$\implies$1) Suppose $f$ is irreducible in $F[X]$. For $f$ to be irreducible, it must be of degree greater than zero and therefore $(f)$ is a proper ideal. Assume there exists an ideal $J$ such that $(f) \unlhd J\unlhd F[X]$. By Theorem \ref{thm:idealPrinciplas}, $F[X]$ is a PID, so $\exists g \in F[X]$ s.t. $J = (g)$. Since $f \in (f) \subseteq (g)$ $\exists h \in F[X]$ s.t. $f = gh$. Since $f$ is irreducible, either $g$ or $h$ is a unit. Suppose $g$ is a unit. Then $\exists g^{-1} \in F[X]$ s.t. $gg^{-1} = 1$, so $1\in(g)\implies J = F[X]$. Suppose now that $h$ is a unit. Then $\exists h^{-1}\in F[X]$ with $hh^{-1}=1$, so $f=gh\implies g=fh^{-1}$, which means $(g)\subseteq(f)\implies(g)=(f)$. We conclude $(f)$ is maximal.
\end{proof}

This theorem gives us a way of easily constructing new fields.
\begin{example}
  Let $F=\ZZ/3\ZZ$, which is a field. The polynomial $f(X)=X^2+1\in F[X]$ is
  irreducible. It follows that $F[X]/(f)$ is a field. Recall that for every $g\in F[X]$
  there exists $h=a_0+a_1x\in F[X]$ s.t. $g+(f)=h+(f)$. Hence we have
  \[F[X]/f(x) = \left\{ a_0+a_1X \mid a_0,a_1\in F \right\}\]
  Note that it has 9 elements.
\end{example}
\begin{theorem}
  Let $F$ be a field and let $f\in F[X]$ have degree 2 or 3. Then $f$ is reducible if and
  only if it has a root in $F$, i.e. iff there exists $a\in F$ s.t. $f(a)=0$.
  \label{thm:deg2o3Root}
\end{theorem}
\begin{proof}
  If $f$ is reducible, then $f=gh$ for some $0<\deg g<\deg f$, $0<\deg h<\deg f$ and
  $\deg g + \deg h = \deg f$, which implies at least one of the factors has degree 1,
  which is of the form $a_0+a_1X$ and has root $\frac{-a_0}{a_1}\in F$.

  Conversely, assume $f(a)=0$ for some $a\in F$. Then Corollary \ref{cor:rootFactor} implies $\exists h\in F[X]$ with $f=h\cdot(X-a)$. Since $X-a$ has degree 1, $h$ must be of degree 1 or 2, depending on whether $f$ is of degree 2 or 3 respectively, so $f$ is reducible. 
\end{proof}


\begin{example}
  Let $d\in \ZZ$ be non-square. Then the corresponding quadratic number field is defined as
  \[\QQ[X]/(X^2-d) \cong \QQ[\sqrt{d}]=\left\{ a+b\sqrt{d} \mid a,b\in \QQ \right\}\]
  \[a+bX+(X^2-d) \mapsto a+b\sqrt{d},\]
  where the inverse of an element $a+b\sqrt{d}$ is $\frac{a-b\sqrt{d}}{a^2-b^2d}$.
\end{example}

\section{Lecture 27 - 22 Nov 2021}
\subsection{Intermission: Classifying groups of order 21}
\begin{example}
  Let $G$ be a group of order 21. How many possibilities are there for $G$.
\end{example}
\begin{proof}
  Note that $21=7\cdot 3$, and by Cauchy, there are elements of $G$, say $g,h\in G$, with
  order $7$ and $3$, say $g^7=e=h^3$. Let $K=\langle g\rangle$ and $H=\langle h\rangle$.
  Note that $K\cap H=\{1\}$, since by Lagrange we have that $|K\cap H|$ divides $|K|$ and
  $|H|$, and $\gcd(|H|,|K|)=1$, and the intersection is required to be a group itself.
  Moreover, note that $K\trianglelefteq G$ since $[G:K]$ being the smallest prime divisor
  of $G$ implies $K\trianglelefteq G$.

  Next, let $h\in H$ and consider the automorphism $\phi_h:K\to K:g^i\mapsto hg^ih^{-1}$
  (it's rather easy to see that the map is well defined and surjective, by normality, it's
  a homomorphism by the general argument of conjugation as a homomorphism, and it is an
  isomorphism because the inverse is $\phi_{h^{-1}}$). Suppose that $hgh^{-1}=g^r$, and note
  that the choice for $r$ will completely determine the automorphism (since $g$ generates
  $K$). Note that $\phi_h^{j}(g)=h^jgh^{-j}=g^{r^j}$ and we require $j=3\implies h^3=e$
  hence we require $g^{r^j}=g^1$. Hence we require $r^3\cong 1 \mod 7$, which are obiously
  $r=1$, $r=2$, $r=4$. 

  An alternate point of view is that, the function $H\to \Aut K:x\mapsto \phi_x$, where
  $\Aut K$ is a group under composition and the identity is the identity automorphism, and
  note that there are $|K|-1=6$ possible such automorphisms.  Moreover, we have $K$ is
  cyclic of order $7$, i.e. $\Aut K \cong (\ZZ/7\ZZ)^{\times}$, its automorphism group is
  group under composition. More generally we have that $\Aut(\ZZ/n\ZZ)\cong
  (\ZZ/n\ZZ)^{\times}$.  In fact, note that when $n$ is prime, the unit groups before is
  cyclic of order $n-1$. Note also that for a cyclic group of order $n$, every elemenet of an
  order dividing $n$ generates exactly one subgroup of that order. So $\Aut K$ being
  a cylic of order $6$ we have that there is exactly one subgroup of order $3$.

  Hence we have the following possibilities. 
  \emph{Case 1: $\phi_h(g)=g$} -- Then $gh=hg$. Note that the semidirect product using
  this conjugation rule gives $HK=G$ since they intersect trivially and the product of
  their order is the order of the group, and the fact that the elements commute gives
  raise to the fact that the group is cylic with generator $gh$.
  \emph{Case 2: $\phi_h(g)=g^2$} -- Non-commuting where the semidirect product of $H$ and
  $K$ using this conjugation rule gives $K\rtimes H\cong C_7\rtimes C_3$.
  \emph{Case 3: $\phi_h(g)=g^4$} -- Note that this hs the case for the previous
  $\phi_h(g)=g^2$, but $\phi_h^2(g)=\phi_{h^{-1}}=g^4$ (since $H$ has order 3).
\end{proof}<++>

\section{Lecture 28 - 24 Nov 2021}
\subsection{Irreducibility criteria}
\begin{definition}
  A polynomial $f(X)$ ins $\ZZ[X]$ is called primitive if the $\gcd(a_n,\cdots,a_0)=1$,
  i.e. if there exists no prime number $p$ that divides all $a_i$.
  \label{def:primitivePol}
\end{definition}

\begin{theorem}[Gauss's Lemma]
  Let $f\in \ZZ[X]$ be primitive. Then $f$ is reducible in $\QQ[X]$ into polynomials of
  degree $r,s\in \NN_{\leq n-1}$ iff it factorises as a product of degree $r$ and $s$
  polynomials in $\ZZ[X]$.
  \label{thm:GaussLemmaPol}
\end{theorem}
\begin{proof}
  \todo{Do this. Suppose $f$ is primitive and can be expressed as the product of two
  polynomials in $\ZZ[X]$, and show that then it can also be expressed as the product of
two polynomials in $\QQ[X]$, i.e. if it is reducible in $\ZZ[X]$ then it's also reducible
in $\ZZ[X]$.}
\end{proof}
\begin{remark}
  Note that the working may be a bit confusing. All the above thm is saying is that if a
  polynomial is primitive, then it's reducible in $\QQ$ iff it's reducible in $\ZZ$. 
\end{remark}
\begin{example}
  The polynomial $X^4+2$ is irreducible in $\QQ[X]$. If it was reducible, then $X^4+2=fg$,
  then we must have either $\{\deg f, \deg g\}=\{1,3\}$, and so one must have a root in
  $\QQ$. But we have that $X^4+2$ is strictly positive, hence this is absurd. 

  The other posibility is $\deg f=\deg g=2$. By Gauss's Lemma, it suffices to show that
  there's no factorization in $\ZZ[X]$. Assume $f,g\in\ZZ[X]$, s.t. $f=a_2x^2+a_1x+a_0$,
  $g=b_2x^2+b_1x+b_0$, and note that since $X^4$ has coeff $1$, we must have $a_2=b_2=1$
  (the case where they're both $-1$ is equivalent). However, expanding the product it's
  easy to see that we require a square of the integers to be $3$, and that's impossible.
  Hence by Gauss's Lemma, the polynomial is irreducible in $\QQ[X]$.
\end{example}

\begin{theorem}[Eisenstein's Criterion]
  Let $p$ be a prime number, and let $f(X)=a_nX^n+\cdots+a_0\in \ZZ[X]$ be s.t.
  $p\nmid a_n$, $p\mid a_i$ for all $i<n$, and $p^2\nmid a_0$. Then $f$ is
  irreducible in $\QQ[X]$.
  \label{thm:eisensteinCriterion}
\end{theorem}
\begin{proof}
  \todo{I think direct should work. Freighleigh has the proof too, but let's try it
  ourselves.}
\end{proof}

\begin{example}
  We can apply Eisenstein's criterion to $X^4+2$ to get an easy proof. 
\end{example}
\begin{example}
  Let $p$ be a prime number. We have that $f(X)=X^{p-1}+\cdots+X+1$ is irreducible in
  $\QQ[X]$. Note that $f(X)=g(X)h(X)$ iff $f(X+1)=g(X+1)=h(X+1)$ (invertible
  substitution). We can use the reverse direction to find that $f(X+1)$ is irreducible by
  using Eisenstein's criterion. In particular, all degrees $<p-1$ have coefficients
  divisible by $p$ and the constant term is $p$, hence satisfying Eisenstein. I'll skip
  details but the lecturer gives a nice trick by rewriting $f(X)=\frac{x^p-1}{x-1}$ and
  using variable substitution $x'=x+1$, and seeing how the combinatorics argument arises
  naturally and easily.
\end{example}

\section{Lecture 29 - 26 Nov 2021}
\subsection{Ring theory -- Fixing loose ends}
Note that even though we have used the usual function notation for working with
polynomials, polynomials are not really functions. They can be defined as a function, but
that does not hold in general. Consider the ring $R=\ZZ/p\ZZ$ and  the polynomial ring
$R[X]$. Note that $R$ is indeed a field and that for $f(X)=X^p-X\in R[X]$ we have $f(a)=0$
for all $a\in R$. Hence note that a polynomial can be the zero function without being the
$0$ polynomial of the ring. Hence $f$ as $R\to R$ is the $0$ function, but it's not the
$0$ polynomial of $R[X]$.

\begin{theorem}[Universal Property of Polynomial Rings]
Let $\phi: R \longrightarrow S$ be any ring homomorphism and let $s \in S$ be any element of $S$. Then there is a unique ring homomorphism
$$
\psi: R[X] \longrightarrow S,
$$
such that $\psi(X)=s$ and which makes the following diagram commute
\begin{center}
\begin{tikzcd}
R \arrow{r}{\phi} \arrow[swap]{d}{i} & S \\
R[X] \arrow[dashed]{ur}[swap]{\exists ! \psi}
\end{tikzcd}
\end{center}
\end{theorem}

\begin{proof}
    We define the map $\psi: R[X] \longrightarrow S$ as follows, for any $f \in R[X]$ with $f = a_0 + a_1X + \ldots + a_nX^n$ with $a_i \in R$, set $\psi(f) = \phi\left(a_0\right)+\phi\left(a_1\right) s+\phi\left(a_2\right) s^2+\ldots+\phi\left(a_n\right) s^n$. This clearly extends $\phi$ as $\psi(r) = \phi(r)$ for $r \in R$ and it also maps $X$ to $s$. One can then check it is also a ring homomorphism. Let $g=b_0+b_1x+\ldots +b_mx^m.$ The proof for additivity is simple. We check multiplication is preserved,
    \begin{align*}
        \psi(fg)&=\psi((a_0 + a_1X + \ldots + a_nX^n)(b_0+b_1x+\ldots +b_mx^m))
        \\ &=\psi((a_0b_0+a_0b_1x+b_0a_1x+a_1b_1x^2+\ldots +a_nb_mx^{n+m})
        \\ &= \phi(a_0b_0) +\phi(a_0b_1)x+\phi(b_0a_1)x+\phi(a_1b_1)x^2+\ldots+\phi(a_nb_m)x^{n+m}
        \\ &= \phi(a_0)\phi(b_0)+\phi(a_0)\phi(b_1)x+\phi(b_0)\phi(a_1)x+\phi(a_1)\phi(b_1)x^2+\ldots+\phi(a_n)\phi(b_m)x^{n+m},
    \end{align*}
    where the last equality follows from $\phi$ being a ring homomorphism. Now, factoring the equation back out gives,
    \begin{align*}
        \psi(fg)&=(\phi(a_0)+\phi(a_1)x+\ldots+\phi(a_n)x^n)(\phi(b_0)+\phi(b_1)x+\ldots+\phi(b_m)x^m)
        \\ &=\psi(f)\psi(g),
    \end{align*}
    as required. 
\end{proof}

Making this definition slightly less general and taking $\phi$ to be the identity map, we end up recovering $\psi$ as the evaluation maps we have seen previously. If we think of polynomial rings as a pointed $R$-algebra, then there is a unique pointed algebra homomorphism from $R[X]$ to some $S$ which is the evaluation homomorphism.

\begin{remark}
    A pointed set is a set in which we fix a basepoint, so its an ordered pair $(X, x_0)$ where X is a set, in our case our basepoint is the element we use to define our evaluation homomorphism.
\end{remark}

So we can define a polynomial ring via a unique evaluation map, this tells us why it is okay to interchange the two notions that polynomials are either formal sums or functions. As mentioned previously we do have to be careful in the case of finite fields.

\begin{example}
  Consider the group $A=(\ZZ\times\ZZ, +)$. The set of
  homomorphisms from $A$ to $A$ forms a ring under 
  \[(f+g)(a)= f(a)+g(a)\]
  \[(fg)(a) = f(g(a)) \forall a\in A\]
  For all $f,g:A\to A$. A homomorphism from a group to itself is called an
  endomorphism. The set of endomorphisms with these two operations is called the
  endomorphism ring, called $\End(A)$. We claim this ring is noncommutative. Take as
  an example $\phi:(m,n)\mapsto (0,n)$ (a ring homomorphism), and $\psi:(m,n)\mapsto
  (m+n,0)$. Then note $\phi\psi:(m,n)\mapsto (0,0)$, and $\psi\phi:(m,n)\mapsto (n,0)$,
  hence they are not the same: take $(\phi\psi)(0,1)=(0,0)\neq
  (\psi\phi)(0,1)=(1,0)$. In fact, note that these two elements serve as a basis for the
  whole ring -- we can get any $(m,n)$ by $m(1,0)+n(0,1)$.
\end{example}

Note that earlier in the course we talked about Automorphisms, which are really just
invertible endomorphisms. And following the above example, it follows that
$\Aut(A)=(\End(A))^{\times}$, the group of units of $\End(A)$.

\section{Lecture 30}
\subsection{Field Extensions}
\begin{definition}[Field Extension]
  Suppose $K$ is a field, and let $F$ be a field containing $K$. Then $F/K$ is a
  \emph{field extension}.
  \label{def:fieldExt}
\end{definition}

\begin{remark}
% In algebra we don't usually refer to ``extensions'' in the context of rings or groups (when we usually look inward to ideals and subgroups) in the same way we do with fields. 
With groups and rings, typically we find it useful to look inward to substructures (subgroups and ideals), however the interest with fields is to look outward to extensions.
This is because if $F/K$ is a field extension, we can view $F$ as a vector space over $K.$ The notation $F/K$ does not imply a quotient like it might in other algebraic structures. Instead, it is simply a way to describe the field $F$ as a structure that contains $K$ and has a vector space structure relative to $K$.
\end{remark}

\begin{example}
The following is a familar example of a field extension:
  $\CC/\RR$ -- a $2$-dim vector space over $\RR$ (the \emph{complex plane}).

This is a more interesting example to think about:
  $\RR/\QQ$ -- an $\infty$-dim (uncountably) vector space over $\QQ$.
\end{example}

\begin{example}
 A more useful definition of a field extension comes from considering embeddings into bigger fields and not just actual set containment.
 
  Let $K$ be a field and consider an irreducible polynomial $f\in K[X]$. Then we can define $F=K[X]/(f)$; this is a field. 

  Note that $K$ is not contained in $F$ in the regular notion of set containment, but it can be embedded into $F.$
  
  Consider the quotient map $\phi:K[X]\to K[X]/(f)$, $g\mapsto g+(f)$. Observe that
  $K$ is actually a subring of $K[X]$ (constant polynomials).

  Consider the map $\phi$ restricted to $K$, $\phi |_{K}.$
  An irreducible polynomial has degree at least 2, therefore, for $\phi(k)=k+(f)=0,$ $k$ must have degree at least 2. So, $\phi |_{K}$ is injective as it consists only of degree 0 polynomials (and zero).

  We can conclude that $F/K$ is a field extension.
\end{example}

\begin{remark}
    Notice that in the example above, we do not have that the set $K$ is actually contained in $F,$ but rather that it can be naturally embedded into $F.$ Because of this, we can still view $F/K$ as a field extension.
\end{remark}

\begin{example}
  Consider $K=\RR$ and $f=X^2+1\in \RR[X]$, which is irreducible. Define $F=\RR[X]/(f)$
  which is a $2$-dimensional field extension of $\RR$ with basis $1+(X^2+1), X+(X^2+1)$. Let $\alpha$ be the image
  of $X$ under the previous quotient map
  \begin{align*}
      \phi: K[X] &\to K[X]/(X^2+1) \\
      k &\mapsto k + (X^2+1).
  \end{align*}
  % $$K[X]\to K[X]/(X^2+1).$$ 
  Notice $\alpha^2+1=0 + (X^2+1)$ in
  $F$, so $\alpha$ satisfies $g(t)=t^2+1\in K[X]$. Thus $F$ is isomorphic to $\CC$.
\end{example}

\begin{theorem}
  Let $K$ be a field and let $f\in K[X]$ with $\deg f>0$. Then there exists a field
  extension $F/K$ and $\alpha \in F$ s.t. $f(\alpha)=0$.
\end{theorem}
\begin{proof}
  Let $f=gh$ where $g$ is irreducible. Define $F=K[X]/(g)$. Then take $\alpha=X+(g)$, and so $g(\alpha)=g(X)+(g)=0+(g)$, so $f(\alpha) = g(\alpha)h(\alpha)=0$.
\end{proof}

\begin{definition}
  Let $F/K$ be a field extension, and $\alpha\in F$. We say $\alpha$ is \emph{algebraic} over
  $K$ if there exists $f\in K[X]\setminus \{0\}$ s.t. $f(\alpha)=0$. If there is no such
  polynomial, we say that $\alpha$ is \emph{transcendental} over $K$.
\end{definition}

\begin{example}
  For all $d\in \QQ$, $\sqrt{d}\in\CC$ is algebraic over $\QQ$, consider
  $f(x)=x^2-d\in\QQ[X]$.

  For all $n\in\ZZ_{>0}$, $e^{2\pi i/n}\in \CC$ is algebraic over $\QQ$, consider
  $f(x)=x^n-1$.

  Transcendental elements of $\CC$ include $e$ and $\pi$, but proving they are transcendental is very difficult.
\end{example}

\begin{theorem}
  Let $F/K$ be a field extension, and let $\alpha\in F$ be algebraic over $K$.
  Then there exists an irreducible polynomial $p\in K[X]$ with $p(\alpha)=0$, and for
  all $f\in K[X]$ with $f(\alpha)=0$, we have $p$ divides $f$ in $K[X]$. \label{thm:pdivf}
\end{theorem}
\begin{proof}
  First, $\alpha$ being algebraic over $K$ implies that there exists $f\in K[X]\setminus \{0\}$ with $f(\alpha)=0$. Let $p\in K[X]$ be such an $f$ with the smallest degree, so $p(\alpha)=0.$ Assume for contradiction that $p$ is reducible, i.e., there exists $g,h\in\K[X]$ with degrees $r,s<n$ respectively, such that $f=gh.$ This means that $f(\alpha)=g(\alpha)h(\alpha)=0,$ so we must have $g(\alpha)=0$ or $h(\alpha)=0.$ This is a contradiction as we have assumed that $p$ is the smallest such polynomial, therefore $p$ must be irreducible over $K[X].$

  Now suppose that $f \in K[X]$ with $f(\alpha) = 0$, by the division algorithm we can write $f=pq+r$ where $\deg r < \deg p$. We have 2 cases, if $r=0$ then $f = pq$ so $p$ divides $f$. Assume $r \neq 0$ then we have that $r(\alpha) = f(\alpha) - p(\alpha)q(\alpha) = 0$, but as $p$ was minimal degree, it follows that $r = 0$, and so $f=pq$ as required.
\end{proof}

% We call such a $p \in K[X]$ the minimum polynomial of $\alpha$ over $K$.

\begin{corollary}
  Let $F/K$ be a field extension, and let $\alpha\in F$ be algebraic over $K$. Let
  $p,p'\in K[X]$ be irreducible polynomials s.t. $p(\alpha)=p'(\alpha)=0$. Then
  $\exists\lambda\in K$ s.t. $p=\lambda p'$.
\end{corollary}
\begin{proof}
  We have from theorem \ref{thm:pdivf} that $p \mid p'$ and $p' \mid p$. This implies $\deg p=\deg p'$, and the result follows.
\end{proof}

\begin{definition}
  A polynomial $f=a_nx^n+\cdots+a_0\in K[X]$ is called \emph{monic} if $a_n=1$.
  Let $F/K$ be an extension field and $\alpha$ be algebraic over $K$. The unique monic
  irreducible polynomial $p\in K[X]$ s.t. $p(\alpha)=0$ is called the irreducible
  polynomial of $\alpha$ over $K$, or the minimal polynomial of $\alpha$ over $K$, written
  as $\irr (\alpha, K)$.
\end{definition}

\begin{example}
  Recall that if $n\in\ZZ$, then $e^{2\pi i/n}\in\CC$ is algebraic over $\QQ$ since it is a
  root of $x^n-1$.
  
  Let $\alpha =e^{2\pi i/n}$ and suppose that $n$ is prime. Then $\irr (\alpha, K)=
  X^{n-1}+X^{n-2}+\cdots+1$. This polynmial is known to be irreducible over the rationals when $n$ is prime, and it has root $\alpha.$ We can show this by Eisenstein's criterion after using translation by $1$, as in Example \ref{eisensteinex}.
\end{example}



\section{Lecture 31}
\subsection{Field Extensions and Degrees}
\begin{definition}
  Let $F/K$ be a field extension and $\alpha\in F$ be algebraic over $K$. The
  \emph{degree of $\alpha$ over $K$}, written $\deg(\alpha, K)$, is the degree of the
  polynomial $\irr(\alpha, K)$.
  The \emph{degree of $F$ over $K$ }, written $[F:K]$ is the dimension of $F$ as a vector
  space over $K$. An extension $F/K$ is finite if $[F:K]<\infty$.
\end{definition}


% Let $F/K$ be a field extension. Consider the evaluation map $\sigma_c:K[x]\to K[x]$ defined by $\sigma_c(k(x))=k(c).$ This function can be showed to be a homomorphism with image a subfield of $F$. In fact, it is the smallest field containing both $K$ and $c.$ 


\begin{definition}
  Let $K(\alpha)$ denote the smallest subfield of $F$ that contain $K$ and $\alpha$, the
  field generated by $\alpha$ over $K.$
\end{definition}

It can be shown that $K(\alpha)$ is exactly the range of the evaluation map at $\alpha$ for polynomials over $K$. More precisely, 
$K(\alpha) = \{f(\alpha) \mid f \in K[X]\}$. We use this fact to prove the following theorem.

\begin{theorem}
  Let $F/K$ be a field extension, $\alpha\in F$ be algebraic over $K$. Then
  $[K(\alpha):K]=\deg (\alpha, K)$. More precisely, every element of $K(\alpha)$ can be
  uniquely written as $b_0+b_1\alpha +\cdots b_{d-1}\alpha^{d-1}$ where $d=\deg (\alpha,
  K)$ and $b_i\in K$. In other words, $1, \alpha, \ldots, \alpha^{d-1}$ is a basis for
  $K(\alpha)$ over $K$ as a vector space.
\end{theorem}
\begin{proof}
  % For uniqueness, suppose $\exists b_i,b_i'\in K$ s.t. 
  % \[b_0+b_1\alpha+\cdots+b_{d-1}\alpha^{d-1}=b_0'+\cdots+b_{d-1}'\alpha^{d-1}\]
  % And note that then $\alpha$ is a root of the polynomial
  % $(b_0-b_1)+\cdots+(b_{d-1}-b_{d-1}')\alpha^{d-1}$. Since $d=\deg(\alpha,K)$ it
  % follows that $b_i=b_i'$ for all $i$.

  % For existence, suppose $\irr(\alpha, K)=x^d+c_{d-1}x^{d-1}+\cdots+c_0$. Then
  % $\alpha^d=-(c_{d-1}\alpha^{d-1}+\cdots +c_0)$, so it is spanned by the stated basis.
  % Similarly, $\alpha^{d+1}=-(c_{d-1}\alpha^{d} + c_{d-2}\alpha^{d-1}+\cdots+c_0\alpha)$,
  % and note that since we can write $\alpha^d$ as a linear combination of
  % $\alpha^{d-1},\cdots, 1$, so can be $\alpha^{d+1}$. Inductively, this is true for any
  % power of $\alpha$, and so the span of $\langle 1, \alpha, \cdots, \alpha^{d-1}\rangle$
  % is closed under multiplication (and obviously closed under addition too because it's a
  % linear combination). It remains to show that it is closed under multiplicative inverses.
  % Consider the ring homomorphism $K[X]\to F:f(x)\mapsto f(\alpha)$, where the image is
  % $\langle 1,\alpha, \cdots, \alpha^{d-1}\rangle$, and the kernel consists of the
  % multiples of the minimal polynomial, i.e. $(\irr(\alpha, K))$. By the first isomorphism
  % theorem of rings we have $K[X]/(\irr(\alpha,K))\cong \langle 1,\alpha, \cdots,
  % \alpha^{d-1}\rangle$, which is a field since $(\irr(\alpha, K))$ is maximal.

   %  First, to show uniqueness,  suppose $\exists b_i,b_i'\in K$ s.t. 
   % $$b_0+b_1\alpha+\cdots+b_{d-1}\alpha^{d-1}=b_0'+b'_1\alpha \cdots+b_{d-1}'\alpha^{d-1}.$$ We then have,
   % $$(b_0-b'_0)+(b_1-b'_1)\alpha+\cdots+(b_{d-1}-b'_{d-1})\alpha ^{d-1} =0,$$
   % so, $b_0=b'_0$, $b_1=b'_1$, \ldots, $b_{d-1}=b'_{d-1}.$

   %  To show the set $\{1, \alpha, \ldots, \alpha^n\}$ is a basis for $F/K$, we must show it is linearly independent and spans $F/K$.



    %proof of spans
     Let $f \in K$. Since $K(\alpha)$ is the range of the evaluation map, $f(\alpha)$ is an arbitrary element of $K(\alpha)$. By the division algorithm, $\exists ! \: q,r \in K$ s.t. $f = q \cdot \irr(\alpha, K) + r,$ where $\deg(r) < \deg(\alpha, K) = d,$ or $r = 0.$
    

   Evaluating at $\alpha$, we have that $f(\alpha) = 0 + r(\alpha),$ i.e. $\exists \: b_i \in K$ s.t. $f(\alpha) = b_0+b_1\alpha +\cdots + b_{d-1}\alpha^{d-1}.$ Hence $f(\alpha)$ can be expressed as a linear combination of $1,\alpha, \ldots,\alpha^{d-1}$, so the set spans $K(\alpha)$.

   %lin indep
   Suppose $b_0 + b_1\alpha + \ldots + b_{d-1}\alpha^{d-1} = 0,$ where not all $b_i$ are zero. This is saying that $\alpha$ is a root of a polynomial of degree at most $d-1 < d$, which contradicts the assumption that $d = \deg(\alpha, K),$ so the set is linearly independent.
   %bosh

   Suppose two distinct linear combinations were equal, i.e. $\exists b_i,b_i'\in K$ s.t. 
   $$b_0+b_1\alpha+\cdots+b_{d-1}\alpha^{d-1}=b_0'+b'_1\alpha \cdots+b_{d-1}'\alpha^{d-1}.$$
   This would mean that
   $$(b_0-b'_0)+(b_1-b'_1)\alpha+\cdots+(b_{d-1}-b'_{d-1})\alpha ^{d-1} =0,$$
   so $b_0=b'_0$, $b_1=b'_1$, \ldots, $b_{d-1}=b'_{d-1},$ since we have linear independence.
\end{proof}

\section{Lecture 32 - 03 Dec 2021}
\subsection{Algebraic Closure}
\begin{example}
  Consider the field extension $\QQ(2^{1/3})=\{a+b2^{1/3} + c2^{2/3} : a,b,c\in\QQ\}$.
\end{example}
\begin{theorem}
  Let $L/F$ and $F/K$ be field extensions. Then $L/K$ is finite if and only if $L/F$ and
  $F/K$ are both finite. Moreover, if this is the case, we have $[L:K]=[L:F][F:K]$. More
  precisely, if $\alpha_1,\cdots, \alpha_n$ is an $F$-basis for $L$, and $\beta_1,\cdots,
  \beta_m$ is a $K$-basis for $F$, then $\{\alpha_i\beta_j : 1\leq i\leq n, 1\leq j\leq
  m\}$ is a $K$-basis for $L$.
\end{theorem}
\begin{proof}
  TODO: Doable. Should attempt.
\end{proof}

\begin{theorem}
  Let $F/K$ be a field extension. Then the set
  \[\overline{K_F} = \left\{ \alpha\in F : \alpha\text{ is algebraic over } K\right\}\]
  Is a field.
  \label{<+label+>}
\end{theorem}
\begin{proof}
  TODO: Constructive approach: Given $\alpha$ being a root of $f$ and $\beta$ being a root
  of $g$, can we construct a polynomial that has root $\alpha+\beta$ and $\alpha\beta$?
\end{proof}<++>
Example, try $\alpha=\sqrt{2}, \beta=\sqrt{3}$. For example $\overline{\QQ_{\CC}}$ is a
huge extension field of $\QQ$, but still much smaller than $\CC$.
$\overline{K_F}$ is called the algebraic closure of $K$ in $F$.
\begin{definition}
  A field $F$ is called algebraically closed if every $f\in F[X]$ with $\deg f>0$ has a
  root in $F$.
  \label{<+label+>}
\end{definition}

\begin{theorem}[Fundamental Theorem of Algebra]
  The field $\CC$ is algebraically closed.
  \label{thm:FTAlg}
\end{theorem}
The above polynomial implies that a given complex polynomial $f$ can be written as factors
with one root in $\CC$ and then a multiply by another smaller degree polynomial $g$, and
recursively this implies that any polynomial over $\CC$ splits as a product of linear
factors (the roots in $\CC$).
\begin{theorem}
  Every field $K$ is contained in an algebraic extension (in a field extension whose
  elements are algebraic over $K$) that is algebraically closed.
  \label{<+label+>}
\end{theorem}<++>

\end{document}
