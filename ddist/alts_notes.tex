%% 
%% This is file, `alts_notes.tex',
%% generated with the extract package.
%% 
%% Generated on :  2023/09/06,15:03
%% From source  :  notes.tex
%% Using options:  active,generate=alts_notes,extract-env={definition},extract-cmd={section,subsection}
%% 
\documentclass[11pt]{scrartcl}

\begin{document}

\section{Lecture 1}

\begin{definition}
  A group is a pair $(G,*)$ where $G$ is a set and $*:G\times G\to G$ is a binary
  operation with the following axioms:
  \begin{itemize}
      \ii (Associativity) For any $g,h,k\in G$
      \[(g*h)*k = g*(h*k)\]
      \ii (Existence of identity) There exists $e\in G$ such that for any $g\in G$,
      \[e*g=g*e = g\]
      \ii (Existence of inverse) For any $g\in G$ there exists $h\in G$ s.t.
      \[g*h=h*g=e\]
  \end{itemize}
  \label{group}
\end{definition}

\begin{definition}
  A group $G$ is abelian if the group is commutative, i.e. $\forall g,h\in G$ we have
  $gh=hg$.
  \label{abelianGroup}
\end{definition}

\begin{definition}
  We say a group is finite or countable if the underlying set is finite or countable,
  respectively.
\end{definition}

\subsection{Symmetric groups}

\section{Lecture 2 - 22 Sep 2021}

\begin{definition}
   A subgroup of a group G, $H\subset G$, is a subset of $G$ with
   \begin{itemize}
       \ii It contains the $G$ group identity element.
       \ii For all $a,b\in H$ we have $ab\in H$
       \ii for any $a\in H$ we have $a^{-1}\in H$.
   \end{itemize}
   \label{subgroup}
 
\end{definition}

\begin{definition}
  A group $G$ is called cyclic if $\exists g\in G$ such that $G=\left\{ g^n : n\in\ZZ
  \right\}$. If $G$ is cyclic, then an element $g$ as above is called a generator, and we
  say $G$ is generated by $g$, $G=<g>$.
  \label{cyclicGroup}
\end{definition}

\begin{definition}
  The order of an element $g\in G$ of a group $G$, written $|g|$, is the least positive
  integer $n$ s.t. $g^n=e$. If such $n$ doesn't exist, it has infinite order. The order
  of a group $G$, $|G|$, is the cardinallity of the underlying set.
  \label{orderGroup}
\end{definition}

\subsection{Cyclic groups structure}

\section{Lecture 3 - 28 Sep 2021}

\subsection{Multiplication table}

\section{Lecture 4 - 27 Sep 2021}

\subsection{Cosets}

\begin{definition}
  Let $G$ be a group and $H$ a subgroup of $G$. Let $g\in G$. The left coset of $H$
  containing $g$ is the set $gH=\left\{ gh | h\in H \right\}$. The right coset of $H$
  containing $g$ is the set $Hg=\left\{ hg | h\in H \right\}$.
  \label{coset}
\end{definition}

\section{Lecture 5 - 29 Sep 2021}

\subsection{Counting groups and Lagrange}

\begin{definition}
  Let $G$ be a group and $H$ a subgroup. The set of left cosets of $H$ in $G$ is denoted
  by $G/H$, also called the set of equivalence classes generated by the equivalence
  relation defined in Corollary \ref{leftCosetsEqRel} . Similarly the set of right cosets of
  $H$ in $G$ is denoted by $H\setminus G$.
  \label{cosets}
\end{definition}

\begin{definition}
  The number of (left) cosets of $H$ in $G$ is called the index of $H$ in $G$, written
  $[G:H]$.
\end{definition}

\section{Lecture 6 - 4 Oct 2021}

\subsection{Normal subgroups and Quotients}

\begin{definition}
  Let $G$ be a group. A subgroup $H$ of $G$ is called normal, written $H\triangleleft G$
  or $H\trianglelefteq G$, if $\forall g\in G$ we have $gHg^{-1}=H$.
  \label{normalSubgroup}
\end{definition}

\begin{definition}
  Let $G$ be a group, and $N$ be a normal subgroup. The set of left cosets $G/N$ together
  with the binary operation $(gN)(hN)=(gh)N$ for $g,h\in G$ is called the \emph{quotient
  group} or \emph{factor group} of $G$ by $N$.
  \label{quotientGroup}
\end{definition}

\section{Lecture 7 - 6 Oct 2021}

\begin{definition}
  Let $(H,*),(K,\cdot)$ be groups. We define their \emph{direct product} as the group
  with underlying set $H\times K = \left\{ (h,k) : h\in H, k\in K \right\}$ with operation
  of point-wise multiplication
  \[(h,k)(h',k')=(h*h',k\cdot k')\]
  \label{directProduct}
\end{definition}

\section{Lecture 8 - 8 Oct 2021}

\subsection{Group Homomorphisms, Types and Facts}

\begin{definition}
  Let $G,G'$ be groups. A group homomorphism from $G$ to $G'$ is a function $\phi:G\to G'$
  s.t. for all $g,h\in G$ one has $\phi(gh)=\phi(g) \phi(h)$.
  \label{groupHomomorphism}
\end{definition}

\begin{definition}[Morphism Zoo]
  \begin{enumerate}
    \item A group \emph{isomorphism} is a grou homomorphism $\phi:G\to G'$ that has a
      2-sided inverse. This is, $\phi\circ\phi^{-1}=1_{G'},\phi^{-1}\circ\phi=1_G$. If
      there exists a group isomorphism between groups $G,G'$, we say these groups are
      isomorphic and write $G\cong G'$. By the previous results (Theorems
      \ref{groupsCategories} \ref{homIdInv}) we have that being isomorphic is an
      equivalence relation. Two isomorphic groups are structurally the same.

    \item A group automorphism is a group isomorphism to itself.
    \item A group endomorphis is a group homomorphism to itself. E.g. the trivial
      homomorphism to itself is an endomorphism for non-trivial groups.
  \end{enumerate}
  \label{morphismZoo}
\end{definition}

\section{Lecture 9 - 11 Oct 2021}

\subsection{First Isomorphism Theorem}

\begin{definition} [Kernel of Homomorphisms]
  Let $\phi:G\to G'$ be a group homomorphism. Then the \emph{kernel} $\ker\phi$ is the set
  of all elements of $G$ that are mapped to the identity of $G'$,
  \[\ker \phi = \left\{ g\in G : \phi(g)=1_{G'} \right\}\]
  \label{kernel}
\end{definition}
