\section{Lecture 24 - 15 Nov 2021}
\subsection{Prime and maximal ideals - division with reminder in polynomials}
Prime and maximal ideals, examples, division with reminder in polynomial rings.

\begin{corollary}
  Let $R$ be a commucative unital ring. Then every maximal of $R$ is prime.
  \label{cor:maximalPrime}
\end{corollary}
\begin{proof}
  $I$ is an ideal of $R$ and maximal. Hence $R/I$ is a field, so $R/I$ is an integral
  domain, hence $I$ is prime.
\end{proof}

\begin{example}
  The converse is false. For example, $\ZZ[X]$, the ideal $(X)$, since $\ZZ[X]/(X)\cong
  \ZZ$ is an integral domain, but $(X)$ is not maximal, since $\ZZ$ is not a field. Let
  $p$ be a prime number, so the ideal $(p,X)=\{pf+Xg: f,g\in\ZZ[X]\}$ is the polynomials
  where constant terms are multiples of $p$, and is maximal and contains $(X)$.
\end{example}

\begin{definition}
  Let $F$ be a field, and let $f(X)=a_0+a_1x+\cdots + a_d X^d\in F[X]$ be a polynomial
  with $a_d\neq 0$. The degree $\deg f$ of $f$ is defined as $d$.
  \label{<+label+>}
\end{definition}

\begin{theorem}[Division with reminder in polynomial rings]
  Let $F$ be a field, and let $f(X)$, $g(X)\in F[X]$. Then there exist unique $q(X),
  r(X)\in F[X]$ s.t. 
  \[f(X)=g(X)q(X)+r(X)\]
  and s.t. either $r(X)=0$ or $\deg r < \deg g$.
  \label{thm:divisionReminderPolynomial}
\end{theorem}
\begin{proof}
  \emph{Proof of existence.} Note $\deg g> \deg f \implies q=0, r=f$. Next, consider
  $f(x)=a_nx^n+\cdots+a_1x+a_0$, and $g(x)=b_mx^m+\cdots+b_1x+b_0$, with $a_n,b_m\neq 0$
  and $n\geq m$. Assume the statement is true for all $f_0$ with $\deg f_0<\deg f$. Let 
  \[f_0(x)=f-\frac{a_n}{b_m}x^{n-m}\cdot g = a'_{n-1}x^{n-1}+\cdots+a'_0\]
  So $\deg f_0< \deg f$. Then by induction we have that there exists $q_0, r\in F[X]$ with
  either $r=0$ or $\deg r< \deg g$ s.t.  $f_0=q_0g + r \implies f=
  (q_0+\frac{a_n}{bm}x^{n-m})g + r$

  \emph{Proof of uniqueness}. For the sake of contradiction, assume that there exists
  $q,q',r,r'\in F[X]$ s.t. $f(x)=qg+r=q'g+r'$ and each $r,r'=0$ or $\deg r,r'<\deg g$.
  However, not that then 
  \[qg+r=q'g+r' \implies g(q-q')=r'-r.\]
  Note that $\deg g(q-q')\geq \deg g$ since $q\neq q'$ and $g$ is not the $0$ polynomial
  since no polynomial has divisor $0$. Henceforth, $\deg r'-r\geq \deg g$, a
  contradiction.
\end{proof}

\begin{corollary}
  Let $F$ be a field and let $f\in F[X]$, $a\in F$. Then $f(a)=0$ if and only if $\exists
  g\in F[X]$ s.t. $f(X)=(X-a)g(X)$.
  \label{<+label+>}
\end{corollary}
\begin{proof}
  The right to left direction is trivial. On the other hand, to prove its converse, divide
  $f(X)$ by $(X-a)$ with remainder. By the previous theorem we have that $\exists q,r\in
  F[X]$ s.t. $r=0$ or $\deg r <1$, since $g$ in this case is $(X-a)$, a linear term, so
  $r\in F$ and $f = (x-a)q+c$ but we have $f(a)=0=c$, so $c=0$. Hence the result follows.
\end{proof}
