\section{Lecture 10 - 11 Oct 2021}
On isomorphism theorems. The first isomorphism in action.

\begin{example}
  Last time we saw how there exists an homomorphism from $D_{12}$ to $S_3$. The kernel of
  this homomorphism is the subgroup $H$ generated by the rotation by $180^o$, and the
  image is all $S_3$. By FIT we have $D_{12}/H \cong S_3$.
\end{example}
\begin{example}
  Let $\CC^x$ be the non-zero complex numbers as a group under multiplication. The
  function $f:\RR\to\CC:x\mapsto e^{2\pi i x}$ is a group homomorphism. The image is the
  unit circle in $\CC$, denoted $S^1$. The kernel is $\ZZ$. Then,
  \[\RR/\ZZ \cong S^1.\]
  This can be quite easily extended to show that 
  \[\RR^n/\ZZ^n \cong \bigoplus_{i=1}^n S^1.\]
\end{example}

\begin{example}
  Consider the group homomorphism from $\CC^x$ to $\RR_{>0}$ (groups under
  multiplication) given by $z\mapsto |z|$. The image is $\RR_{>0}$ (surjective) and the kernel is
  $S^1$. Hence by FIT,
  \[\CC^x / S^1 \cong \RR_{>0}\]
  In fact,
  \[\CC^x \cong S^1 \times \RR_{>0}\]
  See sheet 2, ex 10.
\end{example}
\begin{example}
  How many distinct homomorphisms are there from a cyclic group of order $4$ to a cyclic
  aroup of order $10$, $\phi:G\to G'$ for $G=\langle g|g^4=1\rangle,G'=\langle
  g'|(g')^{10}=1\rangle$?  
\end{example}
\begin{proof}[Solution]
  We classify them by the choices of kernel. 

  First, we chould have a trivial kernel $\left\{ 1 \right\}$, so $G/\left\{ 1 \right\}
  \cong \Img\phi \leq G$. However, by Lagrange we can't have a subgroup of order $4$ for
  $G$ of order $10$. Hence there're no homomorphisms for a trivial kernel.

  Second, we notice that there're exactly three possible kernels, since if the kernel has
  $g$ or $g^3$, it automatically yields a trivial homomorphism (everything to the
  identity). Hence we have 2 more possibilities. We consider $\ker\phi=\langle
  g^2\rangle$. By the FIT, we have $G/\ker\phi \cong \Img\phi$, which is a cyclic group of
  order 2. How many subgroups of $G'$ of order $2$ are there? Only $\langle
  g^5\rangle$. Hence we can have only one such homomorphism.

  The last option is $\ker\phi=G$, the trivial homomorphism. Then $\Img\phi=\left\{ 1
  \right\}= G/G$. There exists a unique such $phi$. 

  In total, then, we have exactly two homomorphisms.
\end{proof}

From here we find an alternative way of thinking about normal subgroups. The normal
subgroups of $G$ are exactly the subsets that can arise as kernels of homomorphisms from
$G$. We encapsulate this in the following claim.

\begin{proposition}
  A subgroup $H\leq G$ is normal if and only if $H$ is the kernel of some homomorphism
  $\phi: G\to G'$.
\end{proposition}
\begin{proof}
  Note that Every kernel is a normal subgroup by the first isomorphism theorem.
  Conversely, if $N\trianglelefteq G$, we can always define the quotient map $\phi:G\to
  G/N : g\mapsto gN$ is a group homomorphism (we proved this in a previous lecture) and
  the kernel is precisely all $g\in G$ s.t. $\phi(g)=N$, which is all $g$ in $N$ itself.
\end{proof}

Then, taking the quotient by a normal subgroup $N$ is essentially equivalent to applying a
homomorphism from $G$ to some other gorup whose kernel is $N$.
