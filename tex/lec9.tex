\section{Lecture 9}
\subsection{First Isomorphism Theorem}

\begin{definition} [Kernel and Image]
  Let $\phi:G\to G'$ be a group homomorphism. 
  
  Then the \emph{kernel}, $\ker\phi$, is the set of all elements of $G$ that are mapped to the identity of $G'$, written as 
  \[\ker \phi = \left\{ g\in G \mid \phi(g)=1_{G'} \right \}.\]

  The \emph{image}, $\Img\phi$, is the set of all elements of $G'$ mapped to under $\phi$. Formally,
  $$
  \Img\phi = \{\phi(g) \mid g \in G\}.
  $$
  \label{kernel}
\end{definition}

\begin{theorem}
  Let $\phi:G\to G'$ be a group homomorphism. Then $\phi$ is injective if and only if
  $\ker \phi = \left\{ 1_G \right\}$, i.e. the kernel is trivial.
  \label{kernelHomomorphism}
\end{theorem}
\begin{proof}
  ($\implies$) 
  % Let $g,h \in G$, with $\phi(g) = \phi(h)$. Then by properties of homomorphisms, we have that $g h^{-1} \in \ker\phi.$ By injectivity, $g=h$, so $1_G \in \ker\phi.$
  Suppose $\phi$ is injective. Since $\phi$ is a homomorphism, $1_G \in \ker\phi.$
  If $k \in \ker\phi,$ then $$\phi(k) = 1_{G'} = \phi(1_{G}),$$ but $\phi$ is injective, so $k = 1_G$. Hence $\ker\phi$ is trivial.
  
  % Suppose $\exists g,h\in G$ s.t. $\phi(g)=\phi(h)$. Then we have
  % $\phi(gh^{-1})=1_{G'}$ and so $gh^{-1}\in\ker\phi$. The proof follows by injectivity. 
  % Symbolically we have
  % \[ (\phi(g)=\phi(h) \implies g=h) \iff (gh^{-1}\in\ker\phi \implies g=h) \iff \ker\phi
  % = \left\{ 1_{G'} \right\}\]

  ($\impliedby$) Suppose $\ker\phi = \{1_G\}$ and suppose $\phi(g) = \phi(h).$ Then by properties of homomorphisms, $\phi(gh^{-1})=1_G,$ so $g h^{-1} \in \ker\phi,$ so $g h^{-1} = 1_G,$ which means that $g=h,$ so $\phi$ is injective.
  % $gh^{-1}\in\ker\phi \iff \phi(g)=\phi(h)$. The fact that the kernel is
  % trivial automatically implies injectivity.
\end{proof}

\begin{remark} [Groups as symmetries, formalised]
  If groups are thought of as symmetries, then homomorphisms are changes in the focus of
  attention of the same symmetry.
\end{remark}

\begin{example}
  Consider the dihedral group $D_{12}$ of order 12. We can describe this by the presentation
  \[D_{12}=\langle \sigma, \tau \mid \sigma^{6}=\tau^2=1, \quad \sigma\tau = \tau\sigma^{-1}\rangle .\]
  If we label the vertices of the hexagon as $1,2,\ldots,6$ we can view this dihedral
  group inside $S_6$, through its action on the set of vertices (see group actions). Then
  we observe how the rotation $\sigma$ corresponds to $(1 2 3 4 5 6)$ and reflection
  $\tau$ to $(1 6)(2 5)(3 4)$, say. This is really saying that an injective group
  homomorphism $D_{12}\to S_6$ exists, defined by $\sigma\to (1 2 3 4 5 6), \tau\to
  (1 6)(2 5)(3 4)$.
  This map should feel natural, as we are essentially just changing the focus from the symmetries of the whole hexagon to permutations of the vertices.
  We can realise $D_{12}$ within $S_6$ in this way: 
  \[D_{12}=\langle (1 2 3 4 5 6),(1 6)(2 5)(3 4)\rangle \leq S_6.\]

  However, we can also consider the action of $D_{12}$ on the set of diagonals of an
  hexagon. Then, for diagonals $1,2,3$, we have 
  \[\sigma\mapsto (1 2 3)\]
  \[\tau\mapsto (1 3).\]
  In other words, this action gives us a homomorphism $D_{12}\to S_3$, which is non-injective
  by the pigeonhole principle. Hence we must have a non-trivial element in the kernel, namely $\sigma^3$ which
  rotates $180$ degrees, however in $S_3$ it gives back the identity. Although the
  homomorphism is non-injective, it is surjective. 
  % In fact, $\ker\phi = \left\{ 1,\sigma^3 \right\}$.
  \label{D12}
\end{example}
% This example illustrated the idea of changing the focus of attention, like deciding which
% question you're going to ask about your object.

\begin{example}
  Recall that for a group $G$ and $N$ a normal subgroup of $G$, the quotient map, $\pi:G\to G/N$,  $g\mapsto gN,$ is a surjective homomorphism.

  Let $H\leqG$. The inclusion map $\iota: H\to G,$ $h\mapsto h,$ is an injective group homomorphism. 
\end{example}

% \begin{example}
%   Let $G$ be a group s.t. $\left\{ g_1,g_2,\cdots\right\}$ (although it may not be
%   countable). Let us keep the group operation but relabel all elements to form a new set
%   $H$ with elements $\left\{ h_1,h_2,\cdots \right\}$ and define a binary operation on $H$
%   as follows. Whenever $i,j,k$ are such that $g_ig_j=g_k$, define $h_k:=h_ih_j$. Then $H$
%   is isomorphic to $G$ by the homomorphism $g_i\mapsto h_i$ for all $i$.
% \end{example}
\begin{remark}
Note that all isomorphisms are essentially a relabelling of the elements of the group.
\end{remark}
% Then, the first big theorem of this group is a combination of the above examples of this
% lecture: Every homomorphism is a combination of quotient maps, injections, and
% isomorphisms.

This is the most important theorem of the course. The main use of this theorem is identify what groups a quotient is isomorphic to.

\begin{theorem} [First isomorphism theorem, Part 1]
  Let $\phi:G\to G'$ be a group homomorphism. Then 
  \begin{enumerate}
    \item $\Img\phi \leq G',$
    \item $\ker\phi \trianglelefteq G.$
  \end{enumerate}
  \label{firstIso1}
\end{theorem}
\begin{proof} 
  We first show that $\Img \phi$ is a subgroup of $G.$ Note that $\Img \phi$ is nonempty since $\phi(1_G)=1_{G'}\in \Img \phi.$ Let $x,y\in \Img \phi,$ then $\phi(g)=x$ and $\phi(h)=y$ for some $g,h\in G.$ We have 
  $$xy^{-1}=\phi(g)(\phi(h))^{-1}=\phi(g)\phi(h^{-1})=\phi(gh^{-1})\in \Img \phi,$$
  therefore, $\Img\phi$ is a subgroup by the subgroup test.
  % \[x,y\in\Img\phi \implies \exists g,h\in G \text{ s.t. } \phi(g)=x,\phi(h)=y \implies
  % \phi(gh^{-1})=xy^{-1}\]
  % And recall that by the subgroup test, since $1\in\Img\phi$ by definition of
  % homomorphism, and $xy^{-1}\in\Img\phi$, then $\Img\phi$ is a subgroup of $G'$, as
  % required.

  To prove that $\ker\phi \trianglelefteq G$, we will first show that it is indeed a subgroup. Note that $e_G\in\ker\phi$. Moreover, for $x,y\in\ker\phi$, we have $\phi (x) = \phi (y) =e_{G'}.$ Therefore, $\phi (xy^{-1})=\phi(x)(\phi(y))^{-1}=e_{G'},$ so  $xy^{-1}\in \ker \phi$ and $\ker \phi$ is a subgroup of $G$ by the subgroup test. Next, we claim that $\ker \phi$ is closed under conjugation by elements of $G$. Let $ g\in G$ and $k \in \ker \phi.$
 We have,
 $$\phi(gkg^{-1})=\phi(g)\phi(k)\phi(g^{-1})=\phi(gg^-1)=e_{G'},$$
 therefore, $gkg^{-1} \in \ker \phi,$ as required.
\end{proof}

Our goal is now, given the kernel $K$ of some homomorphism out of a group $G$, to understand the structure of the quotient group $G/K$.

Note that any homomorphism $\phi:G\to G'$ is surjective onto its image. In addition, $\phi$ may not be injective, but we can define an equivalence relation on $G$ as 
\[g\sim g' \iff \phi(g)=\phi(g')\iff gg^{-1}\in \ker\phi.\]
% Then $\phi(g)$ depends only on the equivalence class of $g$, and is injective if we modify it to be a
% function on equivalence classes. 

So if instead we modify the map to be a function on equivalence classes, this will be injective. We can think of this new map as placing the elements of $G$ into boxes, represented by the different outputs of $\phi$, and then mapping each box to this element of the image.

We have that $K$ is a subgroup of $G$, so 
$$
g\sim g' \iff  gg'^{-1}\in \ker\phi \iff gK = g'K.
$$

So a function from the set of cosets, the quotient group, to the image of $\phi$, sending $gK$ to $\phi(g)$, will be both surjective and injective.



% \begin{remark}
%     It turns out this map is an isomorphism of groups, since $K$ is normal in $G$. This means we can understand the potentially complicated quotient group $G/K$ as the image of a homomorphism out of $G$ with kernel $K$.
% \end{remark}

% However, now think about what equivalence classes are:
% quotient groups! We have by the definition $g\sim g' \iff gg'^{-1}\in \ker\phi$. We already
% said $\ker\phi$ is a subgroup. So we have $gg'^{-1}\in \ker\phi \iff
% g\ker\phi=g'\ker\phi$. Writing $N=\ker\phi$ we have,
% \[g\sim g' \iff gN = g'N\]
% Observe that by defining the above we're saying that $\phi$ is well-defined on cosets, and
% is injective in the set of cosets. In Theorem \ref{firstIso1} we said the kernel of a
% homomorphism is a normal subgroup of $G$.



\begin{theorem} [First Isomorphism Theorem]
  Let $\phi:G\to G'$ be a group homomorphism. Then,
  \begin{enumerate}
    % \item The image $\Img\phi=\left\{ \phi(g) : g\in G \right\}$ is a subgroup of $G',$
    % \item The kernel $\ker\phi$ is a normal subgroup of $G,$ 
    \item $\Img\phi \leq G',$
    \item $\ker\phi \trianglelefteq G,$ 
    
    \item The mapping
      \begin{align*}
          \psi: G/\ker\phi&\to\Img\phi 
          \\ g \ker\phi&\mapsto \phi(g)
      \end{align*}
      is a well-defined bijection. Moreover, it defines a group isomorphism,
      \[G/\ker\phi \cong \Img\phi.\]
  \end{enumerate}
  \label{firstIso}
\end{theorem}
\begin{proof}
   We proved statements $1$ and $2$ in Theorem \ref{firstIso1}. To prove the third statement, we first check that $\psi$ is well-defined. 
   
   Let $g,g'\in G$ s.t. $g\ker\phi=g'\ker\phi$. Then we have $g=g'k$ for some $k\in\ker\phi$, and so $\phi(g)=\phi(g'k)$, therefore $\phi(g)=\phi(g')$.
  Hence we have that
  $g\ker\phi=g'\ker\phi \implies \phi(g)=\phi(g')$, i.e. $\psi$ is well-defined.

    To show that $\psi$ is a homomorphism, let $g_1,g_2 \in G.$ We have,
    $$\psi(g_1\ker \phi \cdot g_2\ker \phi)=\psi(g_1g_2\ker\phi)=\phi(g_1g_2)=\phi(g_1)\phi(g_2)=\psi(g_1\ker\phi)\psi(g_2\ker\phi),$$
    since $\phi$ is a group homomorphism.
   
   Furthermore, we claim that $\psi$ is surjective and injective. Let $g'\in \Img \phi,$ then $\exists g\in G$ s.t. $g'=\phi(g).$ We compute $\psi(g\ker \phi)=\phi(g)=g',$ therefore $\psi$ is surjective.

   Now suppose $g\in G$ s.t. $g\ker\phi\in \ker \psi.$ We have $\psi(g\ker\phi)=\phi(g)=e$. Thus $g \in \ker\phi$ so $g\ker\phi=\ker\phi$ by absorption of cosets, meaning $\ker\psi$ is trivial and $\psi$ is injective. 
   
    We have a bijective homomorphism and therefore an isomorphism.
\end{proof}

\begin{center}
\[\begin{tikzcd}
	G && {G'} \\
	\\
	{G/\ker \varphi} && {\operatorname{Im}\varphi}
	\arrow["\varphi", from=1-1, to=1-3]
	\arrow["\pi"', two heads, from=1-1, to=3-1]
	\arrow["{\psi}"', dashed, from=3-1, to=3-3]
	\arrow["\iota"', hook, from=3-3, to=1-3]
\end{tikzcd}\]
\end{center}

\begin{figure}[htpb]
    \centering
    \caption{First isomorphism theorem, illustrated.} 
    \label{fig:isomorphism-theorem}
\end{figure}
\begin{remark}
    Every homomorphism is really a composition of a surjective quotient map and a bijective homomorphism from the quotient to the image.
\end{remark}