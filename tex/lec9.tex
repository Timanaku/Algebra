\section{Lecture 9 - 11 Oct 2021}
\subsection{First Isomorphism Theorem}
More group homomorphisms and the first isomorphism theorem.
\begin{definition} [Kernel of Homomorphisms]
  Let $\phi:G\to G'$ be a group homomorphism. Then the \emph{kernel} $\ker\phi$ is the set
  of all elements of $G$ that are mapped to the identity of $G'$, 
  \[\ker \phi = \left\{ g\in G : \phi(g)=1_{G'} \right\}\]
  \label{kernel}
\end{definition}

\begin{theorem}
  Let $\phi:G\to G'$ be a group homomorphism. Then $\phi$ is injective if and only if
  $\ker \phi = \left\{ 1_G \right\}$, i.e. the kernel is trivial.
  \label{kernelHomomorphism}
\end{theorem}
\begin{proof}
  ($\Rightarrow$) Suppose $\exists g,h\in G$ s.t. $\phi(g)=\phi(h)$. Then we have
  $\phi(gh^{-1})=1_{G'}$ and so $gh^{-1}\in\ker\phi$. The proof follows by injectivity. 
  Symbolically we have
  \[ (\phi(g)=\phi(h) \implies g=h) \iff (gh^{-1}\in\ker\phi \implies g=h) \iff \ker\phi
  = \left\{ 1_{G'} \right\}\]

  ($\Leftarrow$) $gh^{-1}\in\ker\phi \iff \phi(g)=\phi(h)$. The fact that the kernel is
  trivial automatically implies injectivity.
\end{proof}

\begin{remark} [Groups as symmetries formalised]
  If groups are thought of as symmetries, then homomorphisms are changes in the focus of
  attention of the same symmetry.
\end{remark}

\begin{example}
  Consider the dihedral group $D_{12}$ of order 12. We can describe this by the
  presentation
  \[D_{12}=\langle \sigma, \tau | \sigma^{6}=\tau^2=1, \tau\sigma\tau^{-1} =
  \sigma^{-1}\rangle\]
  If we label the vertices of the hexagon as $1,2,\cdots,6$ we can view this dihedral
  group inside $S_6$, through its action on the set of vertices (see group actions). Then
  we observe how the rotation $\sigma$ corresponds to $(1,2,3,4,5,6)$ and reflecion
  $\tau$ to $(1,6)(2,5)(3,4)$, say. This is really saying that an injective group
  homomorphism $D_{12}\to S_6$ exists, defined by $\sigma\to (1,2,3,4,5,6), \tau\to
  (1,6)(2,5)(3,4)$. Observe that we haven't really done much. We just changed the focus of
  attention from the whole hexagon to only the set of vertices,
  \[D_{12}=\langle (1,2,3,4,5,6),(1,6)(2,5)(3,4)\rangle \leq S_6\]
  However, we can also consider the action of $D_12$ on teh set of diagonals of an
  hexagon. Then, for diagonals $1,2,3$, we have 
  \[\sigma\mapsto (1,2,3)\]
  \[\tau\mapsto (1,3)\]
  I.e. this action gives us a homomorphism $D_{12}\to S_3$, which is non-injective
  clearly. Hence we must have a non-trivial element in the kernel, naming $\sigma^3$ which
  rotates $180$ degrees, however in $S_3$ it gives back the identity. Although the
  homomorphism is non-injective, it is surjective. In fact, $\ker\phi = \left\{ 1,\sigma^3
  \right\}$.
\end{example}
This example illustrated the idea of changing the focus of attention, like deciding which
question you're going to ask about your object.

\begin{example} [Quotient homomorphisms, Injections]
  Recall that for $G$ group and $N$ normal subgroup, the quotient map is a surjective
  homomorphism, $\phi:G\to G/N : g\mapsto gN$.

  Let $H< G$. The map $H\to G:h\mapsto h$ is an injective group homomorphism.
\end{example}

\begin{example}[A silly isomorphism]
  Let $G$ be a group s.t. $\left\{ g_1,g_2,\cdots\right\}$ (although it may not be
  countable). Let us keep the group operation but relabel all elements to form a new set
  $H$ with elements $\left\{ h_1,h_2,\cdots \right\}$ and define a binary operation on $H$
  as follows. Whenever $i,j,k$ are such that $g_ig_j=g_k$, define $h_k:=h_ih_j$. Then $H$
  is isomorphic to $G$ by the homomorphism $g_i\mapsto h_i$ for all $i$.
\end{example}
Note that all isomorphisms are essentially like the example above!

Then, the first big theorem of this group is a combination of the above examples of this
lecture: Every homomorphism is a combination of quotient maps, injections, and
isomorphisms.
\begin{theorem} [First isomorphism theorem, Part 1]
  Let $\phi:G\to G'$ be a group homomorphism. Then 
  \begin{enumerate}
    \item $\Img\phi < G'$
    \item $\ker\phi \trianglelefteq G$
  \end{enumerate}
  \label{firstIso1}
\end{theorem}
\begin{proof}
  For the first claim, we have
  \[x,y\in\Img\phi \implies \exists g,h\in G s.t. \phi(g)=x,\phi(h)=y \implies
  \phi(gh^{-1})=xy^{-1}\]
  And recall that by the subgroup test, since $1\in\Img\phi$ by definition of
  homomorphism, and $xy^{-1}\in\Img\phi$, then $\Img\phi$ is a subgroup of $G'$, as
  required.

  To prove the second claim, note that $e_G\in\ker\phi$. Moreover, let $x,y\in\ker\phi$.
  Then we have $\phi (x) = \phi (y) =e_{G'}\iff \phi (xy^{-1}) = e_{G'} \implies
  xy^{-1}\in\ker\phi$. So $\ker\phi$ is a subgroup. Next, we claim $\forall g\in G,
  g\ker\phi g^{-1}= \ker\phi$, or $\forall g\in G, h\in\ker\phi,ghg^{-1}\in\ker\phi$.
  Indeed we observe that 
  \[\phi(ghg^{-1})= \phi (g)\phi (h) \phi (g)^{-1} = (\phi g)e_{G'} (\phi g)^{-1} = e_{G'}\]
  So $ghg^{-1}\in\ker\phi$, as required.
\end{proof}

Now, note that for a some homomorphism $\phi:G\to G'$, it may not be surjective, but it
will be surjective to its image (that's the definition of image). It may not be injective,
but we can brutally define an equivalence relation as 
\[g\sim g' \iff \phi(g)=\phi(g')\iff gg^{-1}\in \ker\phi\]
Then $\phi(g)$ depends only on the equivalence class of $g$, and is injective as a
function on equivalence classes. However, now think about what equivalence classes are:
quotient groups! We have by the definition $g\sim g' \iff gg'^{-1}\in \ker\phi$. We already
said $\ker\phi$ is a subgroup. So we have $gg'^{-1}\in \ker\phi \iff
g\ker\phi=g'\ker\phi$. Writing $N=\ker\phi$ we have,
\[g\sim g' \iff gN = g'N\]
Observe that by defining the above we're saying that $\phi$ is well-defined on cosets, and
is injective in the set of cosets. In Theorem \ref{firstIso1} we said the kernel of a
homomorphism is a normal subgroup of $G$.

An attempt to illustrate this has been made in Figure 1:

\begin{center}
\[\begin{tikzcd}
	G && {G'} \\
	\\
	{G/\ker \varphi} && {\operatorname{Im}\varphi}
	\arrow["\varphi", from=1-1, to=1-3]
	\arrow["\pi"', two heads, from=1-1, to=3-1]
	\arrow["{\exists ! \widetilde{\varphi}}"', dashed, from=3-1, to=3-3]
	\arrow["\iota"', hook, from=3-3, to=1-3]
\end{tikzcd}\]
\end{center}

\begin{figure}[htpb]
    \centering
    \caption{First isomorphism theorem, illustrated. Note how every homomorphism is really
    a composition of a trivial and surjective quotient map, an bijective (injective and
    surjective) homomorphism from the quotient to the image.} 
    \label{fig:isomorphism-theorem}
\end{figure}


\begin{theorem} [First Isomorphism theorem]
  Let $\phi:G\to G'$ be a group homomorphism. Then,
  \begin{enumerate}
    \item The image $\Img\phi=\left\{ \phi(g) : g\in G \right\}$ is a subgroup of $G'$
    \item The kernel $\ker\phi$ is a normal subgroup of $G$ (also written
        $\phi^{-1}(1_{G'})$ denoting the pre-image of the identity).
    \item The bijection 
      \[\psi: G/\ker\phi\to\Img\phi : g \ker\phi\mapsto \phi(g)\]
      Is well defined. Moreover, it defines a group isomorphism,
      \[G/\ker\phi \cong \Img\phi\]
  \end{enumerate}
  \label{firstIso}
\end{theorem}
\begin{proof}
  Note that we proved statements $1,2$ in Theorem \ref{firstIso1}. We're left with proving
  the third statement. Note that the operation $\psi$ is well-defined. Observe that for
  any $g,g'\in G$ s.t. $g\ker\phi=g'\ker\phi$. Then we have $g=g'h$ for some
  $h\in\ker\phi$, and so $\phi(g)=\phi(g'h) \iff \phi(g)=\phi(g')$. Hence we shown that
  $g\ker\phi=g'\ker\phi \implies \phi(g)=\phi(g')$, i.e. $\psi$ is well-defined.

  Furthermore, we claim that $\psi$ is surjective and injective. Note that surjectivity
  follows from the definition of $\Img\phi$. To prove injectivity, let us proceed by
  contradiction. Assume $g,g'\in G$ are s.t. $g\ker\phi\neq g'\ker\phi$ and
  $\phi(g)=\phi(g')$. So $g\neq g'h$ for any $h\in\ker\phi$, and it follows that
  $g^{-1}g'h=l$ for some non-identity element $l$ of $G$ not necessarily in $\ker\phi$. Hence,
  \[\phi(g^{-1}g'h) = \phi(l) \iff \phi(g^{-1})\phi(g') = \phi(l) \iff
  \phi(g)\phi(l)=\phi(g')\]
  But $\phi(l)\neq 1_{G'}$, a contradiction

  Alternative direct proof for injectivity (nicer imo)
  
  Assume $g,g'\in G$ and $\psi(g\ker\phi) = \psi(g'\ker\phi)$. Hence,
  \[\phi(g) = \phi(g') \iff 1_{G'} = \phi(g)^{-1}\phi(g') = \phi(g^{-1}g') \iff g^{-1}g' \in \ker\phi \iff g\ker\phi = g'\ker\phi\]
  Therefore, $\psi$ is a bijection.
\end{proof}
