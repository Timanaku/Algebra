\section{Lecture 25}
\subsection{Prime ideals in polynomial rings}
\begin{theorem}
  Let $F$ be a field and let $f\in F[X]$ be a non-zero polynomial with $\deg f = n$. Then
  $f$ has at most $n$ roots.
  \label{<+label+>}
\end{theorem}
\begin{proof}
  Assume that $f$ has at least $n+1$ roots in $F$, and $k$ distinct roots, say $a_1, a_2, \ldots, a_{k}$. Let each root $a_i$ have multiplicity $m_i$, so that $\sum_{i=1}^k m_i \geq n+1$. 
%   By the previous corollary, for each root $a_i$, there exists a polynomial $g_i(X) \in F[X]$ such that:
% $$ f(X) = (X - a_i)^{m_i}g_i(X) \quad \text{for } i = 1, 2, \ldots, k. $$

Since $a_1$ is a root of $f$, by Corollary \ref{cor:rootFactor}, $\exists \: g_1 \in F[X]$ s.t.
$$ f(X) = (X - a_1)^{m_1}g_1(X). $$

Now, because $a_2 \neq a_1$ is also a root of $f$ and $F[X]$ has no zero divisors, we must have $g_1(a_2) = 0,$ so $\exists \: g_2 \in F[X]$ s.t.
$$ g_1(X) = (X - a_2)^{m_2}g_2(X). $$

Continuing this process, we construct a (finite) sequence of non-zero polynomials $g_1, g_2, \ldots, g_k$ such that
$$ f(X) = (X - a_1)^{m_1}(X - a_2)^{m_2}\cdots(X - a_k)^{m_k}g_k(X). $$


% However, since $a_{n+1}$ is also a root and it's distinct from all previous roots $a_1, a_2, \ldots, a_n$, the polynomial $g_n(X)$ must also have $a_{n+1}$ as a root. By the same argument as before we have,
% $$ g_n(X) = (X - a_{n+1})^{m_{n+1}}g_{n+1}(X). $$

% Substituting back into the equation for $f(X)$ gives us:
% $$ f(X) = (X - a_1)^{m_1}(X - a_2)^{m_2}\cdots(X - a_n)^{m_n}(X - a_{n+1})^{m_{n+1}}g_{n+1}(X). $$

% Now, notice that the degree of the polynomial on the right-hand side is at least $n+1$ because it is a product of $n+1$ distinct linear factors, and $g_{n+1}(X)$ is a non-zero polynomial. This contradicts our initial assumption that $\deg f=n$.

% Therefore, our assumption that $f$ has more than $n$ roots must be false and $f$ has at most $n$ roots.

Considering the RHS, we must have $\deg f \geq \sum_{i=1}^k m_i \geq n+1$, contradicting that $\deg f = n$. Hence, $f$ must have at most $n$ roots.
\end{proof}

\begin{definition}
  Let $f\in F[X]$ be a polynomial over a field $F$ with $\deg f>0$. Then $f$ is
  \emph{irreducible} over $F$ if whenever $g,h\in F[X]$ are such that $f=gh$, one has either $\deg
  g=0$ or $\deg h=0$. If $f$ is not irreducible, then we say it is \emph{reducible}.
\end{definition}
\begin{remark}
  The notion of irreducibility is relative to the field you're working within; the following example demonstrates this.
\end{remark}
\begin{example}
  Consider $f(x)=X^2+1$. This is irreducible in $\RR[X]$ and reducible in $\CC[X]$, since we can write $f(x)=(X-i)(X+i)$.
  % hence $f$ is reducible in $\CC[X]$.

  The polynomial $p(x)=x^2-2$ is irreducible over $\QQ[X]$ but reducible over $\RR[X]$ as it can be factored into $p(x)=(x-\sqrt{2})(x+\sqrt{2}),$ where $\pm \sqrt{2}\in \RR \setminus \QQ$.
\end{example}

\begin{theorem}
  Let $F$ be a field and let $I\subseteq F[X]$ be an ideal. Then there exists $f\in
  F[X]$ such that $I=(f)=\{gf \mid g\in F[X]\}$.
  \label{thm:idealPrinciplas}
\end{theorem}
\begin{proof}
  % Next, consider $g\in I\setminus\{0\}$ be s.t. $\forall g'\in
  % I\setminus\{0\}$, $\deg g \leq \deg g'$. Then since $I$ is an ideal,  $\forall h\in
  % I\exists f\in F[X]$ s.t.  $gf=h$. In order words, 
  % \[I= \{ 0, g, f_1g, f_2g, \cdots f_i\in F[X] \}\]
  % Since $I$ is an ideal, we have that for every $h=f_k g \in I$ and for every $l\in F[X]$
  % we have $lh\in I$, so $lfg \in I$. In other words, $I=\{fg \mid f\in F[X]\}$, or
  % $I=(g)$.
  
  If $I=\{0\}$, then $I=(0),$ so suppose $I$ is non-trivial. 
  
  Let $f \in I$ be of minimal degree, so that $\deg f \leq \deg g$ for all $g \in I\setminus\{0\}.$ We claim that $I = (f),$ that is every $i \in I$ can be expressed as $fh$ for some $h \in F[X].$

  Certainly $(f) \subseteq I$ since $I$ is an ideal. Let $g \in I.$ Then by the Theorem \ref{thm:divisionReminderPolynomial}, $\exists q,r \in F[X]$ s.t. $g = qf + r,$ where $\deg r < \deg f$ or $r = 0.$ Since $I$ is an additive subgroup, $g - qf = r \in I,$ but we assumed $f$ was of minimal degree in $I,$ so $r = 0$. This means that $g = qf,$ so $g\in (f).$ Hence we have shown $I = (f).$
\end{proof}
% \begin{proof}[Bartel's proof]
%   If $I=\{0\}$, then $I=(0)$. Consider $f\in I\setminus \{0\}$ be of the smallest degree
%   that is non-zero. We claim $I=(f)$. Let $g\in I$ and note that by Theorem
%   \ref{thm:divisionReminderPolynomial} we have that there exists unique $q,r\in F[X]$ s.t.
%   $g=qf+r$ and $r=0$ or $\deg r < \deg f$. Since we're considering the ideal $I$, it
%   follows that $qf\in I$ and $g\in I$ by assumption, hence $r\in I$ and $r=g-qf$. But since
%   we have $\deg f$ is the minimal, it follows that $r=0$ (because $0<\deg r<\deg f$ would
%   contradict that $f$ was the smallest non-zero degree). Since $g$ was arbitrary, it
%   follows that $I=(f)$.
% \end{proof}
\begin{remark}
  The previous theorem says that $F[X]$ is a \emph{principal ideal domain}; that is, all ideals can be generated by a single element in $F[X]$.
  This does not hold if $F$ is not a field (or more precisely a ring with a valid Euclidean function). 
\end{remark}

\begin{example}
  Let $I$ be the ideal of $\QQ[X]$ generated by $(X^2+X)$ and $(X^4+X^3+X)$, i.e., 
  \[I=\{f(X^2+X) + g(X^4+X^3+X) \mid f,g\in \QQ[X]\}.\]
  By Theorem \ref{thm:idealPrinciplas}, we know that there exists $h \in \QQ[X]$ such that $I=(h)$. The gcd of $f$ and $g$ will generate the whole ideal, and in this case it is easy to see that $\gcd(f,g)=X,$ so $I=(X).$ 
\end{example}

\begin{remark}
    The single generator for an ideal of $F[X]$ generated by two polynomials $f$ and $g$ will be their $\gcd$. It may seem like any common factor of the two polynomials would work, and although this would be an ideal that contains the original ideal, it will contain additional elements that are not a linear combination of $f$ and $g.$

    As a simple example, 
    %X^2, X example
    the ideal generated by $X^2$ and $X^3$, is also just generated by $X^2,$ but $X$ is not a generator for the ideal since we get extra elements of degree 1 that would otherwise not be present.
    % When finding a single generator for an ideal that is generated by two polynomials, we find the gcd of the two polynomials. You may think that any common factor of the two polynomials would work, and although this would be an ideal that contains the original ideal, it will be a bigger ideal. Include example? Consider an ideal generated by two polynomials with gcd $x^2.$ If we consider an ideal generated by $x,$ it has additional elements with $x$ terms whereas the original ideal does not.
\end{remark}