\section{Lecture 25 - 17 Nov 2021 }
\subsection{Prime ideals in polynomial rings}
\begin{theorem}
  Let $F$ be a field and let $f\in F[X]$ be a non-zero polynomial with $\deg f = n$. Then
  $f$ has at most $n$ roots.
  \label{<+label+>}
\end{theorem}
\begin{proof}
  Assume that $f$ has at least $n+1$ roots in $F$, and $k$ distinct roots, say $a_1, a_2, \ldots, a_{k}$. Let each root $a_i$ have multiplicity $m_i$, so that $\sum_{i=1}^k m_i \geq n+1$. 
%   By the previous corollary, for each root $a_i$, there exists a polynomial $g_i(X) \in F[X]$ such that:
% $$ f(X) = (X - a_i)^{m_i}g_i(X) \quad \text{for } i = 1, 2, \ldots, k. $$

Since $a_1$ is a root of $f$, by the previous corollary, $\exists \: g_1 \in F[X]$ s.t.
$$ f(X) = (X - a_1)^{m_1}g_1(X). $$

Now, because $a_2 \neq a_1$ is also a root of $f$ and $F[X]$ has no zero divisors, we must have $g_1(a_2) = 0,$ so $\exists \: g_2 \in F[X]$ s.t.
$$ g_1(X) = (X - a_2)^{m_2}g_2(X). $$

Continuing this process, we construct a sequence of polynomials $g_1, g_2, \ldots, g_k$ such that
$$ f(X) = (X - a_1)^{m_1}(X - a_2)^{m_2}\cdots(X - a_k)^{m_k}g_k(X). $$


% However, since $a_{n+1}$ is also a root and it's distinct from all previous roots $a_1, a_2, \ldots, a_n$, the polynomial $g_n(X)$ must also have $a_{n+1}$ as a root. By the same argument as before we have,
% $$ g_n(X) = (X - a_{n+1})^{m_{n+1}}g_{n+1}(X). $$

% Substituting back into the equation for $f(X)$ gives us:
% $$ f(X) = (X - a_1)^{m_1}(X - a_2)^{m_2}\cdots(X - a_n)^{m_n}(X - a_{n+1})^{m_{n+1}}g_{n+1}(X). $$

% Now, notice that the degree of the polynomial on the right-hand side is at least $n+1$ because it is a product of $n+1$ distinct linear factors, and $g_{n+1}(X)$ is a non-zero polynomial. This contradicts our initial assumption that $\deg f=n$.

% Therefore, our assumption that $f$ has more than $n$ roots must be false and $f$ has at most $n$ roots.

Considering the RHS, we must have $\deg f \geq \sum_{i=1}^k m_i \geq n+1$, contradicting that $\deg f = n$. Hence, $f$ must have at most $n$ roots.
\end{proof}

\begin{definition}
  Let $f\in F[X]$ be a polynomial over the field $F$ with $\deg f>0$. Then $f$ is
  \emph{irreducible} if whenever $g,h\in F[X]$ are such that $f=gh$, one has either $\deg
  g=0$ or $\deg h=0$. If $f$ is not irreducible, then it's called \emph{reducible}.
\end{definition}
\begin{remark}
  The notion of irreducability is relative to the field you're working on.
\end{remark}
\begin{example}
  Consider $f(x)=X^2+1$. This is irreducible in $\RR[X]$. However, in $\CC[X]$, we can
  write $f(x)=(X-i)(X+i)$, hence $f$ is reducible in $\CC[X]$.
\end{example}

\begin{theorem}
  Let $F$ be a field and let $I\subseteq F[X]$ be an ideal. Then there exists $f\in
  F[X]$ s.t. $I=(f)=\{gf : g\in F[X]\}$.
  \label{thm:idealPrinciplas}
\end{theorem}
\begin{proof}
  If $I=\{0\}$, then $I=(0)$. Next, consider $g\in I\setminus\{0\}$ be s.t. $\forall g'\in
  I\setminus\{0\}$, $\deg g \leq \deg g'$. Then since $I$ is an ideal,  $\forall h\in
  I\exists f\in F[X]$ s.t.  $gf=h$. In order words, 
  \[I= \{ 0, g, f_1g, f_2g, \cdots f_i\in F[X] \}\]
  Since $I$ is an ideal, we have that for every $h=f_k g \in I$ and for every $l\in F[X]$
  we have $lh\in I$, so $lfg \in I$. In other words, $I=\{fg \mid f\in F[X]\}$, or
  $I=(g)$.
\end{proof}
\begin{proof}[Bartel's proof]
  If $I=\{0\}$, then $I=(0)$. Consider $f\in I\setminus \{0\}$ be of the smallest degree
  that is non-zero. We claim $I=(f)$. Let $g\in I$ and note that by Theorem
  \ref{thm:divisionReminderPolynomial} we have that there exists unique $q,r\in F[X]$ s.t.
  $g=qf+r$ and $r=0$ or $\deg r < \deg f$. Since we're considering the ideal $I$, it
  follows that $qf\in I$ and $g\in I$ by assumption, hence $r\in I$ and $r=g-qf$. But since
  we have $\deg f$ is the minimal, it follows that $r=0$ (because $0<\deg r<\deg f$ would
  contradict that $f$ was the smallest non-zero degree). Since $g$ was arbitrary, it
  follows that $I=(f)$.
\end{proof}
\begin{remark}
  This theorem does not hold if $F$ is not a field. The above theorem proves that $F[X]$ is a principle ideal domain. That is all ideals are generated by 1 element in $F[X]$.
\end{remark}

\begin{example}
  Let $I$ be the ideal of $\QQ[X]$ generated by $(X^2+X)$ and $(X^4+X^3+X)$, 
  \[I=\{f(X^2+X) + g(X^4+X^3+X) : f,g\in \QQ[X]\}\]
  By the above theorem, we have that 
  \[I=(X)\]
\end{example}
