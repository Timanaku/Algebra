\section{Lecture 4}
\subsection{Cosets}
\begin{definition}
  Let $G$ be a group and $H$ a subgroup of $G$. Let $g\in G$. The \emph{left coset} of $H$
  containing $g$ is the set $gH=\left\{ gh \mid h\in H \right\}$. Similarly, the \emph{right coset} of $H$
  containing $g$ is the set $Hg=\left\{ hg \mid h\in H \right\}$.
  \label{coset}
\end{definition}

\begin{example}
  Let $G$ be the symmetric group $S_4$ and let $H=\langle(1,2,3)\rangle=\left\{
e,(1,2,3), (1,3,2) \right\}$ and let $g=(1,4)$. Then we have 
\[gH = \left\{ (1,4), (1,2,3,4), (1,3,2,4) \right\} \subset G.\]
Note that this is not a group itself since it doesn't have the identity; it is simply a
subset of $G$. Moreover, the right coset
\[Hg = \left\{ (1,4), (1,4,2,3), (1,4,3,2) \right\}\]
is a different coset that only has in common $g$. So $gH\neq Hg$.
\end{example}

\begin{theorem}
  Let $G$ be a group, $H\leq G$, and $g,g'\in G$. Then one has $gH=g'H \iff g'^{-1}g\in H$.
\end{theorem}
\begin{proof}
  ($\implies$) Suppose $gH=g'H$. Then there
  exists $h\in H$ s.t. $g=g'h$, and so $g'^{-1}g=h \implies g'^{-1}g\in H$, as required.

  ($\impliedby$) Let $gh\in gH$. We have that $g'^{-1}g = h'$, for some $h'\in H,$ so $g = g'h',$ meaning $gh = g'h'h \in g'H,$ so $gH \subseteq g'H$. We can use a symmetric argument to conclude that $g'H \subseteq gH,$ so $gH = g'H$.
  % Suppose we have $g'^{-1}g\in H$.  Then $ \exists h\in H \text{ s.t. } g'^{-1}g =h$, so $g=g' h$. Similarly, $\exists h' \in H$ s.t. $g'=gh'.$ Therefore $gH=g'H.$ 
\end{proof}

\begin{example}
  Let $G=(\RR,+)$ and $H=\ZZ$. We see that for $x,y \in \RR$, we have $x+\ZZ=y+\ZZ$ if and only if
  $y-x\in \ZZ$, i.e.  $x$ and $y$ differ by an integer.
\end{example}
% \begin{example}
%   A very illustrating example is the group $G=(\RR^3, +)$ and $H=\langle(1,0,0),(0,1,0)\rangle$ (note
%   $H$ is just the $xy$-plane). Let $g,g'\in G$ be vectors. Following the above theorem,
%   we have $g+H=g'+H \iff g-g'\in H$, so we must have $g_3=g'_3$ (so that $g$ and $g'$
%     cancel out in the z-axis and land on the xy-plane).
% \end{example}

\begin{corollary}[Absorption rule]
  Let $G$ be a group and $H\leq G$, and $g\in G$. Then $gH=H \iff g\in H$.
\end{corollary}

\begin{corollary}
  The relation $\sim$ on $G$ defined by $g\sim g'$ iff $gH=g'H$ is an equivalence relation.
  % and so this partitions the whole group $G$ into equivalence classes, which
  % are all the cosets of $H$.
  In particular, the equivalence classes are all the left cosets of $H$ in $G$.
  \label{leftCosetsEqRel}
\end{corollary}

\begin{proof}
     The relation is trivially reflexive, symmetric and transitive by properties of equality.
\end{proof}

\begin{remark}
  Note that this gives the natural conception that cosets partition the group --- in a sense it allows us to look at the group on a larger scale. However, this partition is only another set; it need not necessarily conserve group structure.
\end{remark}


\begin{theorem}
  All cosets of $H$ have the same cardinality. That is $\forall g\in G$, $|gH|=|H|$.
  \label{cosetsCardinality}
\end{theorem}
\begin{proof}
  % We claim there exists a bijection $H\to gH$. This bijection can be $h\mapsto gh$.
  % This is surjective almost by definition, since $gH=\left\{ gh \mid h\in H \right\}$.
  % Injectivity rises from $gh=gh'\implies h=h'$.

  Fix $g \in G$ and let $\varphi:H \longrightarrow gH$ be the mapping $h \longmapsto gh$. We claim that $\varphi$ is a bijection.

  This is surjective since every element of $gH$ can be recovered as $gh$ for some $h \in H$.

  If we suppose $gh = gh',$ then clearly $h = h'$ after we left multiply by $g^{-1}$, so $\varphi$ is injective.
\end{proof}
