\section{Lecture 4 - 27 Sep 2021}
\subsection{Cosets}
\begin{definition}
  Let $G$ be a group and $H$ a subgroup of $G$. Let $g\in G$. The left coset of $H$
  containing $g$ is the set $gH=\left\{ gh | h\in H \right\}$. The right coset of $H$
  containing $g$ is the set $Hg=\left\{ hg | h\in H \right\}$.
  \label{coset}
\end{definition}

\begin{ex}
  Let $G$ be the symmetric group on $4$ letters $S_4$ and let $H=\langle(1,2,3)\rangle=\left\{
e,(1,2,3), (1,3,2) \right\}$ and let $g=(1,4))$. Then we have 
\[gH = \left\{ (1,4), (1,2,3,4), (1,3,2,4) \right\} \subset G\]
Note that this is not a group itself since it doesn't have the identity. It's simply a
subset of $G$. Moreover, the right coset
\[Hg = \left\{ (1,4), (1,4,2,3), (1,4,3,2) \right\}\]
Is a different coset that only has in common $g$. So $gH\neq Hg$.
\end{ex}

\begin{theorem}
  Let $G$ be a group, $H<G$, and $g,g'\in G$. Then one has $gH=g'H \iff g'^{-1}g\in H$.
\end{theorem}
\begin{proof}
  ($\Rightarrow$) since we have $gH=g'H$, we must have that there must
  exists $h\in H$ s.t. $g=g'h$, and so $g'^{-1}g=h \implies g'^{-1}g\in H$, as required.

  ($\Leftarrow$) Since we have $g'^{-1}g\in H \implies \exists h\in H ; g'^{-1}g =h$, so
  we must have $g=g' h$, hence $gH=g'H$. This is because we can now get arbitrary $h'\in
  H$ and get \[gh'=g'hh'\] 
\end{proof}

\begin{example}
  We see that for $G=(\RR,+)$ and $H=\ZZ$, we have $g+H=g'+H$ if and only if
  $g'^{-1}g\in H$, i.e.  $g',g$ differ by an integer.
\end{example}
\begin{example}
  A very illustrating example is the group $G=(\RR^3, +)$ and $H=<(1,0,0),(0,1,0)>$ (note
  $H$ is just the xy-plane). Let $g,g'\in G$ be some vectors. Following the above theorem,
  we have $g+H=g'+H \iff g-g'\in H$, so we must have $g_3=g'_3$ (so that $g$ and $g'$
    cancel out in the z-axis and land on the xy-plane).
\end{example}

\begin{corollary}
  Let $G$ be a group and $H< G$, and $g\in G$. Then $gH=H$ iff $g\in H$.
\end{corollary}

\begin{corollary}
  The relation $\sim$ on $G$ defined by $g\sim g'$ iff $gH=g'H$ is an equivalence
  relation, and so this partitions the whole group $G$ into equivalence classes, which
  are all the cosets of $H$.
  \label{leftCosetsEqRel}
\end{corollary}

\begin{remark}
  Note that this gives the natural conception that cosets partition the group -- in a sense, it
  looks at the group at a larger \emph{scale}. However, this partition is only another
  set. It does not necessarily conserve group structure.
\end{remark}


\begin{theorem}
  All cosets of $H$ have the same cardinality. We have $\forall g\in G$, $|gH|=|H|$.
  \label{cosetsCardinality}
\end{theorem}
\begin{proof}
  We claim there exists a bijection $H\to gH$. This bijection can be $h\mapsto gh$.
  This is surjective almost by definition, since $gH=\left\{ gh | h\in H \right\}$.
  Injectivity rises from $gh=gh'\implies h=h'$.
\end{proof}
