\section{Lecture 1 - 21 Sep 2021} 
\begin{definition}
  A group is a pair $(G,*)$ where $G$ is a set and $*:G\times G\to G$ is a binary
  operation with the following axioms:
  \begin{itemize}
      \ii (Associativity) For any $g,h,k\in G$
      \[(g*h)*k = g*(h*k)\]
      \ii (Existence of identity) There exists $e\in G$ such that for any $g\in G$,
      \[e*g=g*e = g\]
      \ii (Existence of inverse) For any $g\in G$ there exists $h\in G$ s.t. 
      \[g*h=h*g=e\]
  \end{itemize}
  \label{group}
\end{definition}
Note that the second axiom implies that the identity element is unique, since assuming
that two different identity elements exist, it automatically leads to a contradiction
saying that the two identity elements are the same. Moreover, note that closure is implied
by the definition of binary operation.

\begin{theorem}
  Let $G$ be a group with identity $e\in G$ and an element $g\in G$ with left inverse
  $h\in G$ and right inverse $h'\in G$ s.t. $hg = gh'=e$. Then $h=h'$.
\end{theorem}
\begin{proof}
  We have $hg=gh'$ so $hhg=hgh'$, which implies by associativity $hhg=he=h=eh'=h'$.
\end{proof}
Note that the above also implies that the inverse is unique.


\begin{theorem}
  Let $G$ be a group. Then
  \begin{itemize}
      \ii $\forall g\in G$ one has $(g^{-1})^{-1} = g$.
      \ii $\forall g,h,k\in G$ the following are equivalent:
      \begin{itemize}
        \ii $gh = gk$
        \ii $h=k$
        \ii $hg=kg$
      \end{itemize}
      \ii $\forall  g,h\in G$, we have $(gh)^{-1} = h^{-1} g^{-1}$.
  \end{itemize}
\end{theorem}

\begin{proof}
    Let $g,h,k \in G$
    \begin{itemize}
        \ii By definition of an inverse $g^{-1} (g^{-1})^{-1}=1_G=(g^{-1})^{-1}g^{-1}$ and left multiplying (or right) by $g$ gives $(g^{-1})^{-1} = g$.
        \ii $gh=gk \iff h=k \iff hg=kg$ by left cancellation and right multiplication (backwards implication follows by right cancellation and left multiplication)
        \ii $h^{-1} g^{-1} gh = 1_G = gh h^{-1} g^{-1}$ so by definition $h^{-1} g^{-1}$ is the inverse of $gh$ so $h^{-1} g^{-1} = (gh)^{-1}$
    \end{itemize}
\end{proof}

\begin{definition}
  A group $G$ is abelian if the group is commutative, i.e. $\forall g,h\in G$ we have
  $gh=hg$.
  \label{abelianGroup}
\end{definition}

\begin{definition}
  We say a group is finite or countable if the underlying set is finite or countable,
  respectively.
\end{definition}
Note that we usually use multiplication notation (powers, etc) to denote group operation.
We use additive notation instead if the group is Abelian.

\subsection{Symmetric groups}
Since using examples of symmetric groups is recurrent in this course, I thought convenient
to include the Baker course notes on this topic. 

The n-th symmetric group $S_n$ consists of all bijections (or permutations) of the set
${1,2,\cdots , n}$ under function composition. The group has $n!$ elements (permutations).
Each of these elements can be represented by the disjoint cycle notation. Each $\sigma\in
S_n$ can be written as a composition of permutations $\rho$, of the form
\[r_1\to r_2=\rho(r_1) \to \cdots \to r_l=\rho(r_{l-1})\]
Omitting the $r_k\to r_k=\rho(r_k)$. The usual name of this $l$-cycle $\rho$ is
$(r_1, r_2, \cdots, r_l)$. It's usual to omit $1$-cycles, since they correspond to the
identity element. Every element in $S_n$ can be written as a composition of $l$-cycles
(disjoint) of length $1\leq l \leq n$. This factorisation of $\rho$ is unique.

