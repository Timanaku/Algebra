\section{Lecture 1} 
\begin{definition}
  A \emph{group} is a pair $(G,*)$, where $G$ is a set and $*:G\times G\to G$ is a binary
  operation, that satisfies the following axioms:
  
       (G1) Associativity: for any $g,h,k\in G$,
      \[(g*h)*k = g*(h*k);\]
      
       (G2)  Existence of identity: there exists $e\in G$ such that for any $g\in G$,
      \[e*g=g*e = g;\]
      
       (G3) Existence of inverses: for any $g\in G$ there exists $h\in G$ s.t. 
      \[g*h=h*g=e.\]
  
  \label{group}
\end{definition}
% When notating groups, we omit the binary operation if it is well-understood.
% Note that the second axiom implies that the identity element is unique, since assuming
% that two different identity elements exist, it automatically leads to a contradiction
% saying that the two identity elements are the same. Moreover, note that closure is implied
% by the definition of binary operation.

We should note that we are assuming $G$ is closed under the binary operation, but 
%(this is something that must be verified about an operation on element). 
in general we should verify that the product of two elements of $G$ stays in $G$. Often we will omit the binary operation so that $g*h$ is shortened to just $gh$. 

Also, from the second axiom we can immediately deduce that the identity in a group must be unique. We will typically use $e$ or $1$ to denote the identity element of a group, but when there is the possibility of ambiguity we will use $1_{G}$ to denote the identity element in the group $G$.

% \begin{theorem}
%   Let $G$ be a group with identity $e\in G$ and an element $g\in G$ with left inverse
%   $h\in G$ and right inverse $h'\in G$ s.t. $hg = gh'=e$. Then $h=h'$.
% \end{theorem}

\begin{theorem}
    Let $G$ be a group and $g \in G$ with left inverse $h$ and right inverse $h'$, i.e. $$h*g = g*h'=e \in G.$$ Then $h=h'$. 
\end{theorem}

\begin{proof}
  % We have $hg=gh'$ so $hhg=hgh'$, which implies by associativity $hhg=he=h=eh'=h'$.

  We have that
  $$h = h*e = h*(g*h') = (h*g)*h' = e*h' = h',$$
  by direct application of the group axioms.
\end{proof}
Note that the above also implies that the inverse is unique, and we need not make the distinction between left and right inverses in a group. We denote the inverse of an element $g\in G$ as $g^{-1}.$


% \begin{theorem}
%   Let $G$ be a group. Then
%   \begin{itemize}
%       \ii $\forall g\in G$ one has $(g^{-1})^{-1} = g$.
%       \ii $\forall g,h,k\in G$ the following are equivalent:
%       \begin{itemize}
%         \ii $gh = gk$
%         \ii $h=k$
%         \ii $hg=kg$
%       \end{itemize}
%       \ii $\forall  g,h\in G$, we have $(gh)^{-1} = h^{-1} g^{-1}$.
%   \end{itemize}
% \end{theorem}
\begin{proposition}
     Let $G$ be a group and $g,h \in G$. Then we have
     \begin{enumerate}
        \item $(g^{-1})^{-1} = g$,
        \item $(gh)^{-1} = h^{-1} g^{-1}.$
    \end{enumerate}
\end{proposition}
\begin{proof}
    % \begin{enumerate}
        (1) The inverse of $g^{-1}$ is an element $x \in G$ that satisfies $xg^{-1} = g^{-1}x = e.$ 
        
        Taking the second equality, we can left multiply by $g$ to get
        $$
        gg^{-1}x = g \implies x = g.
        $$
        % and we can notice that $g$ is exactly the element that satisfies this condition.
        (2) We can verify that
        \begin{align*}
            (gh)( h^{-1} g^{-1}) &= g(h h^{-1})g^{-1} \\
            &= g e g^{-1} \\
            &= g g^{-1} \\
            &= e,
        \end{align*}
        meaning that the element $h^{-1} g^{-1}$ is the unique inverse of $gh$.
    % \end{enumerate}
\end{proof}

\begin{proposition}
    Let $G$ be a group and $g,h,k \in G$. Then the following are equivalent:
    \begin{enumerate}
        \item $gh = gk,$
        \item $h=k,$
        \item $hg=kg$.
    \end{enumerate}
\end{proposition}

\begin{proof}
    If $gh = gk,$ left multiplying by $g^{-1}$ and applying the inverse axiom we see that $h = k$. Similarly, given that $h=k$, we can left multiply by $g$ so that $gh = gk.$

    Applying a similar argument for right multiplication shows that all three statements are equivalent.
\end{proof}

% \begin{proof}
%     Let $g,h,k \in G$
%     \begin{itemize}
%         \ii By definition of an inverse $g^{-1} (g^{-1})^{-1}=1_G=(g^{-1})^{-1}g^{-1}$ and left multiplying (or right) by $g$ gives $(g^{-1})^{-1} = g$.
%         \ii $gh=gk \iff h=k \iff hg=kg$ by left cancellation and right multiplication (backwards implication follows by right cancellation and left multiplication)
%         \ii $h^{-1} g^{-1} gh = 1_G = gh h^{-1} g^{-1}$ so by definition $h^{-1} g^{-1}$ is the inverse of $gh$ so $h^{-1} g^{-1} = (gh)^{-1}$
%     \end{itemize}
% \end{proof}

\begin{definition}
  A group $G$ is \emph{abelian} if the group operation is commutative, i.e. $\forall g,h\in G$ we have
  $gh=hg$.
  \label{abelianGroup}
\end{definition}
When a group is abelian, we often use additive notation to denote the group operation, as opposed to the typical multiplicative notation (e.g. powers).


\begin{definition}
  We say a group is \emph{finite} or \emph{countable} if the underlying set is finite or countable,
  respectively.
\end{definition}


\subsection{Symmetric groups}
% Since using examples of symmetric groups is recurrent in this course, I thought convenient
% to include the Baker course notes on this topic. 

% The n-th symmetric group $S_n$ consists of all bijections (or permutations) of the set
% ${1,2,\cdots , n}$ under function composition. The group has $n!$ elements (permutations).
% Each of these elements can be represented by the disjoint cycle notation. Each $\sigma\in
% S_n$ can be written as a composition of permutations $\rho$, of the form
% \[r_1\to r_2=\rho(r_1) \to \cdots \to r_l=\rho(r_{l-1})\]
% Omitting the $r_k\to r_k=\rho(r_k)$. The usual name of this $l$-cycle $\rho$ is
% $(r_1, r_2, \cdots, r_l)$. It's usual to omit $1$-cycles, since they correspond to the
% identity element. Every element in $S_n$ can be written as a composition of $l$-cycles
% (disjoint) of length $1\leq l \leq n$. This factorisation of $\rho$ is unique.

One recurring class of groups we will consider are symmetric groups.

\begin{definition} 
    Let $X$ be a set. The \emph{symmetric group} of $X$, denoted $\operatorname{Sym}(X)$ is the set of all permutations of the elements of $X$. Symbolically, $$\operatorname{Sym}(X) = \{\sigma: X \to X \mid \sigma \text{ is a bijection}\}.$$ 

    If $X = \{1, 2, \ldots, n\}$ for some $n \in \NN$, we say $\operatorname{Sym}(X)$ is the symmetric group on $n$ letters, denoted $S_n$.
\end{definition}

\begin{remark}
    Symmetric groups are indeed groups under composition of permutations. It is straightforward to check that the composition of permutations is indeed a permutation, and that the remaining group axioms hold.
    We will later see that all finite groups can be understood within the context of a symmetric group.
\end{remark}

%cycle stuff...
Any permutation can be decomposed into a product of disjoint cycles. For example, the bijection 
% $$\{1,2,3,4,5\} \longmapsto \{2,3,1,5,4\}$$ 
$$
\begin{pmatrix}
1 & 2 & 3 & 4 & 5\\
2 & 3 & 1 & 5 & 4
\end{pmatrix}
$$

is equivalent to the composition of cycles $(123)(45).$ The notation $(123)$ is shorthand for the permutation sending 1 to 2, 2 to 3, and 3 to 1.
