\section{Lecture 19 - 3 Nov 2021}
\subsection{Subrings, ideals, and quotients}
\begin{definition}
  Let $R$ be a ring. A subring is an additive subgroup $S\subset R$ s.t. for every
  $a,b\in S$, $ab\in S$. We usually write $S\leq R$. If $R$ is unital, a subring $S\leq R$
  is called a unital subring if $1\in S$. If $R$ is unital, then we assume every subring
  to be unital, unless otherwise stated.
  \label{<+label+>}
\end{definition}
There's a counter example to the claim that every subring of a unital ring is unital. 

\begin{example}
  We have $\ZZ \leq \QQ \leq \RR \leq \CC \leq \HH$.

  Another interesting example is if $R$ is a commutative ring, then $R$ itself is a
  subring of the ring of polynomials $R[X]$.

  Let $S$ be a subring of the unital ring $R$. Then for every $n\in\ZZ_{>0}$ we have
  $M_{n\times n}(S)$ is a subring of the unital ring $M_{n\times n}(R)$.

  Consider the set of all functions $f:\NN\to\RR$, i.e. all sequences of real numbers.
  Define the pointwise addition and multiplication of elements of the sequence. The subset
  of all sequence that converge to $0$ is a subring itself. The sequence of zeroes is the
  additive identity. This subring is not unital. Hence a counter example to the claim
  under the defintion above, even though the ring is unital, namaing the multiplicative
  identity to be the sequence with all ones. 
\end{example}


Question: What substructures do we need to consider to form quotients?
That depends on the kind of quotient structure we aim for: additive $(a+S)$ or
multiplicative cosets $(bS)$? We can look at existing cosets, naming $\ZZ/n\ZZ$. We will
try to build cosets whose elements consist of additive cosets. From the special case of
additive groups, we know that for the operation of cosets to be well defined we need the
substructure to be abelian. Note that as a group, rings are abelian, so every subring will
also be an abelian subgroup. Let $I\subset R$ be an additive subgroup, and let $a,b\in R$.
We want to define $(a+I)(b+I)=(ab)+I$. Let $a'=a+x, b'=b+y$ for $x,y\in I$ so that $a'$ is
a representative of the same coset that $a$ form, and similarly for $b'$. Then
\[(a+x+I)(b+y+I)= (ab+ay+xb+xy) + I\]
In the one hand, if we take $y=0$, we require $xb\in I$ to have $ab+xb+I=ab+I$. If we take
$x=0$, we require $ab+ay+I=ab+I$, i.e. $ay\in I$. Note that if we have $xb\in I$ and
$ay\in I$ it follows that $xy\in I$.

\begin{definition}
  Let $R$ be a ring. A \emph{left ideal} of $R$ is an additive subgroup $I$ of $R$ s.t.
  for every $a\in I$ and $r\in R$, $ra\in I$. That is, for every $r\in R$, we have
  $rI\subseteq I$.

  A \emph{right ideal} of $R$ is an additive subgroup $I$ of $R$ s.t. for every $a\in I$
  and $r\in R$, $ar\in I$. That is, for every $r\in R$, we have $rI\subseteq I$.

  A \emph{two-sided ideal} is an additive subgroup $I$ of $R$ s.t. it's a left ideal and a
  right ideal. We write $I\trianglelefteq R$.

  An ideal of $R$ is called \emph{proper} if it's not equal to $R$.
  \label{<+label+>}
\end{definition}
Spoiler alert: The two-sided ideals are exactly the conditions we need to create
well-defined addition of cosets. 
\todo{Is it the only condition? Do we need more conditions? Is there a way of building
addition of cosets in a different way than two-sided ideals?}
Note that for an ideal to be an proper, it's necessary for it to not contain the
multiplicative identity, since if it did, then for some $r\in R$ not in the ideal will
produce $r1=r$, which we said is not in the ideal. Hence it can't be an ideal. 

\begin{definition}
  Let $R$ be a unital ring, and $I$ be a proper two-sided ideal in $R$. The
  \emph{quotient ring} has, as its underlying set, the set of cosets $\left\{ r+I : r\in R
  \right\}$. We define the addition of cosets as $(r+I)+(s+I)=(r+s)+I$ and multiplication
  as $(r+I)(s+I)=(rs)+I$.
  \todo{Think about how two-sided proper ideals implies that the operations are
  well-defined, i.e. do not depend on quotient representatives, where two elements
$r+I,s+I$ are said to be the same if $s=r+x \exists x\in I$.}
  \label{<+label+>}
\end{definition}

Note that the quotient ring, which was guided and motivated by $\ZZ/n\ZZ$ is a bit easier
to think about than group quotients, since quotient rings will look like $\ZZ/n\ZZ$, which
is rather easy to visualize, in contrast with quotient groups.
\begin{example}
  Consider the polynomial ring $R=\RR[X]$. The subset $I=X^2 R$, the set of real
  polynomials of degree $2$ or greater, is ideal. Think that for every element $r$ of $R$, the
  product $rh$ for some $h\in I$ will still be in $I$, since the degree of $rh$ will be at
  least $2$ (by $h$). Two polynomials $f=a_0+a_1x+\cdots$, $g=b_0+b_1x+\cdots$ are in the
  same quotient $R/I$ if and only if $a_0=b_0$, $a_1=b=1$ so that $f-g\in I$. I.e. for
  some $a,b\in \RR$, the coset $a+bx+I$ is unique. For any $c\in\RR$, the coset
  $a+bx+cx^2+I$ will be equal for every $c$. Thus $R/I$ can be seen as a $2$-dimensional
  vector space while $R$ is an infinite dimensional vector space, over $\RR$. Here $R/I$
  is spanned by $\hat{1}=1+I$, $\hat{X}=X+I$, with property $\hat{X}^2=0$, since
  $X^2+I=0+I=I$.
\end{example}
