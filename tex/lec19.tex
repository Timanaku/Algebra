\section{Lecture 19}
\subsection{Subrings, ideals, and quotients}
\begin{definition}
  Let $R$ be a ring. A \emph{subring} of $R$ is an additive subgroup $S\subset R$ s.t. for every
  $a,\text{ } b\in S$, one has $ab\in S$. We usually write $S\leq R$ to say that $S$ is a subring of $R$. If $R$ is unital, a subring $S\leq R$ is called a unital subring if $1\in S$. 
  %If $R$ is unital, then we assume every subring
  %to be unital, unless otherwise stated.
  \label{<+label+>}
\end{definition}
%There's a counter example to the claim that every subring of a unital ring is unital. 
\begin{remark}
    Note that although we mainly consider unital subrings, a subring of a unital ring need not be unital. For example, 
    $2\ZZ \leq \ZZ$ does not contain $1$.
\end{remark}


\begin{example}
    We have the chain of subrings $\ZZ \leq \QQ \leq \RR \leq \CC \leq \HH$. Here are some more interesting examples.
    \begin{itemize}
        \item If $R$ is a commutative ring, then $R$ itself is a
  subring of the ring of polynomials $R[X].$
        \item Let $S$ be a subring of the (unital) ring $R$. Then for every $n\in\ZZ_{>0}$ we have $M_{n}(S)$ is a subring of the (unital) ring $M_{n}(R)$.
        \item Consider the set of all functions $f:\NN\to\RR$, i.e. all sequences of real numbers. %meow
        Define addition and multiplication pointwise with elements of the sequence. The sequence of zeroes is the additive identity, and the subset of all sequences that converge to $0$ is a subring. This subring is not unital, since the multiplicative identity would have to be the sequence of ones, which does not converge to $0$.
    \end{itemize} 

  % Hence a counterexample to the claim
  % under the definition above, even though the ring is unital, naming the multiplicative
  % identity to be the sequence with all ones. 
\end{example}


We will now move our attention to forming quotient rings. Recall that normal subgroups form quotient groups. What substructures do we need to consider to form quotient rings?
That depends on the kind of quotient structure we aim for: additive $(r+S)$ or
multiplicative cosets $(rS)$?

We can look at existing cosets, namely $\ZZ/n\ZZ$. We will
try to build cosets whose elements consist of additive cosets. 

Note that as a group, rings are abelian, so every subring will
also be an abelian subgroup. We know then that the operation for additive cosets of rings is well defined as all subgroups of an abelian group are normal. We now need a condition to ensure the well definedness of multiplication of cosets for rings.

Let $I\subset R$ be an additive subgroup, and let $a,\text{ }b\in R$.
We want to define multiplication of cosets as $(r+I)(s+I)=(rs)+I$. So when is this well defined?

Let $r + I = r' + I$ and $s + I = s'+ I$. This gives that there exists some $i,j \in I$ such that $r' = r + i$ and $s' = s + j$. We need $r's' + I = rs + I$ for the operation to be well defined.
Consider
$$
r's' - rs = (r+i)(s+j) - rs = rj + is + ij.
$$
As $r's'-rs \in I \implies r's' + I = rs + I$, we need to ensure $rj + is + ij\in I.$

We have now motivated our definition for an ideal; we want that for every $i \in I$ and $r \in R$ one has $ri, ir \in I$. We can split this into 3 types of ideals: left, right and two-sided.

\begin{remark}
    Spoiler alert: The two-sided ideals are exactly the conditions we need to create well-defined multiplication of cosets. 
\end{remark}


% Let $a'=a+x, b'=b+y$ for $x,y\in I$ so that $a'$ is
% a representative of the same coset that $a$ form, and similarly for $b'$. Then
% \[(a+x+I)(b+y+I)= (ab+ay+xb+xy) + I\]
% In the one hand, if we take $y=0$, we require $xb\in I$ to have $ab+xb+I=ab+I$. If we take
% $x=0$, we require $ab+ay+I=ab+I$, i.e. $ay\in I$. Note that if we have $xb\in I$ and
% $ay\in I$ it follows that $xy\in I$.

\begin{definition}
  Let $R$ be a ring. A \emph{left ideal} of $R$ is an additive subgroup $I$ of $R$ s.t.
  for every $a\in I$ and $r\in R$, $ra\in I$. That is, for every $r\in R$, we have
  $rI\subseteq I$.

  A \emph{right ideal} of $R$ is an additive subgroup $I$ of $R$ s.t. for every $a\in I$
  and $r\in R$, $ar\in I$. That is, for every $r\in R$, we have $rI\subseteq I$.

  A \emph{two-sided ideal} is an additive subgroup $I$ of $R$ that is a left ideal and a
  right ideal. We write $I\trianglelefteq R$ and say $I$ is an ideal of $R$.

  An ideal of $R$ is called \emph{proper} if it is not equal to $R$. 
  \label{<+label+>}
\end{definition}
% \begin{remark}
% Note that for an ideal to be an proper, it is necessary for it to not contain the multiplicative identity. If it did, consider some $r\in R$ that is not in the ideal. We have that $r1=r$ must be in the ideal as 1 is in the ideal; a contradiction. 
% \end{remark}

\begin{remark}
If a two-sided ideal $I$ of $R$ contains $1$, then $I = R$, since by properties of ideals, $1r = r \in I,$  $\forall r \in R.$
This means that no proper ideal can be a unital subring.
\end{remark}

\begin{remark}
    In any ring $R$, the trivial and maximal subrings $\{0_R\}$ and $R$ are ideals. In a field $F$, if a non-zero element $x \in F$ is present in an ideal, then so is $xx^{-1} = 1.$ This means that these are the only ideals in a field. 
\end{remark}

\begin{definition}
  Let $R$ be a unital ring, and $I$ be a proper two-sided ideal in $R$. The
  \emph{quotient ring} $R/I$ has, as its underlying set, the set of cosets $\left\{ r+I \mid r\in R
  \right\}$. We define the addition of cosets as $(r+I)+(s+I)=(r+s)+I$ and multiplication
  as $(r+I)(s+I)=(rs)+I$. 
  \label{<+label+>}
\end{definition}

Note that we have an equivalent condition as with group cosets for two elements of $R/I$ having different representatives but being the same coset. We have $r + I = s + I \iff r - s \in I$.

\begin{example}
  Let $R=\RR[X]$, and consider the subset $I=X^2 R=\{a_2X^2+a_3X^3+\cdots a_dX^d\in \RR[X]: d\in \ZZ_{\geq 2},\text{ } a_i\in \RR\}$. We can check that this is an additive subgroup of $R.$ Now let $r\in R$ and $i\in I$. As $i$ has degree at least 2, the product $ri$ will also have degree at least 2, so $ri=ir \in I.$ Therefore $X^2R$ is an ideal of $R.$ 

  Let us now consider the quotient ring $R/I$. Two polynomials $f=a_0+a_1x+\cdots$, $g=b_0+b_1x+\cdots \in R$ represent the same coset in $R/I$ if and only if $a_0=b_0$, $a_1=b_1$, so that $f-g\in I$. In other words, for each $a,b\in \RR$, the coset $a+bx+I$ is unique. Thus, $R/I$ can be thought of as a plane where each point, or coset, is uniquely determined by a linear polynomial $a+bX+I.$  Therefore, $R/I$ can be can be seen as a $2$-dimensional
  vector space spanned by $\hat{1}=1+I$, $\hat{X}=X+I$, with the property that $\hat{X}^2=0$, since
  $X^2+I=0+I=I$.
\end{example}
