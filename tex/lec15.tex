\section{Lecture 15}
\subsection{Applications of Orbit-Stabiliser Theorem and Cauchy's Theorem}

\begin{definition}
    Let XX be a GG-set. We write XG:={x∈X∣∀g∈G,g⋅x=x}X^G:=\{x \in X \mid \forall g \in G, g \cdot x=x\}
    % g \cdot x=x \:\: \forall g \in G\} \subset X$ 
    to denote the set of fixed points of the action. In other words, XGX^G consists of all xx in XX with \OrbG(x)={x}.\Orb_G(x)=\{x\}. 
    % so x∈XG⟹\OrbG(x)={x}x \in X^G \Longrightarrow \Orb_G(x)=\{x\}.
    In particular, if XX is finite then we have
    $$
    |X|=\left|X^G\right| + \sum_{i=1}^r \left|\Orb{}_{G}(x_i)\right|,
    $$
    where the sum runs over those orbits whose size are greater than one. This formula is called the class equation of the group action. \label{classeq}
\end{definition}

\begin{lemma}
    Let GG be a group of order pnp^n for some prime pp and n≥1n \geq 1. If GG acts on a finite set XX, then
    $$
        \left|X^G\right| \equiv|X| \bmod p. \label{usefullemma}
    $$
\end{lemma}
\begin{proof}
    Since $G$ is a finite group, by the orbit-stabiliser theorem $|\Orb_G(x)|$ divides $|G| = p^n$ for all $x$. In particular, if $\Orb_G(x_i)$ is an orbit with $|\Orb_G(x_i)| > 1$ then we must have $|\Orb_G(x_i)| = p^k$ for some $1 \leq k \leq n$. Hence the result follows from considering the class equation of the group action (\ref{classeq}):
    $$ 
    |X|-\left|X^G\right| =\underbrace{\sum_{i=1}^r \left|\Orb(x_i)\right|}_{\textrm{multiple of pp}}
    \implies \left|X^G\right| \equiv |X| \bmod p.
    $$
\end{proof}

\begin{theorem}[Cauchy's Theorem]
  Let $G$ be a finite group and $p$ be a prime divisor of $|G|$. Then $G$ contains an element of order $p.$
  \label{thm:cauchy}
\end{theorem}
\begin{proof}
    Consider the set
    $$
    X=\left\{\left(x_1, \ldots, x_p\right) \in G^p \mid x_1 \cdots x_p=1\right\}
    $$
    of $p$-tuples of elements $x_i$ of $G$ whose product is the identity. Notice that such a $p$-tuple is uniquely determined by $p-1$ of its components. Indeed, if $x_1, \ldots, x_{p-1}$ is an arbitrary collection of elements in $G$ then $x_p$ is forced to be $x_p=\left(x_1 \cdots x_{p-1}\right)^{-1}$. Thus, we see that $X$ has $|G|^{p-1}$ elements and hence $|X|$ is divisible by $p$ (as $p$ is a divisor of $|G|$).
    
    Now observe that the cyclic group $\mathbb{Z}_p=\langle\sigma\rangle$, for $\sigma=(123 \cdots p) \in S_p$, acts on $X$ by
    $$
    \sigma \cdot\left(x_1, \ldots, x_p\right)=\left(x_{\sigma(1)}, \ldots, x_{\sigma(p)}\right)=\left(x_2, \ldots, x_p, x_1\right)
    $$
    The RHS does indeed remain in $X$
    since $x_1 \cdots x_p=1$ implies $x_1^{-1}=x_2 \cdots x_p$ and hence $x_2 \cdots x_p \cdot x_1=1$. Moreover, we can apply Lemma \ref{usefullemma} to this action to conclude that $\left|X^{\mathbb{Z}_p}\right| \equiv 0 \bmod p$. 
    The set of fixed points of this action is given by
    % $$
    % X^{\mathbb{Z}_p}=\{(x, \ldots, x) \in X \mid x \in G\}.
    % $$
    $$
    X^{\mathbb{Z}_p} = \{(x, \ldots, x) \in G^p \mid x\cdots x = 1\}
    $$
    since $\sigma$ fixes $\left(x_1, \ldots, x_p\right)$ iff $x_1=x_2=\cdots=x_p$. This set is non-empty, since $(1, \ldots, 1) \in X^{\mathbb{Z}_p}$, and so $\left|X^{\mathbb{Z}_p}\right| \geq p$. 

    This implies that there exists an $x \in G$ with $x \neq 1$ such that $(x, \ldots, x) \in X^{\mathbb{Z}_p}$; that is, $x^p=1$.
\end{proof}

\begin{theorem}
  Let $G$ be a finite group and $p$ be the smallest prime dividing $|G|$. Let $H$ be a
  subgroup of index $p$. Then $H$ is normal in $G$.
  \label{thm:cauchyGeneral}
\end{theorem}
\begin{proof}
    Consider the action of $H$ on the set of left cosets $G/H$. By the orbit-stabiliser theorem, the size of every orbit of cosets divides $|H|$, and hence also divides $|G|$. Since there are exactly $p$ elements of $G/H$, any orbit must simultaneously divide $|G|$ and have cardinality at most $p$, so either we have a single orbit of size $p$ or there are $p$ different orbits of size 1, since $p$ is the smallest prime divisor of $|G|$.
    
    
    
    % cosets of $H$, and $p$ is the smallest prime dividing $|G|$, we must have that there is either a single orbit of size $p$ or there are $p$ different orbits of size 1.
    
    Clearly the first option is impossible, since $H \in G/H$ is a fixed point under the action; 
    % for every $h \in H, hH = H$. Which means the action fixes the coset corresponding to the identity. 
    there is an orbit of size 1, so they must all be of size 1. This means that all of our cosets are fixed points, and for every $h \in H,\text{ } g \in G,$ we have $h g^{-1} H=g^{-1} H.$ So, $\exists h' \in H \text { s.t. } h g^{-1}=g^{-1} h',$ and hence $g h g^{-1}=h' \in H,$ so $H$ is normal in $G.$
    % $$
    %     \begin{gathered}
    %     h g^{-1} H=g^{-1} H \Longrightarrow \exists h' \in H \text { s.t. } h g^{-1}=g^{-1} h' \Longrightarrow \\
    %     g h g^{-1}=h' \in H.
    %     \end{gathered}
    % $$
\end{proof}
\begin{remark}
    We are already familiar with this result for index 2 subgroups by Theorem \ref{thm:index2}. We should 
    note that a subgroup of index $p$ is not guaranteed to exist.
\end{remark}
