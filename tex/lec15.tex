\section{Lecture 15 - 25 Oct 2021}
\subsection{Applications of Orbit-Stabiliser Theorem and Cauchy's Theorem}
Orbit-Stabiliser theorem and applications.
Note that stabilizers of different points in $X$ will be the same if and only if the
subgroup $H$ is normal. Recall the theorem from last lecture

\begin{theorem}
  Let $G$ be a group and let $X$ be a $G$-set. Let $x\in X$. Then
  \begin{enumerate}
    \item The $G$-orbit of $x$ is a transitive $G$-set (14.2.1)
    \item The stabiliser of $x$ in $G$ is a subgroup of $G$ (14.2.2)
    \item Every transitive $G$-set is isomorphic to some $G/H$, for $H\leq G$. (Consequence of Orbit-Stabiliser)
    \item The $G$-sets $G/H$ and $G/K$ are isomorphic if and only if the subgroups are
      conjugate. (14.6)
  \end{enumerate}
\end{theorem}
Note that even though the $G$-set is transitive, the stabiliser of every point may indeed be
different sets.

\begin{definition}
    We write $X^G:=\{x \in X: g \cdot x=x$ for all $g \in G\} \subset X$ to denote the set of fixed points of the action then $x \in X^G \Longrightarrow \Orb_G(x)=\{x\}$. In particular, if $X$ is finite then we have
    \begin{center}
    $
    |X|=\left|X^G\right|+\sum_{i=1}^r\left|\Orb_{G}(x)\right|
    $
    \end{center}
    where the sum runs over those orbits whose size are greater than one. This last formula is called the class equation of the group action. \label{classeq}
\end{definition}

\begin{lem}[The Useful Lemma]
    Let $G$ be a group of order $p^n$ for some prime $p$ and $n \geq 1$. If $G$ acts on a finite set $X$ then
    $$
        \left|X^G\right| \equiv|X| \bmod p. \label{usefullemma}
    $$
\end{lem}
\begin{proof}
    Since $G$ is a finite group, the orbit-stabiliser theorem shows that $|\Orb_G(x)|$ divides $|G| = p^n$ for all $x$. In particular, if $\Orb_G(x_i)$ is an orbit with $|\Orb_G(x_i)| > 1$ then we must have $|\Orb_G(x_i)| = p^k$ for some $1 \leq k \leq n$. This implies that p divides $|\Orb_G(x_i)|$ for all $i$. Hence the result follows from considering the class equation of the group action (\ref{classeq}).
    \begin{center}
    $
    |X|=\left|X^G\right|+\underbrace{\sum_{i=1}^r\left|\Orb(x)\right|}_{\textrm{multiple of $p$}}
    \implies \left|X^G\right| \equiv |X| \bmod p.
    $
    \end{center}
\end{proof}

\begin{theorem}[Cauchy's Theorem]
  Let $G$ be a finite group and $p$ be a prime divisor of $|G|$. Then $G$ contains an
  element of order $p$
  \label{thm:cauchy}
\end{theorem}
\begin{proof}
    Consider the set
    $$
    X=\left\{\left(x_1, \ldots, x_p\right) \in G^p \mid x_1 \cdots x_p=1\right\}
    $$
    of $p$-tuples of elements $x_i$ of $G$ whose product is the identity. Notice that such a $p$-tuple is uniquely determined by $p-1$ of its components. Indeed, if $x_1, \ldots, x_{p-1}$ is an arbitrary collection of elements in $G$ then $x_p$ is forced to be $x_p=\left(x_1 \cdots x_{p-1}\right)^{-1}$. Thus, we see that $X$ has $|G|^{p-1}$ elements and hence $|X|$ is divisible by $p$. (As $p$ is a prime divisor of $|G|$)
    
    Now observe that the cyclic group $\mathbb{Z}_p=\langle\sigma\rangle$, for $\sigma=(123 \cdots p) \in S_p$, acts on $X$ by
    $$
    \sigma \cdot\left(x_1, \ldots, x_p\right)=\left(x_{\sigma(1)}, \ldots, x_{\sigma(p)}\right)=\left(x_2, \ldots, x_p, x_1\right)
    $$
    since $x_1 \cdots x_p=1$ implies $x_1^{-1}=x_2 \cdots x_p$ and hence $x_2 \cdots x_p \cdot x_1=1$. Moreover, we can apply Lemma \ref{usefullemma} to this action to get $\left|X^{\mathbb{Z}_p}\right| \equiv 0 \bmod p$. However, the set of fixed points of this action is given by
    $$
    X^{\mathbb{Z}_p}=\{(x, \ldots, x) \in X \mid x \in G\}
    $$
    since $\sigma$ fixes $\left(x_1, \ldots, x_p\right)$ iff $x_1=x_2=\cdots=x_p$. This set is non-empty, since $(1, \ldots, 1) \in X^{\mathbb{Z}_p}$, and so $\left|X^{\mathbb{Z}_p}\right| \geq p$. This implies that there exists an $x \in G$ with $x \neq 1$ such that $(x, \ldots, x) \in X^{\mathbb{Z}_p}$; that is, $x^p=1$.
  % Consider $X=\left\{ (g_1,\cdots,g_p) \in G^p : \Pi g_i =e\right\}$. Let $C_p =\langle
  % \sigma \rangle$ be a cyclic group of order $p$. Let $C_p$ act on $X$ by $\sigma
  % (g_1,\cdots,g_p)\mapsto (g_p,g_1\cdots, g_{p-1})$. Repeating this $p$ times, we get back
  % the identity. We still need to verify that $(g_p,g_{1},\cdots, g_{p-1})\in X$ since we
  % don't know if $G$ is abelian. However, indeed we have 
  % \[g_p g_1 \cdots g_{p-1} = g_p(g_1\cdots g_p)g_p^{-1}= g_p e g_p^{-1}=e\]
  % Hence we have $(g_{p}, g_{1}, \cdots, g_{p-1})\in X$. Hence the action $C_p$ on $X$ is
  % well defined.

  % By Orbit-Stabiliser theorem we have that $|\Orb(x)|$ must divide the size of the group,
  % hence every orbit must be either $p$ or $1$. Then we have that $X$ is a disjoint union of orbits
  % of size $1$ and size $p$,
  % \[|X|= 1\#\{x\in X | \sigma^i x=x \forall i\in\ZZ\} + p \#\{\text{orbits of size $p$}\}\]
  % Note that evey orbit of size $p$ will have $p$ elements, and that's why we multiply by
  % $p$ above.

  % We claim that $|X|=|G|^{p-1}$. Note that we can choose
  % $g_1,\cdots,g_{p-1}$ arbitarily, but $g_p$ is forced to be the inverse of the product up
  % to $g_{p-1}$. In particular, $p\Big | |X|$, and so $p$ divides the number of orbits of
  % size 1.  Then by the above expression, we know that there must be either $0$ or
  % $pk\exists k\in\NN$ orbits of size $1$. Let us describe this set of sets by $Y$ (set of
  % orbits of size 1).  Notice that there can't be $0$ orbits of size $1$, since
  % $(e,e,\cdots, e)$ is already of size $1$.
  % \todo{Shouldn't there be more elements, since the number of orbits of size 1 should
  % divide $p$. In particular, shouldn't there be at least $p$ such elements?}
  % Hence, there must be at least one more element, say $(g_1,\cdots, g_p)$ such that
  % $\sigma (g_1,\cdots,g_p)= (g_p, g_1, \cdots, g_{p-1})= (g_1,\cdot,g_p)$ since this
  % element must have order $1$. Then it implies that $g_1=\cdots=g_p$. However, we have
  % that it must also be in $X$, hence $g_1g_2\cdots g_p=g_1^p=e$.
\end{proof}

\begin{theorem}
  Let $G$ be a finite group and $p$ be the smallest prime dividing $|G|$. Let $H$ be a
  subgroup of index $p$. Then $H$ is normal in $G$.
  \label{thm:cauchyGeneral}
\end{theorem}
\begin{proof}
    Consider the action of $H$ on the set of left cosets $G/H$. By the orbit-stabiliser theorem the size of every orbit of cosets divides $|H|$ and hence also divides $|G|$. Since there are exactly $p$ cosets of $H$ and $p$ is the smallest prime dividing $|G|$, we must have that there is either a single orbit of size $p$ or there are $p$ different orbits of size 1.
    Clearly the first option is impossible, since for every $h \in H, hH = H$. Which means the action fixes the coset corresponding to the identity. Hence there is an orbit of size 1, so they must all be of size 1.
    This means that for every $h \in H, g \in G$ we have
    $$
        \begin{gathered}
        h g^{-1} H=g^{-1} H \Longrightarrow \exists h_1 \in H \text { s.t. } h g^{-1}=g^{-1} h_1 \Longrightarrow \\
        g h g^{-1}=h_1 \in H
        \end{gathered}
    $$
\end{proof}
\begin{remark}
    Note that a subgroup of index $p$ is not guaranteed to exist.
\end{remark}
