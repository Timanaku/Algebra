\section{Lecture 14}
\subsection{Orbit-Stabiliser Theorem}
% Orbit-stabilizer theorem.
\begin{definition}[Orbit and Stabiliser]
  Let $G$ be a group and $X$ be a $G$-set. Let $x\in X$. The \emph{orbit} of $x$ under the action of $G$ is
  $\Orb_G (x)= G\cdot x = \left\{ g\cdot x \mid g\in G\right\}$.
  The \emph{stabiliser} of $x$ in $G$ is $\Stab_G (x)=\left\{ g\in G \mid g\cdot x=x \right\}.$
  \label{def:orbStab}
\end{definition}


\begin{theorem}
  Let $G$ be a group and $X$ be a $G$-set, and let $x\in X.$ Then 
  \begin{enumerate}
    \item The $G$-orbit of $x$ is a transitive $G$-set,
    \item The stabiliser of $x$ is a subgroup of $G$.
  \end{enumerate}
  \label{<+label+>}
\end{theorem}
\begin{proof}
   (1) We have to show that $\Orb_G(x)$ is indeed a $G$-set and that it is
  transitive. Let $\tilde{x}\in G\cdot x,$ then $\tilde{x}=g\cdot x$ for some $g\in G.$ We now check the two axioms for a group action.
  First,
  $$1_G\cdot \tilde{x}=1_G\cdot (g\cdot x)=(1_G\cdot g)\cdot x=g\cdot x=\tilde{x},$$
  where the equalities follow from $X$ being a $G$-set.

  Now for the second axiom, let $g_1,g_2\in G.$ We check,
  $$g_1\cdot (g_2 \cdot \tilde{x})=g_1\cdot (g_2\cdot g\cdot x)=(g_1\cdot g_2)\cdot g\cdot x=(g_1\cdot g_2)\cdot \tilde{x},$$
  as required.
  
  %To show that it is a $G$-set, note that $\forall g\in G, y \in\Orb_G(x)$ we
  %have $g\cdot y = g\cdot (h\cdot x)$ for some $h\in G$.
  
  %By Definition
  %\ref{def:leftAction} it follows that $g\cdot y = (g\cdot h)\cdot x \in\Orb_G(x)$, as
  %required. 
  Moreover, we claim that the $G$-orbit of $x$ is transitive, that is, for all $\tilde{x},\tilde{y} \in G \cdot x,$ there exists some $g\in G$ such that $\tilde{y}=g\cdot \tilde{x}.$ We have $\tilde{x}=g_1\cdot x,\text{ } \tilde{y}=g_2\cdot x$ for some $g_1,\text{ }g_2\in G$. It follows that 
  $$\tilde{y}=g_2\cdot x = g_2 \cdot (g_{1}^{-1} \cdot (g_1 \cdot x)) = (g_2g_1^{-1}) \cdot \tilde{x},$$ 
  as required.

   (2) We now show that the stabiliser is indeed a subgroup. We clearly have $1_G\in\Stab_G(x),$ so the set is non-empty. Let $g_1,g_2\in\Stab_G(x)$, so $g_1\cdot x=g_2\cdot x=x.$ 
   From the first equality we have $(g_2^{-1} g_1)\cdot x = x,$ so $g_2^{-1} g_1 \in \Stab_G(x)$, meaning $\Stab_G(x)$ is a subgroup by the subgroup test.
\end{proof}


\begin{theorem}[Orbit-Stabiliser Theorem]
  Let $G$ be a group and $X$ be a $G$-set. Let $x\in X$. Then there is an isomorphism of $G$-sets given by
  \begin{align*}
      \phi:G/\Stab{}_{G}(x)&\to \Orb{}_G(x)
      \\ g\Stab{}_G(x)&\mapsto g\cdot x.
  \end{align*}
  \label{thm:orbStab}
\end{theorem}
\begin{proof}
    %well-defined, injective, surjective, G-set iso
    We must check that $\phi$ is well defined and a $G$-set isomorphism.

    Let $g,g' \in G$ and suppose $g\Stab_{G}(x) = g'\Stab_{G}(x)$. Then $g^{-1}g'\in \Stab_{G}(x),$ so $(g^{-1}g')\cdot x = x.$ Acting on the left by $g$, we see that $$g'\cdot x = (g g^{-1} g')\cdot x = g \cdot (g^{-1} g')\cdot x = g \cdot x,$$ so $\phi$ is well defined.

    Now for injectivity, suppose $\phi(g\Stab_G(x)) = \phi(g'\Stab_G(x))$. This means $g \cdot x = g' \cdot x,$ and we can apply the above logic in reverse to conclude that $g\Stab_{G}(x) = g'\Stab_{G}(x),$ so $\phi$ is injective.

    Let $y \in \Orb_G(x),$ then $\exists g \in G$ s.t. $y = g \cdot x.$ Then $\phi(g\Stab_G(x)) = g \cdot x = y,$ so $\phi$ is surjective.

    Finally, to prove $\phi$ is a $G$-set isomorphism, let $g,h \in G$. We want to show that $g \cdot (\phi(h\Stab_G(x))) = \phi(g \cdot (h\Stab_G(x))).$

    We have that
    \begin{align*}
        g \cdot (\phi(h\Stab{}_G(x))) &= g \cdot (h \cdot x) \\
        &= (gh)\cdot x \\
        &= \phi(gh\Stab{}_G(x)) \\
        &= \phi(g \cdot (h\Stab{}_G(x))),
    \end{align*}
    as required.
  % Let $g, g' \in G$. We first show that $\phi$ is well defined (doesn't depend on coset representatives) and
  % then show how it's a bijective map with $\phi(g\cdot x)=g\cdot \phi(x)$. We show
  % well-definedness and injectivity in one statement. So
  % \begin{center}
  % $
  % gH=g'\Stab_G(x) \iff g'^{-1}g\in \Stab_G(x) \iff (g'^{-1}g) \cdot x = x \iff g\cdot x = g\cdot x.
  % $
  % \end{center}
  % Note that surjectivity is clear by the definition of $\Orb_G(x)$. Hence $\phi$ is a
  % bijection. Next, observe that $\phi(g \cdot g'\Stab_G(x))= (gg')\cdot x = g \cdot g' \cdot x = g \cdot \phi(g'\Stab_G(x))$. So we have that $\phi$ is a $G$-Set isomorphism as required.
\end{proof}


\begin{corollary}
  Let $G$ be a group and $X$ be a $G$-set. Let $x\in X$. We have $|\Orb_{G}(x)|=
  [G:\Stab_G(x)]$. In particular, if $G$ is finite, then $|\Orb_{G}(x)|=|G|/|\Stab_G(x)|$,
  by Lagrange, and the size of every orbit divides $|G|$.
  \label{cor:orbStab}
\end{corollary}

\begin{proof}
    Let $\phi$ be as in the statement of Theorem \ref{thm:orbStab}. Since $\phi$ is a bijection, we have $|\Orb_G(x)| = |G/\Stab{}_{G}(x)| = [G:\Stab_G(x)]$. If $G$ is finite Lagrange's Theorem says $|G| = |\Stab{}_{G}(x)|\cdot |\Orb_G(x)|,$ so the size of the orbit of $x$ divides $|G|.$
\end{proof}


\begin{theorem}
  Let $G$ be a group and $X$ be a $G$-set. Then lying in the same orbit is an equivalence
  relation on $X$. In particular, $X$ is a union of disjoint orbits (equivalence classes).
  \label{thm:eqRelOrb}
\end{theorem}
\begin{proof}
    We have that $x \sim y \iff \exists g\in G$ s.t. $g\cdot x = y.$ 

    The relation is reflexive since $1_G \cdot x = x.$ 
    
    If $x \sim y,$ then $\exists g\in G$ s.t. $g \cdot x = y \implies g^{-1} \cdot y = x,$ so $y \sim x$ and the relation is symmetric.

    If $x \sim y$ and $y \sim z,$ then $\exists g,g' \in G$ s.t. $g \cdot x = y \text{ and } g'\cdot y = z.$ Then $((g'g) \cdot x) = z,$ so $x \sim z$ and the relation is transitive.
    % Proving reflexivity, symmetry and transitivity in that order we have,
    % for any $x \in X$ the identity element $1_G \in G$ satisfies $1_G \cdot x = x$ so $x$ is in the same orbit as itself thus $x \sim x$.
    % For $x,y \in X$, if $x \sim y$ then there exists a $g \in G$ such that $g \cdot x = y$ so we have $g^{-1} \cdot g \cdot x = g^{-1} \cdot y \implies g^{-1} \cdot y = x$ so $y \sim x$.
    % Suppose $x \sim y$ and $y \sim z$. This means there exists a $g_1, g_2 \in G$ such that $g_1 \cdot x = y$ and $g_2 \cdot y = z$. So we have $g_2 \cdot g_1 \cdot x = z$. So $x \sim z$.
\end{proof}


\begin{theorem}
  Let $G$ be a group, and let $X$ be a transitive $G$-set. Then any two point stabilisers are conjugate in $G$. That is, for any $x,y\in X$ there exists $g\in G$ with
  $\Stab_G(x)=g\Stab_G(y)g^{-1}$.
  \label{conjugatestab}
\end{theorem}
\begin{proof}
  Let $x,y \in X$. Since $X$ is transitive, there is a $g\in G$ with $x=g \cdot y$. We prove this by double inclusion.
  
 Let $h\in\Stab_G(x)$, we want to show that $h\in g\Stab{}_G(y)g^{-1}$.
  
  % ghg^{-1} \in \Stab_G(y).$ 
  Since $h \in \Stab_G(x)$, $h\cdot x=x=g\cdot y,$ so we have
  $$
  h \cdot x = h \cdot (g \cdot y) = (hg) \cdot y = (g \cdot y) \implies (g^{-1}hg) \cdot y = y  
  $$
  and so $h \in g\Stab{}_G(y)g^{-1}$ hence $\Stab{}_G(x) \subseteq g\Stab{}_G(y)g^{-1}.$

  
  % \begin{align*}
  %    h \cdot x = x &\implies h \cdot (g \cdot y) = g \cdot y \\
  %     &\implies (g^{-1}hg)\cdot y = y \\
  %     &\implies g^{-1}hg \in \Stab{}_G(y) \\
  %     &\implies \Stab{}_G(x) \subseteq g\Stab{}_G(y)g^{-1}.
  % \end{align*}

  % Now, given $h' \in \Stab{}_G(y),$ we aim to show that $gh'g^{-1} \in \Stab{}_G(x).$ %ghg^{-1}.x = x
  % We have that 
  % $$
  % (ghg^{-1})\cdot x = g\cdot (h' \cdot x) = g \cdot y
  % $$
    % and observe
  % \[h \cdot x=x=h\cdot g\cdot y=g\cdot y\]
  % \[\iff (g^{-1}hg)\cdot y = y\]
  % \begin{center}
  %     $\iff g^{-1}hg\in\Stab_G(y)$
  % \end{center}
  % \begin{center}
  %     $\iff h\in g\Stab_G(y)g^{-1}$.
  % \end{center}
  % Hence $\Stab_G(x)=g\Stab_G(y)g^{-1}$, as required.

 Now, given $h' \in g\Stab{}_G(y)g^{-1}$, we aim to show that $h' \in \Stab{}_G(x)$.

We have that there exists some $h \in \Stab{}_G(y)$ such that $h' = ghg^{-1}$. To show that $h'$ stabilises $x$, consider
$$
h' \cdot x = (ghg^{-1}) \cdot x = (gh) \cdot (g^{-1} \cdot x) = (gh) \cdot y = g \cdot y = x
$$
and so $h' \in \Stab{}_G(x)$ and  $g\Stab{}_G(y)g^{-1} \subseteq \Stab{}_G(x).$ Hence $\Stab_G(x)=g\Stab_G(y)g^{-1}$, as required.
\end{proof}

All transitive $G$-sets look like sets of left cosets, $G/H$ for a suitable $H$. What $H$?
Given a transitive $G$-set say $X$, then the subgroup it corresponds to (the conjugacy
class of subgroups really) is the conjugacy class of point stabilisers.
Note that if you take two different points in $X$, the stabiliser of each point are
conjugate. Two conjugate subgroups give isomorphic $G$-sets, and hence a bijection arises for
transitive $G$-sets and conjugate classes of subgroups of $G$. This leads us to the following corollary and theorem where we prove these notions formally. This will become clearer after you read the following proofs.

\begin{corollary}[Consequence of Orbit Stabiliser]
    Let $G$ be a group and let $X$ be a transitive $G$-set. Then
    $X$ is isomorphic to $G/H$, for some $H\leq G$.
\end{corollary}

\begin{proof}
    Let $x\in X.$ As $X$ is transitive, $\Orb{}_G(x) = X$, and so by the Orbit Stabiliser theorem we have $X \cong G/\Stab{}_{G}(x)$, where $ \Stab{}_{G}(x) = H \leq G.$ 
\end{proof}
    % We want to show that $G/\Stab{}_{G}(x)$ is independent of our choice of $x$. For any $x' \in X$ we have by theorem \ref{conjugatestab} any two point stabilisers are conjugate, and thus by theorem \ref{thm:cosetsIsomorphic} $G/\Stab{}_{G}(x) \cong G/\Stab{}_{G}(x')$. So taking $\Stab{}_{G}(x) = H$ gives the result.
\begin{remark}
    In the above proof, we can take $H=\Stab{}_{G}(x)$ for any $x\in X.$ This makes sense since by Theorem \ref{conjugatestab} we have that any two point stabilisers are conjugate, and by Theorem \ref{thm:cosetsIsomorphic}, for any $x' \in X,$ $G/\Stab{}_{G}(x) \cong G/\Stab{}_{G}(x').$
\end{remark}

In general, even if a $G$-set $X$ is transitive, its points may have different stabilisers. In fact, stabilisers of different points in $X$ will be the same if and only if the subgroup $\Stab{}_G(x)$ is normal in $G.$ This can easily be shown.

\begin{corollary}
    Let $X$ be a transitive $G$-set and let $x \in X$. Then $\Stab{}_G(x)$ is normal in $G$ if and only if $\Stab{}_G(x)=\Stab{}_G(y)$ for all $y\in X.$
\end{corollary}
\begin{proof}
    First, assume that $\Stab{}_G(x)$ is normal in $G,$ and let $y\in X.$ By Theorem \ref{conjugatestab}, there exists $g\in G$ such that $\Stab{}_G(y)=g\Stab{}_G(x)g^{-1},$ and since $\Stab{}_G(x)$ is closed under conjugation, $\Stab{}_G(y)=\Stab{}_G(x),$ as required.

Now assume that $\Stab{}_G(x)=\Stab{}_G(y)$ for all $x,y\in X.$ Let $h\in \Stab{}_G(x).$ Then $h\cdot x=x,$ and since $X$ is transitive there exists $g\in G$ such that $x=g\cdot y.$
We then have $$h\cdot g\cdot y=g\cdot y\implies g^{-1}hg\cdot y=y,$$ so $g^{-1}hg\in \Stab{}_G(y)=\Stab{}_G(x),$ or equivalently $h\in g\Stab{}_G(x)g^{-1}.$ So we have $\Stab{}_G(x)\subseteq g\Stab{}_G(x)g^{-1}.$ Since $y$ and therefore $g$ is arbitrary, the previous statement is true for all $g\in G,$ so $\Stab{}_G(x)$ is normal in $G.$
\end{proof}








% Recall that since orbits define equivalence relations, we can define equivalence classes,
% and since the orbit of $x$ is isomorphic to the set of left cosets of the stabilizer of
% $x$ in $G$, $G/\Stab_G(x)$, we can find equivalence classes also in $G/\Stab_G(x)$
\begin{theorem}
  Let $G$ be a group. Then there is a bijection between conjugacy classes of subgroups of
  $G$ and isomorphism classes of transitive $G$-sets.
  \label{<+label+>}
\end{theorem}
% \begin{proof}
%   We first claim that the assignment from subgroups $H$ of $G$ to the set of left cosets
%   $G/H$ is well defined, i.e. if I have two subgroups $H,K$ that are conjugate, then they
%   will map to the same isomorphism class of transitive $G$-sets. By Theorem \ref{thm:cosetsIsomorphic} if $H,K$
%   are conjugate, we have that the $G$-sets $G/H$ and $G/K$ are isomorphic. Hence this is
%   given.

%   Moreover, if $X$ is a transitive $G$-set, then $\Stab_G(x)$ for any $x\in X$ is a
%   well-defined conjugacy class (independent of $x$) by Theorem \ref{conjugatestab}.
   
%   Finally we claim the two assignments above are inverses of each other. Fix $H\leq
%   G$ and consider the map $\phi:H\mapsto G/H$. Note how the map
%   $\psi:G/H\mapsto\Stab_G(1\cdot H)$ is $\psi=\phi^{-1}$, since $g\cdot 1H=H \iff g\in H$.
%   Conversely, consider the map from a transitive $G$-set $X\mapsto \Stab(x)$, which has
%   inverse $\Stab(x)\mapsto G/\Stab(x)$ where by the Orbit-Stabilizer theorem we have
%   $G/\Stab(x)\cong X$.
% \end{proof}
\begin{proof}
Let $\mathcal{C}$ denote the set of conjugacy classes of subgroups of
$G$ and $\mathcal{I}$ denote the set of isomorphism classes of transitive $G$-Sets.
Define the map $\varphi: \mathcal{C} \rightarrow \mathcal{I}$ by sending the conjugacy class of a subgroup $H$ to the isomorphism class of the transitive $G$-set $G/H$. Specifically, for a subgroup $H \leq G$, $\varphi([H]) = [G/H]$.

Conversely, define the map $\psi: \mathcal{I} \rightarrow \mathcal{C}$ by sending the isomorphism class of a transitive $G$-set $X$ to the conjugacy class of the stabiliser of any point $x \in X$. That is, for a transitive $G$-set $X$, $\psi([X]) = [\Stab_G(x)]$.

Note that these maps are well defined by Theorem \ref{thm:cosetsIsomorphic} and \ref{conjugatestab} respectively.

To show that $\varphi$ and $\psi$ are inverses, we prove that $\varphi \circ \psi = \text{id}_{\mathcal{I}}$ and $\psi \circ \varphi = \text{id}_{\mathcal{C}}$:

For $\varphi \circ \psi = \text{id}_{\mathcal{I}}$,
let $[X] \in \mathcal{I}$. Then $\psi([X]) = [\Stab_G(x)]$ for some $x \in X$. Applying $\varphi$, we get $\varphi(\psi([X])) = \varphi([\Stab_G(x)]) = [G/\Stab_G(x)]$. Since $X$ is isomorphic to $G/\Stab_G(x)$ by the Orbit-Stabiliser Theorem, we have $\varphi(\psi([X])) = [X]$, proving that $\varphi \circ \psi$ is the identity on $\mathcal{I}$.

For $\psi \circ \varphi = \text{id}_{\mathcal{C}}$,
let $[H] \in \mathcal{C}$. Applying $\varphi$, we get $\varphi([H]) = [G/H]$. Then applying $\psi$, we have $\psi(\varphi([H])) = \psi([G/H]) = [\Stab_G(H)]$. By definition, the stabiliser of $H$ in $G/H$ is $H$ itself, so $\psi(\varphi([H])) = [H]$, showing that $\psi \circ \varphi$ is the identity on $\mathcal{C}$.

Therefore, $\varphi$ and $\psi$ are inverse bijections, establishing the claimed correspondence.
\end{proof}
