\section{Lecture 14 - 22 Oct 2021}
\subsection{Orbit-Stabiliser Theorem}
Orbit-stabilizer theorem.
\begin{definition}[Orbit and Stabiliser]
  Let $G$ be a group and $X$ be a $G$-set. Let $x\in X$. The orbit of $x$ in $G$ is
  $\Orb_G (x)= G\cdot x = \left\{ g\cdot x :g\in G\right\}$.
  The stabiliser of $x$ in $G$ is $\Stab_G (x)=\left\{ g\in G | g\cdot x=x \right\}$
  \label{def:orbStab}
\end{definition}


\begin{theorem}
  Let $G$ be a group and $X$ be a $G$-set. Then 
  \begin{enumerate}
    \item The $G$-orbit of $x$ is a transitive $G$-set.
    \item The stabiliser of $x$ is a subgroup of $G$.
  \end{enumerate}
  \label{<+label+>}
\end{theorem}
\begin{proof}
  For the first claim, we have to show that $\Orb_G(x)$ is indeed a $G$-set and that it's
  transitive. To show that it's a $G$-set note that $\forall g\in G, y\in\Orb_G(x)$ we
  have $g\cdot y = g\cdot (h\cdot x)$ for some $h\in G$. By Definition
  \ref{def:leftAction} it follows that $g\cdot y = (g\cdot h)\cdot x \in\Orb_G(x)$, as
  required. Moreover, we claim it's transitive, that is, $\forall a,b\in\Orb_G(x) \, \exists h'\in G : b=h'\cdot a$. Note that $b=g\cdot x, a=g'\cdot x$ for some $g,g'\in G$, so it
  follows that $a=g' \cdot x = g' \cdot (g^{-1} \cdot (g \cdot x)) = (g'g^{-1}) \cdot b$ as required.

  For part 2 we show that the stabilizer is indeed a subgroup. Note that the identity
  $e_G\in\Stab_G(x)$, so the set is non-empty. Next, consider some $a,b\in\Stab_G(x)$, so
  $a\cdot x=x=b\cdot x \iff (b^{-1}a)\cdot x =x$, so $b^{-1}a\in\Stab_G(x)$, as required.
\end{proof}


\begin{theorem}[Orbit-Stabiliser Theorem]
  Let $G$ be a group and $X$ be a $G$-set. Let $x\in X$. Then there is an isomorphism of
  $G$-sets as $\phi:G/\Stab_{G}(x)\to \Orb(x):g\Stab_G(x)\mapsto g\cdot x$.
  \label{thm:orbStab}
\end{theorem}
\begin{proof}
  Let $g, g' \in G$. We first show that $\phi$ is well defined (doesn't depend on coset representatives) and
  then show how it's a bijective map with $\phi(g\cdot x)=g\cdot \phi(x)$. We show
  well-definedness and injectivity in one statement. So
  \begin{center}
  $
  gH=g'\Stab_G(x) \iff g'^{-1}g\in \Stab_G(x) \iff (g'^{-1}g) \cdot x = x \iff g\cdot x = g\cdot x.
  $
  \end{center}
  Note that surjectivity is clear by the definition of $\Orb_G(x)$. Hence $\phi$ is a
  bijection. Next, observe that $\phi(g \cdot g'\Stab_G(x))= (gg')\cdot x = g \cdot g' \cdot x = g \cdot \phi(g'\Stab_G(x))$. So we have that $\phi$ is a $G$-Set isomorphism as required.
\end{proof}


\begin{corollary}
  Let $G$ be a group and $X$ be a $G$-set. Let $x\in X$. We have $|\Orb_{G}(x)|=
  [G:\Stab_G(x)]$. In particular, if $G$ is finite, then $|\Orb_{G}(x)|=|G|/|\Stab_G(x)|$,
  by Lagrange, and the size of every orbit divides $|G|$.
  \label{cor:orbStab}
\end{corollary}


\begin{theorem}
  Let $G$ be a group and $X$ be a $G$-set. Then lying in the same orbit is an equivalence
  relation on $X$. In particular, $X$ is a union of disjoint orbits (equivalence classes).
  \label{thm:eqRelOrb}
\end{theorem}
\begin{proof}
    Proving reflexivity, symmetry and transitivity in that order we have,
    for any $x \in X$ the identity element $1_G \in G$ satisfies $1_G \cdot x = x$ so $x$ is in the same orbit as itself thus $x \sim x$.
    For $x,y \in X$, if $x \sim y$ then there exists a $g \in G$ such that $g \cdot x = y$ So we have $g^{-1} \cdot g \cdot x = g^{-1} \cdot y \implies g^{-1} \cdot y = x$ so $y \sim x$.
    Suppose $x \sim y$ and $y \sim z$. This means there exists a $g_1, g_2 \in G$ such that $g_1 \cdot x = y$ and $g_2 \cdot y = z$. So we have $g_2 \cdot g_1 \cdot x = z$. So $x \sim z$.
\end{proof}


\begin{theorem}
  Let $G$ be a group, and let $X$ be a transitive $G$-set. Then any two point stabilisers are
  conjugate in $G$. That is, for any $x,y\in X$ there exists $g\in G$ with
  $\Stab_G(x)=g\Stab_G(y)g^{-1}$.
  \label{conjugatestab}
\end{theorem}
\begin{proof}
  Let $x,y \in X$. Since $X$ is transitive, there is a $g\in G$ with $x=g \cdot y$. Let $h\in\Stab_G(x)$, and
  observe
  \[h \cdot x=x=h\cdot g\cdot y=g\cdot y\]
  \[\iff (g^{-1}hg)\cdot y = y\]
  \begin{center}
      $\iff g^{-1}hg\in\Stab_G(y)$
  \end{center}
  \begin{center}
      $\iff h\in g\Stab_G(y)g^{-1}$.
  \end{center}
  Hence $\Stab_G(x)=g\Stab_G(y)g^{-1}$, as required.
\end{proof}

Recall that since orbits define equivalence relations, we can define equivalence classes,
and since the orbit of $x$ is isomorphic to the set of left cosets of the stabilizer of
$x$ in $G$, $G/\Stab_G(x)$, we can find equivalence classes also in $G/\Stab_G(x)$
\begin{theorem}
  Let $G$ be a group. Then there is a bijection between conjugacy classes of subgroups of
  $G$ and isomorphism classes of transitive $G$-sets.
  \begin{enumerate}
    \item ($\rightarrow$) Given a subgroup $H\leq G$, assign to it the set $G/H$ of left
      cosets of $H$ in $G$ -- which will be a transitive $G$-set by Theorem \ref{thm:orbStab}.
    \item ($\leftarrow$) Given a transitive $G$-set $X$, assign to it $H=\Stab_G(x) \quad \forall
      x\in X$.
  \end{enumerate}
  \label{<+label+>}
\end{theorem}
\begin{proof}
  We first claim that the assignment from subgroups $H$ of $G$ to the set of left cosets
  $G/H$ is well defined, i.e. if I have two subgroups $H,K$ that are conjugate, then they
  will map to the same isomorphism class of transitive $G$-sets. By Theorem \ref{thm:cosetsIsomorphic} if $H,K$
  are conjugate, we have that the $G$-sets $G/H$ and $G/K$ are isomorphic. Hence this is
  given.

  More over, if $X$ is a transitive $G$-set, then $\Stab_G(x)$ for any $x\in X$ is a
  well-defined conjugacy class (independent of $x$) by \ref{conjugatestab}.
   
  Finally we claim the two assignments above are inverses of each other. Fix $H\leq
  G$ and consider the map $\phi:H\mapsto G/H$. Note how the map
  $\psi:G/H\mapsto\Stab_G(1\cdot H)$ is $\psi=\phi^{-1}$, since $g\cdot 1H=H \iff g\in H$.
  Conversely, consider the map from a transitive $G$-set $X\mapsto \Stab(x)$, which has
  inverse $\Stab(x)\mapsto G/\Stab(x)$ where by the Orbit-Stabilizer theorem we have
  $G/\Stab(x)\cong X$.
\end{proof}
All transitive $G$-sets look like sets of left cosets, $G/H$ for a suitable $H$. What $H$?
Given a transitive $G$-set say $X$, then the subgroup it corresponds to (the conjugacy
class of subgroups really) is the conjugacy class of point stabilizers.
Note that if you take two different points in $X$, the stabilizer of each point are
conjugate. Two conjugate subgroups give isomorphic $G$-sets, and hence a bijection arises for
transitive $G$-sets and conjugate classes of subgroups of $G$.
