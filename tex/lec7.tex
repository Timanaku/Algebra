\section{Lecture 7}
In this section we will focus on examples of quotient groups.

\begin{example}
  Let $G=S_n$ and $N=A_n$. There are exactly two left cosets of $A_n$ in $S_n$: $1A_n,
  (1,2)A_n$, the latter consisting of all odd permutations. Hence the quotient $S_n/A_n$
  is cyclic of order $2$.
\end{example}

\begin{example}
  Consider the group $S_4$. The subgroup $V_4 = \{e, (12)(34), (13)(24), (14)(23) \}$ is normal. The set $X=\left\{ e,(1,2),(1,3),(2,3),(1,2,3),(1,3,2) \right\}$ is a \emph{full set of left coset representatives of $V_4$ in $S_4$}, i.e. every coset of $V_4$ in $S_4$ contains exactly one element in $X$. This happens to be a subgroup of $S_4$, actually it's $S_3$. Hence we immediately identify the quotient group $S_4/V_4$ is isomorphic to $S_3$.
\end{example}

\begin{example}
  Consider the group $\ZZ$ and the normal subgroup $n\ZZ$ for $n\in\NN$. The quotient
  $\ZZ/n\ZZ=\left\{ 0+n\ZZ, 1+n\ZZ, \cdots, n-1+n\ZZ \right\}$ is cyclic of order $n$,
  generated by $1+n\ZZ$. This quotient group may be written as $C_n$.
\end{example}


