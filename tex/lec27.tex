\section{Lecture 27}
\subsection{Intermission: Classifying groups of order 21}
\begin{example}
  Let $G$ be a group of order 21. How many possibilities are there for $G$?
  %We claim that there are only two possibilities for $G$ up to isomorphism.
\end{example}
% \begin{proof}
%   Note that $21=7\cdot 3$, and by Cauchy, there are elements of $G$, say $g,h\in G$, with
%   order $7$ and $3$, say $g^7=e=h^3$. Let $K=\langle g\rangle$ and $H=\langle h\rangle$.
%   Note that $K\cap H=\{1\}$, since by Lagrange we have that $|K\cap H|$ divides $|K|$ and
%   $|H|$, and $\gcd(|H|,|K|)=1$, and the intersection is required to be a group itself.
%   Moreover, note that $K\trianglelefteq G$ since $[G:K]$ being the smallest prime divisor
%   of $G$ implies $K\trianglelefteq G$.

%   Next, let $h\in H$ and consider the automorphism $\phi_h:K\to K,$ $g^i\mapsto hg^ih^{-1}$
%   (it's rather easy to see that the map is well defined and surjective, by normality, it's
%   a homomorphism by the general argument of conjugation as a homomorphism, and it is an
%   isomorphism because the inverse is $\phi_{h^{-1}}$). Suppose that $hgh^{-1}=g^r$, and note
%   that the choice for $r$ will completely determine the automorphism (since $g$ generates
%   $K$). Note that $\phi_h^{j}(g)=h^jgh^{-j}=g^{r^j}$ and we require $j=3\implies h^3=e$
%   hence we require $g^{r^j}=g^1$. Hence we require $r^3 \equiv 1 \mod 7$, which has solutions
%   % which are obiously meoww
%   $r=1$, $r=2$, $r=4$. 

%   An alternate point of view is that, the function $H\to \Aut K:x\mapsto \phi_x$, where
%   $\Aut K$ is a group under composition and the identity is the identity automorphism, and
%   note that there are $|K|-1=6$ possible such automorphisms.  Moreover, we have $K$ is
%   cyclic of order $7$, i.e. $\Aut K \cong (\ZZ/7\ZZ)^{\times}$, its automorphism group is
%   group under composition. More generally we have that $\Aut(\ZZ/n\ZZ)\cong
%   (\ZZ/n\ZZ)^{\times}$.  In fact, note that when $n$ is prime, the unit groups before is
%   cyclic of order $n-1$. Note also that for a cyclic group of order $n$, every element of an
%   order dividing $n$ generates exactly one subgroup of that order. So $\Aut K$ being
%   a cyclic of order $6$ we have that there is exactly one subgroup of order $3$.

%   Hence we have the following possibilities. 
  
%   \emph{Case 1: $\phi_h(g)=g$} -- Then $gh=hg$. Note that the semi-direct product using
%   this conjugation rule gives $HK=G$ since they intersect trivially and the product of
%   their order is the order of the group, and the fact that the elements commute gives
%   raise to the fact that the group is cyclic with generator $gh$.
  
%   \emph{Case 2: $\phi_h(g)=g^2$} -- Non-commuting where the semi-direct product of $H$ and
%   $K$ using this conjugation rule gives $K\rtimes H\cong C_7\rtimes C_3$.
  
%   \emph{Case 3: $\phi_h(g)=g^4$} -- Note that this is the case for the previous
%   $\phi_h(g)=g^2$, but $\phi_h^2(g)=\phi_{h^{-1}}=g^4$ (since $K$ has order 7, $(\phi_h \circ \phi_{h^{-1}})(g) = g^8 = g$). These automorphisms define a faithful map $\varphi : C_3 \rightarrow \Aut(C_7)$ by 
%   \begin{align*}
%       & e \mapsto \phi_{e}(g) = g, \\
%       & a \mapsto \phi_{a}(g) = g^2, \\
%       & a^2 \mapsto \phi_{a^2}(g) = g^4.
%   \end{align*}
%   where $e,a \in C_3$ and $g \in C_7$. Which gives rise to the semi-direct product $C_7\rtimes_{\varphi} C_3$
% \end{proof}

\begin{proof}
We begin by noting that $21 = 7 \cdot 3$. According to Cauchy's theorem, there must exist elements in $G$, say $g, h \in G$, of orders 7 and 3 respectively, such that $g^7 = e = h^3$. Define two subgroups: $K = \langle g \rangle$ and $H = \langle h \rangle$.

Observe that $K \cap H = \{1\}$, as by Lagrange's theorem, we have the size of $|K \cap H|$ divides both $|K|$ and $|H|$. Since $\gcd(|H|, |K|) = 1$, the intersection must be trivial. Furthermore, $K$ is normal in $G$, as $[G:K]$ is the smallest prime divisor of $|G|$.

Consider a mapping from $H$ to $\Aut(K)$ given by $h \mapsto \phi_h$. We aim to classify how many mappings we have between $H$ and $\Aut(K)$ (equivalently how many options we have for $\phi_h$) as this will tell us how many distinct groups we can construct via a semi-direct product of $K$ and $H$. Since $K$ is cyclic of order 7, $\Aut(K) \cong (\mathbb{Z}/7\mathbb{Z})^{\times}$, which is cyclic of order 6. One can verify there is exactly 2 mappings from $H \cong \ZZ/3\ZZ$ to $\Aut(K) \cong \ZZ/6\ZZ$, which define our semi-direct products. So we conclude there are 2 groups of order 21. 

Now we aim to find explicitly what these 2 mappings are. Consider an automorphism $\phi_h$ of $K$ defined by $\phi_h(g^i) = hg^ih^{-1}$. This map is an automorphism because it is well-defined, surjective, and its inverse is $\phi_{h^{-1}}$. We must have $hgh^{-1} = g^r$ for some $r \in \ZZ$ as $K$ is cyclic, meaning $r$ uniquely determines $\phi_h$. 

Consider $\phi_h^{j}(g) = h^jgh^{-j} = g^{r^j}$; our goal is to try and find a congruence equation that will allow us to solve for $r$ and find all compatible automorphisms. Looking at $j = 3$, we require $g{r^j} = g$, leading to the congruence $r^3 \equiv 1 \mod 7$ which has solutions $r = 1, 2, 4$. 

We know there are only 2 possible mappings, and what we have found above is the elements of $\Aut(K)$ which are compatible with our mapping. 

First define a mapping $\varphi_1 : C_3 \rightarrow \Aut(C_7)$ as:
$$
a \mapsto \phi_a, \text{ where } \phi_a(g) = g, \hspace{5pt} \forall g \in C_7,
$$
that is $\varphi_1$ is the trivial map. We saw in the semi-direct product section that if our mapping is trivial then the definition reduces to the familiar direct product. So we have $C_7 \times C_3 \cong C_{21}$ as 7 and 3 are coprime.

Now, define a mapping $\varphi_2 : C_3 \rightarrow \Aut(C_7)$ as:
\begin{align*}
    & e \mapsto \phi_{e}(g) = g, \\
    & a \mapsto \phi_{a}(g)
= g^2, \\
& a^2 \mapsto \phi_{a^2}(g) = g^4, \hspace{5pt} \forall g \in C_7
\end{align*}

where $e, a \in C_3$, and $g \in C_7$. This mapping leads to the semidirect product $C_7 \rtimes_{\varphi_2} C_3$. So the only groups of order 21 are $C_{21}$ and $C_7 \rtimes_{\varphi_2} C_3$.
\end{proof}
