\section{Lecture 2}
\subsection{Subgroups and their properties}
 \begin{definition}
    Let $G$ be a group. A subset $H \subseteq G$ is called a \emph{subgroup} of $G$ if we have the following:

    \begin{enumerate}
        \item The identity element $e$ is contained in $H$.
        \item If $a,b \in H,$ then $ab \in H$.
        \item If $a \in H$, then $a^{-1} \in H.$
    \end{enumerate}
   % A subgroup of a group G, $H\subset G$, is a subset of $G$ with 
   % \begin{itemize}
   %     \ii It contains the $G$ group identity element.
   %     \ii For all $a,b\in H$ we have $ab\in H$
   %     \ii for any $a\in H$ we have $a^{-1}\in H$.
   % \end{itemize}
   \label{subgroup}
 \end{definition}
In other words, $H$ is a subset of $G$ that is a group in its own right, with the same identity and same group operation. We write $H\leq G$ to denote a subgroup.

% Note that we can package conditions 2 and 3 into a single one and say that if $a,b \in H$, then $ab^{-1} \in H$. This is what is typically known as the subgroup test.

\begin{proposition}[Subgroup test]
  Let $H$ be a subset of a group $G$. Then $H$ is a subgroup if and only if the following hold:
  \begin{enumerate}
    \item $H$ is non-empty.
    \item If $x,y\in H$ then $x^{-1}y\in H$.
  \end{enumerate}
  \label{subgroupTest}
\end{proposition}
\begin{proof}
  ($\implies$) The first holds since $e\in H$. Let $x,y \in H.$ Since $H$ is a group, it is closed under taking inverses, so $x^{-1}\in H$. Also, $H$ is closed under the group operation of $G$, so $x^{-1}y\in H,$ as desired.
  %then we must have $x\in H \implies
  %x^{-1}\in H$ and $x^{-1}y\in H$ for some $y\in H$.

    ($\impliedby$) We have that $H$ is non-empty and that for any $x,y\in H$ we have $x^{-1}y\in H$. We claim that $H$ contains the identity, is closed, associative, and every element has an inverse in $H$. 

  Since $H$ is non-empty, let $x\in H.$
  By the second property with $y=x$ we have $x^{-1}x=e\in H.$
  If $x\in H,$ we have $x^{-1}e=x^{-1}\in H$, so $H$ is closed under taking inverses. Now for two elements $x,y \in H,$ we have shown that $x^{-1} \in H,$ so  $(x^{-1})^{-1}y=xy\in H,$ and hence $H$ is closed under the binary operation. Finally, associativity holds since the binary operation in $G$ is associative.
  %prime Allan Perez moment
  % Finally, to
  % show associativity, we claim that for $a,b,c\in H$ we have $(ab)c=a(bc)$. Note that given the
  % above, we have $(ab)c=a(bc)\iff a'(ab)c=a'a(bc)=bc=(a'a)bc \iff a'(ab)c=(a'a)bc= a'abc$.
\end{proof}

\begin{definition}
  A group $G$ is called \emph{cyclic} if there exists a $g\in G$ such that $$G=\left\{ g^n \mid n\in\ZZ
  \right\}.$$ If $G$ is cyclic, then such an element $g$ is called a generator of $G$. We
  say $G$ is generated by $g$, and write $G=\langle g \rangle$.
  \label{cyclicGroup}
\end{definition}
% Equivalently, any element of a cyclic group can be expressed as some integer power of the generator.

\begin{remark}
    If $g$ is a generator for $G$, then so is $g^{-1}$, since if $x\in G$ can be expressed as $g^k$, then $x = (g^{-1})^{-k}$.
\end{remark}

\begin{theorem}
  Every cyclic group is abelian.
\end{theorem}
\begin{proof}
Let $G=\langle g\rangle$ be a cyclic group, and let $x,y \in G.$ Then $x=g^n$ and $y=g^m$ for some $n,m \in \NN.$ Now, 
  $$xy=g^n g^m = g^{n+m} = g^{m+n} = g^m g^n=yx.$$ Therefore $G$ is abelian.
\end{proof}

\begin{theorem}
  All subgroups of a cyclic group are cyclic.
\end{theorem}
\begin{proof}
  Let $H\leq G$ where $G$ is a cyclic group with generator $g$. We aim to show $\exists h\in H$ s.t. $H=\langle h\rangle = \left\{
  h^n \mid n\in\ZZ  \right\}$. When $H=\left\{ e \right\}$, the proof is trivial. 
  
  
  If $\exists \: n\in\ZZ_{>0}$ s.t. $g^n\in H$, assume such $n$ is the lowest possible, without loss of generality. We claim that $H = \langle g^n \rangle.$ 
  
  Assume $g^a\in H$ for some $a=qn+r$ for $q,r\in \ZZ$ and $0\leq r < n$. Then $g^a=(g^n)^q g^r$, and we know $g^nq\in H$ since $H$ is a group on its own. This means we must have $g^r\in H$, but $n$ was the smallest
  possible power of $g$, so $r=0$, i.e. $g^r=e$, and hence $g^a = g^{nq} = (g^n)^q$, i.e. $H$ is
  cyclic with generator $g^n$.
\end{proof}


% \begin{definition}
%   The \emph{order} of an element $g\in G$ of a group $G$, written $|g|$, is the least positive
%   integer $n$ s.t. $g^n=e$. If such $n$ doesn't exist, we say $g$ has infinite order. 
%   \label{orderGroup}
% \end{definition}

\begin{definition}
    The \emph{order} of a group $G$, written $|G|$, is the cardinality of the underlying set. If $G$ is an infinite group, we say $|G| = \infty$.

    If $g \in G,$ we say the order of $g$, written $|g|$ or $\ord(g)$, is the smallest positive integer $n$ such that $g^n = e$. We say $|g| = \infty$ if no such $n$ exists.

    \label{orderGroup}
\end{definition}

\begin{theorem}
  Let $G$ be a group, and let $g\in G$. Then the order of $g$ is equal to the order of the subgroup $\langle g \rangle \leq G$.
\end{theorem}
\begin{proof}
  % If $|g|=\infty$, then $g^i=g^k \iff i=k$, since if $i < k,$ then $g^{k-i} = e \implies k-i= 0$. Hence, there is no $n \in \ZZ_{>0}$ s.t. $g^n = e$.
% then $g^n \neq e \forall n \in \Z
    % If $|g|=\infty, Z{>0}$. This means that the order of $\langle g \rangle$ is also infinite.
  
  If $|g|=m<\infty$, then $\langle g \rangle =\left\{ 1,g,g^2,
  \cdots, g^{m-1} \right\}$, so $|g|=m.$ 
  
  If the order of $g$ is infinite, then $\langle g\rangle =\{e,g,g^2,g^3,\ldots \}$ with no repeats, since if $i < k$ with $g^i=g^k,$ then $g^{k-i} = e \implies k-i= 0$.
  Therefore 
  %as there is no $n \in \ZZ$ such that $g^n = e$. Therefore, 
  $|\langle g \rangle| = \infty$. 
\end{proof}

\begin{theorem}
  Let $G$ be a group, and let $g\in G$, and $n\in\ZZ$. Then $g^n=e$ if and only if $n$ is a multiple
  of $|g|$.
\end{theorem}
\begin{proof}

($\implies$) Suppose $g^n=e$. There exists integers $k,r$ with $0\leq r<m$ such that $n=km+r.$ We have $g^n=g^{km+r}=g^{km}g^r=e.$ We know $g^{km}=e,$ so we must have $g^n=g^r=e.$ However, since $r<m$, $r=0$. Therefore, $n=km,$ as required.

($\impliedby$) Let $|g|=m$. Suppose $n$ is a multiple of $m,$ then \exists $k\in \ZZ$ s.t. $n=km.$ We have,
$$g^n=g^{mk}=(g^m)^k = e^k = e,$$
as required. 
\end{proof}

% \subsection{Cyclic groups structure}
% For a cyclic group of infinite order, the structure of the group resembles
% $(\ZZ,+)$, whereas for a cyclic group of finite order $n$, the structure resembles
% $(\ZZ_n,
% +\mod n)$.

% \begin{remark}
%     The structure of cyclic groups are very limited to how they 

%     All cyclic groups are
% \end{remark}
