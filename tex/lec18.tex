\section{Lecture 18 - 1 Nov 2021}
\subsection{Intro to rings}
\begin{definition}
  A ring is a collection $(R,+,\cdot)$ where $R$ is a set, and $+,\cdot$ are two binary
  operations on $R$ such that 
  \begin{enumerate}
    \item $(R,+)$ is an abelian group (labeled as the additive identity $0$)
    \item The operation $\cdot$ is associative, i.e. for any $a,b,c\in R$ one has
      $(a\cdot b)\cdot c = a\cdot (b\cdot c)$
    \item $\cdot$ distributes over $+$, i.e. for all $a,b,c\in R$ one has 
      \[(a+b)\cdot c = a\cdot c+b\cdot c \land c\cdot (a+b)=c\cdot a + c\cdot b\]
  \end{enumerate}
  A \emph{unital ring} or a \emph{ring with unity} is a ring as above such that there
  exists $1\in R\setminus \{0\}$ satisfying $1\cdot a = a\cdot 1 = a$ for all $a\in R$.
  A ring is called \emph{commutative} if the operation $\cdot$ is commutative. Note that
  $+$ is always commutative.
  \label{def:ring}
\end{definition}
\begin{remark}
  Note that many authors will refer to a unital ring simply as a ring, and they would
  state otherwise if the ring in question is not unital.
\end{remark}

\begin{example}
  We have already seen examples of rings such as $\ZZ,\QQ,\RR,\CC$. Other more interesting
  ones are,
  \begin{enumerate}
    \item For every $n\in\ZZ_{>1}$, the set of cosets $\ZZ/n\ZZ$ forms a ring under
      addition and multiplication $\mod n$.
    \item If $R$ is commutative ring, then one can form the ring $R[X]$ of polynomials
      over $R$ in one variable. The underlying set is just a set of symbols, not
      necessarily any definite operation. The polynomial hence is just a formal finite
      linear combination of symbols of the form $X^i$ for $i\in\NN$. Multiplication would
      be defined as $X^iX^j=X^{i+j}$. The ring consists of elements of the form
      $\sum_{i=o}^d a_iX^i$, $d\in\ZZ_{\geq 0}$ and $a_i\in R$ for all $i$, with the usual
      addition and multiplication rules.
    \item if $R$ is a ring, then one can form the ring $M_n(R)$ of $n\times n$ matrices
      over $R$ under matrix addition and multiplication.
    \item If $R$ is a ring and $S$ is some set, the set $R^S$ of functions $f:S\to R$ is
      aring under pointwise addition and multiplication: $(f+g)(s)=f(s)+g(s), (f\cdot
      g)(s)=f(s)\cdot g(s)$. If $S=\NN$ then the function $f$ assigns every natural number
      some element of $R$, i.e. some finite sequence, and we can add and multiply
      pointwise elements of the sequence.
  \end{enumerate}
\end{example}

\begin{remark}
  Let $R$ be a ring, and $a,b\in R$, $n\in\ZZ\setminus \{0\}$.
  \begin{enumerate}
    \item Like in the case of additive groups, we write $-a$ for the additive inverse of
      $a$, $na=a+a+\cdots+a$ if $n>0$ and $na=-a-a-\cdots-a$ if $n<0$.
    \item Write $a-b=a+(-b)$.
    \item If $R$ is unital, we set $a^0=1$ for any $a\in R\setminus\{0\}$. Note however
      that $a^{-1}$ won't exist in general.
    \item We usually frop $\cdot$ and instead write $ab$ rather than $a\cdot b$.
  \end{enumerate}
\end{remark}

\begin{remark}
  Note that in rings, $ab$ do not need to commute, the multiplicative identity need does
  not exist, and the cancellation rule does not need to hold. Be careful with these
  intuitive ideas. Try to derive everything from first principles. Go to the axioms and
  prove it.
\end{remark}


\begin{theorem}
  Let $R$ be a ring, let $a,b\in R$, and let $m,n\in\ZZ$. Then we have 
  \begin{enumerate}
    \item $0a = a0 =0$
    \item $a(-b)=(-a)b=-(ab)$
    \item $(-a)(-b)=ab$
    \item $(m+n)a = ma+na$ (not in the axioms, mind that these are integers, not ring
      elements)
    \item $(mn)a=m(na)$
    \item $m(a+b)=ma+mb$
    \item $m(ab)=(ma)b = a(mb)$
    \item $(ma)(nb)=(mn)(ab)$
  \end{enumerate}
  \label{<+label+>}
\end{theorem}

\begin{definition}
  Let $R$ be a unital ring. An element $u$ of $R$ is called a \emph{unit} if there exists
  $u^{-1}\in R$ such that $uu^{-1}u^{-1}u=1$. The set of units of $R$ is denoted by
  $R^{\times}$. The set of units of $R$ forms a group under multiplication.
  \todo{Note that we defined $R$ to be closed under multiplication but it's not obvious
  that $R^{\times}$ is closed}
  \label{<+label+>}
\end{definition}

\begin{definition}
  A unital ring in which every non-zero element is a unit is called a \emph{division
  ring}. A commutative division ring is called a \emph{field}.
  \label{<+label+>}
\end{definition}

\begin{example}
  The Hamilton quaternions $\HH$ are real vector space with basis $1, \vi, \vj, \vk$, with
  multiplication defined by 
  \[\vi^2=\vj^2=\vk^2=1 \quad \vi\vj=-\vj\vi=\vk, \vj\vk=-\vk\vj=\vi, \vk\vi=-\vi\vk=\vj\]
  And extended to $\RR$-linear combinations by distributivity. The inverse of an element
  $a+b\vi+c\vj+d\vk \neq 0$ has inverse $\frac{a-b\vi-c\vj-d\vk}{a^2+b^2+c^2+d^2}$.
\end{example}

