\section{Lecture 18}
\subsection{Intro to rings}
\begin{definition}
    A ring is a collection $(R,+,\cdot)$ where $R$ is a set, and $+$ and $\cdot$ are two binary operations on $R$ such that 
  
    (R1): $(R,+)$ is an abelian group (call the additive identity $0$);
   
    (R2): The operation $\cdot$ is associative, i.e. for any $a,b,c\in R$ one has $$(a\cdot b)\cdot c = a\cdot (b\cdot c);$$
    
    (R3): $\cdot$ distributes over $+$, i.e. for all $a,b,c\in R$ one has 
      $$(a+b)\cdot c = a\cdot c+b\cdot c \text{ and } c\cdot (a+b)=c\cdot a + c\cdot b.$$
  
  A \emph{unital ring} or a \emph{ring with unity} is a ring as above such that there exists $1\in R\setminus \{0\}$ satisfying $1\cdot a = a\cdot 1 = a$ for all $a\in R$.
  A ring is called \emph{commutative} if the operation $\cdot$ is commutative. 
  \label{def:ring}
\end{definition}
\begin{remark}
  Note that many authors will refer to a unital ring simply as a ring, and will call a ring without a unit a \emph{rng}.
\end{remark}



\begin{example}
We have already seen examples of rings such as $\ZZ,\QQ,\RR,\CC$. The following are some more interesting examples.
  \begin{itemize}
    \item For every $n\in\ZZ_{>1}$, the set of cosets $\ZZ/n\ZZ$ forms a ring under
      addition and multiplication modulo $n.$

    \item If $R$ is a commutative ring, then one can form the ring $R[X]$ of polynomials over $R$ in one variable. These polynomials are expressed as formal sums of the form $\sum_{i=0}^{d} a_i X^i,$ where $ d \in \mathbb{Z}_{\geq 0} $, $a_i \in R $ for all $i$, and the addition and multiplication are defined as the usual addition and multiplication for polynomials..
    
    \item if $R$ is a ring, then one can form the ring $M_n(R)$ of $n\times n$ matrices over $R$ under matrix addition and multiplication.
    
    \item If $R$ is a ring and $S$ is an arbitrary set, the set $R^S$ of functions $f:S\to R$ is a ring under pointwise addition and multiplication: $(f+g)(s)=f(s)+g(s), (f\cdot
      g)(s)=f(s)\cdot g(s)$. 
      
      If $S=\NN$ then the function $f$ assigns every natural number some element of $R$, i.e. elements of $R^S$ are sequences of elements of $R$, and we can add and multiply pointwise elements of the sequence.

      \item The most boring ring is the trivial ring, containing a single element. This is indeed commutative and unital, and the only ring in which 0 and 1 coincide.
  \end{itemize}
\end{example}

\begin{remark}
  Let $R$ be a ring, and $a,b\in R,$ $n\in \ZZ\setminus \{0\}$. We use the following conventions and notation.
  \begin{enumerate}
    \item The notation $n \cdot a$ is shorthand for $$\underbrace{a+a+\cdots+ a}_{n \text{ times}}$$ in the ring. Similarly, premultiplication by a negative integer denotes repeated addition of the additive inverse of $a$. In other words, if $n\in \ZZ_{<0},$ then $n \cdot a$ represents $$\underbrace{-a-a-\cdots- a}_{\lvert n \rvert \text{ times}}.$$
    \item Write $a-b=a+(-b)$.
    \item If $R$ is unital, we set $a^0=1$ for any $a\in R\setminus\{0\}$. Note however that $a^{-1}$ won't exist in general.
    \item We usually drop $\cdot$ and instead write $ab$ rather than $a\cdot b$.
  \end{enumerate}
\end{remark}

\begin{theorem}
  Let $R$ be a ring, let $a,b\in R$, and let $m,n\in\ZZ$. Then we have 
  \begin{enumerate}
    \item $0a = a0 =0$;
    \item $a(-b)=(-a)b=-(ab)$;
    \item $(-a)(-b)=ab$;
    \item $(m+n)\cdot a = m\cdot a+n\cdot a$; 
    \item $(mn)\cdot a=m\cdot(n\cdot a)$;
    \item $m\cdot(a+b)=m\cdot a+m\cdot b$;
    \item $m\cdot(ab)=(m\cdot a)b = a(m\cdot b)$;
    \item $(m\cdot a)(n\cdot b)=(mn)\cdot(ab)$.
  \end{enumerate}
  \label{<+label+>}
\end{theorem}
\begin{proof}
    Exercise.
\end{proof}

\begin{remark}
  Note that in rings, $ab$ do not need to commute, the multiplicative identity may not exist, and the cancellation rule does not need to hold. Be careful with these intuitive ideas; try to derive everything directly from the axioms.
  
  Also, note that item 4 above is not the distributivity rule from the axioms as $m$ and $n$ are integers and not ring elements.
\end{remark}

\begin{definition}
  Let $R$ be a unital ring. An element $u$ of $R$ is called a \emph{unit} if there exists
  $u^{-1}\in R$ such that $uu^{-1}=u^{-1}u=1$. The set of units of $R$ is denoted by $R^{\times}$ and forms a group under the multiplication operation in $R$.
 
  \label{<+label+>}
\end{definition}

\begin{definition}
  A unital ring in which every non-zero element is a unit is called a \emph{division
  ring}. A commutative division ring is called a \emph{field}.
  \label{<+label+>}
\end{definition}

\begin{example}
  The Hamilton quaternions $\HH$ are a real vector space with basis $1, \vi, \vj, \vk$, with
  multiplication defined by 
  \[\vi^2=\vj^2=\vk^2=1, \quad \vi\vj=-\vj\vi=\vk, \vj\vk=-\vk\vj=\vi, \vk\vi=-\vi\vk=\vj\]
  and extended to $\RR$-linear combinations by distributivity. The inverse of an arbitrary element
  $a+b\vi+c\vj+d\vk \neq 0$ is $$\frac{a-b\vi-c\vj-d\vk}{a^2+b^2+c^2+d^2}.$$

  This is a division ring but not a field. The non-commutativity of the multiplication is the only property that keeps the quaternions from being a field.
\end{example}

