\section{Lecture 12}
On the 3rd isomorphism theorem.
\subsection{Third Isomorphism Theorem}
\begin{lemma}
  Let $\phi:G\to G'$ be a homomorphism and $H'<G'$. We claim that the preimage
  $\phi^{-1}(H')<G$ is a subgroup of $G$. 
  \label{lem:preImgSubgroup}
\end{lemma}
\begin{proof}
  Since $H'$ is a subgroup of $G',$ we have $1_{G'}\in H'$, so $\phi(1_{G})=1_{G'}$ and it follows that $1_{G}\in \phi^{-1}(H')$. Let $x,y\in\phi^{-1}(H'),$ then we have that $x=\phi^{-1}(h'),\text{ }y=\phi^{-1}(h)$ for some $h,h',\in H'$, or equivalently, $h'=\phi(x)$ and $h=\phi(y).$ It follows that $$h'h^{-1}=\phi(x)\phi(y)^{-1} = \phi(xy^{-1})\in H',$$ therefore $xy^{-1}\in\phi^{-1}(H')$,
  as required.
\end{proof}

% \begin{lemma}
%   Let $\phi:G\to G'$ be a surjective homomorphism and $H'<G'$. We claim 
%   \[H'\trianglelefteq G' \iff \phi^{-1}(H')\trianglelefteq G\]
%   \label{lem:preImgNormSub}
% \end{lemma}
% \begin{proof}
%   Let $g\in G$ and $h\in \phi^{-1}(H')$. Then define $g'=\phi(g), h'=\phi(h)\in H'$ and $\bar{h'} = \phi(\bar{h})$.
%   We have $g'h' = \bar{h}'g'$ for some $\bar{h}'\in H$,
%   \[g'h' = \bar{h}'g' \iff \phi(g)\phi(h)  =\phi(\bar{h})\phi(g) \iff \phi(ghg^{-1}) = g'h'(g')^{-1} \in H'.\]
%   \[\iff ghg^{-1}\in \phi^{-1}(H')\]
%   Hence $\phi^{-1}(H')\trianglelefteq G$, as required.
% \end{proof}

\begin{theorem} [Third Isomorphism Theorem]
If $N$ and $K$ are normal in $G$ with $N \subseteq K$, then we have an isomorphism of groups $G/K \cong (G/N)/(K/N).$
%  Let $G$ be a group, and $N\trianglelefteq G$. Define the quotient map $q_N:G\to G/N$.
%  Then
%  \begin{enumerate}
%    \item For every subgroup $U$ of $G/N$, the preimage $q_N^{-1}(U)$ is a subgroup of
%      $G$.
%    \item If $U$ is a subgroup of $G/N$, then $U$ is normal in $G/N$ if and only iff
%      $q_N^{-1}(U)$ is normal in $G$.
%    \item The function $U\mapsto q_N^{-1}(U)$ defines a bijection between the set of
%      subgroups of $G/N$ and the set of subgroups of $G$ that contain $N$.
 %   If $K\trianglelefteq G$ containing $N$, then there exists a unique isomorphism $\widetilde{f}: G/K \to (G/N)/(K/N)$ which gives 
   % $$
   %     G/K \cong (G/N)/(K/N)
  %  $$
   % By $gK\mapsto (gN)(K/N)$.
%  \end{enumerate}
  \label{thm:tit}
\end{theorem}

\begin{proof}
  % We'll prove these part by part. For the first part, we have that
  % $N=\pi^{-1}(1N)\subset\pi^{-1}(U)$ for every subset of $U$, since $U$ is a subgroup of
  % $G/N$ it must contain the identity element, naming $N$. Note that the pre-image of $N$
  % is $N$ itself (recall the absorption rule from 2F). Moreover, we have that for a
  % homomorphism $\phi:G\to G/N$, the preimage of any subgrup of $G/N$ is a subgroup of $G$
  % by Lemma \ref{lem:preImgSubgroup}.

  % For the second part, we have that the quotient map is surjective by definition. The
  % proof then follows by Lemma \ref{lem:preImgNormSub}.

  % For the third part, we find an inverse of the function $U\to \pi^{-1}(U)$ to show that
  % it's indeed a bijection. Let $\SH$ be the set of subgroups of $G$ that contain $N$, i.e.
  % $\SH=\{H<G | H\cap N = N\}$, and let $\SH'$ be the st of subgroups of $G/N$, i.e.
  % $\SH'=\{gN | g\in G\}$. Then we find $\SH\to\SH'$ by $H\mapsto \pi (H)=H/N$. Hence the
  % defined function is a bijection.
  We claim that  $\widetilde{f}: G/K \to (G/N)/(K/N),$ defined by $gK\mapsto gN(K/N),$ is an isomorphism of groups.
  First, to check the map is well defined, suppose $gK=g'K$ for some $g,g'\in G$. We have,
  \begin{align*}
      gK=g'K &\iff g^{-1}g'\in K
      \\ &\iff g^{-1}g' N \in K/N
      \\ &\iff gN(K/N) = g'N(K/N),
  \end{align*}
  as required. 
  The equivalences above read backwards give us the injectivity of $\widetilde{f}$. 
  
  The map is clearly surjective; for all $gN(K/N)\in (G/N)(K/N),$ we have $gK\in G/K$ with $\widetilde{f}(gK)=gN(K/N).$ 
  %therefore, $\widetilde{f}$ is surjective. 


  We are left to prove that $\widetilde{f}$ is a homomorphism. We have
  \begin{align*}
      \widetilde{f}(gKg'K)&=\widetilde{f}(gg'K)=gg'N(K/N)
      \\&=gN(K/N)g'N(K/N)=\widetilde{f}(gK)\widetilde{f}(g'K).
  \end{align*}
  We have a bijective homomorphism and therefore an isomorphism.
\end{proof}

This is easier to understand with the following commutative diagram:

\[\begin{tikzcd}
	G &&& {G/K} \\
	\\
	{G/N} &&& {(G/N)/(K/N)}
	\arrow["{q_{N}}"{description}, two heads, from=1-1, to=3-1]
	\arrow["{q_{K/N}}", two heads, from=3-1, to=3-4]
	\arrow["{\widetilde{f}}"{description}, dashed, from=3-4, to=1-4]
	\arrow["{q_{K}}", two heads, from=1-1, to=1-4]
\end{tikzcd}\]

\begin{remark}
 The third isomorphism theorem says that the quotient homomorphism $q_K : G \rightarrow G/K$ can be factored through the quotient homomorphism $q_N : G \rightarrow G/N$ to give $q_K \simeq q_{K/N} \circ q_N$.
\end{remark}

% \begin{proof}
    
% \end{proof}

An intuitive way of thinking about the 3rd Isomorphism Theorem is as follows: suppose that $G$ is a group, $K$ is a normal subgroup, and $\phi: G \rightarrow G'$ is a group homomorphism to some other group $G'$ whose kernel $N = \ker \phi$ is contained in $K$. Then the kernel of $\phi$ restricted to $K$ is still $N$. By FIT, $\phi$ induces isomorphisms $G/N \cong \phi(G)$ and $K/N \cong \phi(K)$.
It can be shown that the 3rd Isomorphism Theorem is saying that one has an isomorphism 
$$
{G/K}\cong{\phi(G)/\phi(K)}. 
$$

It is not true in general that $G\cong G'$ and $N \cong N'$ implies $G/N \cong G'/N'$ (see ex sheet 3 Q14); there is something to prove here.
We need to show that for $G,K,N$ as defined above, we have $(G/N)/(K/N)\cong \phi(G)/\phi(K).$ 
Consider the map
$$\psi : \phi(G)\to (G/N)/(K/N)$$ defined by $\phi(g)\mapsto gN(K/N).$ This map is clearly surjective; for any $gN(K/N),$ we have $\phi(g) \in \phi(G),$ so we have $\psi(\phi(g))=gN(K/N).$ Now, 
\begin{align*}
    \ker{\psi}&=\{\phi(g)\mid \psi(\phi(g))=K/N\}
    \\ & =\{\phi(g)\mid gN(K/N)=K/N\}
    \\ & =\{\phi(g)\mid gN\in K/N\}
    \\ & =\{\phi(g)\mid g\in K\} = \phi(K).
\end{align*}
Therefore, by the first isomorphism theorem, we have $\phi(G)/\phi(K) \cong (G/N)/(K/N),$ and by the third isomorphism theorem, $G/K \cong (G/N)/(K/N) \cong \phi(G)/\phi(K),$ as required.


In other words, if we want to understand some quotient $G/K$, then we may apply any homomorphism $\phi$ to $G$ whose kernel is contained in $K$, and it suffices to understand the quotient ``on the other side'' of $\phi$.


\begin{example}
  Let $G=\GL_{2}(\RR)$ and let $K=\GL_2^+(\RR)=\{X\in\GL_2(\RR) : \det X>0\}$. It is
  easy to see that $K$ is closed under conjugation by elements of $G,$ so $K$ is normal in $G$. We want to study $G/K$ using the Third Isomorphism Theorem. We have $N=\SL_2(\RR)\subset K,$ and $N\trianglelefteq G$ as it is the kernel of the determinant map. Consider the surjective group homomorphisms
  $\det: G\to\RR^{\times}$ and $\det: \GL_2^+(\RR)\to\RR_{>0}$. Both have kernel $N,$ so by the
  FIT we have $G/N\cong \RR^{\times}$ and $K/N\cong \RR_{>0}$. Now, by the 3rd Iso. Thm. we have
  $G/K\cong \det\GL_2(\RR)/\det\GL_2^+(\RR)\cong \RR^{\times}/\RR_{>0}\cong \{\pm 1\}$.
\end{example}

% Get used to thinking about normal subgroups as kernels of homomorphisms, and quotients
% $G/N$ as images $\phi(G)$ of homomorphisms $\phi$ whose kernels are that normal subgroup
% $N=\ker\phi.$

\begin{lemma}
    % Let $G$ be a group and $H \leq G$ and $N \unlhd G$. Then every subgroup $U$ of $G/N$ is of the form $H/N$. 

    Let $G$ be a group and $N \unlhd G$. Then every subgroup $U$ of $G/N$ is of the form $H/N$ for some $H \leq G$ containing $N$. 
    \label{lem:subgpG/N}
\end{lemma}

\begin{proof}
     Let $H = \pi^{-1}(U)$. 
     By Lemma \ref{lem:preImgSubgroup}, $H \leq G$. Furthermore, because $N = 1_{G/N} \subseteq U,$ we have $\pi^{-1}(N)=N \subseteq H$. Hence, the quotient $H/N$ is well defined.
     
     % $N \subseteq H$ since $N\subseteq U,$ and since $N=1_{G/N},$ we have $\pi^{-1}(N)=N\subset H$.

     %U subgroup means N in U
     %H is preimage of U, so preimage of N = N is contained in H
     
     % containing $N$ since $N \in U$,
     % , meaning  
     % $\pi^{-1}(1N) = N$
      We go by way of double inclusion.

    ($H/N \subseteq U$) Let $hN \in H/N$. As $H = \pi^{-1}(U),$ we have $\pi(h) = hN \in U.$

    ($U \subseteq H/N$) Let $u \in U$, then $u = gN$ for some $g \in G$.
    Since $gN \in U, \pi(g) \in U,$ so $g \in H$ and thus $u \in H/N$.

    So every subgroup of $G/N$ is of the form $H/N$.
\end{proof}

\begin{theorem}[Correspondence Theorem]
    Let $G$ be a group, let $N$ be a normal subgroup, and let $\pi$ be the quotient map $G \rightarrow G / N$. Then:
    \begin{enumerate}
        \item For subgroup $U$ of $G/N$ the assignment $U \mapsto \pi^{-1}(U)$ defines a bijection between the set of subgroups of $G / N$ and the set of those subgroups of $G$ that contain $N$;
        \item The above assignment defines a bijection between the set of normal subgroups of $G / N$ and the set of normal subgroups of $G$ containing $N$.
    \end{enumerate}
\end{theorem}


\begin{proof}
% (1) Let $\operatorname{Sub}_{\subset N}$ denote the set of all subgroups containing $N$ and $\operatorname{Sub}_{G/N}$ denote the set of subgroups of $G/N$. We know from Lemma \ref{lem:subgpG/N} that every element of this set is of the form $H/N$. So we have our pre-image map $\pi^{-1} : \operatorname{Sub}_{G/N} \rightarrow \operatorname{Sub}_{\subset N}$ defined by $H/N \mapsto \{g \in G \mid gN \in H/N\} = H \leq G$ by Lemma \ref{lem:preImgSubgroup}. I claim that the map $\alpha : \operatorname{Sub}_{\subset N} \rightarrow \operatorname{Sub}_{G/N}$ defined by $H \mapsto H/N$ is the inverse of $\pi^{-1}$.

%     Showing right sided inverse, we have that $\pi^{-1} \circ \alpha(H) = \pi^{-1}(H/N) = \{g \in G \mid gN \in H/N\} = H$ so $\pi^{-1} \circ \alpha(H) = \id_{\operatorname{Sub}_{\subset N}}$.

%     Now for left sided inverse we have $\alpha \circ \pi^{-1}(H/N) = \alpha(\{g \in G \mid gN \in H/N\}) = \alpha(H) = H/N$ so $\alpha \circ \pi^{-1} = \id_{\operatorname{Sub}_{G/N}}$.

%     So there is a bijection between $\operatorname{Sub}_{G/N}$ and $\operatorname{Sub}_{\subset N}$.

(1) Let \( \mathcal{S}^{G}_N \) denote the set of all subgroups of \( G \) that contain \( N \), and let \( \mathcal{S}^{G/N} \) denote the set of subgroups of \( G/N \). We know from Lemma \ref{lem:subgpG/N} that every element of \( \mathcal{S}^{G/N} \) is of the form \( H/N \). So we have our pre-image map \( \pi^{-1}: \mathcal{S}^{G/N} \rightarrow \mathcal{S}^{G}_N \) defined by \( H/N \mapsto \{g \in G \mid gN \in H/N\} = H \leq G \), by Lemma \ref{lem:preImgSubgroup}. We claim that the map \( \theta: \mathcal{S}^{G}_N \rightarrow \mathcal{S}^{G/N} \) defined by \( H \mapsto H/N \) is the inverse of \( \pi^{-1} \).

To show that \( \theta \) is the right inverse, we have that \( \pi^{-1}(\theta(H)) = \pi^{-1}(H/N) = \{g \in G \mid gN \in H/N\} = H \), so \( \pi^{-1} \circ \theta = \id_{\mathcal{S}^{G}_N} \), the identity map on \( \mathcal{S}^{G}_N \).

Now for the left inverse, we have \( \theta(\pi^{-1}(H/N)) = \theta(\{g \in G \mid gN \in H/N\}) = \theta(H) = H/N \), so \( \theta \circ \pi^{-1} = \id_{\mathcal{S}^{G/N}} \), the identity map on \( \mathcal{S}^{G/N} \).

 Thus, there is a bijection between \( \mathcal{S}^{G/N} \) and \( \mathcal{S}^{G}_N \).


(2) From sheet 2, q5, since $\pi$ is a surjective homomorphism, we have that $\pi^{-1}(U)$ is a normal subgroup of $G$ if and only if $U$ is normal in $G/N.$ 
Note that because $U$ is a subgroup of $G/N$, $\pi^{-1}(U)$ contains $N$. Hence every normal subgroup of $G$ containing $N$ has an associated normal subgroup of $G/N$ and vice versa, i.e. there is a bijection between these two sets.
% The statement implies that for every normal subgroup of $G/N,$ the preimage is normal in $G,$ and vice versa. Therefore, there is a bijection between these two sets.
\end{proof}



