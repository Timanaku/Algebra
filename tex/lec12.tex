\section{Lecture 12 - 18 Oct 2021}
On the 3rd isomorphism theorem. Last lecture about group theory. From next lecture on
we'll talk about group actions. 
\subsection{Third Isomorphism Theorem}
\begin{lemma}[Sheet 2, Q5]
  Let $\phi:G\to G'$ be a homomorphism and $H'<G'$. We claim that the preimage
  $\phi^{-1}(H')<G$ is a subgroup of $G$. 
  \label{lem:preImgSubgroup}
\end{lemma}
\begin{proof}
  Note that we have $1_{G'}\in H'$ since $H'$ is a subgroup of $G'$, so
  $\phi(1_{G})=1_{G'}$ and it follows tht $1_{G}\in \phi^{-1}(H')$. Next we claim that for
  any $x,y\in\phi^{-1}(H')$ we have $xy^{-1}\in\phi^{-1}(H')$. Note that we have, for any
  $h,h'\in H'$, $h'h^{-1}\in H'$ since $H'$ is a subgroup. Moreover, we have that
  $h=\phi(x),h'=\phi(y)$ for some $x,y,\in\phi^{-1}(H')$, hence it follows that
  $\phi(x)\phi(y)^{-1} = \phi(xy^{-1})\in H'$, or equivalently $xy^{-1}\in\phi^{-1}(H')$,
  as required.
\end{proof}

\begin{lemma}
  Let $\phi:G\to G'$ be a surjective homomorphism and $H'<G'$. We claim 
  \[H'\trianglelefteq G' \iff \phi^{-1}(H')\trianglelefteq G\]
  \label{lem:preImgNormSub}
\end{lemma}
\begin{proof}
  Let $g\in G$ and $h\in \phi^{-1}(H')$. Then define $g'=\phi(g), h'=\phi(h)\in H'$ and $\bar{h'} = \phi(\bar{h})$.
  We have $g'h' = \bar{h}'g'$ for some $\bar{h}'\in H$,
  \[g'h' = \bar{h}'g' \iff \phi(g)\phi(h)  =\phi(\bar{h})\phi(g) \iff \phi(ghg^{-1}) = g'h'(g')^{-1} \in H'.\]
  \[\iff ghg^{-1}\in \phi^{-1}(H')\]
  Hence $\phi^{-1}(H')\trianglelefteq G$, as required.
\end{proof}

\begin{theorem} [Third Isomorphism Theorem]
If $N$ and $K$ are normal in $G$ with $N \subseteq K$, then we have an isomorphism of groups $G/K \cong (G/N)/(K/N)$
%  Let $G$ be a group, and $N\trianglelefteq G$. Define the quotient map $q_N:G\to G/N$.
%  Then
%  \begin{enumerate}
%    \item For every subgroup $U$ of $G/N$, the preimage $q_N^{-1}(U)$ is a subgroup of
%      $G$.
%    \item If $U$ is a subgroup of $G/N$, then $U$ is normal in $G/N$ if and only iff
%      $q_N^{-1}(U)$ is normal in $G$.
%    \item The function $U\mapsto q_N^{-1}(U)$ defines a bijection between the set of
%      subgroups of $G/N$ and the set of subgroups of $G$ that contain $N$.
 %   If $K\trianglelefteq G$ containing $N$, then there exists a unique isomorphism $\widetilde{f}: G/K \to (G/N)/(K/N)$ which gives 
   % $$
   %     G/K \cong (G/N)/(K/N)
  %  $$
   % By $gK\mapsto (gN)(K/N)$.
%  \end{enumerate}
  \label{thm:tit}
\end{theorem}

This is easier to understand with the following commutative diagram:

\[\begin{tikzcd}
	{G} &&&&& {G/K} \\
	\\
	& {G/N} &&& {(G/N)/(K/N)}
	\arrow["{q_K}", two heads, from=1-1, to=1-6]
	\arrow["{q_{K/N}}", two heads, from=3-2, to=3-5]
	\arrow["{q_N}"{description}, two heads, from=1-1, to=3-2]
	\arrow["{\widetilde{f}}"{description}, dashed, from=3-5, to=1-6]
\end{tikzcd}\]

\begin{remark}
    In other words, the third isomorphism theorem says that the quotient homomorphism $q_K : G \rightarrow G/K$ can be factored through the quotient homomorphism $q_N : G \rightarrow G/N$ to give $q_K \simeq q_{K/N} \circ q_N$.
\end{remark}


\begin{proof}
  % We'll prove these part by part. For the first part, we have that
  % $N=\pi^{-1}(1N)\subset\pi^{-1}(U)$ for every subset of $U$, since $U$ is a subgroup of
  % $G/N$ it must contain the identity element, naming $N$. Note that the pre-image of $N$
  % is $N$ itself (recall the absorption rule from 2F). Moreover, we have that for a
  % homomorphism $\phi:G\to G/N$, the preimage of any subgrup of $G/N$ is a subgroup of $G$
  % by Lemma \ref{lem:preImgSubgroup}.

  % For the second part, we have that the quotient map is surjective by definition. The
  % proof then follows by Lemma \ref{lem:preImgNormSub}.

  % For the third part, we find an inverse of the function $U\to \pi^{-1}(U)$ to show that
  % it's indeed a bijection. Let $\SH$ be the set of subgroups of $G$ that contain $N$, i.e.
  % $\SH=\{H<G | H\cap N = N\}$, and let $\SH'$ be the st of subgroups of $G/N$, i.e.
  % $\SH'=\{gN | g\in G\}$. Then we find $\SH\to\SH'$ by $H\mapsto \pi (H)=H/N$. Hence the
  % defined function is a bijection.
  We claim that  $\widetilde{f}: G/K \to (G/N)/(K/N)$ is an isomorphism of groups.
  First we check it's well defined, suppose $gK=g'K$ for some $g,g'\in G$. Then
  $g^{-1}g'\in K$, so $g^{-1}g' N \in K/N$ hence $gN(K/N) = g'N(K/N)\in (G/N)/(K/N)$.
  Surjectivity is given since every coset representative of $G/N$ is hit by some
  representative of $G/K$. Injectivity proof is the same as the well-definedness proof but
  read backwards (notice how the proof goes both ways!)
  \todo{provide alternate proof using FIT}
\end{proof}
The above theorem says that if $G$ is a group, $K$ is a normal subgroup, and $\phi$ is a
homomorphism from $G$ whose kernel ($N$ above, since every normal subgroup is a kernel of
a homomorphism) is contained in $K$, then $G/K \cong \phi(G)/\phi(K)$. Note that
$\phi(G)=\Img \phi:G\to G/N$ and $\phi(K)=\Img \phi_{K}:K\to K/N$. This, in order to
understand $G/K$ we may apply any homomorphism $\phi$ whose kernel ($N$) is contained in
$K$ and instead understand that quotient on the other side of the homomorphism, which is
often simpler.

\begin{example}
  Let $G=\GL_{2}(\RR)$ and let $K=\GL_2^+(\RR)=\{X\in\GL_2(\RR) : \det X>0\}$. Then it's
  easy to see that $K$ is normal in $G$. What is $G/K$? THe gorup $K$ contains the normal
  subgroup $N=\SL_2(\RR)=\ker\det$. We have surjective group homomorphisms
  $G\to\RR^{\times}$ and $\GL_2^+(\RR)\to\RR_{>0}$, with kernel $\SL_2(\RR)$ and by the
  FIT we have $G/N\cong \RR^{\times}$ and $K/N\cong \RR_{>0}$. By the TIT we have
  $G/K\cong \det\GL_2(\RR)/\det\GL_2^+(\RR)\cong \RR^{\times}/\RR_{>0}\cong \{\pm 1\}$.
\end{example}

Get used to thinking about normal subgroups as kernels of homomorphisms, and quotients
$G/N$ as images $\phi(G)$ of homomorphisms $\phi$ whose kernels are that normal subgroup
$N=\ker\phi$.
