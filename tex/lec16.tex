\section{Lecture 16 - 27 Oct 2021}
\subsection{Not Burnside's Lemma}
Applications of Cauchy's Theorem and Burnside's lemma (not Burnside's work!)
\begin{example} [Classify groups of order 6]
  Let $G$ be a group of order $6$. Then $G$ is either cyclic or isomorphic to $S_3\equiv
  D_{2\cdot 3}$.
\end{example}
\begin{proof}
  Cauchy Theorem suggest that there exists an element of $h\in G$ of order $2$ and an
  element $k\in G$ of order $3$. Notice that $K=\langle k \rangle$ will have index $2$ by
  Lagrange, and so by Theorem \ref{thm:cauchyGeneral} it must be normal in $G$. Since $K$
  is normal, it means that conjugation by $h$ will preserve the subgroup meaning that
  $hkh^{-1}\in K$. In particular, conjugation by $h$ defines an automorphism on $K$. 

  We claim that $\phi_K:K\to K: k^i\to hk^ih^{-1}$ is an isomorphism. Since $K$ is normal,
  we're guaranteed to have $hk^ih^{-1}\in K$ for any $i$. Furthermore, we claim we have an
  homomorphism. For $k',k''\in K$ we have $\phi_K(k'k'')=hk'k''h^{-1}=
  hk'h^{-1}hk''h^{-1}=\phi_K(k')\phi_K(k'')$, hence we indeed have a group homomorphism.
  Finally, note that the inverse of $\phi_K$ is just $\phi_K$ since $h$ has order $2$,
  $\phi_K^2(k)=h(hkh^{-1})h^{-1}= h^2 k h^{-2} = k$, hence $\phi_K$ is indeed bijection. But
  the set of automorphism on $K$ of prime order must have size $p-1$ (we proved this in week
  2 exercise sheet), hence we should have $2$ automorphisms. Indeed we have $k\mapsto k$ and
  $k\mapsto k^2$. We found that $\phi_K$ is an automorphism so it must be one of these two.
  In \emph{case 1} we consider $hkh^{-1}=k$, so $hk=kh$, so 
  \[G=\langle k,h | k^3=e=h^2, kh=hk \rangle\]
  So $G=\langle k \rangle \times \langle h \rangle$, which is cyclic by exercise sheet 2.
  In particular, $G=\langle kh \rangle$.

  In \emph{case 2} we consider $hkh^{-1}=k^{2}=k^{-1}$, so we have
  \[G=\langle k,h | k^3=e=h^2, hk=k^{-1}h\rangle\]
  Which is the Dihedral group of order $6$, and this is isomorphic to $S_3$. 
\end{proof}

\begin{definition}
    Let $G$ be a group and $g \in G$ and $X$ the associated $G$-Set. Then we denote $X^{g} := \{x \in X : g \cdot x = x\}$.
\end{definition}

\begin{definition}
  Let $G$ be a group and $X$ be a $G$-set. The set of $G$-orbits of $X$ is denoted by
  $X/G$. Thus we have $X=\bigsqcup_{\mathcal{O}\in X/G} \mathcal{O}$ (note, disjoint union).
\end{definition}

\begin{theorem}[non-Burn's lemma]
  Let $G$ be a group and $X$ be a $G$-set. We have 
  \[|X/G|=\frac{1}{|G|}\sum_{g\in G} |X^g|\]
  \label{thm:notBurnside}
\end{theorem}
\begin{proof}
  Consider the set $S=\left\{ (g,x)\in G\times X : g \cdot x=x \right\}$. We aim to count $|S|$.
  On the one side, we can go over each element of $G$ and find every $x$ for which that
  $g$ has $g \cdot x=x$, and that's exactly $|X^g|$, hence $|S|=\sum_{g\in G} |X^g|$.

  On the other hand, we can go over each element $x\in X$ and find every $g\in G$ for which
  that $x$ has $g\cdot x=x$. That's just $|\Stab x|= |G|/|\mathcal{O}|$ (by the Orbit Stabilizer
  theorem). Then, 
  \[|S|= \sum_{x\in X} |\Stab x|= \sum_{x\in X} \frac{|G|}{|\mathcal{O}|}= |G|
  \sum_{\mathcal{O}\in X/G} \sum_{x\in \mathcal{O}} \frac{1}{|\mathcal{O}|}= |G|\sum_{\mathcal{O}\in X/G} 1 = |G||X/G|\]
  Hence
  \[|G||X/G| = \sum_{g\in G} |X^g|\]
  As required.
\end{proof}

Recall that an orbit of a $G$-set is the set of points that becomes
\emph{indistinguishable} under the symmetries of the set under the group action, since the
actions themselves encode symmetries of the set. So,
\begin{remark}
  Orbit counting is just counting the objects of a $G$-set up to symmetry.
\end{remark}
\begin{example}
  How many essentially different ways are there of coloring the sides of a triangle using
  four colours?  Essentially different in that one coloring cannot be obtained from the
  other by applying symmetries of the triangle.
\end{example}
\begin{proof}
  A naive way of approaching this is counting each possible coloring and checking them.
  This is exactly the case for not Burnside's lemma. Let $X$ be the set of all genuinely
  different coloring for the triangle sitting still. Then $|X|=4^3=64$. The group $D_6$
  acts on $X$. Two colourings are essentially different if one cannot be obtained from the
  other by applying symmetries of the triangle, i.e. if they lie in different orbits under
  the action of $D_6$. So the number of essentially different colourings is just the
  number of orbits, $|X/G|$.
  \[|X/G|=\frac{1}{|G|}\sum_{\sigma\in D_6}|X^{\sigma}|\]
  Consider the symmetries,
  \begin{enumerate}
    \item $\sigma=e\implies |X^{\sigma}|=|X|=64$.
    \item $\sigma$ is a reflection, so we require the two faces we're swapping to be the
      same colour. We have 4 choices in the first face, then the opposite wrt the axis of
      symmetry is decided (no choice), and we have 4 choices for the third face. Hence
      $|X^{\sigma}|=16$. Note we're summing over every element of $G$, so we have to
      account this for every reflection: 3 reflections, so there are $3\cdot 16=48$.
    \item $\sigma$ is a (single) rotation. For the colour to be the same, we require all 3 sides to
      be the same. We have $4$ colours, so $|X^{\sigma}|=4$. We have 2 different rotations
      (the third rotation in the triangle is just the identity), so there $2\cdot 4= 8$.
  \end{enumerate}
  Then, by Burnside's,
  \[|X/G|=\frac{1}{6}(64+8+48) = 20\]
\end{proof}
