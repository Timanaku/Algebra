\section{Lecture 16}
\subsection{Not Burnside's Lemma}
Applications of Cauchy's Theorem and Burnside's lemma (not Burnside's work!)
\begin{example} [Classify groups of order 6]
  Let $G$ be a group of order $6$. Then $G$ is either cyclic or isomorphic to $S_3\equiv
  D_{6}$.
\end{example}
\begin{proof}
  Let $G$ be a group of order 6. By Cauchy's Theorem, there exists an element $h\in G$ of order $2$ and an
  element $k\in G$ of order $3$. Notice that $K=\langle k \rangle$ has index $2$ by
  Lagrange, and so by Theorem \ref{thm:cauchyGeneral} it must be normal in $G$. 
  %say here that G is generated by h and k, how to prove?
  Also, considering orders we can see that $G = \{1_G, k, k^2, h, hk, hk^2 \}$ since each of these elements must be distinct. In other words, $G = \langle h, k \rangle.$
  
  % $\langle h \rangle \cap \langle k \rangle = \{1_{G}\}$
  
  Recall that an automorphism is an isomorphism from a group to itself. Since $K$ is of prime order $p$, the set of automorphisms of $K,\text{ } \Aut(K)$, must have size $p-1$ (proved in ex sheet 2 Q11), so $|\Aut(K)|=2$;
  note that we have actually seen this directly using multiplication tables in Example \ref{ex:multTable}. Let $\varphi \in \Aut(K)$ be the identity map $k \mapsto k$ and $\psi \in \Aut(K)$ be the map defined by $k \mapsto k^2.$
  
  %conjugation by h defines an auto
  %case 1: identity auto
  %case 2: other auto

  Furthermore, we claim that 
  \begin{align*}
      \phi_K: K &\to K \\
      \tilde{k} &\mapsto h \tilde{k} h^{-1}
  \end{align*}
  is an automorphism of $K$, i.e. a bijective homomorphism. 
  % If we can show this, then we have only two options for $\phi_K$.

  Since $K$ is normal in $G$, we have $h\tilde{k}h^{-1}\in K$ for $\tilde{k} \in K$, so $K$ is indeed the codomain. Furthermore, for any two $k_1,k_2\in K,$ we have
  $$\phi_K(k_1k_2)=h(k_1k_2)h^{-1} = 
  hk_1(h^{-1}h)k_2h^{-1} = \phi_K(k_1)\phi_K(k_2),$$ hence $\phi_K$ is a group homomorphism.
  Finally, because $h$ has order $2$,
  $$\phi_K^2(\tilde{k}) = h(h\tilde{k}h^{-1})h^{-1}= h^2 \tilde{k} h^{-2} = \tilde{k},$$ and so $\phi_K$ is self-inverse (bijective), so it is indeed an automorphism.
  % shows that the inverse of $\phi_K$ is itself, so $\phi_K$ is indeed a bijection.

  Now since $\phi_K \in \Aut(K) = \{\varphi, \psi\},$ we have two cases to check.

  First, suppose $\phi_K = \varphi,$ i.e. $hkh^{-1} = k$. Then $hk = kh$, and so $h$ and $k$ commute.
  %say more stuff, conclude that G is cyclic
  %multiplication table????
  Imposing this condition, we can see that the element $hk \in G$ has order 6, since $(hk)^n = h^nk^n,$ and so $|hk| = \lcm(|h|,|k|) = 6.$
  Therefore $hk$ is a generator for $G$, and hence $G$ is cyclic.
  % The fact that $h$ and $k$ commute makes it straightforward to construct the multiplication table of $G$:

  % and one is able to check that this is the multiplication table for $Z_6$, so $G$ is cyclic.

  On the other hand, suppose $\phi_K = \psi,$ i.e. $hkh^{-1} = k^2 = k^{-1}.$
  %cook and conclude that G is D_6
  Then $$G = \langle h, k \mid h^2 = k^3 = 1_G, hk = k^{-1}h \rangle \cong D_6 \cong S_3.$$

  
  
  % We claim that conjugation by $h$ defines an automorphism on $K$. 

  % We will now show that the following is an automorphism,
  % \begin{align*}
  %     \phi_K:K&\to K
  %     \\ k^i &\mapsto hk^ih^{-1}.
  % \end{align*}
  % Since $K$ is normal, we have $hk^ih^{-1}\in K$ for any $i$. Furthermore, for $k_1,k_2\in K,$ we have, 
  % $$\phi_K(k_1k_2)=hk_1k_2h^{-1}=
  % hk_1h^{-1}hk_2h^{-1}=\phi_K(k_1)\phi_K(k_2),$$ hence $\phi_K$ is a group homomorphism.
  % Finally, because $h$ has order $2$,
  % $\phi_K^2(k)=h(hkh^{-1})h^{-1}= h^2 k h^{-2} = k$. This shows that the inverse of $\phi_K$ is itself, so $\phi_K$ is indeed a bijection.
  
  % But since $K$ is of prime order $p$,
  % the set of automorphisms of $K,\text{ } \Aut(K)$, must have size $p-1$ (proved in ex sheet 2 Q11), so $|\Aut(K)|=2$. We have automorphisms defined by $k\mapsto k$ and
  % $k\mapsto k^2$. We showed that $\phi_K\in \Aut(K).$ 
  
  % First, we consider $hkh^{-1}=k$, so $hk=kh$, so 
  % \[G=\langle k,h \mid k^3=e=h^2, kh=hk \rangle\]
  % So $G=\langle k \rangle \times \langle h \rangle$, which is cyclic by exercise sheet 2.
  % In particular, $G=\langle kh \rangle$.

  % Then we consider $hkh^{-1}=k^{2}=k^{-1}$, so we have
  % \[G=\langle k,h \mid k^3=e=h^2, hk=k^{-1}h\rangle\]
  % Which is the dihedral group of order $6$, and this is isomorphic to $S_3$. 
\end{proof}

\begin{definition}
    Let $G$ be a group, $g \in G$ and $X$ the associated $G$-Set. Then we denote $X^{g} := \{x \in X \mid g \cdot x = x\}$, that is, all the points in $X$ that are fixed by a specific element $g\in G.$
\end{definition}
\begin{remark}
    Do not confuse this with $X^G,$ the set of fixed points of an action.
\end{remark}
\begin{definition}
  Let $G$ be a group and $X$ be a $G$-set. The set of $G$-orbits of $X$ is denoted by
  $X/G$. Thus we have $X=\bigsqcup_{\mathcal{O}\in X/G} \mathcal{O}$.
\end{definition}
\begin{remark}
    Note that the above definition uses disjoint union.
\end{remark}
\begin{theorem}[Not Burnside's Lemma]
  Let $G$ be a group and $X$ be a $G$-set. We have 
  \[|X/G|=\frac{1}{|G|}\sum_{g\in G} |X^g|.\]
  \label{thm:notBurnside}
\end{theorem}
\begin{proof}
  Consider the set $S=\left\{ (g,x)\in G\times X \mid g \cdot x=x \right\}$. We will show the required equality by counting $S$ in two different ways.
  On the one side, for each element $g\in G$, we can count the number of elements $x\in X$ satisfying
  %for which that $g$ has 
  $g \cdot x=x$; this is exactly $|X^g|$ by definition. 
  But in the definition of $S$ we are free to choose $g \in G$, so 
  %is defined to be
  % doing this for all g, so |S| = ....
  % Hence, all possible such tuples is 
  $|S|=\sum_{g\in G} |X^g|$.

  On the other hand, we can count the $g\in G$ for each $x\in X$ such that 
  $g\cdot x=x$. Notice that this is just $|\Stab{}_G(x)|,$ which is $|G|/|\Orb_G(x)|$ by the Orbit-Stabiliser Theorem. Let $\mathcal{O}=\Orb_G(x)$ for this section only. Then, 
  % \begin{align*}
  $$
      |S|= \sum_{x\in X} |\Stab{}_G(x)|= \sum_{x\in X} \frac{|G|}{|\mathcal{O}|}.
  $$
  We can translate this sum over all elements of $X$ to a sum over all elements of each orbit:
    
      
  $$ |S| = |G|
  \sum_{\mathcal{O}\in X/G} \sum_{x\in \mathcal{O}} \frac{1}{|\mathcal{O}|}= |G|\sum_{\mathcal{O}\in X/G} 1 = |G||X/G|.
  $$
  % \end{align*}
  Hence
  \[|G||X/G| = \sum_{g\in G} |X^g|,\]
  as required.
\end{proof}


Recall that an orbit of a G-set is a collection of elements that can be transformed into each other by the group action. In other words, it is a set of points that become
indistinguishable under the symmetries of the set under the group action. This is important as it allows us to count each distinct arrangement only once, omitting any symmetrical equivalents. 
\begin{remark}
    An important application of group actions and orbit counting is simplifying counting problems. The next example demonstrates this by counting the distinct ways to color a triangle, considering its symmetries.
\end{remark}
\begin{example}
  How many essentially different ways are there of coloring the sides of a triangle using
  four colours?  Essentially different in that one coloring cannot be obtained from the
  other by applying symmetries of the triangle.
\end{example}
\begin{proof}
  This can be done using non-Burnside's Lemma. Let $X$ be the set of all genuinely different colourings for the triangle sitting still. Then $|X|=4^3=64$ as we have four colours to choose from for each side. 
  Consider the group $D_6$ acting on $X$. Two colourings are essentially different if one cannot be obtained from the
  other by applying symmetries of the triangle, i.e. if they lie in different orbits under
  the action of $D_6$. So the number of essentially different colourings is just the number of orbits, $|X/G|$, which is given by
  \[|X/G|=\frac{1}{|G|}\sum_{\sigma\in D_6}|X^{\sigma}|.\]

    We will now count $|X^{\sigma}|$ for each $\sigma \in D_6.$ In other words, we are counting the number of triangle colourings that are fixed by each $\sigma \in D_6.$ 
  
    First consider $\sigma=e.$ The identity fixes all $x\in X,$ so $|X^e|=|X|=64$.
    
    Next, consider $\sigma$ a reflection through an axis of symmetry. For the triangle to be fixed, we require the two faces we're swapping to be the same colour. We therefore have 4 choices in the first face, then the opposite w.r.t. the axis of symmetry is decided (no choice), and we have 4 choices for the third face. Hence $|X^{\sigma}|=16$. Note that we have three such $\sigma \in D_6$ corresponding to each axis of symmetry so we count $3\cdot 16=48$.

    Finally, consider $\sigma$ a (single) rotation. For the triangle to stay the same, we require all 3 sides to be the same; we have one choice from four colors. Therefore, $|X^{\sigma}|=4$. We have 2 different rotations (the third rotation in the triangle is just the identity), so we count $2\cdot 4= 8$.
  
  Then, by Burnside's,
  \[|X/G|=\frac{1}{6}(64+8+48) = 20.\]
\end{proof}
