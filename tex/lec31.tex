\section{Lecture 31}
\subsection{Field Extensions and Degrees}
\begin{definition}
  Let $F/K$ be a field extension and $\alpha\in F$ be algebraic over $K$. The
  \emph{degree of $\alpha$ over $K$}, written $\deg(\alpha, K)$, is the degree of the
  polynomial $\irr(\alpha, K)$.
  The \emph{degree of $F$ over $K$ }, written $[F:K]$ is the dimension of $F$ as a vector
  space over $K$. An extension $F/K$ is finite if $[F:K]<\infty$.
\end{definition}


% Let $F/K$ be a field extension. Consider the evaluation map $\sigma_c:K[x]\to K[x]$ defined by $\sigma_c(k(x))=k(c).$ This function can be showed to be a homomorphism with image a subfield of $F$. In fact, it is the smallest field containing both $K$ and $c.$ 


\begin{definition}
  Let $K(\alpha)$ denote the smallest subfield of $F$ that contain $K$ and $\alpha$, the
  field generated by $\alpha$ over $K.$
\end{definition}

It can be shown that $K(\alpha)$ is exactly the range of the evaluation map at $\alpha$ for polynomials over $K$. More precisely, 
$K(\alpha) = \{f(\alpha) \mid f \in K[X]\}$. We use this fact to prove the following theorem.

\begin{theorem}
  Let $F/K$ be a field extension, $\alpha\in F$ be algebraic over $K$. Then
  $[K(\alpha):K]=\deg (\alpha, K)$. More precisely, every element of $K(\alpha)$ can be
  uniquely written as $b_0+b_1\alpha +\cdots b_{d-1}\alpha^{d-1}$ where $d=\deg (\alpha,
  K)$ and $b_i\in K$. In other words, $1, \alpha, \ldots, \alpha^{d-1}$ is a basis for
  $K(\alpha)$ over $K$ as a vector space.
\end{theorem}
\begin{proof}
  % For uniqueness, suppose $\exists b_i,b_i'\in K$ s.t. 
  % \[b_0+b_1\alpha+\cdots+b_{d-1}\alpha^{d-1}=b_0'+\cdots+b_{d-1}'\alpha^{d-1}\]
  % And note that then $\alpha$ is a root of the polynomial
  % $(b_0-b_1)+\cdots+(b_{d-1}-b_{d-1}')\alpha^{d-1}$. Since $d=\deg(\alpha,K)$ it
  % follows that $b_i=b_i'$ for all $i$.

  % For existence, suppose $\irr(\alpha, K)=x^d+c_{d-1}x^{d-1}+\cdots+c_0$. Then
  % $\alpha^d=-(c_{d-1}\alpha^{d-1}+\cdots +c_0)$, so it is spanned by the stated basis.
  % Similarly, $\alpha^{d+1}=-(c_{d-1}\alpha^{d} + c_{d-2}\alpha^{d-1}+\cdots+c_0\alpha)$,
  % and note that since we can write $\alpha^d$ as a linear combination of
  % $\alpha^{d-1},\cdots, 1$, so can be $\alpha^{d+1}$. Inductively, this is true for any
  % power of $\alpha$, and so the span of $\langle 1, \alpha, \cdots, \alpha^{d-1}\rangle$
  % is closed under multiplication (and obviously closed under addition too because it's a
  % linear combination). It remains to show that it is closed under multiplicative inverses.
  % Consider the ring homomorphism $K[X]\to F:f(x)\mapsto f(\alpha)$, where the image is
  % $\langle 1,\alpha, \cdots, \alpha^{d-1}\rangle$, and the kernel consists of the
  % multiples of the minimal polynomial, i.e. $(\irr(\alpha, K))$. By the first isomorphism
  % theorem of rings we have $K[X]/(\irr(\alpha,K))\cong \langle 1,\alpha, \cdots,
  % \alpha^{d-1}\rangle$, which is a field since $(\irr(\alpha, K))$ is maximal.

   %  First, to show uniqueness,  suppose $\exists b_i,b_i'\in K$ s.t. 
   % $$b_0+b_1\alpha+\cdots+b_{d-1}\alpha^{d-1}=b_0'+b'_1\alpha \cdots+b_{d-1}'\alpha^{d-1}.$$ We then have,
   % $$(b_0-b'_0)+(b_1-b'_1)\alpha+\cdots+(b_{d-1}-b'_{d-1})\alpha ^{d-1} =0,$$
   % so, $b_0=b'_0$, $b_1=b'_1$, \ldots, $b_{d-1}=b'_{d-1}.$

   %  To show the set $\{1, \alpha, \ldots, \alpha^n\}$ is a basis for $F/K$, we must show it is linearly independent and spans $F/K$.



    %proof of spans
     Let $f \in K$. Since $K(\alpha)$ is the range of the evaluation map, $f(\alpha)$ is an arbitrary element of $K(\alpha)$. By the division algorithm, $\exists ! \: q,r \in K$ s.t. $f = q \cdot \irr(\alpha, K) + r,$ where $\deg(r) < \deg(\alpha, K) = d,$ or $r = 0.$
    

   Evaluating at $\alpha$, we have that $f(\alpha) = 0 + r(\alpha),$ i.e. $\exists \: b_i \in K$ s.t. $f(\alpha) = b_0+b_1\alpha +\cdots + b_{d-1}\alpha^{d-1}.$ Hence $f(\alpha)$ can be expressed as a linear combination of $1,\alpha, \ldots,\alpha^{d-1}$, so the set spans $K(\alpha)$.

   %lin indep
   Suppose $b_0 + b_1\alpha + \ldots + b_{d-1}\alpha^{d-1} = 0,$ where not all $b_i$ are zero. This is saying that $\alpha$ is a root of a polynomial of degree at most $d-1 < d$, which contradicts the assumption that $d = \deg(\alpha, K),$ so the set is linearly independent.
   %bosh

   Suppose two distinct linear combinations were equal, i.e. $\exists b_i,b_i'\in K$ s.t. 
   $$b_0+b_1\alpha+\cdots+b_{d-1}\alpha^{d-1}=b_0'+b'_1\alpha \cdots+b_{d-1}'\alpha^{d-1}.$$
   This would mean that
   $$(b_0-b'_0)+(b_1-b'_1)\alpha+\cdots+(b_{d-1}-b'_{d-1})\alpha ^{d-1} =0,$$
   so $b_0=b'_0$, $b_1=b'_1$, \ldots, $b_{d-1}=b'_{d-1},$ since we have linear independence.
\end{proof}
