\section{Lecture 30}
\subsection{Field Extensions}
\begin{definition}[Field Extension]
  Suppose $K$ is a field, and let $F$ be a field containing $K$. Then $F/K$ is a
  \emph{field extension}.
  \label{def:fieldExt}
\end{definition}

\begin{remark}
% In algebra we don't usually refer to ``extensions'' in the context of rings or groups (when we usually look inward to ideals and subgroups) in the same way we do with fields. 
With groups and rings, typically we find it useful to look inward to substructures (subgroups and ideals), however the interest with fields is to look outward to extensions.
This is because if $F/K$ is a field extension, we can view $F$ as a vector space over $K.$ The notation $F/K$ does not imply a quotient like it might in other algebraic structures. Instead, it is simply a way to describe the field $F$ as a structure that contains $K$ and has a vector space structure relative to $K$.
\end{remark}

\begin{example}
The following is a familar example of a field extension:
  $\CC/\RR$ -- a $2$-dim vector space over $\RR$ (the \emph{complex plane}).

This is a more interesting example to think about:
  $\RR/\QQ$ -- an $\infty$-dim (uncountably) vector space over $\QQ$.
\end{example}

\begin{example}
 A more useful definition of a field extension comes from considering embeddings into bigger fields and not just actual set containment.
 
  Let $K$ be a field and consider an irreducible polynomial $f\in K[X]$. Then we can define $F=K[X]/(f)$; this is a field. 

  Note that $K$ is not contained in $F$ in the regular notion of set containment, but it can be embedded into $F.$
  
  Consider the quotient map $\phi:K[X]\to K[X]/(f)$, $g\mapsto g+(f)$. Observe that
  $K$ is actually a subring of $K[X]$ (constant polynomials).

  Consider the map $\phi$ restricted to $K$, $\phi |_{K}.$
  An irreducible polynomial has degree at least 2, therefore, for $\phi(k)=k+(f)=0,$ $k$ must have degree at least 2. So, $\phi |_{K}$ is injective as it consists only of degree 0 polynomials (and zero).

  We can conclude that $F/K$ is a field extension.
\end{example}

\begin{remark}
    Notice that in the example above, we do not have that the set $K$ is actually contained in $F,$ but rather that it can be naturally embedded into $F.$ Because of this, we can still view $F/K$ as a field extension.
\end{remark}

\begin{example}
  Consider $K=\RR$ and $f=X^2+1\in \RR[X]$, which is irreducible. Define $F=\RR[X]/(f)$
  which is a $2$-dimensional field extension of $\RR$ with basis $1+(X^2+1), X+(X^2+1)$. Let $\alpha$ be the image
  of $X$ under the previous quotient map
  \begin{align*}
      \phi: K[X] &\to K[X]/(X^2+1) \\
      k &\mapsto k + (X^2+1).
  \end{align*}
  % $$K[X]\to K[X]/(X^2+1).$$ 
  Notice $\alpha^2+1=0 + (X^2+1)$ in
  $F$, so $\alpha$ satisfies $g(t)=t^2+1\in K[X]$. Thus $F$ is isomorphic to $\CC$.
\end{example}

\begin{theorem}
  Let $K$ be a field and let $f\in K[X]$ with $\deg f>0$. Then there exists a field
  extension $F/K$ and $\alpha \in F$ s.t. $f(\alpha)=0$.
\end{theorem}
\begin{proof}
  Let $f=gh$ where $g$ is irreducible. Define $F=K[X]/(g)$. Then take $\alpha=X+(g)$, and so $g(\alpha)=g(X)+(g)=0+(g)$, so $f(\alpha) = g(\alpha)h(\alpha)=0$.
\end{proof}

\begin{definition}
  Let $F/K$ be a field extension, and $\alpha\in F$. We say $\alpha$ is \emph{algebraic} over
  $K$ if there exists $f\in K[X]\setminus \{0\}$ s.t. $f(\alpha)=0$. If there is no such
  polynomial, we say that $\alpha$ is \emph{transcendental} over $K$.
\end{definition}

\begin{example}
  For all $d\in \QQ$, $\sqrt{d}\in\CC$ is algebraic over $\QQ$, consider
  $f(x)=x^2-d\in\QQ[X]$.

  For all $n\in\ZZ_{>0}$, $e^{2\pi i/n}\in \CC$ is algebraic over $\QQ$, consider
  $f(x)=x^n-1$.

  Transcendental elements of $\CC$ include $e$ and $\pi$, but proving they are transcendental is very difficult.
\end{example}

\begin{theorem}
  Let $F/K$ be a field extension, and let $\alpha\in F$ be algebraic over $K$.
  Then there exists an irreducible polynomial $p\in K[X]$ with $p(\alpha)=0$, and for
  all $f\in K[X]$ with $f(\alpha)=0$, we have $p$ divides $f$ in $K[X]$. \label{thm:pdivf}
\end{theorem}
\begin{proof}
  First, $\alpha$ being algebraic over $K$ implies that there exists $f\in K[X]\setminus \{0\}$ with $f(\alpha)=0$. Let $p\in K[X]$ be such an $f$ with the smallest degree, so $p(\alpha)=0.$ Assume for contradiction that $p$ is reducible, i.e., there exists $g,h\in\K[X]$ with degrees $r,s<n$ respectively, such that $f=gh.$ This means that $f(\alpha)=g(\alpha)h(\alpha)=0,$ so we must have $g(\alpha)=0$ or $h(\alpha)=0.$ This is a contradiction as we have assumed that $p$ is the smallest such polynomial, therefore $p$ must be irreducible over $K[X].$

  Now suppose that $f \in K[X]$ with $f(\alpha) = 0$, by the division algorithm we can write $f=pq+r$ where $\deg r < \deg p$. We have 2 cases, if $r=0$ then $f = pq$ so $p$ divides $f$. Assume $r \neq 0$ then we have that $r(\alpha) = f(\alpha) - p(\alpha)q(\alpha) = 0$, but as $p$ was minimal degree, it follows that $r = 0$, and so $f=pq$ as required.
\end{proof}

% We call such a $p \in K[X]$ the minimum polynomial of $\alpha$ over $K$.

\begin{corollary}
  Let $F/K$ be a field extension, and let $\alpha\in F$ be algebraic over $K$. Let
  $p,p'\in K[X]$ be irreducible polynomials s.t. $p(\alpha)=p'(\alpha)=0$. Then
  $\exists\lambda\in K$ s.t. $p=\lambda p'$.
\end{corollary}
\begin{proof}
  We have from theorem \ref{thm:pdivf} that $p \mid p'$ and $p' \mid p$. This implies $\deg p=\deg p'$, and the result follows.
\end{proof}

\begin{definition}
  A polynomial $f=a_nx^n+\cdots+a_0\in K[X]$ is called \emph{monic} if $a_n=1$.
  Let $F/K$ be an extension field and $\alpha$ be algebraic over $K$. The unique monic
  irreducible polynomial $p\in K[X]$ s.t. $p(\alpha)=0$ is called the irreducible
  polynomial of $\alpha$ over $K$, or the minimal polynomial of $\alpha$ over $K$, written
  as $\irr (\alpha, K)$.
\end{definition}

\begin{example}
  Recall that if $n\in\ZZ$, then $e^{2\pi i/n}\in\CC$ is algebraic over $\QQ$ since it is a
  root of $x^n-1$.
  
  Let $\alpha =e^{2\pi i/n}$ and suppose that $n$ is prime. Then $\irr (\alpha, K)=
  X^{n-1}+X^{n-2}+\cdots+1$. This polynmial is known to be irreducible over the rationals when $n$ is prime, and it has root $\alpha.$ We can show this by Eisenstein's criterion after using translation by $1$, as in Example \ref{eisensteinex}.
\end{example}


