\section{Lecture 30 - 29 Nov 2021}
\subsection{Field Extensions (Non-examinable from now on)}
\begin{definition}[Field Extension]
  Suppose $K$ is a field, and let $F$ be a field containing $K$. Then $F/K$ is a
  \emph{field extension}.
  \label{def:fieldExt}
\end{definition}
We don't talk about ring extensions or group extensions. Why is it
different with fields? Note that if $F/K$ is a field extension, then $F$ can be viewed as
a vector space over $K$. As a note of warning, the notation $F/K$ does not make reference
to the quotient, but rather is just a formal expression to denote the $F$ over $K$, a
field $F$ that behaves like a vector field of some dimension over a smaller (embedded)
field $K$.

\begin{example}
  $\CC/\RR$ -- a $2$-dim vector space over $\RR$ (the \emph{complex plane}).
  
  $\RR/\QQ$ -- an $\infty$-dim (uncountably) vector space over $\QQ$.
\end{example}
\begin{example}
  Let $K$ be a field and consider an irreducible polynomial $f\in K[X]$. Then we can
  define $F=K[X]/(f)$, and we find that $F/K$ is a field extension. But why is $K$
  contained in $F$? Consider the quotient map $\phi:K[X]\to K[X]/(f)$, and observe that
  $K$ is actually a subring of $K[X]$ (constant polynomials), where $\phi$ when restricted
  to $K$ is injective. In fact, observe that for it to not be injective, we'd need some
  element of $K$ to map to the $0$ polynomial, but for this to be the case that element
  would need to be a polynomial of degree at least $1$, which is impossible, hence the map
  is injective (only the $0$ polynomial is in the kernel).
\end{example}
\begin{example}
  Consider $K=\RR$ and $f(x)=X^2+1\in \RR[X]$ which is irreducible. Define $F=\RR[X]/(f)$
  which is a $2$-dim extension field of $\RR$ with basis $1, X$. Let $\alpha$ be the image
  of $X$ under the previous quotient map $K[X]\to K[X]/(X^2+1)$. Notice $\alpha^2+1=0$ in
  $F$, so $\alpha$ satisfies $g(t)=t^2+1\in K[X]$. This $F$ is isomorphic to $\CC$.
\end{example}

\begin{theorem}
  Let $K$ be a field and let $f\in K[X]$ with $\deg f>0$. Then there exists a field
  extension $F/K$ and $\alpha \in F$ s.t. $f(\alpha)=0$.
\end{theorem}
\begin{proof}
  Let $f=gh$ where $g$ is irreducible. Define $F=K[X]/(g)$. Then take $\alpha=X+(g)$, and
  so $g(\alpha)=g(X)+(g)=0+(g)$, so $f(\alpha)=g(\alpha)h(\alpha)=0$.
\end{proof}

\begin{definition}
  Let $F/K$ be a field extension, and $\alpha\in F$. We say $\alpha$ is algebraic over
  $K$ if there exists $f\in K[X]\setminus \{0\}$ s.t. $f(\alpha)=0$. If there's not such
  polynomial, we say that $\alpha$ is trascendental over $K$.
\end{definition}

\begin{example}
  For all $d\in \QQ$, $\sqrt{d}\in\CC$ is algebraic over $\QQ$, consider
  $f(x)=x^2-d\in\QQ[X]$.

  For all $n\in\ZZ_{>0}$, $e^{2\pi i/n}\in \CC$ is algebraic over $\QQ$, consider
  $f(x)=x^n-1$.

  Trascendental elements of $\CC$ include $e$ and $\pi$, but proving this is very
  difficult.
\end{example}

\begin{theorem}
  Let $F/K$ be a field extension, and let $\alpha\in F$ be algebraic over $K$.
  Then there exists an irreducible polynomial $p\in K[X]$ with $p(\alpha)=0$, and for
  all $f\in K[X]$ with $f(\alpha)=0$, we have $p$ divides $f$ in $K[X]$.
\end{theorem}
\begin{proof}
  For part 1 $\alpha$ being algebraic over $K$ implies that there exists $f\in K[X]$
  non-zero with $f(\alpha)=0$. Let $p\in K[X]$ have minimal degree over all such $f$s, and
  we show that $p$ is irreducible.

  For part 2, suppose that $f\in K[X]$ with $f(\alpha)=0$, then write $f=pq+r$ s.t. $r=0$
  or $\deg r <\deg p$, in which case then show that
  $r(\alpha)=f(\alpha)-pq(\alpha)=0\implies r=0$ by minimality of $p$.
\end{proof}

% We call such a $p \in K[X]$ the minimum polynomial of $\alpha$ over $K$.

\begin{corollary}
  Let $F/K$ be a field extension, and let $\alpha\in F$ be algebraic over $K$. Let
  $p,p'\in K[X]$ be irreducible polynomials s.t. $p(\alpha)=p'(\alpha)=0$. Then
  $\exists\lambda\in K$ s.t. $p=\lambda p'$.
\end{corollary}
\begin{proof}
  We have $p|p'$ and $p'|p$, and the result follows since this implies $\deg p=\deg p'$.
\end{proof}

\begin{definition}
  A polynomial $f=a_nx^n+\cdots+a_0\in K[X]$ is called \emph{monic} if $a_n=1$.
  Let $F/K$ be an extension field and $\alpha$ be algebraic over $K$. The unique monic
  irreducible polynomial $p\in K[X]$ s.t. $p(\alpha)=0$ is called the irreducible
  polynomial of $\alpha$ over $K$, or the minimal polynomial of $\alpha$ over $K$, written
  as $\irr (\alpha, K)$.
\end{definition}

\begin{example}
  Recall that if $n\in\ZZ$, then $e^{2\pi i/n}\in\CC$ is algebraic over $\QQ$ since it's a
  root of $x^n-1$. Suppose $n$ is a prime. Then $\irr (\alpha, K)=
  X^{n-1}+X^{n-2}+\cdots+1$, by Eisenstein's criterion using translation by $1$.
\end{example}

\begin{exercise}
  What is the minimum polynomial of $e^{2\pi i /6}$ and $e^{2\pi i /8}$ over $\QQ$?
\end{exercise}

