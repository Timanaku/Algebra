\section{Lecture 20}
\subsection{Ring homomorphisms}
\begin{example}
  Let $R = \RR[X]$ and consider the subring $I=(X^2+1)R$, which is a two-sided ideal. It can be shown that every coset $f+I$ contains a unique polynomial of the form $a_0+a_1x$. It
  follows that $R/I$ is a 2-dimensional vector space $\RR$ spanned by $\hat{1}=1+I$,
  $\hat{X}=X+I$. We have the additional property that $\hat{X}^2 = -\hat{1},$ since $\hat{X}^2= X^2 + I = (X^2+1)+I - (1+I)$. 
  % (why? think about it in terms
  % of sets, the answer follows), 
  % so $\hat{X}^2=I-(1+I)= -\hat{1}$.
\end{example}

\begin{definition}
  Let $R,\text{ }S$ be rings. A \emph{ring homomorphism} from $R$ to $S$ is a function $\phi:R\to S$
  such that for all $a,\text{ }b\in R$ we have 
  \begin{enumerate}
    \item $\phi(a+b)=\phi(a)+\phi(b);$ 
    \item $\phi(ab)=\phi(a)\phi(b)$. 
    
    %(note no necessary a group homomorphism bc $R$ is not
      %a multiplicative groups, but it must respect the operation).
  \end{enumerate}

  If $R$ and $S$ are unital, then a ring homomorphism is called \emph{unital} if $\phi(1_R)=1_S$. Unless otherwise stated, a homomorphism between unital rings will be assumed to be unital.

  \end{definition}
  
  \begin{remark}
     A ring homomorphism is a homomorphism of additive groups. It is not necessarily a homomorphism of multiplicative groups (as rings are not groups under multiplication in general), but it must still respect the operation. %bosh
  \end{remark}
  
  
\begin{definition}
  We say that a ring homomorphism is an \emph{isomorphism} if it has a two-sided inverse
  $\phi\inv:S\to R$, also a ring homomorphism. Symbolically, we have 
  \[\phi \circ \phi\inv = \id_S \quad \phi\inv\circ\phi = \id_R.\]
  \label{def:ringHomUnital}
\end{definition}

\begin{theorem}
  A ring homomorphism is an isomorphism if and only if it is bijective.
  \label{<+label+>}
\end{theorem}
\begin{proof}
  Clearly any isomorphism must be bijective, so there is nothing to check in the forward direction --- we must only check that a bijective ring homomorphism is necessarily an isomorphism.
  
  ($\impliedby$) We have that if $\phi:R\to S$ is a bijective ring homomorphism then $\phi\inv$ is also a group
  homomorphism of addition; it remains to check multiplication is preserved.

  We claim that $\phi\inv$ also preserves multiplication, i.e.
  $\forall g,h\in S$, $$\phi\inv(gh)=\phi\inv(g)\phi\inv(h).$$
  If $g,h \in S,$ then by the surjectivity of $\phi$ we have $x,y \in R$ such that $g =\phi(x)$ and $h=\phi(y)$. Then
  $$
  \phi\inv(gh) = \phi\inv(\phi(x)\phi(y)) = \phi\inv(\phi(xy)) = xy = \phi\inv(g) \phi\inv(h),
  $$
  as required.
  % Note that we have
  % $g=\phi(x),h=\phi(y)$ for some $x,y\in R$, hence it follows that
  % $gh=\phi(x)\phi(y)=\phi(xy)$. Note that $x=\phi\inv(g),y=\phi\inv(h)$, hence
  % $\phi\inv(gh)=\phi\inv(\phi(xy))=xy$ and $xy=\phi\inv(g)\phi\inv(h)$, hence it follows
  % that $\phi\inv(gh)=\phi\inv(g)\phi\inv(h)$.
  % The reverse direction is trivial. 
\end{proof}

\begin{example}
  Let $R$ be a unital ring and $I$ be a proper two-sided ideal. The quotient map $R\to
  R/I$,  $r\mapsto r+I$ is a ring homomorphism.

  % Another example is for a commutative ring $R$. Let $r\in R$. The evaluation map
  % $\phi_r:R[X]\to R$, $f\mapsto f(r)$ is a ring homomorphism.
 
  Recall that for $I=(X^2+1)\RR[X]$, we had the
  property that $\hat{X}^2=-\hat{1}$, hence it might not be surprising that
  $\RR[X]/I\to\CC$ sending $\hat{X}$ to $i\in\CC$ is a ring isomorphism. 
\end{example}

\begin{definition}
%[Evaluation Map]
    Let $R\leq S$ be rings and $s \in S$. Then the \emph{evaluation map} at $s$ is defined by 
    \begin{align*}
        \phi_{s}: R[X] &\to S \\
        f &\mapsto f(s).
    \end{align*}
    If $f = a_0 + a_1X + \cdots + a_dX^d$, then we say $f(s) = a_0 + a_1s + \cdots + a_ds^d$.
\end{definition}

\begin{remark}
    If $R$ is commutative, then the evaluation map is a ring homomorphism. We will see that this is a common way to induce ring isomorphisms.
\end{remark}
