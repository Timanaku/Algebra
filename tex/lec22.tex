\section{Lecture 22 - 10 Nov 2021}
\subsection{Integral domains}
\begin{definition}
  A commutative unital ring with no zero divisors is called an \emph{integral domain}.
  \label{def:integralDomain}
\end{definition}

\begin{theorem}
  Let $R$ be a unital ring. Let $u\in R$ be a \emph{unit} (recall then $\exists u'\in R$
  s.t. $u'u=uu'=e_R$). Then $u$ is is not a zero divisor.
  \label{<+label+>}
\end{theorem}
\begin{proof}
  Let $u$ be a unit and assume there exists $v\in R$ s.t. $uv=0$ or $vu=0$. Since $u$ is a unit, $\exists u^{-1}$ s.t. $u^{-1} u=1$. We have,
  $$uv=0 \implies u^{-1}uv=u^{-1}0=0\implies 1v=0 \implies v=0.$$ Similarly, $vu=0 \implies v=0$. Therefore, $u$ is not a zero divisor.
\end{proof}

\begin{corollary}
  Every field is an integral domain
  \label{<+label+>}
\end{corollary}
\begin{proof}
  Recall that a field is a commutative division unital ring. Hence the result follows by
  the above theorem.
\end{proof}
\begin{remark}
  The converse is not true. An example of an integral domain which is not a field is
  $\ZZ$.
  \label{<+label+>}
\end{remark}

\begin{theorem}
  A finite integral domain is a field.
\end{theorem}
\begin{proof}
  It suffices to show that every element of a finite integral
  domain has an inverse. Let $R$ be a finite integral domain, and let $r\in R\setminus\{0\}$. Consider the set $\{r^n \mid n\in\NN\}\subseteq R$. Since $R$ is
  finite, there exists $k\in\NN_{\geq 2}$ s.t. $r^k=r$, so $r(r^{k-1}-1)=0$. Since
  $R$ is an integral domain, we must have $r^{k-1}-1 = 0$ since $r \neq 0.$ Equivalently, we can say that $r^{k-2}r=1$, so $r$ is invertible.
  \end{proof}

\begin{corollary}
  Let $p$ be a prime number. Then $\ZZ/p\ZZ$ is a field.
  \label{<+label+>}
\end{corollary}
\begin{proof}
  Since $\ZZ/p\ZZ$ is finite, it suffices to show that it is an integral domain. Let
  $a+p\ZZ, b+p\ZZ\in \ZZ/p\ZZ$. Then observe that $ab+p\ZZ=0+p\ZZ$ if and only if $ab\in
  p\ZZ$, thus either $p \mid a $ or $p \mid b$ i.e. $a \in p\ZZ$ or $b \in p\ZZ$, so either $a+p\ZZ = 0 + p\ZZ$ or $b+p\ZZ = 0 +p\ZZ $.
  % so $ab=np$ for some $n\in\ZZ$. Since $p$ is a prime, it must be the case that
  % either $a=p$, $b=p$, or $n=0$ and $b=0$ or $a=0$. Hence either $a+p\ZZ = 0+p\ZZ$ or
  % $b+p\ZZ=0+p\ZZ$. 
  Hence no zero divisor exists. Therefore $\ZZ/p\ZZ$ is an integral
  domain.
\end{proof}

This is exactly what it means to be prime, as if $ab$ is a multiple of $p$, then one of $a$ or $b$ must be a multiple of $p$. We generalise this definition below in terms of ideals.

\begin{definition}[Prime Ideal]
  Let $R$ be a ring. An ideal $I$ of $R$ is called \emph{prime} if $I$ is a proper ideal,
  and whenever $a,b\in R$ are s.t. $ab\in I$, then one has $a\in I$ or $b\in I$.
  \label{<+label+>}
\end{definition}

\begin{definition}[Maximal ideal]
  Let $R$ be a ring. An ideal $I$ of $R$ is called \emph{maximal ideal} if $I$ is a proper
  ideal and for every other ideal $J$ s.t. $I\subseteq J \subseteq R$ we have that $J=I$
  or $J=R$.
  \label{def:maximalIdeal}
\end{definition}


