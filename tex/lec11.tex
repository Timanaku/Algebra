\section{Lecture 11 - 13 Oct 2021}
\subsection{Second Isomorphism Theorem}
On Isomorphism theorems. We will continue applying the FIT. First we will apply it to the
second isomorphism theorem.
\begin{theorem}[Second Isomorphism Theorem]
  Let $G$ be a group, $H<G$, let $N\trianglelefteq G$. Then,
  \begin{enumerate}
    \item The set $HN=\left\{ hn |h\in H,n\in N \right\} <G$ .
    \item The intesection $H\cap N \trianglelefteq H$, not necessarily of $G$.
    \item There is an isomorphism $H/H\cap N \cong HN/N$.
  \end{enumerate}
  \label{secondIso}
\end{theorem}
\begin{proof}
  \todo{Part 1 and 2 are exercise sheet 2, q 2}
  To prove the third point, we use the FIT. We claim there exists an homomorphism $\phi$
  where $\Img\phi= HN/N$ and $\ker\phi = H\cap N$. Consider the composition $\phi:H\xhookrightarrow{}
  G\to G/N: h\mapsto hN$. Note that the operation is a homomorphism because it's a
  composition of two homomorphisms: the embedding of $H$ in $G$ and the quotient map.
  Moreover note that $h\in\ker\phi \iff h\in H \text{ and } hN=N \iff h\in H \text{ and } h\in N$, i.e.
  $h\in H\cap N$. On the other hand, note that $xN\in\Img\phi \iff \exists h\in H : hN=xN
  \iff  h^{-1}x\in N \iff x = hn \in HN$.
  Therefore $\Img\phi = HN/N$, and by the FIT it follows
  that $H/H\cap N \cong HN/N$. 
\end{proof}
Recall from the previous lecture that the first isomorphism theorem says that taking the
quotient of $G$ by some normal subgroup $N$ is equivalent to applying a homomorphism $\phi$
from $G$ whose kernel is $N$.

The second isomorphism theorem addresses the question: What if I apply the same $\phi$ to
a subgroup $H$ of $G$? 

\begin{example}
  Let $GL_2(\RR)$ be a multiplicative group. Let $SL_2(\RR)\leq GL_2(\RR)$ be the subgroup
  of matrices with determinant $1$. This is a normal subgroup. What's the structure of
  $GL_2(\RR)/SL_2(\RR)$? By the first isom theorem we have that $SL_2(\RR)$ must be the
  kernel of some group homomorphism, namely the determinant map. The image of the
  determinant is all $\RR^{\times}$, so by the FIT, we deduce that
  $GL_2(\RR)/SL_2(\RR)\cong \RR^{\times}$. Consider the subgroup
  \[Z= \left\{ \diag (a,a) : a\in \RR^{\times} \right\} \leq GL_{2}(\RR)\]
  Observe that $z\in Z, |z|=a^2$, so we see that $Z\cap SL_{2}(\RR) = \left\{ \pm
  \diag(1,1) \right\}$. 

  We claim that $Z\cdot SL_2(\RR)$ is the subgroup $GL_2^+(\RR)$
  consisting of all matrices with positive determinant. The proof is left as an exercise
  to the reader (it's not too hard). By the SIT applied with $H=Z$ and $N=SL_2(\RR)$ we
  have
  \[ Z/\left\{ \pm \diag(1,1) \right\} \cong GL_2^+(\RR)/ SL_2(\RR) \cong \RR_{>0}\]

\end{example}

\subsection{Non-examinable aside}
Bartel makes a non-examinable aside to try to explain how SIT will be useful eventually.
First, he introduces the idea of \emph{atoms of group theory}. If we want to understand a
big and complicated group $G$, we can do so by finding a normal subgroup $N$ and
understand that and its quotient $G/N$ (which is in essence a group with similar structure
to $G$ but with less elements). Note that his reduction step relies on the existence of a
proper non-trivial normal subgroup. The SIT is very useful in this approach
\begin{definition}[Simple Group]
  A group is \emph{simple} if it is non-trivial and has no proper non-trivial normal
  subgroups.
\end{definition}

If we understand all simple groups, and how bigger groups are made up of smaller normal
groups and their quotients, we would understand all groups. All finite simple groups have
been classified. For instance, for any $n\geq 5$, the alternate group $A_n$ is simple.

\begin{theorem}
  Let $G$ be a metabelian group, meaning there exists an abelian normal subgroup $N$ of
  $G$ sucht that $G/N$ is also abelian. Then every subgroup of $G$ is also metabelian.
  \label{<+label+>}
\end{theorem}

\begin{proof}
    Let $H \leq G$. Since $N$ is a normal abelian subgroup of $G$, $H \cap N$ is abelian as its a subgroup of $N$. Using the SIT we have that $H/H \cap N \cong HN/N$, since $G/N$ is abelian $HN/N$ is abelian and so $H/H \cap N$ is abelian. So H has an abelian normal subgroup $H \cap N$ such that the quotient is abelian, and as $H$ was an arbitrary subgroup of $G$ every subgroup of $G$ is metabelian.
\end{proof}
