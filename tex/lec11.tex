\section{Lecture 11}
\subsection{Second Isomorphism Theorem}
\begin{theorem}[Second Isomorphism Theorem]
  Let $G$ be a group, $H \leq G$, $N\trianglelefteq G$. Then,
  \begin{enumerate}
    \item $HN=\left\{ hn \mid h\in H,n\in N \right\} \leq G$,
    \item $H\cap N \trianglelefteq H$,
    \item $H/H\cap N \cong HN/N$.
  \end{enumerate}
  \label{secondIso}
\end{theorem}
\begin{proof}
    % (1) First, we have $1_G\in HN$, so $HN$ is nonempty. Let $x,y \in HN,$ meaning we have $h_1,h_2 \in H$ and $n_1,n_2 \in N$ such that $x = h_1n_1$, $y = h_2n_2.$ Then $$xy^{-1} = (h_1n_1)(h_2n_2)^{-1} = h_1 n_1 n_2^{-1} h_2^{-1}.$$ 

    % Now if we let $h_1 = h h_2$ for some $h \in H$ (such a $h$ exists since $H$ is a group) and $n_1 n_2^{-1} = n \in N,$ we have that $$xy^{-1} = h (h_2 \cdot n \cdot h_2^{-1}) \in HN,$$ since $N \trianglelefteq G$. Therefore, $HN\leq G$ by the subgroup test. 
    
    (1) We will start by showing that $HN=NH.$ Let $hn\in HN.$ Since $N\trianglelefteq G,$ $hnh^{-1}\in N,$ so there exists $n'\in N$ such that $hnh^{-1}=n'.$ Therefore $hn=n'h,$ so $HN\subset NH.$ It can be shown similarly that $NH\subset HN,$ so $HN=NH.$ 
    
    We now show that $HN$ is a subgroup of $G$ by the subgroup test.
    We have $1_G\in HN$, so $HN$ is nonempty.
    Now let $x=h_1n_1\in HN$ and $y=h_2n_2\in HN.$ Then
    $$xy^{-1} = (h_1n_1)(h_2n_2)^{-1} = h_1 n_1 n_2^{-1} h_2^{-1}.$$ As we have shown $HN=NH,$ we can write $n_2^{-1} h_2^{-1}=\hat{h}\hat{n}$ for some $\hat{h}\hat{n} \in HN,$ so $$xy^{-1}=h_1 n_1\hat{h}\hat{n}.$$ Again, we can write $n_1\hat{h}=\tilde{h}\tilde{n}$ for some $\tilde{h}\tilde{n}\in HN,$ so $$xy^{-1}=h_1\tilde{h}\tilde{n}\hat{n}\in HN,$$ as required.

    (2) We want to show $H\cap N$ is closed under conjugation by elements of $H.$ Let $h \in H$ and $x \in H \cap N$, i.e. $x\in H$ and $x\in N$. We have $hxh^{-1}\in H$ as $H$ is a group. Also, since $N \trianglelefteq G$, $hxh^{-1} \in N$. Therefore, $hxh^{-1}\in H\cap N,$ as required.

    (3) We define 
    % \begin{align*}
    %    \varphi: H \xhookrightarrow{} & G \to G/N \\
    %    h \mapsto & h \longmapsto hN.
    % \end{align*}
    \[
    \begin{array}{cccccc}
         \varphi: & H & \xhookrightarrow{} \: G \: \to & G/N \\
         & \vin & & \vin \\
         & h & \longlonglongmapsto & hN.
    \end{array}
    \]
    
    %need homomorphism, surjective
    Since $\varphi$ is the composition of an inclusion map and a natural projection, which are both homomorphisms, it is also a homomorphism.
    %The map is surjective, since if $gN \in G/N,$
    
    We have that $\ker \varphi = \{h \in H \mid \varphi(h) = N \} = \{h \in H \mid h \in N \} = H \cap N.$

    Also, 
    \begin{align*}
        \Img \varphi &= \{\varphi(h) \mid h \in H\} \\
        &= \{gN \mid gN=\varphi(h) \text{ for some } h \in H\} \\
        &= \{gN \mid gN = hN \text{ for some } h \in H \} \\
        &= \{gN \mid h^{-1}g \in N, h\in H\} \\
        &= \{gN \mid g\in HN\} \\
        &= HN/N.
    \end{align*}

    Hence, by the first isomorphism theorem,
    $$
    H/H\cap N \cong HN/N.
    $$
  % The first and second point are exercises in sheet 2 (question 2).
  % To prove the third point, we use the FIT. We claim there exists a homomorphism $\phi$
  % where $\Img\phi= HN/N$ and $\ker\phi = H\cap N$. Consider the composition $\phi:H\xhookrightarrow{}
  % G\to G/N: h\mapsto hN$. Note that the operation is a homomorphism because it's a
  % composition of two homomorphisms: the embedding of $H$ in $G$ and the quotient map.
  % Moreover note that $h\in\ker\phi \iff h\in H \text{ and } hN=N \iff h\in H \text{ and } h\in N$, i.e.
  % $h\in H\cap N$. On the other hand, note that $xN\in\Img\phi \iff \exists h\in H : hN=xN
  % \iff  h^{-1}x\in N \iff x = hn \in HN$.
  % Therefore $\Img\phi = HN/N$, and by the FIT it follows
  % that $H/H\cap N \cong HN/N$. 
\end{proof}
% Recall from the previous lecture that the first isomorphism theorem says that taking the
% quotient of $G$ by some normal subgroup $N$ is equivalent to applying a homomorphism $\phi$
% from $G$ whose kernel is $N$.

% The second isomorphism theorem addresses the question: What if I apply the same $\phi$ to a subgroup $H$ of $G$? 

% \begin{tikzpicture}[>=latex]

% \node[above] at (2,3) {$G$};

% \fill[red!20] (0,3) rectangle (1,2);

% \fill[green!20] (0,3) rectangle (4,2.5);

% \begin{scope}
%   \clip (0,3) rectangle (4,2.5);
%   \fill[red!20, opacity=0.5] (0,3) rectangle (1,2);
% \end{scope}

% \node at (0.5,2.5) {$N$};
% \node at (3.5,2.75) {$H$};

% \draw[step=1cm, draw=black] (0,0) grid (4,3);

% \end{tikzpicture}


% \begin{tikzpicture}[>=latex]

% \fill[green!20] (0,1) rectangle (4,1.5);
% \fill[red!20, opacity=0.5] (0,1.5) rectangle (1,1);
% \node at (0.5,1.25) {$H \cap N$};

% \draw[step=1cm, draw=black] (0,1) grid (4,1.5);
% \draw (0,1) -- (4,1); 

% \end{tikzpicture}



\begin{example}
  Let $\operatorname{GL}_2(\RR)$ be the multiplicative group of invertible $2\times2$ matrices over $\RR$. Let $SL_2(\RR)\leq \operatorname{GL}_2(\RR)$ be the subgroup
  of matrices with determinant $1$. This is a normal subgroup. What is the structure of
  $\operatorname{GL}_2(\RR)/SL_2(\RR)$? 
  
  By the first isomorphism theorem we have that $\operatorname{SL}_2(\RR)$ must be the
  kernel of some group homomorphism, namely the determinant map. The image of the
  determinant is all $\RR^{\times}$, so we deduce that
$\operatorname{GL}_2(\RR)/\operatorname{SL}_2(\RR)\cong \RR^{\times}$. 
  
  Consider the subgroup
  \[Z= \left\{ \diag (a,a) \mid a\in \RR^{\times} \right\} \leq \operatorname{GL}_{2}(\RR)\]
  of diagonal matrices. We wish to understand $Z$ in the context of $\operatorname{GL}_2(\RR)/\operatorname{SL}_2(\RR)$ using the SIT.
  Observe that $\diag(a,a) \in Z$ has determinant $a^2$, so we see that $Z\cap \operatorname{SL}_{2}(\RR) = \left\{ \pm
  \diag(1,1) \right\},$ a group consisting of 2 elements. 

  The remaining component of the SIT to understand is $Z\cdot \operatorname{SL}_2(\RR).$ We claim without proof this is $\operatorname{GL}_2^+(\RR) = \{X \in \operatorname{GL}_{2}(\RR) \mid \det(X) > 0\}.$

  Applying the SIT with  $H=Z$ and $N=\operatorname{SL}_2(\RR),$ we have
  $$
  Z/\left\{ \pm \diag(1,1) \right\} \cong \operatorname{GL}_2^+(\RR)/ \operatorname{SL}_2(\RR) \cong \RR_{>0}.
  $$
  % We claim that $Z\cdot \operatorname{SL}_2(\RR)$ is the subgroup $\operatorname{GL}_2^+(\RR)$
  % consisting of all matrices with positive determinant. The proof is left as an exercise
  % to the reader (it's not too hard). By the SIT applied with $H=Z$ and $N=SL_2(\RR)$ we
  % have
  % \[ Z/\left\{ \pm \diag(1,1) \right\} \cong \operatorname{GL}_2^+(\RR)/ \operatorname{SL}_2(\RR) \cong \RR_{>0}\]
\end{example}

\subsection{Non-examinable aside}
To explain how SIT will be useful eventually, we introduce the idea of \emph{atoms of group theory}. If we want to understand a big and complicated group $G$, we can do so by finding a normal subgroup $N$ and understanding its quotient $G/N$ (which is in essence a group with similar structure
to $G$ but with less elements). Note that this reduction step relies on the existence of a proper non-trivial normal subgroup. The SIT is very useful in this approach.
\begin{definition}[Simple Group]
  A group is \emph{simple} if it is non-trivial and has no proper non-trivial normal
  subgroups.
\end{definition}

If we understand all simple groups, and how bigger groups are made up of smaller simple groups and their quotients, we would understand all groups. All finite simple groups have
been classified. 

\begin{definition}
    A group $G$ is called \emph{metabelian} if it has a normal subgroup $N$ such that both $N$ and $G/N$ are abelian.
\end{definition}
\begin{theorem}
  Let $G$ be a metabelian group. Then every subgroup of $G$ is also metabelian.
  \label{<+label+>}
\end{theorem}

\begin{proof}
   Let $G$ be metabelian, let $N$ be a normal subgroup such that $N$ and $G/N$ are abelian, and let $H$ be an arbitrary subgroup of $G$. We claim that $H$ is also metabelian. Indeed, $H \cap N$ is a normal subgroup of $H$, and it is abelian, since it is a subgroup of the abelian group $N$; moreover, by SIT we have $H/(H \cap N) \cong HN/N \leq G/N,$ so $H/(H \cap N)$ is abelian, being (isomorphic to) a subgroup of an abelian group. Thus $H \cap N$ is the required normal subgroup of $H$.
\end{proof}
