\section{Lecture 23 - 12 Nov 2021}
\subsection{Prime and maximal ideals}
\begin{definition}
  Let $R$ be a ring, and $r\in R$. The set $Rr=\{xr : x\in R\}$ is called the \emph{left
  ideal generated by $r$}, and this is the smallest left sided ideal containing $r$. The
  \emph{two sided ideal generated by $r$} is
  \[(r)=RrR= \{\text{finite sums }\sum_i x_i r y_i : x_i,y_i\in R\}\]
  This is the smallest two-sided ideal containing $r$. 
   Moreover, if $S$ is a subset of $R$, then
  the ideal generated by $S$ is
  \[(S)=RSR = \{\text{finite sums }\sum_{s\in S} x_s s y_s : x_s,y_s\in R \}\]
\end{definition}

\begin{definition}
     Let $I\unlhd R.$ Then $I$ is called a principle ideal if there exists an $r\in R$ such that $I=(r).$
  
\end{definition}
\begin{example}
  Every ideal of $\ZZ$ is principal. Let $x,n\in \ZZ$. Then $x\in(n)=n\ZZ$ iff $n\mid x$. We
  deduce that $(n)$ is a prime ideal iff $\pm n$ is a prime number. Moreover, note that
  $(n)$ is prime if and only if it is maximal.
\end{example}

\begin{theorem}
  Let $R$ be a ring and let $I,J$ be ideals of $R$. Then $I+J=\{i+j\mid i\in I, j\in J\}$ is
  an ideal of $R$. Furthermore, it is the smallest ideal containing $I$ and $J$.
  \label{<+label+>}
\end{theorem}
\begin{proof}
  Let $x,y\in I+J$, so there exists $i,i'\in I$ and $j,j'\in J$ s.t. $x=i+j$ and $y=i'+j'$. Then $$x-y=(i+j)-(i'+j')=(i-i')+(j-j')\in I+J,$$ so $I+J$ is an additive subgroup by the subgroup test. For $r\in R$, we have that
  $$r(i+j)=ri+rj\in I+J,$$
  since $I$ and $J$ are ideals.
  Therefore, $I+J$ is closed
  under left multiplication. Similarly, $I+J$ is closed under right multiplication, hence it is an ideal.
\end{proof}

\begin{theorem}
  Let $R$ be a commutative unital ring, and let $I$ be a proper ideal. Then $I$ is prime
  if and only if $R/I$ is an integral domain.
  \label{<+label+>}
\end{theorem}
\begin{proof}
  Suppose that $R/I$ is an integral domain. Let $a,b\in R$ be s.t. $(a+I)(b+I)=ab+I=0+I$,
  but since $R/I$ is an integral domain, we must have that either $a\in I$ or $b\in I$, which is
  exactly the definition of $I$ being prime.

  Conversely, assume that $I$ is prime and $ab \in I$. Let $a+I, b+I\in R/I$ then it follows that
  $ab+I=(a+I)(b+I)=0+I.$ From our assumption we have that either $a\in I$ or $ b\in I$, or equivalently, $a+I=0+I$ or $b+I=0+I$, as required.
\end{proof}

\begin{example}
  Let $R=\ZZ[X]$ and consider $I=(X)$. Then note that $R/I\cong\ZZ$ by the FIT since the
  homomorphism $f\mapsto f(0)$ has kernel $I$ and image $\ZZ$. So $I$ is a prime ideal.
\end{example}

\begin{theorem}
  Let $R$ be a commutative unital ring, and let $I$ be a proper ideal. Then $I$ is maximal
  if and only if $R/I$ is a field.
  \label{<+label+>}
\end{theorem}
\begin{proof}
  Suppose $I$ is maximal. Let $a+I\in R/I$ and assume $a\not\in I$. Consider $I+(a) \supset I$, but since $I$ is
  maximal, then $I+(a)= R$. In particular, $1\in I+(a)$, so there exists $i\in I, r\in R$
  s.t. $1=i+ra\implies ra+I=1+I \implies (r+I)(a+I)=1+I$. Hence $R/I$ is a field.

  % Suppose $R/I$ is a field. Let $I \subset J$ be an ideal. So there exists $a \in J$ which implies $a+I \eq 0 + I$. As $R/I$ is a field then there exists a $b + I \in R/I$ such that $(a+I)(b+I) = 1+I$. So there exists a $c \in I$ such that $1 = i+ab$

  Suppose $R/I$ is a field. Let $I\subset J$ be an ideal. Then there exists $b\in J$ s.t.
  $b\not\in I$, so $b+I\neq 0+I\in R/I$. Since $R/I$ is a field, $\exists b^{-1}+I\in R/I$
  s.t. $(b+I)(b^{-1}+I)=bb^{-1}+I = 1+I$. In other words, there exists $i\in I$ s.t. $bb^{-1}+i=1$. But $b\in J\implies bb^{-1}\in J$, and 
  $i\in I\subset J$, hence $bb^{-1}+i\in J\implies 1\in J$. Since for all $x\in R$ we have
  $1 x=x\in J$, this implies $J=R$.
\end{proof}
