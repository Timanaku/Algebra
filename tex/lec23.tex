\section{Lecture 23}
\subsection{Prime and maximal ideals}
\begin{definition}
  Let $R$ be a ring, and $r\in R$. The set $Rr=\{xr \mid x\in R\}$ is called the \emph{left
  ideal generated by $r$}, and this is the smallest left sided ideal containing $r$. The
  \emph{two sided ideal generated by $r$} is
  \[(r)=RrR= \{\text{finite sums }\sum_i x_i r y_i \mid x_i,y_i\in R\}.\]
  This is the smallest two-sided ideal containing $r$. 
   Moreover, if $S$ is a subset of $R$, then
  the ideal generated by $S$ is
  \[(S)=RSR = \{\text{finite sums }\sum_{s\in S} x_s s y_s \mid x_s,y_s\in R \}.\]
\end{definition}

\begin{definition}
     Let $I\unlhd R.$ Then $I$ is called a \emph{principal ideal} if there exists an $r\in R$ such that $I=(r).$
  
\end{definition}
\begin{example}
  Every ideal of $\ZZ$ is principal. This means for any ideal in $\mathbb{Z}$, there exists an integer $n$ such that the ideal can be expressed as $(n) = \{ n \cdot k \mid k \in \mathbb{Z} \}$, the set of all multiples of $n$.
  
  Furthermore, an ideal $(n)$ is a prime ideal if and only if $|n|$ is a prime number. This is because if $|n|$ is not prime, say $n = pq$ with $p,\text{ }q\in \ZZ_{>1}$ both less than $|n|$, then $pq$ belongs to $(n)$ but neither $p$ nor $q$ would be in $(n)$, contradicting the definition of a prime ideal.
  
  Also, note that $(n)$ is prime if and only if it is maximal. This is because any other proper ideal that contains $(n)$ would have to be generated by a divisor of $n,$ and if $n$ is prime, its only divisors are 1 and itself.
\end{example}

\begin{theorem}
  Let $R$ be a ring and let $I,J$ be ideals of $R$. Then $I+J=\{i+j\mid i\in I, j\in J\}$ is
  an ideal of $R$. Furthermore, it is the smallest ideal containing $I$ and $J$.
  \label{<+label+>}
\end{theorem}
\begin{proof}
  Let $x,y\in I+J$, so there exists $i,i'\in I$ and $j,j'\in J$ s.t. $x=i+j$ and $y=i'+j'$. Then $$x-y=(i+j)-(i'+j')=(i-i')+(j-j')\in I+J,$$ so $I+J$ is an additive subgroup by the subgroup test. For $r\in R$, we have that
  $$r(i+j)=ri+rj\in I+J,$$
  since $I$ and $J$ are ideals.
  Therefore, $I+J$ is closed
  under left multiplication. Similarly, $I+J$ is closed under right multiplication, hence it is an ideal.

  Let $L$ be an ideal containing $I$ and $J$. Then clearly $I+J \subseteq L$ since an ideal is closed under addition, so $I+J$ must be the smallest one.
\end{proof}

\begin{theorem}
  Let $R$ be a commutative unital ring, and let $I$ be a proper ideal. Then $I$ is prime
  if and only if $R/I$ is an integral domain.
  \label{<+label+>}
\end{theorem}
\begin{proof}
  Suppose that $R/I$ is an integral domain. Let $a,b\in R$ and suppose 
  $$(a+I)(b+I)=0+I.$$
  Since $R/I$ is an integral domain, we must have that either $a\in I$ or $b\in I$, which is exactly the definition of $I$ being prime.

  Conversely, assume that $I$ is prime and $ab \in I$ for $a,b \in R$. 
  % Let $a+I, b+I\in R/I$, 
  It follows that $$ab+I=(a+I)(b+I)=0+I,$$
  and since $I$ is prime we must have either $a \in I$ or $b \in I.$ Equivalently, $a+I=0+I$ or $b+I=0+I$, as required.
  % From our assumption we have that either $a\in I$ or $ b\in I$, or equivalently, $a+I=0+I$ or $b+I=0+I$, as required.
\end{proof}


\begin{example}
  Let $R=\ZZ[X]$ and consider $I=(X)$. Consider the evaluation map
  \begin{align*}
  \phi: \ZZ[X]&\to \ZZ
  \\ f &\mapsto f(0).
  \end{align*}
  We have that $R/I\cong\ZZ$ by the FIT since $\phi$ has kernel $I$ and image $\ZZ$. Since $\ZZ$ and therefore $R/I$ is an integral domain, $I$ is a prime ideal.
\end{example}

\begin{theorem}
  Let $R$ be a commutative unital ring, and let $I$ be a proper ideal. Then $I$ is maximal
  if and only if $R/I$ is a field.
  \label{<+label+>}
\end{theorem}
\begin{proof}
  Suppose $I$ is maximal. Let $a+I\in R/I$ and assume $a\not\in I$. Consider $I+(a) \supset I$, but since $I$ is
  maximal, then $I+(a)= R$. In particular, $1\in I+(a)$, so there exists $i\in I, r\in R$
  s.t. $1=i+ra\implies ra+I=1+I \implies (r+I)(a+I)=1+I$, meaning $a+I$ is invertible. Hence, since our choice of $a$ was arbitrary, $R/I$ is a field.

  % Suppose $R/I$ is a field. Let $I \subset J$ be an ideal. So there exists $a \in J$ which implies $a+I \eq 0 + I$. As $R/I$ is a field then there exists a $b + I \in R/I$ such that $(a+I)(b+I) = 1+I$. So there exists a $c \in I$ such that $1 = i+ab$

  Suppose $R/I$ is a field and let $J\supset I$ be an ideal of $R$. Then there exists $x\in J$ s.t.
  $x\not\in I$, so $x+I\neq 0+I\in R/I$. Since $R/I$ is a field, $$\exists y+I\in R/I
 \text{ s.t. } (x+I)(y+I)=xy+I = 1+I.$$ In other words, $\exists i\in I$ s.t. $xy+i=1$. But $x\in J\implies xy\in J$, and 
  $i\in I\subset J$, hence $xy+i\in J\implies 1\in J \implies J=R$, so $I$ is a maximal ideal.
\end{proof}


% The following is an immediate consequence of the previous theorems.
\begin{corollary}
  % Let $R$ be a commutative unital ring. Then every maximal ideal of $R$ is prime.
  Every maximal ideal in a commutative unital ring is prime.
  \label{cor:maximalPrime}
\end{corollary}
\begin{proof}
  Let $R$ be a commutative unital ring and let $I$ be a maximal ideal of $R$. Then $R/I$ is a field, so $R/I$ is an integral
  domain, hence $I$ is prime.
\end{proof}

\begin{remark}
  The converse is false in general. For example, consider $\ZZ[X]$ and the the prime ideal $(X)$. We know $\ZZ[X]/(X)\cong \ZZ$ is an integral domain, but $(X)$ is not maximal since $\ZZ$ is not a field. 
\end{remark}
\begin{example}
    Let us consider an ideal that is maximal in $\ZZ[X]$. Let
  $p$ be a prime number, so the ideal $(p,X)=\{pf+Xg \mid f,g\in\ZZ[X]\}$ is the polynomials
  where constant terms are multiples of $p$. The ideal is maximal as $\ZZ[X]/(p,X)$ is a field.
\end{example}
