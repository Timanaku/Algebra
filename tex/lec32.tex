\section{Lecture 32 - 03 Dec 2021}
\subsection{Algebraic Closure}
\begin{example}
  Consider the field extension $\QQ(2^{1/3})=\{a+b2^{1/3} + c2^{2/3} : a,b,c\in\QQ\}$.
\end{example}
\begin{theorem}
  Let $L/F$ and $F/K$ be field extensions. Then $L/K$ is finite if and only if $L/F$ and
  $F/K$ are both finite. Moreover, if this is the case, we have $[L:K]=[L:F][F:K]$. More
  precisely, if $\alpha_1,\cdots, \alpha_n$ is an $F$-basis for $L$, and $\beta_1,\cdots,
  \beta_m$ is a $K$-basis for $F$, then $\{\alpha_i\beta_j : 1\leq i\leq n, 1\leq j\leq
  m\}$ is a $K$-basis for $L$.
\end{theorem}
\begin{proof}
  TODO: Doable. Should attempt.
\end{proof}

\begin{theorem}
  Let $F/K$ be a field extension. Then the set
  \[\overline{K_F} = \left\{ \alpha\in F : \alpha\text{ is algebraic over } K\right\}\]
  Is a field.
  \label{<+label+>}
\end{theorem}
\begin{proof}
  TODO: Constructive approach: Given $\alpha$ being a root of $f$ and $\beta$ being a root
  of $g$, can we construct a polynomial that has root $\alpha+\beta$ and $\alpha\beta$?
\end{proof}<++>
Example, try $\alpha=\sqrt{2}, \beta=\sqrt{3}$. For example $\overline{\QQ_{\CC}}$ is a
huge extension field of $\QQ$, but still much smaller than $\CC$.
$\overline{K_F}$ is called the algebraic closure of $K$ in $F$.
\begin{definition}
  A field $F$ is called algebraically closed if every $f\in F[X]$ with $\deg f>0$ has a
  root in $F$.
  \label{<+label+>}
\end{definition}

\begin{theorem}[Fundamental Theorem of Algebra]
  The field $\CC$ is algebraically closed.
  \label{thm:FTAlg}
\end{theorem}
% The above theorem implies that a given complex polynomial $f$ can be written as factors
% with one root in $\CC$ and then a multiply by another smaller degree polynomial $g$, and
% recursively this implies that any polynomial over $\CC$ splits as a product of linear
% factors (the roots in $\CC$).

By the above theorem, if $f$ is a complex polynomial of degree $n$, then $f$ has a root $\alpha_1$, i.e. we can write $f$ as $(x-\alpha_1) f_1(x)$. And again, $f_1$ must have a root $\alpha_2$, so $f = (x-\alpha_1) (x-\alpha_2) f_2(x)$. Repeating inductively, we are able to factorise $f$ completely as $(x-\alpha_1)(x-\alpha_2)\cdots (x-\alpha_n)$. In other words, a polynomial of degree $n$ over $\CC$ has exactly $n$ roots in $\CC$.
\begin{theorem}
  Every field $K$ is contained in an algebraic extension (in a field extension whose
  elements are algebraic over $K$) that is algebraically closed.
  \label{<+label+>}
\end{theorem}<++>
