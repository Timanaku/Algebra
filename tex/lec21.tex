\section{Lecture 21}
\subsection{Isomorphism of rings and cancellation}

Recall that a subring $U\subset R$ is an additive subgroup of $R$ such that for all
$r,r'\in U$ we have $rr'\in U$. 

% A \emph{two-sided ideal $I$ in a ring $R$ } is an additive subgroup of $R$ such that for
% all $x\in I, r\in R$ we have $xr\in I$, $rx\in I$. Note that if $1\in I$, an ideal, then
% for every $i\in I$ and $r\in R$, then $ir\in R$, but note that if $i=1$, then $1x=x$, so
% we'd require that $I=R$, not a proper ideal. Hence, being an ideal is not a stronger
% condition than being a subgroup.



\begin{definition}
  Let $\phi:R\to S$ be a ring homomorphism. The \emph{kernel} of $\phi$ is defined as
  $\ker\phi=\left\{ r\in R \mid \phi(r)=0_S \right\}$ and the \emph{image} of $\phi$ is
  defined as $\Img\phi=\left\{ \phi(r) \mid r\in R \right\}\subseteq S.$
\end{definition}

\begin{theorem}[First Isomorphism Theorem for Rings]
  Let $\phi:R\to S$ be a ring homomorphism of unital rings. Then
  \begin{enumerate}
    \item The kernel $\ker\phi$ is a proper ideal of $R$;
    \item The image $\Img\phi$ is a subring of $S$;
    \item The mapping 
    \begin{align*}
        \psi: R/\ker\phi &\to \Img\phi
        \\ r+\ker\phi&\mapsto \phi(r)
    \end{align*}
    is a well-defined isomorphism.
  \end{enumerate}
\end{theorem}
\begin{proof}
  (1) Note that $\ker\phi$ is an additive subgroup by FIT of groups. Moreover,
  note that for $k\in\ker\phi$ and $r\in R$ we have $\phi(rk)=\phi(r)\phi(k)=\phi(r)0=0$
  so $rk\in\ker\phi$ and similarly $kr\in\ker\phi$. Also note that $1\not\in\ker \phi$
  since we have that $\phi(1_R)=1_S\neq 0_S$. Hence $\ker\phi\lhd R$.

  (2) Note that $\Img\phi$ is an additive subgroup by FIT of groups. Moreover, let
  $s,s'\in\Img\phi$, so $\exists r,r' \in R$ s.t. $\phi(r) = s, \phi(r') = s'.$ Since $\phi$ is a ring homomorphism, 
  $$
  \phi(rr') = \phi(r)\phi(r') = ss',
  $$
  so $ss' \in \Img\phi.$
  
  
  
  % $\phi(rr')=\phi(r)\phi(r')=ss'\in S$ since $S$ is a ring the
  % product of two elements of the ring will still be in $S$. Hence the claim follows.

  (3) By FIT for groups, we have that the isomorphism exists for the
  additive group structure with $\psi:R\to \operatorname{Im} \phi$ defined by $\psi(r+\ker \phi)=\phi(r)$. As $\psi$ is bijective, we are left to show that $\psi$ preserves multiplication.
  % multiplication is preserved. 
  Let $r_1,r_2 \in R.$ Then
  % the homomorphism also holds for
  % the multiplicative operation. Consider $r_1,r_2\in R$,
\begin{align*}
    \psi(r_1+\ker\phi)\psi(r_2+\ker\phi)&=\phi(r_1)\phi(r_2)
    \\&=\phi(r_1 r_2)
    \\&=\psi(r_1 r_2+\ker\phi)
    \\&=\psi((r_1+\ker\phi)(r_2+\ker\phi)).
\end{align*}
Therefore, we have a bijective ring homomorphism, hence the claim follows.
\end{proof}

\begin{example}[In algebraic number theory]
  Consider the homomorphism $\phi: \RR[X]\to \CC,$ $f\mapsto f(i)$. This is a surjective ring homomorphism onto $\CC$.  
  % Clearly the image is all of $\CC$, and its kernel is all multiples of $X^2+1$. 
  Furthermore, $\ker\phi$ is the ideal generated by $X^2+1$, since this is the set of all real polynomials with $i$ as a root.
  Hence
  % we have $\ker\phi=(X^2+1)\RR[X]$ and
  by the FIT, $\RR[X]/(x^2+1)\RR[X]\cong \CC$. 
  
  Note that as a vector space over $\RR$, $\CC$ is 2-dimensional, while $\RR[X]$ is infinitely dimensional. Hence we just need two elements of $\RR$ to represent every coset. This should make sense, as we often express an arbitrary complex number $z$ as $a + ib,$ where $a,b \in \RR.$
\end{example}
\begin{example}[In analysis]
  Consider the set of continuous functions on $\RR$, $C^0(\RR)$. This is a ring under pointwise addition and multiplication. The evaluation map 
  \begin{align*}
      \phi:C^0(\RR)&\to \RR
      \\ f &\mapsto f(0)
  \end{align*}
 is a ring homomorphism, where the image is $\RR$ and the kernel is
  $$\ker\phi=\{f\in C^0(\RR)\mid f(0)=0\},$$ i.e. all the functions that pass through the
  origin. Hence $C^0(\RR)/\ker\phi \cong \RR$.
\end{example}


\begin{theorem}
  Let $F$ be a field and $R$ a non-trivial ring. Then every
  homomorphism $F\to R$ is injective.
\end{theorem}
\begin{proof}
  Let $\phi:F\to R$ be a ring homomorphism. Assume that $\ker\phi$ is not trivial, and hence $\phi$ is not injective. Then there exists a non-zero element $f\in F$, s.t. $f\in\ker \phi$, i.e. $\phi(f)=0_R$. Since $F$ is a field, we have $f^{-1}\in F$ s.t. $f^{-1}f=1_F$. So,
  $$1_R=\phi(1_F)=\phi(f f^{-1})=\phi(f)\phi(f^{-1})=0_R.$$ Hence $R$ must be the trivial ring, a contradiction.
\end{proof}
\begin{remark}
  Note that we did not use the fact that $F$ is commutative. In fact, $F$ does not need to be field, but just a division ring.
\end{remark}

\begin{definition}
  Let $R$ be a ring. An element $a\in R$ is a \emph{left zero divisor} if $a\neq 0$ and
  there exists $b\in R\setminus \{0\}$ s.t. $ab=0$. A right zero divisor is defined analogously.
\end{definition}

\begin{example}
  Consider $\ZZ/8\ZZ$. Then $2+8\ZZ$ is a zero divisor since it is non-zero and $(2+8\ZZ)(4+8\ZZ)=8+8\ZZ=0+8\ZZ$. 

  Consider $R=\RR[X]/X^2\RR[X]$. The element $X+X^2\RR[X]$ is a zero divisor since taking its square gives $0+X^2\RR[X]$.
\end{example}


\begin{theorem}
  Let $R$ be a ring. The following are equivalent
  \begin{enumerate}
    \item $R$ has no left zero divisors;
    \item $R$ has no right zero divisors;
    \item For all $a,b,c\in R$ with $a\neq 0$, one has $ab=ac$ if and only if $b=c$;
    \item For all $a,b,c\in R$ with $a\neq 0$, one has $ba=ca$ if and only if $b=c$.
  \end{enumerate}
  \label{<+label+>}
\end{theorem}
\begin{proof}
  % We have 1 $\iff$ 2 trivially.
  
  % We have 1 $\implies$ 3 since for $ab=ac$ then $ab-ac=a(b-c)=0$,
  % where $a\neq 0$ and not a zero divisor, so must have $b-c = 0$
  % and hence $b=c.$
  
  % Next, we show that 4 $\implies$ 1 by using the contrapositive.

  % %tinka <3
  % %not(p iff q) is
  % %p and not q or q and not p
  % Negating 1, we have at least one left zero divisor $y \in R$ s.t $\exists x\in R\setminus \{0\}$ s.t. $xy = 0.$
  % Then $xy = 0 = 0y,$ but $x \neq 0,$ so we have the negation of 4.

  % % $\exists a,b,c \in R (a \neq 0)$ s.t. $ab=ac$ and $b \neq c$ 
  
  % % Suppose that $a$ is a zero divisor, so $ab=ac$. Let $b$ be the element such that $ab=0$ which is non-zero. Then the case where
  % % $c=0$ we have $ab=0=ac$ but $b\neq 0$. 

  % % We show $1\iff 4$ using a symmetric argument to $1\iff 3.$

  % By a symmetric argument, $3 \implies 2,$ so all four statements are equivalent.

% ($(1) \implies (2)$) Assume for contradiction that $R$ has a right zero divisor, then there exists $a,b \in R$ with $a \neq 0_{R}$ with $ab = 0$. Now consider $(ca)b = c(ab) = c0 = 0$, this shows $ca$ is a left zero divisor, a contradiction.

($1 \iff 2$) This is trivial by the definition of a zero divisor.

($2 \implies 3$) Suppose $ab = ac$ for some $a,b,c \in R$ with $a \neq 0$. We have $a(b-c) = 0$, and since $R$ has no right zero divisors and $a \neq 0$, we must have $b-c = 0 \implies b = c,$ as required.

($3 \implies 2$) Assume for contradiction that there is a right zero divisor in $R$. Then there exists $a,b \in R \setminus \{0\}$ with $ab=0.$ Now consider $c=0\in R.$ We have $ac=a0=0=ab,$ and from (3) we have $b=c=0,$ a contradiction. So, $R$ has no right zero divisors.

($1 \implies 4$) Shown by a symmetric argument to ($2 \implies 3$).

% Suppose $ba = ca$ for some $a,b,c \in R$ with $a \neq 0$. Then we have $(b-c)a = 0$ and since $R$ has no left zero divisors and $a \neq 0$ we have $b-c = 0 \implies b = c.$

($4 \implies 1$) Shown by a symmetric argument to ($3 \implies 2$).

% Assume for contradiction there is a left zero divisor in $R$. Then there exists $a,b \in R \setminus {0}$ with $ba = 0$. Now let $c = 0$, we then have $ca = 0$. From our assumption we have that $ba = ca \text{ and } a \neq 0 \implies b = c$. But $b \neq c$ as $b$ is a left zero divisor, a contradiction.

From above, we have $(3) \implies (2) \implies (1) \implies (4),$ and therefore we have the chain $(1) \implies (2) \implies (3) \implies (4) \implies (1),$
so all four statements are equivalent.
\end{proof}

\begin{theorem}
  Let $R$ be a unital ring and let $u\in R$ be a unit. 
  % \emph{unit} (recall then $\exists u'\in R$ s.t. $u'u=uu'=e_R$). 
  Then $u$ is is not a zero divisor.
  \label{thm:unitZero}
\end{theorem}
\begin{proof}
  Let $u$ be a unit and assume there exists $v\in R$ s.t. $uv=0$ or $vu=0$. Since $u$ is a unit, $\exists u^{-1} \in R$ s.t. $u^{-1} u=1$. We have,
  $$uv=0 \implies u^{-1}uv=0\implies 1v=0 \implies v=0.$$ Similarly, $vu=0 \implies v=0$. Therefore, $u$ is not a zero divisor.
\end{proof}