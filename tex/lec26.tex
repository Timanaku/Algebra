\section{Lecture 26}
\subsection{Irreducible polynomials}
\begin{theorem}
  Let $F$ be a field, and let $f\in F[X]$ be non-zero. Then the following are equivalent
  \begin{enumerate}
    \item The ideal $(f)=\{fg \mid g\in F[X]\}$ is maximal;
    \item The ideal $(f)$ is prime;
    \item The polynomial $f$ is irreducible in $F[X]$.
  \end{enumerate}
  \label{<+label+>}
\end{theorem}
\begin{proof}
(1$\implies$2) Suppose $(f)$ is maximal. Then $(f)$ is maximal $\implies$ $F[X]/(f)$ is a field $\implies$ $F[X]/(f)$ is an integral domain $\implies$ $(f)$ is prime. (Corollary \ref{cor:maximalPrime})

(2$\implies$3) Suppose the ideal $(f)$ is prime. Consider $f = gh$ for some $g,h \in F[X]$, so $g \in (f)$ or $h \in (f)$. Assume without loss of generality that $g \in (f)$, i.e., there exists $k \in F[X]$ such that $g = fk$. Now, we have that $\deg f = \deg g + \deg h$ and $\deg g = \deg f + \deg k$. Hence, $\deg h = \deg k = 0$, so $h,k \in F$ and are units. Therefore, $f$ is irreducible in $F[X]$.    

\\(3$\implies$1) Suppose $f$ is irreducible in $F[X]$. For $f$ to be irreducible, it must be of degree greater than zero and therefore $(f)$ is a proper ideal. Assume there exists an ideal $J$ such that $(f) \unlhd J\unlhd F[X]$. By Theorem \ref{thm:idealPrinciplas}, $F[X]$ is a PID, so $\exists g \in F[X]$ s.t. $J = (g)$. Since $f \in (f) \subseteq (g)$ $\exists h \in F[X]$ s.t. $f = gh$. Since $f$ is irreducible, either $g$ or $h$ is a unit. Suppose $g$ is a unit. Then $\exists g^{-1} \in F[X]$ s.t. $gg^{-1} = 1$, so $1\in(g)\implies J = F[X]$. Suppose now that $h$ is a unit. Then $\exists h^{-1}\in F[X]$ with $hh^{-1}=1$, so $f=gh\implies g=fh^{-1}$, which means $(g)\subseteq(f)\implies(g)=(f)$. We conclude $(f)$ is maximal.
\end{proof}

This theorem gives us a way of easily constructing new fields.
\begin{example}
  Let $F=\ZZ/3\ZZ$, which is a field. The polynomial $f(X)=X^2+1\in F[X]$ is
  irreducible. It follows that $F[X]/(f)$ is a field. Recall that for every $g\in F[X]$
  there exists $h=a_0+a_1x\in F[X]$ s.t. $g+(f)=h+(f)$. Hence we have
  \[F[X]/f(x) = \left\{ a_0+a_1X \mid a_0,a_1\in F \right\}\]
  Note that it has 9 elements.
\end{example}
\begin{theorem}
  Let $F$ be a field and let $f\in F[X]$ have degree 2 or 3. Then $f$ is reducible if and
  only if it has a root in $F$, i.e. iff there exists $a\in F$ s.t. $f(a)=0$.
  \label{thm:deg2o3Root}
\end{theorem}
\begin{proof}
  If $f$ is reducible, then $f=gh$ for some $0<\deg g<\deg f$, $0<\deg h<\deg f$ and
  $\deg g + \deg h = \deg f$, which implies at least one of the factors has degree 1,
  which is of the form $a_0+a_1X$ and has root $\frac{-a_0}{a_1}\in F$.

  Conversely, assume $f(a)=0$ for some $a\in F$. Then Corollary \ref{cor:rootFactor} implies $\exists h\in F[X]$ with $f=h\cdot(X-a)$. Since $X-a$ has degree 1, $h$ must be of degree 1 or 2, depending on whether $f$ is of degree 2 or 3 respectively, so $f$ is reducible. 
\end{proof}


\begin{example}
  Let $d\in \ZZ$ be non-square. Then the corresponding quadratic number field is defined as
  \[\QQ[X]/(X^2-d) \cong \QQ[\sqrt{d}]=\left\{ a+b\sqrt{d} \mid a,b\in \QQ \right\}\]
  \[a+bX+(X^2-d) \mapsto a+b\sqrt{d},\]
  where the inverse of an element $a+b\sqrt{d}$ is $\frac{a-b\sqrt{d}}{a^2-b^2d}$.
\end{example}
