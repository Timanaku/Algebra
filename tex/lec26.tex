\section{Lecture 26 - 19 Nov 2021}
\subsection{Irreducible polynomials}
\begin{theorem}
  Let $F$ be a field, and let $f\in F[X]$ be non-zero. Then the following are equivalent
  \begin{enumerate}
    \item The ideal $(f)=\{fg:g\in F[X]\}$ is maximal;
    \item The ideal $(f)$ is prime;
    \item The polynomial $f$ is irreducible in $F[X]$.
  \end{enumerate}
  \label{<+label+>}
\end{theorem}
\begin{proof}
(1) $\Rightarrow$ (2) Suppose $(f)$ is maximal. Then $(f)$ is maximal $\Rightarrow$ $F[X]/(f)$ is a field $\Rightarrow$ $F[X]/(f)$ is an integral domain $\Rightarrow$ $(f)$ is prime. (Corollary \ref{cor:maximalPrime})

(2) $\Rightarrow$ (3) Suppose the ideal $(f)$ is prime. Consider $f = gh$ for some $g,h \in F[X]$, so $g \in (f)$ or $h \in (f)$. Assume without loss of generality that $g \in (f)$, i.e., there exists $k \in F[X]$ such that $g = fk$. Now, we have that $\deg f = \deg g + \deg h$ and $\deg g = \deg f + \deg k$. Hence, $\deg h = \deg k = 0$, so $h,k \in F$ and are units. Therefore, $f$ is irreducible in $F[X]$.

(3) $\Rightarrow$ (1) Suppose $f$ is irreducible in $F[X]$. For $f$ to be irreducible, it must be of degree greater than zero and therefore $(f)$ is a proper ideal. Assume there exists an ideal $J$ such that $(f) \unlhd J\unlhd F[X]$. By theorem \ref{thm:idealPrinciplas}, $F[X]$ is a PID, so $\exists g \in F[X]$ s.t. $J = (g)$. Since $f \in (f) \subset (g)$ $\exists h \in F[X]$ s.t. $f = gh$. Since $f$ is irreducible, either $g$ or $h$ is a unit. Suppose $g$ is a unit. Then $g^{-1} \in F[X]$ s.t. $gg^{-1} = 1$, and since $g \in J$, $g^{-1} \in J$, so $J = F[X]$. Suppose now that $h$ is a unit. $g \in J$, $g^{-1} \in J$, so $f = gh \in J$ (since $h$ is a unit). Thus $(f) \subset J$ so $(f) = J$. Therefore $(f)$ is maximal. Then $ghh^{-1} = fh^{-1} = g \in (f)$, so $(g) = (f)$. Therefore $(f)$ is maximal.
\end{proof}

  % \emph{(1) implies (2):} In Corollary \ref{cor:maximalPrime} we proved that every maximal
  % ideal of a commutative ring is prime. Note that $F[X]$ is a commutative ring, hence
  % the result follows.

  % \emph{(2) implies (3):} Note that $\deg f>0$ since if $\deg f=0$ we would have
  % $(f)=F[X]$. Next, consider $f=gh$, so we have that $g\in (f)$ or $h\in (f)$.  If $g\in
  % (f)$, note that $\deg f \geq\deg g$ and $\deg g \geq \deg f$, so we must have $\deg
  % f= \deg g$, as required, hence $\deg h=0$, so $f$ is irreducible. The case for $h\in
  % (f)$, it follows that $\deg g=0$.

  % \emph{(3) implies (1):} Suppose $f$ is irreducible in $F[X].$ 
  
  
  % Note that for $f$ to be irreducible, we must have $\deg f>0$, and it
  % follows that $(f)$ will be a proper ideal. Moreover, assume we have an ideal $J$ s.t.
  % $I\subseteq J \subseteq F[X]$. In particular, note that $J$ must be principal by Theorem
  % \ref{thm:idealPrinciplas}, hence there exists $g\in F[X]$ s.t. $J=(g)$. Since
  % $I\subseteq J$ we must that some $h\in F[X]$ s.t. $f=gh$. Since $f$ is irreducible, it
  % follows that either $\deg g=0$, in which case $J=F[X]$, or $\deg g=\deg f$, in which
  % case $h\in F$ so it follows that $J=I$.

This theorem gives us a way of easily constructing new fields.
\begin{theorem}
  Let $F$ be a field and let $f\in F[X]$ have degree 2 or 3. Then $f$ is reducible if and
  only if it has a root in $F$, i.e. iff there exists $a\in F$ s.t. $f(a)=0$.
  \label{thm:deg2o3Root}
\end{theorem}
\begin{proof}
  If $f$ is reducible, then $f=gh$ for some $0<\deg g<\deg f$, $0<\deg h<\deg f$ and
  $\deg g + \deg h = \deg f$, which implies at least one of the factrs has degree 1,
  which is of the form $a_0+a_1X$ and has root $\frac{-a_0}{a_1}$.
\end{proof}

\begin{example}
  Let $F=\ZZ/3\ZZ$, which is a field. The polynomial $f(X)=X^2+1\in F[X]$ is
  irreducible. It follows that $F[X]/(f)$ is a field. Recall that for every $g\in F[X]$
  there exists $h=a_0+a_1x\in F[X]$ s.t. $g+(f)=h+(f)$. Hence we have
  \[F[X]/f(x) = \left\{ a_0+a_1X : a_0,a_1\in F \right\}\]
  Note that it has 9 elements.
\end{example}

\begin{example}
  Let $d\in \ZZ$ be non-square. Then the quadratic number field is defined as
  \[\QQ[X]/(X^2-d) \cong \QQ[\sqrt{d}]=\left\{ a+b\sqrt{d} | a,b\in \QQ \right\}\]
  \[a+bX+(X^2-d) \mapsto a+b\sqrt{d}\]
  Where the inverse of an element $a+b\sqrt{d}$ is $\frac{a-b\sqrt{d}}{a^2-b^2d}$.
\end{example}
