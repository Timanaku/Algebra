\section{Lecture 6}
\subsection{Normal subgroups and Quotients}
Recall that $G/H$ is a set and does not necessarily preserve the group structure of $G.$ We want to explore when it does. When do the set of left (or right) cosets form a group under coset multiplication, defined as
$$xH\cdot yH=xyH$$
for $x,y\in G?$ The problem we run into is ensuring well-definedness, that taking different representatives of the same coset gives us the same result. So, when does $xH=x'H$ and $yH=y'H$ $\implies xyH=x'y'H?$

First, let us look at a non-example. In $S_3$, consider the subgroup $H=\langle (12)\rangle =\{e, (12)\}.$ 
We have that $(123)H=(13)H$ and $(132)H=(23)H$.
Now, if we multiply these two cosets together using different representatives, we find that
% and check if the chosen representative matters.
% We have, 
$$(13)H(23)H=(13)(23)H=(132)H,$$ but
$$(123)H(132)H=(123)(132)H=H,$$
so the coset multiplication is not well-defined in this case. 

We can notice that, for example, $$(13)H = \{(13),(123)\} \neq \{(13),(132)\} = H(13).$$

It turns out that equality between left and right cosets $xH=Hx$ (normality) is exactly the condition we need to make this work, as seen later by Theorem \ref{normality}. 


% \\Recall that in general, we have $xH\neq Hx$. %in this case, $xH\neq Hx$
\begin{definition}
    Let $G$ be a group and let $H\leq G.$ The subgroup $H$ is called \emph{normal} in $G$, denoted $H\triangleleft G,$ if for all $x\in G$ we have $xH=Hx.$ 
\end{definition}

The maximal and trivial subgroups of any group are normal. $G \leq G$ has only one coset so the condition is automatically satisfied. Similarly, if $H \leq G$ is trivial, then $$xH = \{xe\} = \{x\} = \{ex\} = Hx.$$

Also, any subgroup of an abelian group is normal, since
$$
xH = \{xh \mid h\in H \} =  \{hx \mid h\in H \} = Hx. 
$$


\begin{theorem}[Equivalent conditions for normality]
    Let $G$ be a group and $H\leq G.$ Then the following are equivalent:
    \begin{itemize}
        \item  $\forall x\in G,$ $xH=Hx$
        \item  $\forall x\in G,  \exists y\in G$ s.t. $xH=Hy$
        \item  $G/H=H\setminus G$
        \item  $\forall x\in G,$ $xHx^{-1}=H$
        \item  $\forall x\in G,$ $xHx^{-1}\subseteq H$
        \item  $\forall x \in G,$ $\forall h\in H,$ $xhx^{-1} \in H$
    \end{itemize}
\end{theorem}
\begin{proof}
    See 2F.
\end{proof}
\begin{theorem}
    Let $G$ be a group and $H\leq G.$ Then multiplication of (left) cosets of $H$ in $G$, defined for all $aH,bH \in G/H$ by
    $$aH\cdot bH=abH$$
    is well defined if and only if $H$ is normal in $G.$
    \label{normality}
\end{theorem}
\begin{proof}
    $(\implies)$ Suppose the operation is well defined, i.e., if $xH=x'H$ and $yH=y'H$ then $xyH=x'y'H.$

    We aim to show that $xH = Hx.$
    Let $x'\in xH,$ so we have that $xH=x'H$.
    Then we can say
    $$x^{-1}x'H=x^{-1}Hx'H=x^{-1}HxH=x^{-1}xH=eH=H.$$
    Therefore, there exists $\widetilde{h}$ such that $x^{-1}x'=h$ so $x'=\widetilde{h}x \in Hx$, meaning $xH \subseteq Hx.$
    We also have $x'\in Hx$ implies $x'\in xH$ by a symmetric argument. Therefore $xH=Hx$.
    
    $(\impliedby)$ Suppose $H\triangleleft G,$ and let $xH=x'H$ and $yH=y'H.$ Then we have $x'=xh$ and $y'=yh'$ for some $h,h'\in H.$ Then
    $$x'y'H=xhyh'H=xhyH.$$ By our normality condition, there exists $\hat{h}$ s.t. $hy=y\hat{h}.$ Therefore, $x'y'H=xhyH=xy\hat{h}H=xyH$ as required.
\end{proof}
We can now define the quotient group of a group by its normal subgroups.
\begin{definition}
  Let $G$ be a group, and $N$ be a normal subgroup. The set of left cosets $G/N$ together
  with the binary operation $(gN)(hN)=(gh)N$ for $g,h\in G$ is called the \emph{quotient
  group} or \emph{factor group} of $G$ by $N$.
  \label{quotientGroup}
\end{definition}
Note that by the definition of normal subgroups, we have that normal subgroup's left
cosets equal that subgroup's right cosets, since $gNg^{-1}=N\iff gN=Ng$.
% This lecture we will start the revision on normal subgroups and quotients.
% Recall from last year that, if $V$ is a vector space and $U$ is a subspace, then the set
% of cosets $\left\{ v+ U: v\in V \right\}$ is a vector space in its own right. For example,
% the addition of cosets is defined by $(v+U)+(v'+U)=(v+v')+U$. The lecturer gave as an
% intuitive example the vector space $\RR^2$, where we take a line throughout the origin as a
% subspace, and then show how adding some vector form $\RR^2$ just shifts the line around
% the plane (no longer through the origin, so not a subspace), and adding these two lines we
% add the two elements used to create that line ($v,v'$), find the new vector ($v+v'$), and
% create a new line.

% In order to be sure that this addition of cosets works, we need to show that it's well
% defined (i.e. it preserves under addition of different representative of the same coset).


% Let us try to generalise this idea of reconstructing the parent structure from quotients,
% but to any group. Let $G$ be a group and $H$ a subgroup of $G$. We would like the set of
% left cosets of $H$ in $G$ to \emph{inherit} the group structure -- spoiler alert a
% homomorphism!. We could try to define for $x,y\in G$, $xH yH=(xy)H$ (observe the
% homomorphic structure). We need to check that this is well defined, as above. Let $e,h\in
% H$ denote the identity element and some element of $H$, respectively. Let $y\in G$ denote
% some element not in $H$. We want that $(eH)(yH)=yH = (hH)(yH)=(hy)H \iff y^{-1}hy\in H$. 

% Hence, a necessary and sufficient condition for multiplication of cosets
% to be well defined is that, for every $g\in G$ and $h\in H$ we get $ghg^{-1}\in H$.
% Another way of stating this condition is,
% \[\forall g\in G, gHg^{-1}\subseteq H \iff H\subseteq g^{-1}H g \iff H=g^{-1}H g\]

% \begin{definition}
%   Let $G$ be a group. A subgroup $H$ of $G$ is called normal, written $H\triangleleft G$
%   or $H\trianglelefteq G$, if $\forall g\in G$ we have $gHg^{-1}=H$.
%   \label{normalSubgroup}
% \end{definition}

% Hence for the set of left (or right) cosets of $H<G$ to have group structure with the
% above operation, we require $H$ to be normal in $G$.

\begin{example}
  Consider the group $S_3$. We claim the subgroup generated by $(1 2 3)$ is normal.
  Indeed, it is the group consisting of the identity and all 3-cycles in $S_3$. Since
  conjugation preserves the cycle type of a permutation, the claim follows (since any
  conjugation will be a 3-cycle, which is contained in the subgroup).

    In contrast, the subgroup generated by $(12)$ is not normal. For instance, we have
  \[(13)(12)(13)=(23)\not\in \langle(12)\rangle. \]

\end{example}

\begin{theorem}
    Let $G$ be a group and $H \leq G$ of index 2. Then $H$ is normal in $G$.
    \label{thm:index2}
\end{theorem}
\begin{proof}
    Let $H$ be a subgroup of $G$ of index 2. Then $[G:H] = 2$ so there are two left cosets and two right cosets of $H$ in $G.$ One of the cosets is $eH=He=H.$ Now, take an element of $G$ that is not in $H,$ $g\in G\setminus H.$ Since cosets form a partition of the group, the other coset must be the rest of the group. In other words, $gH=Hg=G\setminus H.$ We have shown that left and right cosets are equal, therefore $H$ is normal in $G.$ 

    % If $[G:H] = 2$, then there are two left cosets of $H$ in $G$. Let $g \in G \setminus H.$ Since $g \notin H,$ we must have $gH = G \setminus H$ since the set of left cosets are a partition of the group.

    % , namely $H$ and $xH$ for some $x \in G\setminus H.$ Furthermore, $xH = G\setminus H$ by Corollary \ref{leftCosetsEqRel}.

    % If we instead consider the right cosets of $H$ in $G$ ($H$ and $Hy$ for some $y \in G\setminus H$), then by the same argument $Hy = G\setminus H.$

    % %want to maybe just say it's easy to check we have the second normality condition in Thm 6.2
    
    % Let $g \in G.$ If $g \in H,$ then $gH = H = Hg.$ If $g \notin H,$ then $gH = G\setminus H = Hg.$ Since left cosets are equal to right cosets, $H$ is normal in $G$.
    
    % By a similar argument, $Hg = G \setminus H$, so we have $gH = Hg.$ It is also straightforward to check that $hH = Hh, \forall h \in H,$ so $H$ is normal in $G$.
\end{proof}

\begin{example}
  Recall that we can write a permutation as a product of transpositions (2-cycle
  permutations), and the \emph{parity} of the number of transpositions is invariant
  (doesn't depend how you write the product). So a permutation is even if it can be
  written as an even number of transpositions, and it's odd otherwise. Hence we define the
  sign of a permutation to be $+1$ or $-1$ depending on whether it's even or odd,
  respectively. We can note also that the sign of a permutation must be the same as its inverse, since the function
  \begin{align*}
    \sgn&:S_n \to \{+1,-1\} \\
    \sigma &\mapsto
    \begin{cases}
        +1,\quad \text{if } \sigma \text{ is even} \\
        -1, \quad \text{if } \sigma \text{ is odd}
    \end{cases}
  \end{align*}
  
  is a homomorphism.

  Let $n\in\NN$ and let $A_n\subset S_n$ be the set of even permutations. This is a normal
  subgroup, since for any $\sigma, \tau\in S_n$ we have
  \[\sgn (\sigma \tau \sigma^{-1}) =
  \sgn(\sigma)\sgn(\tau)\sgn(\sigma)=\sgn(\tau).\]

  We could have also noted that $A_n$ is index 2 in $S_n$ and applied Theorem \ref{thm:index2} directly.
\end{example}

\begin{example}
  Let $D_{2n}$ be the dihedral group of order $2n$, and let $H$ be the subgroup of $n$
  rotations. Then $H$ is normal. On the other hand, the subgroup generated by a reflection
  is not normal.

\end{example}

% Once we've demonstrated well-definedness and provided some examples of the criteria, we
% have a method for creating a group structure out of the quotients of the group. 


% \begin{remark}
%   Neumann Groups and Geometry book gives a nice and direct view of how normal groups
%   naturally give raise to the \emph{group operation preservation} condition. Since for a
%   normal subgroup $N\trianglelefteq G$ we have for any $a,b$ it follows that $(Na)(Nb)$ is
%   again a coset (recall direct product of groups), since $(Na)(Nb)=N(aN)b=(NN)(ab)=N(ab)$,
%   making $G/N$ a group, the quotient group.
% \end{remark}

\begin{remark}
  Note that, as previously said, cosets partition the group, making a \emph{smaller set}
  where the elements of those sets are collections of many group elements. However, the
  fact that normal subgroups form quotient \emph{groups} makes this partition a group
  itself. This is why it is so useful; it helps with understanding the group in a simpler light.
\end{remark}
