\section{Lecture 3 - 28 Sep 2021}
Today we cover more on cyclic groups and multuplication tables.
\begin{theorem}
  Let $g\in G$ have order $n$, finite, and fix some $k\in\ZZ$. Then
  \[|g^k|= \frac{n}{d} \iff d=\gcd{(n,|k|)}\]
\end{theorem}
\begin{proof}
  The order of $g^k$ is the smallest integer $i$ s.t. $g^{ik}=e$. However, we know that the
  order of $g$ is $n$, so we must have $g^{mn}=e=g^{ik}$, so we must have
  $i=\frac{nm}{k}$, and such $i$ must be the smallest possible. Since we have fixed $n$,
  we have that $i=\frac{n}{d}$ only if $d=\gcd{(n,k)}$, as required.
\end{proof}

\begin{cor}
  Let $G=\langle g \rangle$ be a cyclic group generated by $g$, and have order $n<\infty$. Fix
  $k\in\ZZ$. Then
  \[G=\langle g^k \rangle \iff \gcd{(n,k)}=1\]
\end{cor}
\begin{proof}
  We have that $g^k$ is a generator for $G$ $\iff \langle g^k \rangle = \langle g \rangle \iff |g^k|=n \iff
  \gcd{(k,n)}=1$.
\end{proof}


\subsection{Multiplication table}
Next multuplication tables are explained and how they encode the entire structure of the
group, where the (i,j) entry encodes the result of $g_ig_J$. However, we should be aware
that different multiplication tables can encode essentially the same group. An example is
given, considering the cyclic group of order $4$, where two elements' labels are swapped
and seeing how the multuplication table differs.
\begin{theorem}
  In a multiplation table, each row contains every element only once, and so does each
  column. 
\end{theorem}
\begin{proof}
  We have $g_ig_j = g_i g_k \iff g_j=g_k$, and the same argument can be used for the
  columns. For countable groups, this holds (even if it's infinite).
\end{proof}

Next we're shown an example of how there are only 2 groups of order 4, namely Klein's
4-group and the typical cyclic group of order 4 isomorphic of $\ZZ_4$.
