\section{Lecture 3}

\begin{theorem}
  Let $g\in G$ have order $n<\infty$, and fix some $k\in\ZZ$. Then
  $|g^k|= \frac{n}{d}$ where $d=\gcd{(n,|k|)}.$
\end{theorem}

\begin{proof}
    The order of $g^k$ is the smallest positive integer $i$ such that $(g^k)^i = g^{ki} = e$. As the order of $g$ is $n$, we seek the smallest $i$ such that $ki$ is a multiple of $n$. The smallest such multiple will be when $i|k| = \operatorname{lcm}(n,|k|) = \frac{n|k|}{\gcd(n,|k|)}$.
    This gives $i = \frac{n}{d}$ as required.
    
    % ($\implies$) Let $i$ be the smallest positive integer such that $(g^k)^i = g^{ki} = g^n = e$, we have $i = \frac{n}{d} \implies \operatorname{lcm}(k,n) = ki = \frac{kn}{d} \implies d = \gcd(n,k) $.
\end{proof}

\begin{corollary}
  Let $G=\langle g \rangle$ be a cyclic group generated by $g$, and have order $n<\infty$. Fix
  $k\in\ZZ$. Then
  \[G=\langle g^k \rangle \iff \gcd{(n,k)}=1.\]
\end{corollary}
\begin{proof}
  We have that $g^k$ is a generator for $G$ $\iff \langle g^k \rangle = \langle g \rangle \iff |g^k|=n \iff
  \gcd{(k,n)}=1$.
\end{proof}


\subsection{Multiplication tables}
% Next multiplication tables are explained and how they encode the entire structure of the
% group, where the (i,j) entry encodes the result of $g_ig_j$. However, we should be aware
% that different multiplication tables can encode essentially the same group. An example is
% given, considering the cyclic group of order $4$, where two elements' labels are swapped
% and seeing how the multiplication table differs.
% \begin{theorem}
%   In a multiplication table, each row contains every element only once, and so does each
%   column. 
% \end{theorem}
% \begin{proof}
%   We have $g_ig_j = g_i g_k \iff g_j=g_k$, and the same argument can be used for the
%   columns. For countable groups, this holds (even if it's infinite).
% \end{proof}

% Next we're shown an example of how there are only 2 groups of order 4, namely Klein's
% 4-group and the typical cyclic group of order 4 isomorphic of $\ZZ_4$.


% Notes for rewrite:

% What is a multiplication table?
% Simple example (order 3 probably)
% Theorem
% Classifying groups of order 4 example
% Isomorphism waffle, swapping rows/columns, relabelling

One often helpful way to understand a small group is to construct its multiplication table. 

\begin{definition}
   If a group $G$ has order $n$ and elements $g_1,g_2,\ldots,g_n$, then a \emph{multiplication table} for $G$ is an $n \times n$ grid where the entry $(i,j)$ in the $i^{\text{th}}$ row and $j^{\text{th}}$ column is the product $g_ig_j$ in the group.
\end{definition}

\begin{example}
    Consider the cyclic group of order 3, $C_3 = \{e, g, g^2 \}.$ A multiplication table for $C_3$ is:
    \begin{center}
        \begin{tabular}{c| *{3}{c}}
        \multicolumn{4}{c}{} \\
        $\text{ }$ & $e$ & $g$ & $g^2$ \\
        \hline
        $e$ &  $e$ & $g$ & $g^2$ \\
        $g$ & $g$ & $g^2$ & $e$ \\
        $g^2$ & $g^2$ & $e$ & $g$
        \end{tabular}
    \end{center}

    Note that we could switch $g$ and $g^2$ to get the equivalent table:

    \begin{center}
        \begin{tabular}{c| *{3}{c}}
        \multicolumn{4}{c}{} \\
        $\text{ }$ & $e$ & $g^2$ & $g$ \\
        \hline
        $e$ &  $e$ & $g^2$ & $g$ \\
        $g^2$ & $g^2$ & $g$ & $e$ \\
        $g$ & $g$ & $e$ & $g^2$
        \end{tabular}
        %mention here that this is just a relabelling?
    \end{center}
    \label{ex:multTable}
\end{example}

\begin{theorem}
    Let $G$ be a finite group. Then every row and column of a multiplication table for $G$ contains each element of $G$ exactly once.
\end{theorem}

\begin{proof}
    % We will consider only the case for a row, but the proof for columns is similar.
    
    % Let $G = \{ g_1,g_2,\ldots,g_n \}.$ Suppose the $(i,j)$ and $(i,k)$ entries are the same, i.e. $g_ig_j = g_ig_k$. Left multiplication by $g_i^{-1}$ gives that $g_j = g_k,$ so $j = k.$

    This follows directly from the group axioms. If the $(i,j)$ and $(i,k)$ entries are the same, i.e. $g_ig_j = g_ig_k$, left multiplication by $g_i^{-1}$ gives that $g_j = g_k,$ so $j = k.$
\end{proof}


%[Classifying the groups of order 4]
    We wish to classify all possible groups of order 4 using this method. In other words, we want to find all essentially different multiplication tables for a group with 4 elements.

    Let $G = \{g_1, g_2, g_3, g_4\}$ and choose $g_1$ to be the identity element in $G$. The identity axiom forces 7 entries in the multiplication table:

    \begin{center}
        \begin{tabular}{c| *{4}{c}}
        \multicolumn{5}{c}{} \\
        $\text{ }$ & $g_1$ & $g_2$ & $g_3$ & $g_4$ \\
        \hline
        $g_1$ &  $g_1$ & $g_2$ & $g_3$ & $g_4$ \\
        $g_2$ & $g_2$ & \text{ } & \text{ } & \text{ } \\
        $g_3$ & $g_3$ & \text{ } & \text{ } & \text{ } \\
        $g_4$ & $g_4$ & \text{ } & \text{ } & \text{ }
        \end{tabular}
    \end{center}

    Now we have a choice. We can either pick $g_2g_2 = g_1,$ or $g_2g_2 = g_3$ ($g_4$ leads to the same structure as $g_3$). 
    
    Let us choose the first case, i.e. $g_2$ is self-inverse. If we add this to the multiplication table, the presence of $g_1,g_2, \text{ and } g_3$ in the row and column containing $g_2g_3$ forces this to take the value $g_4$ (we can treat this much like a sudoku puzzle). 
    
    Once we have this breakthrough, we are able to fill in all but the lower right-hand quadrant. Again, we have a choice, so let $g_3g_3 = g_1$ (we will come back to the other possibility later). This completes the multiplication table:
    % $$
    % g_3g_3 = g_3g_2g_4 = g_3g_2g_2
    % $$ 
    
    % We arrive at the following multiplication table:

    \begin{center}
        \begin{tabular}{c| *{4}{c}}
        \multicolumn{5}{c}{} \\
        $\text{ }$ & $g_1$ & $g_2$ & $g_3$ & $g_4$ \\
        \hline
        $g_1$ &  $g_1$ & $g_2$ & $g_3$ & $g_4$ \\
        $g_2$ & $g_2$ & $g_1$ & $g_4$ & $g_3$ \\
        $g_3$ & $g_3$ & $g_4$ & $g_1$ & $g_2$ \\
        $g_4$ & $g_4$ & $g_3$ & $g_2$ & $g_1$
        \end{tabular}
    \end{center}
    % mira how tf does this shit work
    % ah i knwo what to do
    % replace g1 with g2g2 and associativity that bitch

    % (g3g2)g2 = g3
    % g2(g2g3) = g3
    % implies g2 and g2g3 commute

    % g2g2g3g2g2 = g3

    % g3g3 = g2g4 = g2g2g3

    Now, we can go back to the beginning and check the case where $g_2g_2 = g_3.$ By the ``sudoku'' property, we must have $g_4g_2 = g_1.$ This alone forces the rest of the multiplication table:

    \begin{center}
        \begin{tabular}{c| *{4}{c}}
        \multicolumn{5}{c}{} \\
        $\text{ }$ & $g_1$ & $g_2$ & $g_3$ & $g_4$ \\
        \hline
        $g_1$ &  $g_1$ & $g_2$ & $g_3$ & $g_4$ \\
        $g_2$ & $g_2$ & $g_3$ & $g_4$ & $g_1$ \\
        $g_3$ & $g_3$ & $g_4$ & $g_1$ & $g_2$ \\
        $g_4$ & $g_4$ & $g_1$ & $g_2$ & $g_3$
        \end{tabular}
    \end{center}

    The claim now is that these two group tables are the only essentially different ones. To convince ourselves of this, we can return to the previously mentioned possibility in our first complete table, and instead let $g_3g_3 = g_2$. Compare the following multiplication tables:

    \begin{center}
    
        \begin{tabular}{c| *{4}{c}}
        \multicolumn{5}{c}{} \\
        $\text{ }$ & $g_1$ & $g_2$ & $g_3$ & $g_4$ \\
        \hline
        $g_1$ &  $g_1$ & $g_2$ & $g_3$ & $g_4$ \\
        $g_2$ & $g_2$ & $g_1$ & $g_4$ & $g_3$ \\
        $g_3$ & $g_3$ & $g_4$ & $g_2$ & $g_1$ \\
        $g_4$ & $g_4$ & $g_3$ & $g_1$ & $g_2$
        \end{tabular}
        \quad
        \begin{tabular}{c| *{4}{c}}
        \multicolumn{5}{c}{} \\
        $\text{ }$ & $g_1$ & $g_2$ & $g_3$ & $g_4$ \\
        \hline
        $g_1$ &  $g_1$ & $g_2$ & $g_3$ & $g_4$ \\
        $g_2$ & $g_2$ & $g_3$ & $g_4$ & $g_1$ \\
        $g_3$ & $g_3$ & $g_4$ & $g_1$ & $g_2$ \\
        $g_4$ & $g_4$ & $g_1$ & $g_2$ & $g_3$
        \end{tabular}
        
    \end{center}
    \vspace{2mm}
    Notice that the permutation of rows $2\mapsto4\mapsto3\mapsto2$ on the left table gives us the right table, up to relabelling the elements.
    
    In a similar way, any other possible table we could make can be permuted and relabelled to arrive at one of the two distinct multiplication tables we already have. This tells us that we only have two groups of order 4, which we call $C_4$ and $V_4$.
    From the multiplication tables, we can see that both groups are abelian.

    \begin{remark}
        In general, we say two groups of the same order are ``the same'' if their multiplication tables are the same. We will make this notion precise later with the concept of an isomorphism.
    \end{remark}

