\section{Lecture 29 - (Non-Examinable from now on)}
\subsection{An Aside on Free groups}
% % Note that even though we have used the usual function notation for working with
% % polynomials, polynomials are not really functions. 

% % They can be defined as a function,

% We are used to thinking of polynomials as functions, but this is not a good definition in general. 
% % but that does not hold in general.
% Consider the ring $R=\ZZ/p\ZZ$ and  the polynomial ring $R[X]$. Note that $R$ is indeed a field and that for $f(X)=X^p-X\in R[X]$ we have $f(a)=0$
% for all $a\in R$, so we see that a polynomial can be the zero function without being the $0$ polynomial of the ring. 
% % Hence $f$ as $R\to R$ is the $0$ function, but it's not the $0$ polynomial of $R[X]$.

% So its not the best idea to think of polynomials as functions, but we still want to be able to evaluate them at elements of a ring.

% \begin{theorem}[Universal Property of Polynomial Rings]
% Let $\phi: R \longrightarrow S$ be any ring homomorphism and let $s \in S$ be any element of $S$. Then there is a unique ring homomorphism
% $$
% \psi: R[X] \longrightarrow S,
% $$
% such that $\psi(X)=s$ and which makes the following diagram commute
% \begin{center}
% \begin{tikzcd}
% R \arrow{r}{\phi} \arrow[swap]{d}{i} & S \\
% R[X] \arrow[dashed]{ur}[swap]{\exists ! \psi}
% \end{tikzcd}
% \end{center}
% \end{theorem}

% \begin{proof}
%     We define the map $\psi: R[X] \longrightarrow S$ as follows, for any $f \in R[X]$ with $f = a_0 + a_1X + \ldots + a_nX^n$ with $a_i \in R$, set $\psi(f) = \phi\left(a_0\right)+\phi\left(a_1\right) s+\phi\left(a_2\right) s^2+\ldots+\phi\left(a_n\right) s^n$. 
    
%     This clearly extends $\phi$ as $\psi(r) = \phi(r)$ for $r \in R$ and it also maps $X$ to $s$. We must check that $\psi$ is indeed a ring homomorphism.
    
%     Let $f=a_0+a_1X+\ldots +a_nX^n.$ and $g=b_0+b_1X+\ldots +b_mX^m$ $(n \geq m)$ be elements of $R[X].$
    
%     When $n=m$,
%     \begin{align*}
%         \psi(f+g) &= \psi(a_0+b_0+(a_1+b_1)X+\ldots +(a_n+b_n)X^n)\\
%         &= \phi(a_0 + b_0) + \phi(a_1+b_1)s + \ldots + \phi(a_n+b_n)s^n \\
%         &= \phi(a_0) + \phi(a_1)s + \ldots + \phi(a_n)s^n + \phi(b_0) + \phi(b_1)s + \ldots + \phi(b_n)s^n \\
%         &= \psi(f) + \psi(g).
%     \end{align*}

%     And when $n > m$,
%     \begin{align*}
%         \psi(f+g) &= \psi(a_0+b_0+(a_1+b_1)X+\ldots +(a_m+b_m)X^m + \ldots + (a_n+0)X^n)\\
%         &= \phi(a_0 + b_0) + \phi(a_1+b_1)s + \ldots + \phi(a_m+b_m)s^m + \ldots + (a_n+b_n)s^n \\
%         &= \phi(a_0) + \phi(a_1)s + \ldots + \phi(a_n)s^n + \phi(b_0) + \phi(b_1)s + \ldots + \phi(b_m)s^m \\
%         &= \psi(f) + \psi(g),
%     \end{align*}
%     so addition is preserved.
    
%     Similarly, we check multiplication is preserved:
%     \begin{align*}
%         \psi(fg)&=\psi((a_0 + a_1X + \ldots + a_nX^n)(b_0+b_1X+\ldots +b_mx^m))
%         \\ &=\psi((a_0b_0+a_0b_1X+b_0a_1X+a_1b_1X^2+\ldots +a_nb_mX^{n+m})
%         \\ &= \phi(a_0b_0) +\phi(a_0b_1)s+\phi(b_0a_1)s+\phi(a_1b_1)s^2+\ldots+\phi(a_nb_m)s^{n+m}
%         \\ &= \phi(a_0)\phi(b_0)+\phi(a_0)\phi(b_1)s+\phi(b_0)\phi(a_1)s+\phi(a_1)\phi(b_1)s^2+\ldots+\phi(a_n)\phi(b_m)s^{n+m},
%     \end{align*}
%     where the last equality follows from $\phi$ being a ring homomorphism. Now, factoring the equation back out gives,
%     \begin{align*}
%         \psi(fg)&=(\phi(a_0)+\phi(a_1)s+\ldots+\phi(a_n)s^n)(\phi(b_0)+\phi(b_1)s+\ldots+\phi(b_m)s^m)
%         \\ &=\psi(f)\psi(g),
%     \end{align*}
%     as required. 
%     % LEWIS WILL DO UNIQUENESS
% \end{proof}

% Making this definition slightly less general and taking $\phi$ to be the identity map, we end up recovering $\psi$ as the conventional evaluation maps we have seen previously. If we think of polynomial rings as a pointed $R$-algebra, then there is a unique pointed algebra homomorphism from $R[X]$ to some $S$ which is the evaluation homomorphism.

% \begin{remark}
%     A pointed set is a set in which we fix a basepoint, so it is an ordered pair $(X, x_0)$ where $X$ is a set, in our case our basepoint is the element we use to define our evaluation homomorphism.
% \end{remark}

% So we can define a polynomial ring via a unique evaluation map, this tells us why it is okay to interchange the two notions that polynomials are either formal sums or functions. As mentioned previously we do have to be careful in the case of finite fields.

% \begin{example}
%   Consider the group $A=(\ZZ\times\ZZ, +)$. The set of
%   homomorphisms from $A$ to $A$ forms a ring under 
%   \[(f+g)(a)= f(a)+g(a)\]
%   \[(fg)(a) = f(g(a)) \, \, \forall a\in A\]
%   For all $f,g:A\to A$. A homomorphism from a group to itself is called an
%   endomorphism. The set of endomorphisms with these two operations is called the
%   endomorphism ring, called $\End(A)$. We claim this ring is noncommutative. Take as
%   an example $\phi:(m,n)\mapsto (0,n)$ (a ring homomorphism), and $\psi:(m,n)\mapsto
%   (m+n,0)$. Then note $\phi\psi:(m,n)\mapsto (0,0)$, and $\psi\phi:(m,n)\mapsto (n,0)$,
%   hence they are not the same: take $(\phi\psi)(0,1)=(0,0)\neq
%   (\psi\phi)(0,1)=(1,0)$. In fact, note that these two elements serve as a basis for the
%   whole ring -- we can get any $(m,n)$ by $m(1,0)+n(0,1)$.
% \end{example}

% Note that earlier in the course we talked about Automorphisms, which are really just
% invertible endomorphisms. And following the above example, it follows that
% $\Aut(A)=(\End(A))^{\times}$, the group of units of $\End(A)$.



Free groups are an essential concept in group theory, providing a fundamental example of how groups can be constructed. The idea is to start with a set \(S\) and create a group that has the least possible number of relations among elements of \(S\) needed to form a group.

A free group can be thought of as a group formed by all possible "words" created from elements of \(S\) and their inverses, subject only to the most basic requirements of a group. The crucial point is that no additional relations (like commutativity or specific element properties) are imposed beyond those necessary for a group structure.

\begin{definition}[Free Group - Constructive Approach]
Let \(S\) be a set. Form the set \(T\) where each element is the inverse of a unique element of \(S\), and vice versa. Consider the set of all "words" formed by concatenating elements from \(S \cup T\), including an empty word \(e\) as the identity.

The \emph{free group} on \(S\), denoted \(F_S\), consists of all such words reduced to their simplest form via the elimination of adjacent inverse pairs. The group operation is concatenation followed by this reduction, ensuring the group axioms are satisfied.
\end{definition}

\begin{remark}
The constructive definition provided above is intentionally broad and conceptual, aimed at offering an intuitive understanding of free groups. For our purposes, we will primarily utilize the definition based on the universal property, as it proves two key general theorems more efficiently. One can find a rigorous construction of free groups online.
\end{remark}


\begin{definition}[Free Group - Universal Property]
The free group \(F_S\) on a set \(S\) is characterized by the following universal property: for any function \(f\) from \(S\) to a group \(G\), there exists a unique group homomorphism \(\varphi\) from \(F_S\) to \(G\) making the following diagram commute:
\[
\begin{tikzcd}
S \arrow[r, hook, "i"] \arrow[rd, "f"'] & F_S \arrow[d, "\varphi"] \\
 & G
\end{tikzcd}
\]
Here, \(i\) is the inclusion map of \(S\) into \(F_S\).
\end{definition}

\begin{theorem}
    Every group $G$ is the quotient of a free group $F_X$ by some normal subgroup $N$.
\end{theorem}

\begin{proof}
    We know that by the universal property of free groups, $f$ extends to a unique $\varphi,$ where $\varphi:F_X \to G$ is a group homomorphism. Take $f: X \to G$ to be the canonical inclusion map, where $X$ is the underlying set of $G$, and define $\varphi(x) = f(x).$ Then $\varphi$ is clearly surjective onto $G.$ By the FIT, $F_X/\ker \varphi \cong G$. So, $N = \ker \phi$ satisfies the statement of the theorem.
\end{proof}

\begin{theorem}[Uniqueness of Free Groups]
    Let $F$ and $F'$ be free groups on the set $X$. Then $F \cong F'$
\end{theorem}

\begin{proof}
    Suppose \( F\) and \( F' \) are both free groups on the set \( X \). We want to show that \( F \) and \( F' \) are isomorphic.

By the universal property of $F$, for any function $\iota_2: X \to F'$, there exists a unique group homomorphism $\varphi_1: F \to F'$ such that $\varphi_1 \circ \iota_1 = \iota_2$, where $\iota_1: X \to F$ is the inclusion map. Similarly, for the universal property of $F'$ we get $\varphi_2 \circ \iota_2 = \iota_1$. This gives the following diagram.

\begin{center}
\begin{tikzcd}
& X \arrow[ld, "\iota_1"'] \arrow[rd, "\iota_2"] & \\
F \arrow[rr, "\varphi_1", bend left=25] & & F' \arrow[ll, "\varphi_2", bend left=25]
\end{tikzcd}
\end{center}

Substituting gives $\iota_1 = (\varphi_2 \circ \varphi_1) \circ \iota_1$ and $\iota_2 = (\varphi_1 \circ \varphi_2) \circ \iota_2$. So by the uniqueness of the universal property, $\varphi_2 \circ \varphi_1 = \id_{F}$ and $\varphi_1 \circ \varphi_2 = \id_{F'}$, so $\phi_1$ is an isomorphism and $F \cong F'$ as required. 
\end{proof}



